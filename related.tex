\chapter[Trabajos Relacionados]{Trabajos Relacionados}
\label{cap:related-work}

Como se ha mencionado a lo largo de esta tesis, el problema de identificar un conjunto suficiente de métodos generadores de objetos 
para un módulo dado es un desafío recurrente en el ámbito de la ingeniería de software. Este problema
aparece de forma recurrente en distintos análisis de programas, incluyendo, pero no limitado a, la verificación formal 
(\emph{model checking}) y la generación automática de tests.

En trabajos como \cite{DBLP:conf/tacas/NoriRTT09,DBLP:conf/tacas/KhurshidPV03}, en el contexto de verificación 
de software, y \cite{Tillmann:2010,Tillmann:2008,paPacheco07,DBLP:conf/icse/BraioneDMP18}, en el contexto de generación de 
casos de prueba automatizados, se aborda este mismo problema: determinar qué parte de una API debe ser utilizada como entrada 
para el análisis. En general, esta tarea se realiza de forma manual en la mayoría de los enfoques previos.

El uso de técnicas basadas en búsqueda para resolver problemas complejos en ingeniería de software es una estrategia cada vez 
más popular, con aplicaciones exitosas en generación de inputs para pruebas \cite{Fraser:2011}, reparación de programas 
\cite{DBLP:journals/tse/GouesNFW12}, entre otros. Hasta donde tenemos conocimiento, nuestra propuesta representa 
una aplicación novedosa de computación evolutiva en este tipo de contexto.

Un enfoque relacionado, aunque centrado en un problema distinto, es el implementado por la herramienta \textsf{SUSHI} 
\cite{DBLP:conf/icse/BraioneDMP18}. En este caso, se utiliza un algoritmo genético alimentado por condiciones de camino 
(\emph{path conditions}) generadas mediante ejecución simbólica, con el objetivo de construir inputs que las satisfagan, 
utilizando únicamente la API del módulo en cuestión. Este enfoque asume que el subconjunto relevante de métodos ya 
ha sido provisto, lo que contrasta con nuestra propuesta, que precisamente aborda el problema de identificar dicho subconjunto.

Nuestra técnica también requiere un mecanismo para identificar métodos \emph{observadores}, lo cual resolvimos realizando un análisis estático con
la herramienta \emph{Infer}. Si bien existen enfoques previos para identificar observadores, o más precisamente métodos \emph{puros} \cite{Huang:2012,Salcianu:2005}, es importante destacar que el 
foco de nuestro algoritmo evolutivo no está puesto en esa tarea, sino en la construcción de un conjunto mínimo y suficiente 
de métodos generadores de objetos.
Además, nuestro enfoque es independiente del mecanismo utilizado para identificar observadores o métodos puros, por lo que podría 
complementarse con trabajos previos que abordan específicamente esa problemática.

Una vez identificado automáticamente un conjunto suficiente de métodos relevantes —tanto generadores como observadores— es posible aplicar técnicas de generación 
exhaustiva acotada sin necesidad de que el desarrollador proporcione manualmente especificaciones formales. Este paso de identificación previa es clave en nuestra 
propuesta, ya que habilita la integración fluida de enfoques como \textsf{BEAPI} sobre software real sin intervención adicional. 

Los enfoques de generación exhaustiva acotada (BEG, por sus siglas en inglés) han demostrado ser efectivos para alcanzar una alta cobertura de código y para 
encontrar errores, tal como se reporta en diversos trabajos previos \cite{Marinov01,Khurshid01,Boyapati02,Sullivan04}. El segundo objetivo de esta tesis no es 
revisar nuevamente la efectividad de las suites BEG, sino presentar un enfoque que resulte sencillo de aplicar sobre software existente, ya que no requiere 
la escritura manual de especificaciones formales sobre las propiedades de las entradas (por ejemplo, funciones \texttt{repOK}).

Se han propuesto diferentes lenguajes para describir formalmente las restricciones estructurales en BEG, incluyendo la lógica relacional de Alloy (en el llamado 
estilo declarativo), utilizada por la herramienta \textsf{TestEra} \cite{Marinov01}; y el código fuente en un lenguaje imperativo (en el estilo operacional), como en 
\textsf{Korat} \cite{Boyapati02}. El estilo declarativo tiene la ventaja de ser más conciso y más simple para quienes están familiarizados con él; sin embargo, 
este conocimiento no es común entre desarrolladores. El estilo operacional puede ser más extenso, pero como las especificaciones y el código están escritos en el 
mismo lenguaje, suele ser el preferido por quienes desarrollan software.

\textsf{UDITA} \cite{Gligoric10} y \textsf{HyTeK} \cite{Rosner14} proponen emplear una combinación de los estilos declarativo y operacional para escribir las 
especificaciones, dado que ciertas restricciones pueden ser más fáciles de expresar 
en uno u otro estilo. Con especificaciones precisas, ambos enfoques pueden ser utilizados para BEG. Sin embargo, para utilizarlos, los desarrolladores deben 
estar familiarizados con ambos estilos de especificación, y dedicar el tiempo y esfuerzo necesarios para escribir dichas especificaciones.

Verificadores de modelos como \textsf{Java Pathfinder} \cite{Visser05} (\textsf{JPF}) también pueden realizar BEG, pero el usuario debe proporcionar manualmente un 
\emph{driver} para la generación: un programa que el verificador de modelos utiliza para construir las estructuras que luego se pasan al sistema bajo prueba. 
Escribir un driver para BEG a menudo implica invocar rutinas de la API en combinación con operadores no determinísticos propios de \textsf{JPF}, por lo que 
el desarrollador debe familiarizarse con estos operadores y realizar trabajo manual adicional para aplicar este enfoque.
Además, \textsf{JPF} se ejecuta sobre una máquina virtual personalizada en lugar de la máquina virtual estandar de Java (\textsf{JVM}), lo cual introduce una 
sobrecarga significativa en su ejecución comparado con enfoques que utilizan la \textsf{JVM} estándar, como \textsf{BEAPI}. Los resultados de un estudio previo 
\cite{Siddiqui09} muestran que \textsf{JPF} es significativamente más lento que \textsf{Korat} en tareas de BEG. En dicho estudio, \textsf{Korat} fue presentado 
como el enfoque más rápido y escalable al momento de su publicación, lo cual se explica en parte por su eficaz poda del espacio de búsqueda y la eliminación 
de estructuras isomorfas. En contraste, \textsf{BEAPI} no requiere una función de especificación, \texttt{repOK} y opera exclusivamente mediante invocaciones a la API.

Un tipo alternativo de BEG consiste en generar todas las entradas necesarias para cubrir todos los caminos factibles (y acotados) de ejecución del programa.
Esta es la idea detrás de la generación de tests dinámica (también llamado concolic test), una variante de la ejecución 
simbólica \cite{Godefroid18}. Este enfoque ha sido implementado en diversas herramientas \cite{Godefroid12,Godefroid05,Pham19,Christakis15}, y ha logrado 
con éxito generar suites con alta cobertura de código, detectar errores reales en programas y demostrar propiedades de los programas, como seguridad de memoria.
\textsf{Kiasan} \cite{Deng06} y \textsf{FAJITA} \cite{Abad13} también son enfoques de generación de pruebas de tipo \emph{white-box}, que requieren especificaciones 
formales y tienen como objetivo cubrir el comportamiento del sistema bajo prueba.

La técnica de linearización ha sido empleada previamente para eliminar estructuras isomorfas en verificadores de modelos tradicionales \cite{Iosif02,Robby03}, y también 
en verificadores de software como \textsf{JPF} \cite{Visser06}. Un estudio anterior experimentó el uso de equivalencia de estados en \textsf{JPF}, y propuso varias técnicas 
para podar el espacio de búsqueda de entradas mediante linearización, tanto para ejecución concreta como simbólica \cite{Visser06}. 
Como se indicó previamente, la ejecución concreta en \textsf{JPF} requiere que el usuario proporcione un \emph{driver}, 
mientras que el enfoque simbólico busca cubrir caminos del sistema; en cambio, nuestro trabajo se enfoca en la 
generación exhaustiva de entradas válidas (BEG). 
La linearización también ha sido utilizada con éxito para la minimización de suites de pruebas \cite{Xie04}.