\chapter[Introducción]{Introducción}
\label{cap:introduccion}

% Esqueleto de la intro general de la tesis
% Contexto: Explayarse sobre por qué los análisis de software son importantes. Meter énfasis en el testing y la verificación
% El análisis de software es una tarea crucial en el campo de la Ingeniería de
% Software \cite{}. Por ejemplo, la generación automática de tests \cite{} y la verificación \cite{} son muy efectivos para encontrar defectos en el software.


En la actualidad, el software está presente en casi todos los aspectos de la vida diaria, desde sistemas críticos en la industria aeronáutica y en salud, hasta aplicaciones móviles de uso cotidiano \cite{} \pp{Libros de ingenieria de software} La evolución de la tecnología y su integración en nuestra sociedad han hecho que la calidad del software sea un factor clave en el desarrollo tecnológico. La ingeniería de software es una disciplina fundamental en el desarrollo y construcción de sistemas de software complejos y de alta calidad \cite{} \pp{Libros de ingenieria de software}. 
%A medida que la tecnología ha avanzado, el software se ha vuelto omnipresente en nuestra vida cotidiana. 
A medida que el software se vuelve ubicuo gracias a los rápidos avances en tecnología, garantizar la corrección funcional del software es más crucial que nunca. Sin embargo, a menudo sucede que el software desarrollado es defectuoso, lo que causa distintos tipos de inconvenientes a los usuarios finales.
Además, la creciente demanda de sistemas de software crítico ha llevado a la aparición de diversos desafíos y problemas en su desarrollo y mantenimiento.

Para abordar estos problemas, un área de investigación de creciente importancia es la del análisis automático de software, cuyo objetivo es la provisión de herramientas para asistir a los programadores en diversas tareas
%la identificación de defectos en el software. 
La generación automática de tests \cite{Ponzio:2016, Rosner15, Abad13,
Galeotti:2010, Khalek:2011, Pasareanu:2010, Gligoric10, Tillmann:2008,
Pacheco07}, los \emph{model checkers} para software \cite{Visser06, Visser05,
Clarke:2004}, y los análisis estáticos \cite{Calcagno:2011, Itzhaky:2014}, entre
otros, son enfoques destacados en esta línea de investigación. En particular,
los enfoques de generación automática de tests tienen como objetivo asistir a los desarrolladores 
en diversas tareas cruciales del testing de software \cite{Ammann16,Myers11},
como la generación automática de entradas para los tests \cite{Cadar08,Christakis15,Iosif02,Luckow14,Fraser11}, 
y la detección y reporte automáticos de fallas \cite{Pacheco07, Ma15 , Godefroid05, Marinov01, Boyapati02, Godefroid12}.

Aunque estas técnicas implican en muchos casos análisis completamente automatizados, su aplicación suele requerir cierto esfuerzo por parte de los ingenieros de software. Los \emph{model checkers} para software dependen de la definición de \emph{drivers}, programas que permiten construir entradas para el código bajo análisis. De manera similar, en los enfoques de pruebas unitarias parametrizadas \cite{Tillmann:2010} es obligatorio un mecanismo para construir entradas. Algunas herramientas basadas en la ejecución simbólica requieren las llamadas "fábricas de objetos" para construir casos de prueba que incluyan entradas con tipos no primitivos \cite{Tillmann:2008}. Las técnicas automatizadas de generación de pruebas basadas en la API de un módulo pueden usarse para construir entradas de tipos no primitivos \cite{Pacheco07, Fraser11}, automatizando así los problemas de generación de entradas antes mencionados. Sin embargo, suelen presentar dificultades para generar un buen conjunto de entradas diversas para módulos que manipulan estructuras dinámicas complejas (almacenadas en el heap). Esto es aún más difícil para módulos con APIs ricas \cite{Ponzio:2018}. Muchos autores han abordado este problema definiendo diferentes enfoques para guiar la generación de pruebas y crear conjuntos más diversos de entradas \cite{Ponzio:2018, Ciupa:2008}.

En este trabajo, se proponen dos enfoques diferentes para abordar el problema de generar mejores entradas para módulos con estados dinámicos complejos. Por un lado, observamos que la selección de rutinas de la API de módulo, para alimentar una herramienta de generación de entradas y construir estructuras para el análisis de programas (drivers para model checking de software, entradas para tests unitarios, o para tests parametrizados, etc.), tiene un impacto crucial en el análisis. Por esta razón, definimos el concepto de \emph{métodos generadores de objetos} como un subconjunto \(B\) de métodos, extraídos de la API de un módulo \(M\), que pueden emplearse para crear cualquier entrada para \(M\) en un análisis automatizado de programas  (por ejemplo, un driver para model checking de software). 
Claramente, cuanto mayor sea el número de estructuras diferentes que se puedan crear con \(B\), mayores serán las posibilidades de encontrar errores en \(M\). Decimos que un conjunto de métodos generadores de objetos es suficiente si estos pueden usarse en combinación para construir todas las estructuras posibles de \(M\). Por lo tanto, los generadores de objetos suficientes son la mejor opción posible para encontrar errores. Cabe notar que \(B\) puede contener rutinas superfluas. Una rutina superflua \(s\) es aquella tal que los objetos del módulo pueden construirse usando las rutinas en $B - \{s\}$ (el ejemplo más simple son las rutinas que nunca cambian el estado de los objetos, también denominados métodos puros u observadores \cite{}). Llamamos \emph{minimal} a un conjunto de generadores de objetos \(B\) que no contiene rutinas superfluas. La minimalidad es importante porque proporcionar rutinas superfluas a una herramienta de análisis automático suele impactar negativamente en su eficiencia (la cantidad de formas posibles de combinar \(k\) rutinas usualmente aumenta exponencialmente con respecto a \(k\)). 

Identificar un conjunto suficiente y minimal de métodos generadores de objetos de forma manual requiere de un análisis minucioso de las rutinas disponibles y una comprensión profunda de la semántica del módulo. Esto es especialmente difícil para módulos con APIs ricas, donde hay muchas rutinas y mucha redundancia en la API. Para facilitar la tarea del desarrollador, en este trabajo se proponen distintos enfoques automáticos para identificar conjuntos suficientes y minimales de métodos generadores de objetos, con distintas características. Exploramos aquí el uso de un algoritmo evolutivo y de un algoritmo codicioso (basado en hill climbing), con diferentes funciones objetivo que utilizan técnicas de generación automáticas de test. Además de buscar generadores de objetos suficientes y minimales, nuestros enfoques también tienen en cuenta otras características de los métodos, como la cantidad y la complejidad de sus parámetros, de modo que se favorecen las rutinas “más simples” en la búsqueda. El objetivo es elegir métodos generadores que puedan ser utilizados de manera más fácil y eficiente en los análisis de programas.

Dado que el número de instancias de un módulo de software es potencialmente infinito,
% y los análisis de programas que abordamos también están limitados en cuanto al número de estructuras que pueden emplear, 
nuestros algoritmos se limitan a explorar conjunto acotado de estructuras para \(M\) \cite{Boyapati02} (por ejemplo, todas las instancias de una lista enlazada con hasta \(k\) nodos). El valor de la función objetivo para un conjunto de rutinas \(R\) se basa en el número de estructuras acotadas que se pueden generar utilizando combinaciones de estas rutinas, mediante técnicas automáticas de generación de tests. Experimentamos con dos formas distintas de computar la función objetivo, una se basa en una modificación de una herramienta de generación de casos de prueba aleatorios (Randoop \cite{ref24}) para generar la mayor cantidad posible de estructuras acotadas a partir de \(R\), permitiendo como máximo \(k\) objetos de cada tipo en las estructuras (por ejemplo, todas las instancias de listas enlazadas con hasta \(k\) nodos). \pp{Esto es así Cacho?} Otra se basa en una técnica eficiente de generación exhaustiva acotada basada en la API, denominada BEAPI, que definimos en este trabajo (ver más adelante).
Nuestra evaluación experimental muestra que usando cotas razonables, los algoritmos definidos son eficientes, y computan métodos generadores de objetos suficientes y minimales con precisión (siempre computan generadores de objetos suficientes, y en la mayoría de los casos los generadores de objetos identificados son minimales, o tienen muy pocos métodos superfluos).


% TODO: Introducir BEAPI
Por otro lado, se observa que muchos de los enfoques de generación automática de
entradas actuales incluyen componentes aleatorios, los cuales evitan realizar una exploración 
sistemática del espacio de entradas, pero logran una mayor eficiencia en la generación \cite{Pacheco07, Ma15, Fraser11}. 
Aunque estos enfoques han sido útiles para encontrar un gran número de errores en software, podrían no 
explorar ciertos comportamientos defectuosos debido a su naturaleza aleatoria.

Enfoques alternativos buscan explorar sistemáticamente un gran número de
ejecuciones del software bajo test, 
%(\emph{SUT}, por sus siglas en inglés), 
con el objetivo de proporcionar mayores garantías sobre la ausencia de errores
en el código \cite{Marinov01, Boyapati02, Godefroid05, Godefroid18, Cadar08, Luckow14}. 
Algunos de estos enfoques se basan en la generación exhaustiva acotada \cite{Marinov01, Boyapati02}, 
que consiste en generar todas las entradas que pueden construirse utilizando dominios de datos acotados. 
Típicamente, los enfoques de generación exhaustiva acotada se usan para testear
módulos que manipulan estructuras de datos dinámicas complejas (por ejemplo, listas enlazadas, árboles, etc.). 
Usualmente, las estructuras del módulo deben satisfacer ciertas restricciones,
conocidas como el invariante de representación (denominado \textit{repOK})). Este invariante 
describe las propiedades que deben satisfacer las estructuras válidas del
módulo. Además, los métodos del módulo pueden contar con precondiciones, que
describen las entradas aceptables para la ejecución los métodos (adicionalmente,
\textit{repOK} debe cumplirse antes de la ejecución de cada método). 
Los enfoques más utilizados de generación exhaustiva acotada requieren que el
usuario proporcione una especificación formal de las restricciones que las
entradas de un método debe satisfacer \cite{Marinov01, Boyapati02} (típicamente,
la precondición y el invariante de representación). En particular, Korat
requiere que la especificación se provea como un método booleano en Java, que 
retorna true si las estructuras satisfacen la especificación, o falso en caso
contrario.
Dada la especificación y límites en el tamaño de los dominios de datos
\cite{Marinov01, Boyapati02} (a menudo denominados \textit{scopes}), los
enfoques de generación exhaustiva acotada generan todas las entradas acotadas
que satisfacen la especificación.

Varios estudios demuestran que los enfoques de generación exhaustiva acotada son
efectivos para revelar fallas en el software \cite{Marinov01, Khurshid01,
Boyapati02, Sullivan04}. Además, la hipótesis de la cota pequeña (\textit{small
scope hypothesis}) \cite{Andoni02}, que establece que la mayoría de las fallas
en el software pueden revelarse ejecutando el programa con ``datos de entrada
pequeños'', sugiere que los enfoques de generación exhaustiva acotada deberían
descubrir la mayoría de las fallas (si no todas), si se utilizan \textit{scopes}
lo suficientemente grandes. El desafío que enfrentan los enfoques de generación
exhaustiva acotada es cómo explorar eficientemente un espacio de búsqueda
enorme, que a menudo crece exponencialmente con respecto a los \textit{scopes}.
El espacio de búsqueda típicamente incluye una gran cantidad de entradas
inválidas (que no satisfacen la especificación), y entradas isomorfas \cite{15, 28} \pp{citas}. 
Por lo tanto, podar partes del espacio de búsqueda que involucran datos de entrada 
inválidos e isomorfos (redundantes) es clave para que estos enfoques sean eficientes y 
escalables en la práctica \cite{Boyapati02}.

Sin embargo, escribir especificaciones formales adecuadas para la generación
exhaustiva acotada (e.g. para Korat) es una tarea desafiante y que consume mucho tiempo. Las
especificaciones deben capturar con precisión las restricciones sobre las
entradas (propiedades del invariante de representación, precondición, etc.). 
Las especificaciones demasiado restrictivas pueden llevar a no generar
algunas entradas válidas, lo cual podría hacer que el testing posterior no explore 
comportamientos defectuosos del software. Por otro lado, las especificaciones
poco restrictivas pueden llevar a la generación de entradas inválidas, lo que
podría producir falsas alarmas durante el testing. Además, el usuario necesita 
tener en cuenta la forma en que opera el enfoque de generación y escribir las especificaciones 
siguiendo reglas específicas para lograr una buena eficiencia en la generación \cite{Boyapati02}. 
Por último, las especificaciones formales requeridas por estas herramientas raramente están disponibles 
en el software actual, lo que dificulta la usabilidad de los enfoques de generación exhaustiva 
acotada basados en especificaciones.

En este trabajo, definimos un nuevo enfoque para la generación exhaustiva
acotada, denominado BEAPI, que funciona realizando llamadas a métodos de la API
del módulo bajo test. De manera similar a los enfoques de generación de tests basados en API \cite{Pacheco07, Ma15,Fraser11}, BEAPI genera secuencias de invocaciones a métodos de la API, denominadas secuencias de test. 
La ejecución de cada secuencia de test generada por BEAPI produce una entrada en el conjunto resultante de objetos exhaustivo acotado. 
Como es habitual en la generación exhaustiva acotada, BEAPI requiere que el
usuario proporcione los scopes para la generación, que en BEAPI incluyen cotas
en el tamaño de las estructuras a generar, y valores primitivos para invocar a
los métodos de la API (ver Capítulo \ref{}). 

Un enfoque de fuerza bruta intentaría generar todas las secuencias de test con
hasta $m$ métodos, donde $m$ es la longitud máxima permitida para las secuencias de test, 
dado por el usuario. Este enfoque es intrínsecamente combinatorio, y agota los recursos 
computacionales antes de su finalización incluso para valores de $m$ muy pequeños. 
Si BEAPI se basa en la creación de secuencias de tests a partir de métodos de la
API, definimos varias técnicas de poda que son cruciales para la eficiencia de
la generación, y que permiten que BEAPI escale a scopes significativamente mayores. 

Primero, BEAPI ejecuta las secuencias de test y descarta aquellas que corresponden 
a violaciones de las reglas de uso de la API \cite{Pacheco07} (por ejemplo, lanzar 
excepciones que indican un uso incorrecto de la API, como IllegalArgumentException en 
Java \cite{Liskov00, Pacheco07}). 
Segundo, BEAPI implementa la coincidencia de estados (\emph{state matching})
\cite{Iosif02, Politano20, Xie04} para descartar secuencias de test que
producen estructuras ya creadas por secuencias previamente exploradas. 
Tercero, BEAPI emplea solo un subconjunto de los métodos de la API para crear
secuencias de test: un conjunto de métodos identificados automáticamente como
generadores de objetos \cite{Ponzio19} (definidos anteriormente). 
Previo a la generación de tests, BEAPI ejecuta un enfoque automatizado de
identificación de métodos generadores \cite{Ponzio19}, en busca de un
subconjunto pequeño de métodos de la API que sea suficiente para generar el conjunto 
de entradas exhaustivas acotadas (esto es, generadores de objetos suficientes y
minimales). 

Una ventaja de BEAPI respecto de los enfoques basados en especificaciones es que
no requiere especificaciones complejas que describan con precisión las
propiedades de las entradas válidas (por ejemplo, no requiere de la definición
de un invariante de representación). En cambio, en BEAPI 
requiere un esfuerzo mínimo de especificación en la mayoría de los casos (incluidos nuestros estudios de estudios), que consiste en 
hacer que los métodos de la API lancen excepciones en presencia de entradas
inválidas. Este estilo de programación se denomina "programación defensiva", 
fue popularizado por Liskov \cite{Liskov00}, y es ampliamente recomendado para el desarrollo de APIs robustas. 
Otra ventaja de BEAPI respecto a los enfoques basados en especificaciones es que produce secuencias de test para crear las entradas correspondientes utilizando métodos de la API, lo que facilita la creación de tests, y mejora la legibilidad y mantenibilidad de los mismos \cite{Braione17}. Esto es en contraste con las técnicas existentes, en las que las estructuras son codificadas directamente en los tests (como grafos) usando mecanismos de bajo nivel del lenguaje de programación (como reflexión en Java \cite{} \pp{algo de reflection}).


\pp{TODO: Evaluación experimental de builders cuando tengamos la sección
    experimental}
Evaluamos nuestro enfoque experimentalmente en un conjunto de clases Java con estados complejos, extraídas de la literatura. Los resultados muestran que, en nuestros casos de estudio, nuestro enfoque identifica conjuntos de rutinas que son suficientes y mínimos, en un tiempo aceptable \pp{Cuánto en promedio?}. También evaluamos el impacto de nuestro enfoque en un análisis automatizado, concretamente, en la generación de casos de prueba para pruebas parametrizadas \pp{No hicimos esto según entiendo}. Comparamos cómo se comporta la herramienta de generación de casos de prueba aleatorios Randoop cuando se le proporciona la API completa del módulo, frente a proporcionarle únicamente los constructores identificados por nuestro enfoque. Los resultados indican que, en este último caso, Randoop generó más objetos (y más grandes), dentro de un tiempo límite fijo.


\pp{TODO: Evaluación experimental de BEAPI cuando tengamos la sección
    experimental}
Evaluamos experimentalmente BEAPI y mostramos que su eficiencia y escalabilidad son comparables a las del mejor enfoque exhaustivo acotado actual (Korat), sin necesidad de proveer especificaciones formales para la estructuras (repOKs). También mostramos que BEAPI puede ser útil para encontrar fallos en los repOKs, comparando los conjuntos de entradas generados por BEAPI utilizando la API con los conjuntos de entradas generados por Korat a partir de un repOK. Usando este procedimiento, encontramos varios errores en los repOKs empleados en la evaluación experimental de herramientas relacionadas, proporcionando así evidencia sobre la dificultad de escribir repOKs (en particular, los repOKs precisos requeridos por las herramientas de generación acotada).


% Contribuciones:
% 1- Definición y cómputo automático de builders

% Cuales son los análisis que se podrían beneficiar de los builders y por qué

% Definición intuitiva de builders
% Introducción de cómputo automático de builders

% 2- BEAPI
% Importancia de técnicas exhaustivas acotadas

% Por que me conviene generar a partir de la API

% Descripción breve de la técnica


% Breve resumen de los resultados
