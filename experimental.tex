%!TEX root = main.tex
\chapter[Evaluaci\'on]{Evaluaci\'on}
\label{cap:experimental}


En los capítulos \ref{cap:builders} y \ref{cap:beapi} presentamos enfoques para
identificar automáticamente un conjunto de métodos generadores de objetos, y una
técnica novedosa de generación exhaustiva acotada basada en la API. 
En este capítulo, realizamos una evaluación experimental de las técnicas
mencionadas. La sección \ref{sec:experimentalIdentificacion} analiza experimentalmente los
algoritmos de identificación de generadores de objetos, mientras que en la
sección \ref{sec:experimentalBeapi} se evalúa la generación exhaustiva acotada basada en la API.


\section{Algoritmos de identificación de métodos generadores de objetos}
\label{sec:experimentalIdentificacion}

En esta sección, evaluamos experimentalmente los enfoques presentados en el
Capítulo~\ref{cap:builders}. Analizaremos la eficiencia y precision de cada
algoritmo utilizando las funciones de valoración introducidas en la
Sección~\ref{sec:fitness}. Además, mostraremos cómo los métodos identificados
por nuestras técnicas pueden ser aprovechados en el contexto de herramientas
de verificación de software.

Con respecto a esta técnica, las siguientes preguntas de investigación guían esta experimentación:

\begin{itemize}
\item \emph{RQ1}: ¿Qué tan eficientes son los algoritmos propuestos para
    identificar conjuntos de métodos generadores de objetos?
\item \emph{RQ2}: ¿Qué tan precisos son los algoritmos presentados para identificar métodos generadores de objetos?
\item \emph{RQ3}: ¿Cuál es el impacto de utilizar métodos generadores de objetos
    en el contexto del análisis automático de software?
\end{itemize}

\subsection{Configuración experimental}
\label{sec:experimentalIdentificacionConfig}
La evaluación se llevó a cabo sobre un conjunto de implementaciones de
estructuras de datos que sirven como casos de estudio de referencia,
incluyendo: \verb"NCL" de Apache Commons Collections~\cite{apache};
\verb"BinaryTree", \verb"BinomialHeap" y \verb"FibonacciHeap", extraídos
de~\cite{Visser:2006}; y \verb"UnionFind", una implementación de conjuntos
disjuntos tomada de JGrapht~\cite{jgrapht}.

También se incluyeron componentes de proyectos reales de software, tales como
\verb"Lits" de la implementación de Sat4j~\cite{sat4j}, utilizada previamente
en~\cite{Loncaric:2018}. Esta clase representa un conjunto de variables
booleanas, encargado de registrar cuándo se realizó la última asignación y si
existen observadores monitoreando su estado.

Por otro lado, se consideraron estructuras adicionales como \verb"Scheduler",
un planificador de procesos tomado del conjunto de benchmarks SIR~\cite{sir},
y varias colecciones del paquete estándar \verb"java.util" de Java%
\footnote{\url{https://docs.oracle.com/javase/8/docs/api/java/util/
package-summary.html}}, incluyendo \verb"TreeMap", \verb"TreeSet",
\verb"HashMap",\\
 \verb"HashSet" y \verb"LinkedList".

Para evaluar la precisión de nuestros algoritmos, se construyó manualmente un
conjunto de referencia o \emph{ground truth} que contiene los métodos
generadores considerados mínimos y suficientes en cada caso de estudio. Esta
tarea implicó un análisis detallado y manual de cada API, lo que representó un
trabajo considerable, especialmente en los casos más complejos. La
Tabla~\ref{tab:groundTruth} resume este conjunto de referencia, indicando para
cada clase el número total de métodos en su API pública (\#API) y los métodos
identificados manualmente como generadores de objetos. Cabe destacar que en algunos
casos ciertos métodos pueden ser intercambiables (por ejemplo,
\texttt{addFirst} y \texttt{addLast} en \texttt{NCL}).

Todos los experimentos fueron ejecutados en una máquina con procesador Intel
Core i7-6700 a 3.4\,GHz y 8\,GB de RAM, corriendo el sistema operativo
GNU/Linux.

\begin{table}[t!]
\centering
{\scriptsize
\begin{tabular}{l l}
\hline
&Métodos generadores de objectos  \\
\hline
\multirow{2}{*}{\textbf{NCL}} 
 & NCLinkedList(int)  \\
 & addFirst(Object)    \\
 {\scriptsize \#API: 34} & removeFirst()  \\
\hline

\multirow{2}{*}{\textbf{UFind}} 
 & UnionFind()  \\
 & addElement(int)   \\
 {\scriptsize \#API: 9} & union(int,int)   \\
\hline

\multirow{2}{*}{\textbf{FHeap}} 
 & FibonacciHeap()  \\
 & insert(int)  \\
 {\scriptsize \#API: 7} & removeMin()  \\
\hline
\multirow{1}{*}{\textbf{BTree}} 
 & BinTree()  \\
 {\scriptsize \#API: 7} & add(int) \\
\hline

\multirow{1}{*}{\textbf{BHeap}} 
 & BinomialHeap()  \\
 {\scriptsize \#API: 10} & insert(int) \\
 & decreaseKeyValue(int,int)   \\
\hline

\multirow{5}{*}{\textbf{Lits}} 
 & Lits() \\
 & getFromPool(int) \\
 & satisfies(int)  \\
 & forgets(int) \\
 {\scriptsize \#API: 26} & setLevel(int,int)  \\
 & setReason(int)\\
\hline

\multirow{3}{*}{\textbf{Sched.}} 
 & Schedule() \\
 & addProcess(int) \\
{\scriptsize \#API: 10} & blockProcess() \\
%  & quantumExpire() & \\
%   & finishProcess() & \\

\hline

\multirow{1}{*}{\textbf{LinkedList}} 
 & LinkedList() \\
 {\scriptsize \#API: 67} & addFirst(Object)  \\
 \hline

\multirow{2}{*}{\textbf{TreeMap}} 
 & TreeMap() \\
 & put(Object,Object) \\
{\scriptsize \#API: 61} & remove(Object) \\
\hline

\multirow{2}{*}{\textbf{TreeSet}} 
 & TreeSet() \\
 & add(Object) \\
{\scriptsize \#API: 34} & remove(Object) (int) \\
\hline

\multirow{1}{*}{\textbf{HashSet}} 
 & HashSet(int,float) \\
 {\scriptsize \#API: 31} & add(Object) \\
\hline

\multirow{1}{*}{\textbf{HashMap}} 
 & HashMap(int,float)  \\
{\scriptsize \#API: 45} & put(Object,Object)  \\
\hline

\end{tabular}%
}
\caption{Ground truth de métodos generadores por clase y la cantidad de métodos en la API de cada clase.}
\label{tab:groundTruth}
\end{table}


% Como se discutió en el Capítulo~\ref{cap:beapi}, nuestros algoritmos requieren
% la configuración de diversos parámetros que influyen en su eficiencia y
% efectividad. En particular, los parámetros más relevantes para el algoritmo
% genético incluyen la tasa de mutación, la tasa de cruce, el tamaño de la
% población y la estrategia de selección utilizada.

Nuestros algoritmos requieren la configuración de diversos parámetros que influyen en su eficiencia y
efectividad. En particular, los parámetros más relevantes para el algoritmo
genético ((Sección~\ref{sec:approachGA})) incluyen la tasa de mutación, la tasa de cruce, el tamaño de la
población y la estrategia de selección utilizada.

Para las evaluaciones que utilizan la función de valoración basada en Randoop
(Sección~\ref{sec:fitnessRandoop}), es importante considerar el tiempo
disponible para la generación de objetos y el valor de la semilla aleatoria.
Randoop utiliza esta semilla para controlar la generación de pruebas y
permitir la reproducción de resultados. Dado que Randoop introduce un grado
considerable de aleatoriedad, optamos por utilizar una semilla aleatoria
diferente en cada ejecución para evaluar la robustez de nuestro enfoque bajo
distintos escenarios.

En el caso de la función de valoración basada en generación exhaustiva
(Sección~\ref{sec:fitnessGE}), es crucial definir el \emph{scope}, es decir, la
cantidad máxima de objetos por clase que se pueden generar. Además, se omiten
ciertos campos internos de las clases bajo análisis que no afectan la
estructura lógica de los objetos generados, tales como el campo
\texttt{modCount}, común en muchas colecciones de Java, que simplemente cuenta
el número de modificaciones internas pero no altera la estructura observada
del objeto.

Los valores de configuración utilizados en los experimentos fueron determinados empíricamente mediante un proceso de prueba y error,
como es habitual en el diseño y ajuste de algoritmos genéticos. 
Se seleccionaron aquellos parámetros que ofrecieron los mejores resultados en términos de eficiencia y estabilidad de las soluciones encontradas. 
Para el algoritmo genético (descrito en la Sección~\ref{sec:fitnessGE}), se utilizó una tasa de cruce de 0.4 y una tasa de mutación de 0.05. 
La estrategia de selección adoptada fue la de torneo, con cuatro participantes por torneo, y se empleó una población de 100 individuos, 
evolucionando durante 20 generaciones por ejecución.

En lo que respecta a la función de valoración basada en Randoop (descrita en la Sección~\ref{sec:fitnessRandoop}),
se configuró un tiempo máximo de ejecución de 30 segundos por candidato, 
y se emplearon dos semillas distintas en cada ejecución para capturar la variabilidad introducida por 
el carácter aleatorio del generador. Esta configuración permite obtener una medida más robusta del desempeño 
de los algoritmos en contextos con elementos no determinísticos.

Por otro lado, en la función de valoración basada en generación exhaustiva (también discutida en la Sección~\ref{sec:fitnessGE}), 
se estableció un scope de 5, es decir, el número máximo de objetos de cada tipo que se generan para evaluar cada conjunto candidato de métodos
constructores. Además, se ignoraron campos de las clases evaluadas que no alteran la estructura lógica de los objetos generados, tales como el campo \texttt{modCount}.

Para cada combinación de caso de estudio y función de valoración, el algoritmo
genético fue ejecutado cinco veces con los parámetros mencionados. Los
resultados reportados en las siguientes secciones corresponden al promedio de
estas ejecuciones. En el caso particular de la función de valoración basada en
generación exhaustiva, no fue necesario realizar múltiples repeticiones, dado
que dicha evaluación es determinística bajo una misma configuración.


%Casos de estudio, técnicas a evaluar, métricas, parámetros usados, ground truth, etc..



\subsection{RQ1: Eficiencia del cómputo de generadores de objetos}


La eficiencia de nuestros algoritmos es la cantidad de segundos que le lleva a cada uno de nuestros algoritmos con cada función de 
valoración para encontrar el subconjunto de métodos generadores de objetos que sean suficientes. Para evaluar la eficiencia de nuestro algoritmo, 
llevamos a cabo experimentos utilizando conjuntos de datos de muestra y medimos el tiempo de promedio de las 5 ejecuciones necesarias 
para obtener las soluciones. Comparamos los resultados obtenidos con la técnica evolutiva (\emph{GA}) y con la técnica greedy (\emph{HC}) 
para evaluar la eficiencia de ambos enfoques en función de la función de valoración utilizada. Los resultados se presentan en la Tabla \ref{tab:eficiencia}.

Como se observa en la tabla, el algoritmo Hill Climbing (\emph{HC}) es más efectivo que el algoritmo genético (\emph{GA}) para calcular los constructores de métodos en todos los casos de estudio considerados. 
El enfoque de Hill Climbing es razonablemente eficiente, tardando solo 20 minutos en el peor de los casos (NCL, debido a la cantidad y complejidad 
de su API), mientras que el algoritmo genético puede tardar mas de una hora en el peor de los casos. Esto se debe a que los algoritmos 
de Hill Climbing comienzan desde abajo hacia arriba, considerando menos métodos antes de considerar más métodos, a diferencia de los 
algoritmos genéticos, que generan sucesores mediante operadores de cruce y mutación. Esta estrategia favorece al algoritmo Hill Climbing 
para encontrar los métodos mínimos en tiempos más cortos. Cabe mencionar que en casos de estudio donde el algoritmo genético supera al 
Hill Climbing, se debe a que la clase bajo prueba tiene pocos métodos, lo que permite que los algoritmos evolutivos converjan más 
rápidamente que el algoritmo Hill Climbing.


En relación con las funciones de valoración, observamos que la función de valoración basada en el generador exhaustivo (GE) logra 
finalizar la búsqueda de métodos generadores en mucho menos tiempo en comparación con la función de valoración basada en la 
cobertura generada por la suite de pruebas modificada del Randoop (RC). Esto se aplica a todos los algoritmos implementados. La función 
de valoración GE converge antes en los algoritmos debido a que genera objetos hasta el alcance especificado (en este caso, 5 para todos 
los casos) y luego finaliza la ejecución del candidato actual para continuar con los siguientes. Todas tienen un \emph{timeout} de 
30 segundos para el caso de pueda generar muchas estrucutras con el scope dado. Esto permite una convergencia más rápida en comparación 
con la función de valoración basada en Randoop, que agota su presupuesto de tiempo para cada candidato ejecutado en el algoritmo. Es 
importante destacar que todos los algoritmos se ejecutan con 10 hilos simultáneamente, evaluando así 10 candidatos a la vez para cada 
algoritmo.

En conclusión, en términos de eficiencia de nuestro algoritmo, para los casos estudiados en esta tesis, es conveniente utilizar el 
algoritmo Hill Climbing con una función de valoración que cuente los objetos generados por el generador exhaustivo (\emph{GE}).
\setlength{\tabcolsep}{4pt} 

\begin{table}[H]
\centering
\scriptsize
\begin{tabular}{cccccc}
\hline
\multicolumn{2}{c}{\textbf{Casos}} & \multicolumn{2}{c}{\textbf{GA}} & \multicolumn{2}{c}{\textbf{HC}} \\
\cline{3-6}
\multicolumn{2}{c}{} & \textbf{\tiny GE} & \textbf{\tiny RC} & \textbf{\tiny GE} & \textbf{\tiny RC} \\
\hline
\multicolumn{2}{c}{\textbf{NCL}}            & 292/\textbf{374}   & 3622/\textbf{4829} & 40/\textbf{67}   & 645/\textbf{921}  \\
\multicolumn{2}{c}{\tiny \#API: 34, \#MétodosGen: 3} & & & & \\

\multicolumn{2}{c}{\textbf{UFind}}          & 41/\textbf{53}     & 700/\textbf{934}   & 16/\textbf{27}   & 267/\textbf{381}  \\
\multicolumn{2}{c}{\tiny \#API: 9, \#MétodosGen: 3}  & & & & \\

\multicolumn{2}{c}{\textbf{FHeap}}          & 9/\textbf{11}      & 346/\textbf{462}   & 4/\textbf{6}     & 225/\textbf{321}  \\
\multicolumn{2}{c}{\tiny \#API: 7, \#MétodosGen: 3}  & & & & \\

\multicolumn{2}{c}{\textbf{BTree}}          & 4/\textbf{5}       & 174/\textbf{232}   & 1/\textbf{2}     & 168/\textbf{240}  \\
\multicolumn{2}{c}{\tiny \#API: 7, \#MétodosGen: 2}  & & & & \\

\multicolumn{2}{c}{\textbf{BHeap}}          & 24/\textbf{31}     & 989/\textbf{1319}  & 6/\textbf{10}    & 273/\textbf{390}  \\
\multicolumn{2}{c}{\tiny \#API: 10, \#MétodosGen: 3} & & & & \\

\multicolumn{2}{c}{\textbf{Lits}}           & 25/\textbf{32}     & 3429/\textbf{4572} & 7/\textbf{11}    & 394/\textbf{563}  \\
\multicolumn{2}{c}{\tiny \#API: 26, \#MétodosGen: 6} & & & & \\

\multicolumn{2}{c}{\textbf{Sched.}}         & 73/\textbf{94}     & 1685/\textbf{2247} & 15/\textbf{24}   & 262/\textbf{374}  \\
\multicolumn{2}{c}{\tiny \#API: 10, \#MétodosGen: 3} & & & & \\

\multicolumn{2}{c}{\textbf{LinkedList}}     & 69/\textbf{89}     & 1637/\textbf{2182} & 8/\textbf{13}    & 302/\textbf{432}  \\
\multicolumn{2}{c}{\tiny \#API: 67, \#MétodosGen: 2} & & & & \\

\multicolumn{2}{c}{\textbf{TreeMap}}        & 974/\textbf{1248}  & 3872/\textbf{5162} & 68/\textbf{113}  & 484/\textbf{691}  \\
\multicolumn{2}{c}{\tiny \#API: 61, \#MétodosGen: 3} & & & & \\

\multicolumn{2}{c}{\textbf{TreeSet}}        & 75/\textbf{96}     & 2266/\textbf{3021} & 7/\textbf{11}    & 561/\textbf{801}  \\
\multicolumn{2}{c}{\tiny \#API: 34, \#MétodosGen: 3} & & & & \\

\multicolumn{2}{c}{\textbf{HashSet}}        & 115/\textbf{148}   & 428/\textbf{570}   & 12/\textbf{20}   & 2690/\textbf{3842} \\
\multicolumn{2}{c}{\tiny \#API: 31, \#MétodosGen: 2} & & & & \\

\multicolumn{2}{c}{\textbf{HashMap}}        & 1406/\textbf{1803} & 2566/\textbf{3421} & 51/\textbf{85}   & 1202/\textbf{1532} \\
\multicolumn{2}{c}{\tiny \#API: 45, \#MétodosGen: 2} & & & & \\
\hline
\end{tabular}

\caption{Casos de estudio: \textbf{promedio/peor} de tiempo (s) para hallar el subconjunto óptimo de métodos generadores. \textbf{GA} y \textbf{HC} evaluados con \textbf{GE} (generación exhaustiva) y \textbf{RC} (cobertura con Randoop).}
\label{tab:eficiencia}
\end{table}


\subsection{RQ2: Precisión de los algoritmos de cómputo de generadores de objetos}
\label{sec:experimentalIdentificacionPrecision}
La efectividad de nuestros algoritmos se mide en función de cuán cercanos están a identificar el conjunto mínimo de métodos 
generadores de objectos de una API necesario para generar estructuras de datos válidas. 
Cabe destacar que en todos los casos, nuestros algoritmos 
logran identificar al menos un subconjunto suficiente de métodos generadores de objetos. No obstante, cuanto más reducido sea este 
subconjunto, mejor será la calidad del resultado en términos de cobertura de generación y simplicidad de las entradas de prueba.

Los experimentos se realizaron utilizando las estructuras de datos ya detalladas en la primera sección de este capítulo. Se evaluaron 
los dos algoritmos: Hill Climbing (\emph{HC}), un algoritmo genético (\emph{GA}). Para cada uno de ellos, 
se aplicaron dos funciones de valoración: una basada en el generador exhaustivo 
(\emph{GE}) y otra basada en la cobertura de ramas obtenida a partir de una suite de pruebas generada con Randoop (\emph{RC}).
Es importante señalar que la función de valoración basada en Randoop depende de una semilla aleatoria, lo que introduce cierta 
variabilidad en los resultados. Esta función mide la cobertura alcanzada durante la ejecución de la suite generada, y puede arrojar 
diferentes valores dependiendo de la semilla inicial.

En el caso de la estructura \emph{NCL}, el algoritmo \emph{HC} utilizando la función de valoración \emph{GE} logra encontrar el 
conjunto mínimo exacto de métodos generadores en todas las ejecuciones. Sin embargo, cuando se utiliza la función \emph{RC}, 
el mismo algoritmo encuentra en promedio un conjunto con 3.80 métodos, lo que indica que en tres de las cinco ejecuciones se incluyó 
un método adicional no necesario. Por su parte, el algoritmo evolutivo (\emph{GA}) encuentra en todas las ejecuciones los metodos correctos
con ambas funciones de valoracion.

En estructuras como \emph{UnionFind}, \emph{FibonacciHeap}, \emph{BTree}, \emph{BinomialHeap} y \emph{Scheduler}, todos los 
algoritmos evaluados lograron encontrar el conjunto mínimo exacto de métodos \emph{builders}. Esto puede atribuirse a que estas 
estructuras presentan un menor número de métodos en su API, y sus invariantes son relativamente simples en comparación con otras 
clases evaluadas.

Para la estructura \emph{Lits}, la función de valoración basada en el generador exhaustivo (\emph{GE}) tiende a incluir al menos un 
método adicional. Este fenómeno se debe a cómo el generador maneja los valores iniciales de los arreglos internos en la clase 
\emph{List}. En particular, \emph{Lits} forma parte de una biblioteca Java utilizada para resolver problemas de satisfacibilidad 
booleana (SAT), y contiene colecciones que almacenan literales. Algunos métodos permiten crear o modificar arreglos con literales, 
pero el generador exhaustivo, sujeto al \emph{scope} limitado y a los valores iniciales, no logra producir ciertos arreglos sin 
utilizar métodos adicionales. Es importante aclarar que esto no implica que dichos métodos adicionales sean verdaderos generadores, 
sino que bajo las restricciones actuales (como el \emph{scope} configurado), no existe otra vía para construir determinadas 
estructuras sin agregarlos. Si se incrementa el \emph{scope} o se modifica la clase para exponer mejor sus constructores, entonces 
se logra encontrar el conjunto mínimo real. En contraste, la función \emph{RC} encuentra siempre el conjunto mínimo, dado que su 
evaluación no depende del \emph{scope}, sino de la cobertura lograda en ejecución.

En las estructuras pertenecientes al paquete \emph{java.util}, como \emph{TreeMap}, \emph{TreeSet}, \emph{HashMap}, \emph{HashSet} y \emph{LinkedList}, 
el comportamiento de los algoritmos evaluados varía significativamente en función de la complejidad de la estructura, 
el tamaño de la API y la función de valoración utilizada.

El algoritmo genético (\emph{GA}), en combinación con la función de cobertura (\emph{RC}), muestra una mayor tendencia a incluir métodos 
adicionales en el conjunto resultante, especialmente en estructuras donde existen muchas formas alternativas de construir una instancia válida. 
Por ejemplo, en \emph{TreeMap} y \emph{TreeSet}, el promedio de métodos identificados es de 5.6 y 3.2 respectivamente, 
frente al mínimo esperado de 3. Este comportamiento puede atribuirse a la naturaleza estocástica del algoritmo genético,
que depende de la exploración guiada por operadores evolutivos (selección, cruce y mutación) y de la presión de la función de aptitud. 
En escenarios donde múltiples combinaciones conducen a configuraciones aceptables, pero no óptimas, el algoritmo puede converger 
hacia soluciones suficientes pero redundantes, incluyendo métodos no esenciales.

Además, la propia función \emph{RC}, basada en cobertura alcanzada mediante herramientas como \emph{Randoop}, 
puede inducir una saturación temprana al favorecer métodos que rápidamente maximizan cobertura parcial, 
aunque no necesariamente contribuyan a una construcción mínima. Esto puede explicar, por ejemplo, los resultados observados en \emph{HashSet} y \emph{HashMap}, 
donde se identifican más métodos de los requeridos para generar la estructura objetivo. 
En el caso específico de \emph{HashMap}, incluso la función \emph{GE} tiende a incluir el método \texttt{keySet()}, 
necesario para construir configuraciones específicas que de otro modo no serían alcanzables bajo el \emph{scope} definido.

En contraste, el algoritmo \emph{HC} muestra mayor estabilidad en la precisión, 
logrando identificar el conjunto mínimo en la mayoría de los casos, 
independientemente de la función de valoración empleada. Su comportamiento determinista y 
su foco en mejoras incrementales lo hacen menos propenso a incluir métodos redundantes, 
aunque esto también puede limitar su capacidad de escape ante óptimos locales.

En resumen, la precisión alcanzada por los algoritmos depende fuertemente tanto de la estrategia de búsqueda como de la función de valoración. 
La exploración estocástica de \emph{GA} le permite descubrir soluciones diversas, 
pero también lo expone a converger en conjuntos sobredimensionados si la presión selectiva o la función de evaluación no están cuidadosamente diseñadas. 
Estos resultados destacan la necesidad de calibrar adecuadamente los parámetros y de considerar cuidadosamente la métrica de evaluación para lograr una identificación eficiente y precisa de métodos generadores.
\begin{table}[H]
\centering
\label{tab:t1}
\scriptsize
\begin{tabular}{c ccccc}
\midrule
\multicolumn{2}{c}{\multirow{3}{*}{\textbf{Cases}}} & \multicolumn{4}{c}{\textbf{Efectividad}} \\
\cline{3-6}
\multicolumn{2}{c}{} & \multicolumn{2}{c}{\textbf{GA}} & \multicolumn{2}{c}{\textbf{HC}}  \\
\multicolumn{2}{c}{} & \textbf{\tiny{GE}} & \textbf{\tiny{RC}} & \textbf{\tiny{GE}} & \textbf{\tiny{RC}}  \\
\midrule
\multicolumn{2}{c}{\textbf{NCL}} & 3  &   \cellcolor{gray!25} \emph{3.8} & 3 &  3 \\
\multicolumn{2}{c}{\tiny \#API: 34} &  &   & &    \\
\multicolumn{2}{c}{\tiny \#Builders: 3} &  &   & &    \\

\midrule
\multicolumn{2}{c}{\textbf{UFInd}}& 3 & 3  & 3  & 3      \\
\multicolumn{2}{c}{\tiny \#API: 9} &  &   & &   \\
\multicolumn{2}{c}{\tiny \#Builders: 3} &  &   & &    \\
\midrule

\multicolumn{2}{c}{\textbf{FHeap}}& 3 & 3  &  3 &  3   \\
\multicolumn{2}{c}{\tiny \#API: 7} &  &   & &    \\
\multicolumn{2}{c}{\tiny \#Builders: 3} &  &   & &    \\
\midrule
\multicolumn{2}{c}{\textbf{BTree}} & 2 & 2  &  2 &  2   \\
\multicolumn{2}{c}{\tiny \#API: 8} &  &   & &    \\
\multicolumn{2}{c}{\tiny \#Builders: 2} &  &   & &    \\
\midrule
\multicolumn{2}{c}{\textbf{BHeap}}& 3 & 3 &  3 &  3    \\
\multicolumn{2}{c}{\tiny \#API: 10} &  &   & &   \\
\multicolumn{2}{c}{\tiny \#Builders: 3} &  &   & &   \\
\midrule
\multicolumn{2}{c}{\textbf{Lits}} &  7 & 6 & 7 &  6    \\
\multicolumn{2}{c}{\tiny \#API: 26} &  &   & &    \\
\multicolumn{2}{c}{\tiny \#Builders: 6} &  &   & &   \\
\midrule
\multicolumn{2}{c}{\textbf{Sched.}} &  3 & 3   & 3  &  3 \\

\multicolumn{2}{c}{\tiny \#API: 10} &  &   & &    \\
\multicolumn{2}{c}{\tiny \#Builders: 3} &  &   & &    \\
\midrule\multicolumn{2}{c}{\textbf{LinkedList}} & 2 &  2 &  2 &  2  \\
\multicolumn{2}{c}{\tiny \#API: 67} &  &   & &   \\
\multicolumn{2}{c}{\tiny \#Builders: 2} &  &   & &    \\
\midrule
\multicolumn{2}{c}{\textbf{TreeMap}} &  3 & \cellcolor{gray!25}\emph{5.60}  &  3  &  3   \\
\multicolumn{2}{c}{\tiny \#API: 61} &  &   & &  \\
\multicolumn{2}{c}{\tiny \#Builders: 3} &  &   & &  \\
\midrule
\multicolumn{2}{c}{\textbf{TreeSet}} &  3&\cellcolor{gray!25} \emph{3.20}  &  3 &  3  \\
\multicolumn{2}{c}{\tiny \#API: 34} &  &   & &  \\
\multicolumn{2}{c}{\tiny \#Builders: 3} &  &   & &   \\
\midrule
\multicolumn{2}{c}{\textbf{HashSet}} &  2 &3  &  2 &  3 \\
\multicolumn{2}{c}{\tiny \#API: 31} &  &   & &  \\
\multicolumn{2}{c}{\tiny \#Builders: 2} &  &   & &    \\
\midrule
\multicolumn{2}{c}{\textbf{HashMap}} & \cellcolor{gray!25}2.60  & 2  &2   &2     \\
\multicolumn{2}{c}{\tiny \#API: 45} &  &   & &    \\
\multicolumn{2}{c}{\tiny \#Builders: 2} &  &   & &   \\
\hline
\end{tabular}

\caption{Tabla de casos de estudio y efectividad de encontrar el subconjunto óptimo}
\label{tab:efectividad}
\end{table}



\subsection{RQ3: Impacto de los generadores de objetos en el análisis de
software}

En esta sección evaluamos la utilidad práctica de los métodos generadores de objetos (\emph{builders}) 
identificados previamente, en el contexto de la generación automatizada de casos de prueba. 
Estos métodos permiten crear instancias válidas de estructuras de datos, que pueden ser utilizadas como 
entradas en pruebas unitarias o pruebas basadas en propiedades.

Para aquellos casos de estudio cuyas clases permiten medir el tamaño de los objetos generados 
(mediante el método \texttt{size()}) y comparar objetos por igualdad (a través de \texttt{equals()}), 
realizamos un experimento con la herramienta \emph{Randoop}. Generamos casos de prueba en dos 
configuraciones distintas: (i) utilizando toda la API disponible de la clase bajo análisis 
(denominada \texttt{}{AllMethods}), y (ii) utilizando únicamente los métodos generadores de objetos
(\texttt{Builders}) identificados por nuestro enfoque en el experimento anterior 
(ver Tabla~\ref{tab:results-obj}).

Para ambas configuraciones, medimos la \textit{cantidad de objetos distintos 
generados} (\textit{No. of Objs.}). Además, definimos tres valores de presupuesto de tiempo para la generación de 
pruebas: 60, 120 y 180 segundos. Los resultados obtenidos se resumen en la Tabla~\ref{tab:results-obj}.

Los resultados muestran de forma consistente que el uso exclusivo de los métodos generadores 
permite generar una cantidad significativamente mayor de objetos válidos en comparación con el uso 
de la API completa. En promedio, se observa que la cantidad de objetos generados con \texttt{Builders} 
es hasta cinco veces superior a la obtenida con \texttt{AllMethods} en el mismo presupuesto de tiempo. 

Estos hallazgos ponen de manifiesto el impacto positivo de identificar automáticamente métodos 
generadores adecuados. Su utilización en herramientas de prueba como \emph{Randoop} mejora sustancialmente 
la calidad y diversidad de los objetos generados, lo cual es especialmente relevante para clases con 
estado interno complejo. Por lo tanto, concluimos que la identificación automatizada de métodos generadores de objetos 
constituye una estrategia efectiva y valiosa para potenciar la generación automatizada de pruebas.
\begin{table}[H]
\centering
\scriptsize
\begin{tabular}{ c l c c}
\hline
Class & Budget &
\multicolumn{2}{c}{\textsf{No. of Objs}} \\
&& \tiny{\textbf{SoloMetodosGeneradores}} & \tiny{\textbf{TodosLosMetodos}} \\
\hline
\multirow{3}{*}{\textbf{NCL}} 
&	60	&	6648	&	470	\\
&	120	&	9436	&	612	\\
&	180	&	11441	&	703	\\
\hline
\multirow{3}{*}{\textbf{UFind}} 
&	60	&	1033	&	372	\\
&	120	&	1342	&	483	\\
&	180	&	1534	&	555	\\
\hline
\multirow{3}{*}{\textbf{FibHeap}}
&	60	&	6541	&	1766	\\
&	120	&	9270	&	2347	\\
&	180	&	10923	&	2745	\\
\hline
\multirow{3}{*}{\textbf{RBT}}
&	60	&	2634	&	515	\\
&	120	&	3410	&	611	\\
&	180	&	3938	&	676	\\
\hline
\multirow{3}{*}{\textbf{BTree}}
&	60	&	2937	&	975	\\
&	120	&	3820	&	1196	\\
&	180	&	4367	&	1354	\\
\hline
\multirow{3}{*}{\textbf{BHeap}}
&	60	&	6455	&	971	\\
&	120	&	8665	&	1230	\\
&	180	&	10093	&	1401	\\
\hline
\multirow{3}{*}{\textbf{Lits}}
&	60	&	3968	&	3174	\\
&	120	&	5109	&	4142	\\
&	180	&	5848	&	4783	\\
\hline
\multirow{3}{*}{\textbf{Schedule}}
&	60	&	2901    &	2176	\\
&	120	&	3221	&	2756    \\
&	180	&	3437	&	3140    \\
\hline
\multirow{3}{*}{\textbf{LinkedList}} 
&	60	&	8121	&	790	\\
&	120	&	11503	&	1095	\\
&	180	&	13905	&	1323	\\
\hline
\multirow{3}{*}{\textbf{TreeMap}} 
&	60	&	2750	&	748	\\
&	120	&	3754	&	953	\\
&	180	&	4496	&	1107 \\
\hline\multirow{3}{*}{\textbf{TreeSet}}
&	60	&	1129	&	291	\\
&	120	&	1527	&	343	\\
&	180	&	1816	&	381	\\
\hline
\multirow{3}{*}{\textbf{HashSet}}
&	60	&	8208	&	1498	\\
&	120	&	11467	&	2008	\\
&	180	&	13548	&	2366	\\
\hline
\multirow{3}{*}{\textbf{HashMap}}
&	60	&	9581	&	2103	\\
&	120	&	13044	&	3173	\\
&	180	&	15514	&	3784	\\
\hline

\end{tabular}%

\caption{Evaluación del uso de los builders identificados (BLD) frente a toda la API (API) en la generación de casos de test.}
\label{tab:results-obj}
\end{table}

\vspace{10pt}
En un segundo experimento, evaluamos el impacto del uso de métodos \emph{builders} en la eficiencia 
del análisis de programas mediante verificación formal. Para ello, utilizamos \emph{Java PathFinder} 
(JPF)~\cite{Visser:2005}, una herramienta de verificación de modelos con exploración explícita del 
espacio de estados para programas Java.\footnote{\url{https://github.com/javapathfinder/jpf-core}}. 
Este experimento lo evaluamos en los casos de estudio del paquete \texttt{java.util}.

Las técnicas de verificación de modelos requieren definir \emph{controladores de prueba}, es decir, 
secuencias de métodos que permitan construir las entradas sobre las que se analizará el programa. 
Idealmente, estas secuencias deberían generarse a partir del conjunto mínimo de métodos que permitan 
construir todas las instancias posibles dentro de un alcance definido (\emph{scope}). Sin embargo, 
la selección manual de estos métodos representa una barrera significativa para el uso de herramientas 
como JPF, ya que requiere un análisis exhaustivo de la API y una comprensión detallada de la 
semántica del módulo.

JPF permite definir controladores no deterministas que generan todas las secuencias posibles de 
llamadas a métodos de la API hasta una longitud determinada. Internamente, JPF explora todas las 
ejecuciones posibles, almacenando los estados visitados y retrocediendo ante duplicados. Para ello, 
utiliza puntos de elección no determinista definidos a través de métodos especiales como 
\verb|Verify.getInt(int lo, int hi)| y \verb|Verify.random(int n)|, los cuales introducen bifurcaciones 
en la ejecución.

Para ilustrar este mecanismo, consideramos el análisis del método \texttt{put} de \texttt{TreeMap}. 
La generación de entradas puede realizarse mediante un controlador como el mostrado en la immplentación mostrada en ~\ref{lst:driverAPI}, que incluye todos los métodos disponibles en la API de \texttt{TreeMap}. 
Este enfoque asegura cobertura, pero introduce un crecimiento exponencial en el número de ejecuciones 
a explorar, ya que cada paso del controlador puede seleccionar cualquier método de forma no 
determinista.

\begin{lstlisting}[caption={Controlador con todos los métodos},label={lst:driverAPI},language=Java,captionpos=b]
    private static TreeMap generateStructure(int scope) {
       int maxLength = Verify.getInt(0, scope);
       TreeMap t = new TreeMap();
       for (int i = 1; i <= maxLength; i++) {
          switch (Verify.random(n_methods)) {
             case 0:
                t.put(Verify.getInt(0,scope),Verify.getInt(0,scope));
                break;
             case 1:
                t.remove(Verify.getInt(0,scope));
                break;						
             case 2:
                t.clear();
                break;
             case 3:
                t.containsValue(Verify.getInt(0,scope));
                break;
             ...
             case 11: 
                t.putAll(l);
                break;
          }
       }
       return t;
    }
    \end{lstlisting}

Para mitigar esta explosión combinatoria, propusimos utilizar únicamente los métodos generadores 
detectados automáticamente por nuestro enfoque con el algoritmo de \emph{HC} y la funcion de valoracion de \emph{RC} (ver Sección~\ref{sec:experimentalIdentificacionPrecision}). 
Como se muestra en el framento de codigo ~\ref{lst:driverBLD}, el controlador basado exclusivamente en los métodos generadores de objectos de 
\texttt{TreeMap} selecciona sólo dos métodos, suficientes y mínimos, para generar todas las instancias 
válidas dentro del alcance definido. En consecuencia, JPF puede explorar las mismas configuraciones 
de objetos que con la API completa, pero con un espacio de búsqueda significativamente más reducido.

\begin{lstlisting}[caption={Probando el método put de TreeMap con JPF},label={lst:label},language=Java,captionpos=b]
    public static void main(String[] args) {
       int scope = 3;
       TreeMap t = generateStructure(scope);
       t.put(Verify.getInt(0,scope),Verify.getInt(0,scope));
       assert t.repOK();
    }
\end{lstlisting}

Los resultados de los experimentos, presentados en las Tablas~\ref{tab:results-jpf1} y 
\ref{tab:results-jpf2}, demuestran que los controladores construidos a partir de nuestros 
métodos generadores permiten una mejora sustancial en eficiencia y escalabilidad. Se observa una 
reducción considerable en el tiempo de análisis y en la cantidad de estados explorados, sin 
comprometer la cobertura de las estructuras objetivo. Esta evidencia refuerza la utilidad práctica 
de nuestra técnica en entornos de verificación formal, donde el costo computacional de la exploración 
del espacio de estados es un factor crítico.

El patrón se mantiene de forma consistente a través de múltiples estructuras de datos de 
\texttt{java.util}, incluyendo \texttt{HashMap}, \texttt{HashSet}, \texttt{LinkedList}, 
\texttt{TreeMap} y \texttt{TreeSet}. Por ejemplo, en el análisis del método \texttt{remove} 
en \texttt{HashMap}, al aumentar el \textit{scope} de 4 a 6, el número de estados explorados 
con el controlador basado en la API completa crece de 627361 a más de 22 millones, con tiempos 
de ejecución que alcanzan los 3011 segundos (aproximadamente 50 minutos). En cambio, utilizando únicamente los métodos generadores de objetos, 
se recorren menos de 2.3 millones de estados, en un tiempo de análisis significativamente menor.

Este mismo comportamiento se replica en otras estructuras. En el caso de \texttt{LinkedList}, 
el uso de todos los métodos de la API lleva a una explosión de estados (más de 39 millones en 
\textit{scope} 6), mientras que con \emph{builders}, el número de estados se reduce a aproximadamente 
1.1 millones para el mismo scope. En términos de tiempo, esta diferencia se traduce en 2771 segundos contra apenas 221. 
Asimismo, para \texttt{TreeMap} con el método \texttt{put}, el uso de \emph{builders} en 
\textit{scope} 5 permite mantener el análisis por debajo de los 3100 segundos, mientras que 
la versión con todos los métodos alcanza el límite de tiempo (TO que es igual a 5000 segundos).

Además de la reducción en el tiempo de verificación y en los estados explorados, la comparación 
entre ambos controladores confirma que el conjunto reducido de métodos \emph{builders} es suficiente 
para generar todas las configuraciones relevantes de entrada que permiten evaluar correctamente 
las propiedades del programa. Esto implica que, en la práctica, los métodos adicionales disponibles 
en la API no contribuyen a la diversidad estructural de las entradas, sino que solo introducen 
complejidad innecesaria en el espacio de búsqueda.

En resumen, estos resultados evidencian que la identificación automática de métodos generadores no 
solo mejora la calidad de las entradas en contextos como la generación de pruebas, sino que también 
es una estrategia efectiva para escalar técnicas de verificación formal a estructuras de datos más 
complejas. Al eliminar métodos superfluos y reducir el espacio de búsqueda, se mejora significativamente 
el rendimiento de herramientas como JPF, sin sacrificar precisión ni completitud en el análisis.


\begin{table}
\scriptsize
\begin{tabular}{ c| l| c c c c c}
\hline
Class & Method & Scope &
\multicolumn{2}{c}{\textsf{MGO}} &
\multicolumn{2}{c}{\textsf{API}} \\
&&&
\tiny{\textbf{states}} & \tiny{\textbf{time (S)}} &
\tiny{\textbf{states}} & \tiny{\textbf{time (S)} }\\
\hline
\multirow{14}{*}{LinkedList} 
& add
  & 1 & 11  & 0 & 42  & 0 \\
& & 2 & 58  & 0 & 715 & 0 \\
& & 3 & 453 & 0 & 11064 & 2 \\
& & 4 & 4881  & 1 & 153247  & 14  \\
& & 5 & 67183 & 13  & 2291803 & 171 \\
& & 6 & 1120932 & 221 & 39759491  & 2771  \\
& & 7 & 21913097  & 4588  &TO  & \\

\cline{2-7}
 &remove 
  & 1 & 11  & 0 & 42  & 0 \\
& & 2 & 58  & 0 & 715 & 0 \\
& & 3 & 453 & 0 & 11064 & 2\\
& & 4 & 4881  & 1 & 153247  & 14\\
& & 5 & 67183 & 11  & 2291803 & 168\\
& & 6 & 1120932 & 188 & 39759491  & 2713\\
& & 7 & 21913097  & 3998  & TO \\ 

\hline
\multirow{10}{*}{TreeMap} 
% Check
& put
  & 1 & 45  & 0 & 112 & 0 \\
& & 2 & 991 & 0 & 2821  & 0 \\
& & 3 & 16071 & 2 & 71678 & 8 \\
& & 4 & 552807  & 75  & 1757976 & 204 \\
& & 5 & 20601447  & 3007  & TO& \\
\cline{2-7}

& remove
  & 1 & 25  & 0 & 61  & 0 \\
& & 2 & 406 & 0 & 1345  & 0 \\
& & 3 & 8359  & 1 & 33845 & 3 \\
& & 4 & 182732  & 19  & 768564  & 66  \\
& & 5 & 7176075 & 713 & 24775111  & 2341  \\



%  && 5 &  & MO & - & - &MO  &  &-  & -  \\


\hline
\multirow{19}{*}{TreeSet} 
&put
  & 1 & 11  & 0 & 48  & 0 \\
& & 2 & 49  & 0 & 595 & 0 \\
& & 3 & 197 & 0 & 5275  & 1 \\
& & 4 & 806 & 0 & 36427 & 4 \\
& & 5 & 3115  & 0 & 204877  & 19  \\
& & 6 & 12062 & 1 & 1024038 & 90  \\
& & 7 & 47241 & 5 & 4966224 & 452 \\
& & 8 & 501666  & 45  & 23976826  & 2094  \\
& & 9 & 2047285 & 183 &TO  &  \\
\cline{2-7}

& remove
  & 1 & 17  & 0 & 48  & 0 \\
& & 2 & 88  & 0 & 595 & 0 \\
& & 3 & 423 & 0 & 5275  & 1 \\
& & 4 & 1842  & 0 & 36427 & 4 \\
& & 5 & 7455  & 1 & 204877  & 19  \\
& & 6 & 30197 & 3 & 1024038 & 88  \\
& & 7 & 122717  & 10  & 4966224 & 431 \\
& & 8 & 501666  & 43  & 23976826  & 2088  \\
& & 9 & 2047285 & 181 & TO    \\
& & 10  & 8182166 & 719 &    TO  \\
& & 11  & 31473779  & 2738  &  TO     \\
\hline

\end{tabular}%
\caption{Estados generados JPF utilizando el método driver con todos los métodos de la API vs. solo utilizando los métodos generadores de objectos.}
\label{tab:results-jpf1}
 \end{table}


\begin{table}[H]
\scriptsize
\begin{tabular}{ c| l| c c c c c}
\hline
Class & Method & Scope &
\multicolumn{2}{c}{\textsf{Builders}} &
\multicolumn{2}{c}{\textsf{AllMethods}} \\
&&&
\tiny{\textbf{states}} & \tiny{\textbf{time (S)}} &
\tiny{\textbf{states}} & \tiny{\textbf{time (S)} }\\
\hline
\multirow{11}{*}{HashMap} 
& put
 & 1 & 21& 0& 100 & 0 \\
& & 2 & 136& 0& 235 & 0 \\
& & 3 & 2945& 0& 59494 & 9 \\
& & 4 & 64626& 11&  1808536& 299 \\
& & 5 & 1512217& 267& TO&  \\

 \cline{2-7}

& remove
 & 1 &25 & 0& 56 & 0 \\
& & 2 &325 & 0& 1144 &  0\\
& & 3 & 5479& 1& 23966 & 3 \\
& & 4 & 105607& 10& 627361 &  78\\
& & 5 & 2289075& 213& 22547086 &3011  \\
& & 6 & 55335111 & 5017&TO &  \\

\hline

\multirow{14}{*}{HashSet} 
& put
  & 1 & 11  & 0 & 30  & 0 \\
& & 2 & 40  & 0 & 248 & 0 \\
& & 3 & 149 & 0 & 1405  & 0 \\
& & 4 & 531 & 0 & 6259  & 1 \\
& & 5 & 1765  & 0 & 24107 & 3 \\
& & 6 & 5496  & 0 & 84617 & 8 \\
& & 7 & 16217 & 2 & 278241  & 24  \\


\cline{2-7}

& remove
  & 1 & 11  & 0 & 30  & 0 \\
& & 2 & 40  & 0 & 248 & 0 \\
& & 3 & 149 & 0 & 1405  & 0 \\
& & 4 & 531 & 0 & 6259  & 1 \\
& & 5 & 1765  & 0 & 24107 & 3 \\
& & 6 & 5496  & 0 & 84617 & 8 \\
& & 7 & 16217 & 1 & 278241  & 24  \\

\hline

\end{tabular}%
\caption{Estados generados JPF utilizando el método driver con todos los métodos de la API vs. solo utilizando los métodos builders.}
\label{tab:results-jpf2}
 
\end{table}




\section{Generación exhaustiva acotada basada en la API}
\label{sec:experimentalBeapi}

En la evaluación de BEAPI buscaremos responder a las siguientes preguntas de
investigación:

\begin{itemize}
\item \emph{RQ4}: ¿Qué tan eficiente es BEAPI para la generación exhaustiva                                                                                       
    acotada de objetos?
\item\emph{RQ5}: ¿Cuál es el impacto de las optimizaciones propuestas en el
    rendimiento de BEAPI?
\item\emph{RQ6}: ¿Puede BEAPI ayudar a encontrar discrepancias entre
    especificaciones de invariantes de clase (repOK) y la capacidad de generación de objetos de la API?
\item\emph{RQ7}: ¿Qué tan efectivos son los objetos producidos por BEAPI para el
    testing parametrizado?  
\end{itemize}

\subsection{Configuración experimental}

Casos de estudio, técnicas a evaluar, métricas, parámetros usados, ground truth, etc..


\subsection{Configuración experimental}

Como estudios de caso, utilizamos implementaciones de estructuras de datos tomadas de cuatro bancos 
de pruebas reconocidos: \textsf{Korat}~\cite{Boyapati02}, \textsf{Kiasan}~\cite{Deng06}, 
\textsf{FAJITA}~\cite{Abad13}, \textsf{ROOPS}~\cite{??} y un conjunto de clases de proyectos reales de software que la denominamos \textsf{realWorld}
donde incluyen clases de \emph{java.util} (TreeMap, TreeSet, LinkedList, Hasmap), de \emph{Apache Commons Collections} (NodeCachingLinkedList) y un Scheduler de SIR

Estos bancos cubren una amplia variedad de estructuras de datos complejas, tales como listas doblemente enlazadas, árboles binarios de búsqueda, 
árboles rojo-negro, montículos binomiales, heaps de Fibonacci, entre otras. Una característica clave 
de estos bancos es que incluyen especificaciones explícitas de invariante de clase (métodos 
\texttt{repOK}) escritas por sus propios autores, lo que garantiza una base confiable para evaluar 
algoritmos de generación de estructuras válidas.

Los experimentos se llevaron a cabo en una computadora con procesador Intel Core i7-8700 
(3.2 GHz) y 16 GB de RAM, ejecutando Ubuntu Linux 20.04. Para cada ejecución individual se estableció 
un tiempo máximo de espera de 60 minutos. Todos los experimentos fueron diseñados para ser reproducibles, 
y el artefacto experimental con el código fuente se encuentra disponible en: ???.

Cada técnica evaluada se ejecutó sobre los mismos casos de estudio, con parámetros estandarizados 
en términos de alcance (\textit{scope}), número de métodos disponibles, y criterios de generación. 
Para la comparación con \textsf{Korat}, se emplearon los \texttt{repOK}s provistos por los autores 
de cada benchmark, sin modificaciones, a fin de preservar la equivalencia semántica entre enfoques 
y evitar sesgos en la evaluación del rendimiento. En los casos donde se identificaron errores o 
especificaciones incompletas en los \texttt{repOK}s, estos se señalan de forma explícita en los 
resultados y se analizan en la Sección~\ref{sec:existing-specs-analysis}.

\cacho{los parametros usados en GA}


\subsection{RQ4: Eficiencia de BEAPI}

Esta pregunta busca evaluar si \emph{BEAPI} es lo suficientemente eficiente como para ser una 
alternativa viable frente a otras técnicas de generación exhaustiva acotada, en particular 
\textsf{Korat}. Para ello, comparamos el desempeño de ambas herramientas en términos de tiempo 
de ejecución, número de estructuras generadas y exploradas, y escalabilidad con respecto al 
\emph{scope} de entrada.

Los resultados experimentales, presentados en la Tabla~\ref{table:korat-beapi}, cubren una muestra 
representativa de casos tomados de los bancos de pruebas \textsf{Korat}, \textsf{FAJITA}, 
\textsf{ROOPS} y \textsf{Kiasan}. Para cada clase evaluada, informamos los tiempos de generación 
(en segundos), la cantidad de estructuras válidas generadas y el total de estructuras exploradas 
por cada técnica. Incluimos, para cada enfoque, el mayor \emph{scope} alcanzado exitosamente dentro 
del tiempo límite de 60 minutos, indicando en negrita los tiempos correspondientes a los límites 
de escalabilidad. Vale aclarar que no se utilizo el benchamarks \textsf{realWorld} en esta sección, ya que
no se dispone de \texttt{repOK}s para estas clases, lo que impide una comparación justa con \textsf{Korat}.

En estos experimentos, se aseguraron condiciones comparables: los \texttt{repOK}s utilizados con 
\textsf{Korat} fueron los provistos originalmente por los autores de cada benchmark y no se 
modificaron para este estudio, evitando así alterar la eficiencia inherente a la técnica. 
Para \emph{BEAPI}, se probó exhaustivamente que los métodos de la API empleados fueran correctos 
y suficientes para el análisis. Como era de esperarse, pueden existir diferencias en los conjuntos 
de estructuras exploradas por cada técnica, debido a los distintos espacios de búsqueda que inducen. 
Sin embargo, para un mismo caso y alcance, ambas herramientas deberían generar la misma cantidad 
de estructuras válidas, salvo en casos donde hay errores en los \texttt{repOK}s o diferencias sutiles 
en la definición de \emph{scope}.

Los resultados muestran un comportamiento mixto. En estructuras con restricciones fuertes —como los 
árboles rojo-negro (\texttt{RBT}) y los árboles binarios de búsqueda (\texttt{BST})— \emph{BEAPI} 
supera a \textsf{Korat}, al explorar menos estados y terminar en menos tiempo. Esto ocurre porque 
en estas estructuras el número de configuraciones válidas es más bajo, lo que permite a \emph{BEAPI} 
recorrer el espacio de soluciones de forma eficiente. Por el contrario, en estructuras más permisivas, 
como listas doblemente enlazadas (\texttt{DLList}) o montículos binomiales (\texttt{BinHeap}), 
\textsf{Korat} logra un mejor rendimiento gracias a su estrategia de poda temprana basada en 
\texttt{repOK}.

Cabe destacar dos tipos de excepciones en los resultados. En primer lugar, algunos casos incluyen 
\texttt{repOK}s con errores, que impactan negativamente en la eficiencia de \textsf{Korat}; estos 
casos están sombreados en la tabla y se analizan más en detalle en la Sección~\ref{sec:existing-specs-analysis}. 
En segundo lugar, en estructuras como \texttt{RBT} y \texttt{FibHeap}, la diferencia semántica en la 
noción de \emph{scope} entre ambas técnicas produce discrepancias. Por ejemplo, hay estructuras de 
tamaño $n$ que solo pueden obtenerse mediante secuencias que superan temporalmente dicho tamaño 
(inserciones seguidas de eliminaciones y nuevas inserciones para provocar reordenamientos de balanceo). 
Como \emph{BEAPI} descarta automáticamente las secuencias que exceden el \emph{scope} máximo, no puede 
llegar a estas configuraciones, mientras que \textsf{Korat} sí lo logra.

Un aspecto clave que influye en la eficiencia de \emph{BEAPI} es la cantidad de estructuras válidas 
generables para un determinado alcance. A medida que crece esta cardinalidad, aumenta también el 
número de secuencias que \emph{BEAPI} debe evaluar, lo cual impacta directamente en su tiempo de 
ejecución. Sin embargo, un punto a favor de \emph{BEAPI} es que no depende de la eficiencia del 
\texttt{repOK}; esto lo hace más robusto ante especificaciones incorrectas o mal escritas, mientras 
que \textsf{Korat} puede verse severamente afectado por ellas.

En cuanto al rendimiento observado, \textsf{Korat} muestra mejor desempeño en 4 de los 6 casos 
pertenecientes a su propio banco de pruebas. En el benchmark \textsf{FAJITA}, \emph{BEAPI} supera 
a \textsf{Korat} en 3 de 4 casos, mientras que en \textsf{ROOPS} lo hace en 5 de 7. Finalmente, 
en el banco \textsf{Kiasan}, \textsf{Korat} resulta más rápido en 6 de 7 casos. Esta tendencia confirma 
que \emph{BEAPI} ofrece un mejor rendimiento en estructuras altamente restringidas —donde el número 
de configuraciones válidas es reducido— mientras que \textsf{Korat} mantiene la ventaja en dominios 
más amplios, donde la cantidad de estructuras válidas crece de forma acelerada. Esto se debe a que 
en tales escenarios, \emph{BEAPI} debe generar y evaluar muchas más secuencias de entrada en cada 
iteración, lo cual penaliza su desempeño.

Un caso ilustrativo es el de \texttt{BinHeap}, donde \textsf{Korat} alcanza el \emph{scope} 8 utilizando 
el \texttt{repOK} de \texttt{ROOPS}, el \emph{scope} 10 con el de \texttt{FAJITA}, y el \emph{scope} 11 
con su propia especificación, todas ellas equivalentes en términos de estructuras generadas. Esto evidencia 
la sensibilidad de \textsf{Korat} a la forma en que se escriben los \texttt{repOK}s, los cuales, si están 
ajustados específicamente para su motor de búsqueda, pueden mejorar significativamente su rendimiento. 
No obstante, cuando los \texttt{repOK}s presentan errores o están mal diseñados, su impacto negativo 
puede ser severo, como se analiza en profundidad en la Sección~\ref{sec:existing-specs-analysis}.

En conclusión, \emph{BEAPI} es una herramienta eficiente para la generación exhaustiva acotada. 
Ofrece tiempos competitivos y, en varios casos, mejores que \textsf{Korat}, especialmente cuando 
se trata de estructuras con invariantes complejas o bajo número de soluciones. Su independencia 
respecto al \texttt{repOK} lo vuelve una alternativa robusta y complementaria para contextos donde 
la calidad de la especificación no está garantizada.


\subsection{RQ5: Optimizaciones de BEAPI}
\label{sec:optimizations}
\subsection{RQ5: Impacto de las optimizaciones propuestas}

En esta pregunta de investigación evaluamos el impacto que tienen las optimizaciones propuestas 
de \emph{BEAPI} en su rendimiento general frente al enfoque de generación exhaustiva tradicional. 
Concretamente, analizamos cuatro configuraciones de la herramienta: \textsf{SM/BLD}, que habilita tanto 
la coincidencia de estados (State Matching) como la identificación de constructores (\emph{builders}); 
\textsf{SM}, que activa únicamente la coincidencia de estados; \textsf{BLD}, que habilita únicamente 
la identificación de constructores; y \textsf{NoOPT}, que desactiva ambas optimizaciones, funcionando 
como un enfoque de fuerza bruta.

Para este análisis, nos enfocamos en el benchmark \texttt{Real World}, que incluye seis 
implementaciones reales de estructuras de datos ampliamente utilizadas: \texttt{LinkedList} 
(67 métodos en la API), \texttt{TreeSet} (34), \texttt{TreeMap} (61), \texttt{HashMap} (45), \texttt{NCL} (34) y 
\texttt{Schedule} (12). La Tabla~\ref{tab:results-realWorld} 
resume los resultados obtenidos. Cabe mencionar que se realizaron experimentos equivalentes para 
el resto de los benchmarks utilizados en este capítulo, pero se omiten en las tablas en esta tesis por motivos de espacio.
El conjunto \texttt{Real World} representa adecuadamente el comportamiento observado en los demás casos, 
y resulta suficiente para sustentar las conclusiones extraídas en esta sección.

Los resultados muestran de forma contundente que la configuración sin optimizaciones (\textsf{NoOPT}) 
tiene un rendimiento deficiente incluso en los estudios de caso más simples y con alcances bajos. 
Estos \emph{scopes}, si bien pequeños, ya resultan inadecuados para generar conjuntos de prueba de 
calidad y revelan de inmediato la falta de escalabilidad del enfoque de fuerza bruta. Esta limitación 
es representativa de la problemática general que enfrentan las técnicas de generación exhaustiva en 
presencia de APIs amplias y estructuras complejas.

Entre las optimizaciones propuestas, la coincidencia de estados (\textsf{SM}) demuestra ser la más 
impactante a nivel general. Por sí sola, permite escalar a alcances considerablemente mayores y reduce 
los tiempos de ejecución en varios órdenes de magnitud respecto a \textsf{NoOPT}. Su efecto se vuelve 
crítico en aquellas estructuras donde múltiples secuencias de llamadas a métodos pueden generar el mismo estado 
intermedio, lo que genera redundancia innecesaria si no se cuenta con un mecanismo de detección y poda 
de estados previamente explorados.

La segunda optimización, la identificación previa de métodos \emph{builders} (\textsf{BLD}), adquiere 
mayor relevancia a medida que se incrementa la cantidad de métodos disponibles en la API. En 
estructuras reales como \texttt{TreeMap}, \texttt{LinkedList} o \texttt{HashMap}, donde existen 
docenas de métodos públicos con efectos variados, limitar la generación de secuencias a combinaciones 
de constructores válidos tiene un impacto significativo en la eficiencia del proceso de generación. 
Aunque por sí sola esta optimización no siempre permite escalar a scopes altos, su combinación con la 
coincidencia de estados resulta esencial para lograr tiempos razonables de ejecución.

El mayor beneficio se observa en la configuración \textsf{SM/BLD}, donde ambas optimizaciones están 
habilitadas. En esta configuración, \emph{BEAPI} escala significativamente mejor, alcanzando scopes 
mucho mayores en todos los casos analizados, con tiempos de ejecución hasta cientos de veces menores 
que con las demás configuraciones. Por ejemplo, para \texttt{NCL}, se logra alcanzar \textit{scope} 6 
en 73.78 segundos con \textsf{SM/BLD}, mientras que con \textsf{SM} apenas se alcanza \textit{scope} 4 
en más de 3 segundos, y \textsf{NoOPT} no logra ejecutar siquiera scopes bajos. Para \texttt{TreeSet}, 
el contraste es aún más pronunciado: \textsf{SM/BLD} alcanza \textit{scope} 13 en 226 segundos, mientras 
que \textsf{SM} requiere casi 900 segundos para el mismo alcance.

Estos resultados evidencian que las optimizaciones propuestas no solo reducen tiempos de ejecución, 
sino que amplían sustancialmente los límites de escalabilidad de la herramienta. Sin estas mejoras, 
\emph{BEAPI} se comporta como un generador exhaustivo tradicional, enfrentando rápidamente la explosión 
combinatoria del espacio de búsqueda al aumentar el número de métodos o el tamaño de las secuencias. 
En cambio, con las optimizaciones activadas, la técnica logra mantener el proceso dentro de márgenes 
computacionalmente viables incluso en contextos realistas y exigentes.

En conclusión, la activación conjunta de coincidencia de estados e identificación de \emph{builders} 
resulta crucial para obtener un rendimiento competitivo. Estas optimizaciones transforman a \emph{BEAPI} 
en una herramienta capaz de escalar a scopes altos sin sacrificar exhaustividad ni precisión, 
permitiendo su aplicación efectiva en casos de uso reales.

\vspace{8pt}
\noindent
\textbf{Sobre el costo de la identificación de métodos \emph{builders}.} Si bien esta optimización aporta 
beneficios sustanciales en eficiencia, es importante destacar que su costo computacional es relativamente 
bajo en comparación con el proceso completo de generación. La identificación se realiza como una etapa 
preprocesamiento independiente y solo una vez por clase analizada. Además, como se muestra en los experimentos 
previos de RQ1 y RQ2, los métodos seleccionados no solo mejoran el tiempo de ejecución, sino que también 
favorecen la precisión del conjunto generado, al evitar combinaciones innecesarias y preservar la 
suficiencia para construir todos los objetos válidos bajo el \emph{scope} definido.


\begin{table}[H]
\scriptsize

\centering
\begin{tabular}{ l r | r | r | r | r  }
  \toprule
  \multicolumn{6}{c}{\textbf{Real World}} \\
  \midrule 
  \textbf{Class} & \textbf{Scope} & \textbf{SM/BLD} & \textbf{SM}  & \textbf{BLD} & \textbf{NoOPT}  \\
  \midrule
  NCL
&	3	&	.10	&	.47	&	-	&	-	\\
&	4	&	.41	&	3.48	&	-	&	-	\\
&	5	&	3.33	&	-	&	-	&	-	\\
&	6	&	73.78	&	-	&	-	&	-	\\
  \midrule
  TSet
&	3	&	.03	&	.07	&	56.82	&	-	\\
&	4	&	.06	&	.13	&	-	&	-	\\
&	5	&	.11	&	.22	&	-	&	-	\\
&	6	&	.17	&	.42	&	-	&	-	\\
&	7	&	0.31	&	.91	&	-	&	-	\\
&	8	&	0.74	&	2.66	&	-	&	-	\\
&	9	&	2.23	&	7.80	&	-	&	-	\\
&	10	&	6.88	&	26.34	&	-	&	-	\\
&	11	&	21.52	&	86.06	&	-	&	-	\\
&	12	&	69.98	&	276.85	&	-	&	-	\\
&	13	&	226.66	&	887.83	&	-	&	-	\\
    \midrule
  TMap
&	3	&	.11	&	.25	&	-	&	-	\\
&	4	&	.75	&	2.36	&	-	&	-	\\
&	5	&	15.97	&	57.64	&	-	&	-	\\
&	6	&	839.87	&	2901.37	&	-	&	-	\\
  \midrule
  LList
&	3	&	.02	&	.13	&	.64	&	-	\\
&	4	&	.06	&	.38	&		&		\\
&	5	&	.20	&	3.80	&		&		\\
&	6	&	.96	&	258.85	&		&		\\
&	7	&	12.98	&		&		&		\\
&	8	&	286.21	&		&		&		\\
  \midrule
  HMap
&	3	&	.10	&	11.49	&	-	&	-	\\
&	4	&	.55	&		&		&		\\
&	5	&	5.33	&		&		&		\\
&	6	&	119.87	&		&		&		\\
  \midrule
  Schedule
&	3	&	.01	&	.01	&	59.27	&	-	\\
&	4	&	.82	&	45.55	&	-	&	-	\\
&	5	&	1.43	&	-	&	-	&	-	\\
&	6	&	06.01	&	-	&	-	&	-	\\
&	7	&	23.32	&	-	&	-	&	-	\\
  \bottomrule

\end{tabular}
\caption{Tiempos de generacion de acuerdo a las optimizaciones propuestas}
\label{tab:results-realWorld}

\end{table}


\begin{table}[H]
\scriptsize
\centering
\label{tab:results-obj1}
\begin{tabular}{ l r | r | r | r | r  }
  \toprule
  \multicolumn{6}{c}{\textbf{FAJITA}} \\
  \midrule 
  \textbf{Class} & \textbf{Scope} & \textbf{SM/BLD} & \textbf{SM}  & \textbf{BLD} & \textbf{NoOPT}  \\
  \midrule
  BinTree
&	3	&	0.06	&	0.05	&	0.63	&	0.886	\\
& 4	&	0.1	&	0.11	&	150.99	&	141.21	\\
&	5	&	0.26	&	0.31	&	-	&	-	\\
&	6	&	0.52	&	0.52	&	-	&	-	\\
&	7	&	01.01	&	1.15	&	-	&	-	\\
&	8	&	2.59	&	3.55	&	-	&	-	\\
&	9	&	12.38	&	12.96	&	-	&	-	\\
&	10	&	51.34	&	59.45	&	-	&	-	\\
&	11	&	265.57	&	204.13	&	-	&	-	\\
&	12	&	1578.72	&	1560.70	&	-	&	-	\\
  \midrule
  AVL
&	3	&	0.02	&	0.07	&	0.24	&	- \\
&	4	&	0.04	&	0.16	&	81.24	&	- \\
&	5	&	0.09	&	0.31	&	-	&	-	\\
&	6	&	0.14	&	0.76	&	-	&	-	\\
&	7	&	0.27	&	2.43	&	-	&	-	\\
&	8	&	0.46	&	7.71	&	-	&	-	\\
&	9	&	0.86	&	27.89	&	-	&	-	\\
&	10	&	1.92	&	104.60	&	-	&	-	\\
&	11	&	5.80	&	391.28	&	-	&	-	\\
&	12	&	14.58	&	1389.78	&	-	&	-	\\
&	13	&	45.45	&	-	&	-	&	-	\\
&	14	&	-	&	-	&	-	&	-	\\
  \midrule
  RBT
&	3	&	0.02	&	0.05	&	25.21	&	-	\\
&	4	&	0.05	&	0.10	&	-	&	-	\\
&	5	&	0.09	&	0.13	&	-	&	-	\\
&	6	&	0.16	&	0.25	&	-	&	-	\\
&	7	&	0.31	&	0.50	&	-	&	-	\\
&	8	&	0.72	&	1.30	&	-	&	-	\\
&	9	&	2.11	&	3.67	&	-	&	-	\\
&	10	&	6.18	&	11.70	&	-	&	-	\\
&	11	&	19.72	&	37.63	&	-	&	-	\\
&	12	&	63.16	&	122.18	&	-	&	-	\\
&	13	&	206.66	&	394.47	&	-	&	-	\\
  \midrule
  BinHeap
&	3	&	0.03	&	0.04	&	1.35	&	-	\\
&	4	&	0.08	&	0.18	&	-	&	-	\\
&	5	&	0.31	&	1.21	&	-	&	-	\\
&	6	&	1.28	&	18.41	&	-	&	-	\\
&	7	&	44452	&	391.53	&	-	&	-	\\
&	8	&	97.08	&	-	&	-	&	-	\\
  \midrule
  SLList
&	3	&	0.02	&	0.04	&	0.44	&	- \\
&	4	&	0.11	&	0.10	&	71.89	&	-	\\
&	5	&	0.33	&	0.38	&	-	&	-	\\
&	6	&	1.58	&	2.31	&	-	&	-	\\
&	7	&	15.38	&	291.53	&	-	&	-	\\
&	8	&	297.41	&	705.71	&	-	&	-	\\
  \midrule
  DLList
&	3	&	0.01	&	0.19	&	0.63	&	-	\\
&	4	&	0.05	&	0.55	&	155.70	&	-	\\
&	5	&	0.17	&	2.50	&	-	&	-	\\
&	6	&	0.83	&	27.58	&	-	&	-	\\
&	7	&	12.14	&	-	&	-	&	-	\\
&	8	&	269.47	&	-	&	-	&	-	\\
  \midrule
  NCL
&	3	&	0.06	&	0.07	&	08.07	&	-	\\
&	4	&	0.26	&	0.19	&	-	&	-	\\
&	5	&	0.84	&	0.65	&	-	&	-	\\
&	6	&	4.80	&	5.95	&	-	&	-	\\
&	7	&	58.75	&	99.29	&	-	&	-	\\
  \bottomrule
\end{tabular}
\end{table}

\begin{table}
\scriptsize
\centering
\label{tab:results-obj2}
\begin{tabular}{ l r | r | r | r | r  }
  \toprule
  \multicolumn{6}{c}{\textbf{ROOPS}} \\
  \midrule 
  \textbf{Class} & \textbf{Scope} & \textbf{SM/BLD} & \textbf{SM}  & \textbf{BLD} & \textbf{NoOPT}  \\
  \midrule
  AVL
&	3	&	0.02	&	0.04	&	.34	&	-	\\
&	4	&	0.03	&	0.07	&	102.16	&	-	\\
&	5	&	0.05	&	0.11	&	-	&	-	\\
&	6	&	.009	&	0.23	&	-	&	-	\\
&	7	&	0.17	&	0.47	&	-	&	-	\\
&	8	&	0.32	&	0.64	&	-	&	-	\\
&	9	&	0.65	&	2.43	&	-	&	-	\\
&	10	&	1.52	&	07.04	&	-	&	-	\\
&	11	&	5.11	&	24.58	&	-	&	-	\\
&	12	&	19.92	&	55.86	&	-	&	-	\\
&	13	&	46.71	&	657.17	&	-	&	-	\\
  \midrule
  NCL
&	3	&	0.04	&	1.31	&	1.37	&	7.96	\\
&	4	&	0.10	&	9.59	&	52.17	&	-	\\
&	5	&	0.34	&	40.54	&	-	&	-	\\
&	6	&	2.27	&	128.51	&	-	&	-	\\
&	7	&	33.89	&	659.40	&	-	&	-	\\
&	8	&	769.63	&	-	&	-	&	-	\\
  \midrule
  BinTree
&	3	&	0.02	&	0.04	&	0.23	&	105.46	\\
&	4	&	0.05	&	0.08	&	85.32	&	-	\\
&	5	&	0.11	&	0.16	&	-	&	-	\\
&	6	&	0.20	&	0.50	&	-	&	-	\\
&	7	&	0.52	&	1.00	&	-	&	-	\\
&	8	&	1.70	&	3.77	&	-	&	-	\\
&	9	&	7.11	&	16.30	&	-	&	-	\\
&	10	&	34.72	&	82.07	&	-	&	-	\\
&	11	&	181.11	&	431.63	&	-	&	-	\\
&	12	&	966.41	&	2281.42	&	-	&	-	\\
  \midrule
  LList
&	3	&	.03	&	0.09	&	0.26	&	-	\\
&	4	&	.07	&	0.48	&	115.27	&	-	\\
&	5	&	.23	&	11.45	&	-	&	-	\\
&	6	&	01.06	&	1169.05	&	-	&	-	\\
&	7	&	12.62	&	-	&	-	&	-	\\
&	8	&	295.94	&	-	&	-	&	-	\\
  \midrule
  RBT
&	3	&	0.04	&	0.04	&	39.11	&	-	\\
&	4	&	0.11	&	0.09	&	-	&	-	\\
&	5	&	0.22	&	0.14	&	-	&	-	\\
&	6	&	0.40	&	0.35	&	-	&	-	\\
&	7	&	0.70	&	0.52	&	-	&	-	\\
&	8	&	1.43	&	1.13	&	-	&	-	\\
&	9	&	4.11	&	3.42	&	-	&	-	\\
&	10	&	13.92	&	11.34	&	-	&	-	\\
&	11	&	31.02	&	51.05	&	-	&	-	\\
&	12	&	81.03	&	2379.44	&	-	&	-	\\
&	13	&	697.06	&	-	&	-	&	-	\\
  \midrule
  FibHeap
&	3	&	0.04	&	0.09	&	0.94	&	-	\\
&	4	&	0.13	&	0.20	&	-	&	-	\\
&	5	&	0.70	&	1.13	&	-	&	-	\\
&	6	&	6.51	&	12.80	&	-	&	-	\\
&	7	&	129.01	&	243.36	&	-	&	-	\\
  \midrule
  BinHeap
&	3	&	0.05	&	0.11	&	2.03	&	18.38	\\
&	4	&	0.09	&	0.34	&	-	&	-	\\
&	5	&	0.26	&	0.96	&	-	&	-	\\
&	6	&	1.31	&	2.96	&	-	&	-	\\
&	7	&	13.06	&	30.23	&	-	&	-	\\
&	8	&	96.94	&	220.18	&	-	&	-	\\
  \bottomrule
\end{tabular}
\end{table}
\vspace{0.3cm}
\begin{table}[H]
\scriptsize

\centering
\label{tab:results-obj3}
\begin{tabular}{ l r | r | r | r | r  }
  \toprule
  \multicolumn{6}{c}{\textbf{Korat}} \\
  \midrule 
  \textbf{Class} & \textbf{Scope} & \textbf{SM/BLD} & \textbf{SM}  & \textbf{BLD} & \textbf{NoOPT}  \\
  \midrule
  DLList
&	3	&	0.06	&	0.07	&	0.53	&	31.63	\\
&	4	&	0.25	&	0.36	&	113.34	&	-	\\
&	5	&	01.01	&	1.30	&	-	&	-	\\
&	6	&	7.11	&	9.97	&	-	&	-	\\
&	7	&	108.08	&	153.01	&	-	&	-	\\
&	8	&	-	&	-	&	-	&	-	\\
  \midrule
  FHeap
&	3	&	0.04	&	0.18	&	0.85	&	-	\\
&	4	&	0.11	&	0.45	&	-	&	-	\\
&	5	&	0.56	&	2.55	&	-	&	-	\\
&	6	&	5.95	&	26.8	&	-	&	-	\\
&	7	&	115.44	&	-	&	-	&	-	\\
&	8	&	-	&	-	&	-	&	-	\\
  \midrule
  BinHeap
&	3	&	0.11	&	0.10	&	0.68	&	-	\\
&	4	&	0.23	&	0.45	&	-	&	-	\\
&	5	&	0.68	&	5.24	&	-	&	-	\\
&	6	&	44503	&	176.12	&	-	&	-	\\
&	7	&	25.32	&	-	&	-	&	-	\\
&	8	&	163.39	&	-	&	-	&	-	\\
&	9	&	-	&		&		&		\\
  \midrule
  BST
&	3	&	0.04	&	0.02	&	0.24	&	0.55	\\
&	4	&	0.09	&	0.02	&	77.21	&	116.03	\\
&	5	&	0.25	&	0.20	&	-	&	-	\\
&	6	&	0.44	&	0.38	&	-	&	-	\\
&	7	&	0.86	&	0.95	&	-	&	-	\\
&	8	&	03.07	&	2.73	&	-	&	-	\\
&	9	&	11.77	&	10.70	&	-	&	-	\\
&	10	&	49.098	&	49.61	&	-	&	-	\\
&	11	&	199.46	&	171.50	&	-	&	-	\\
&	12	&	1341.86	&	1351.38	&	-	&	-	\\
&	13	&	-	&	-	&	-	&	-	\\
  \midrule
  SSList
&	3	&	0.04	&	0.04	&	0.49	&	14.02	\\
&	4	&	0.13	&	0.21	&	106.38	&	-	\\
&	5	&	0.42	&	0.65	&	-	&	-	\\
&	6	&	1.87	&	2.3	&	-	&	-	\\
&	7	&	17.87	&	25.22	&	-	&	-	\\
&	8	&	256.49	&	352.16	&	-	&	-	\\
&	9	&	-	&	-	&	-	&	-	\\
  \midrule
  RBT
&	3	&	0.05	&	.03	&	25.18	&	-	\\
&	4	&	0.10	&	.06	&	-	&	-	\\
&	5	&	0.25	&	.11	&	-	&	-	\\
&	6	&	0.43	&	.21	&	-	&	-	\\
&	7	&	0.91	&	.40	&	-	&	-	\\
&	8	&	1.31	&	1.01	&	-	&	-	\\
&	9	&	4.63	&	2.90	&	-	&	-	\\
&	10	&	13.88	&	9.19	&	-	&	-	\\
&	11	&	33.42	&	28.84	&	-	&	-	\\
&	12	&	79.45	&	94.46	&	-	&	-	\\
&	13	&	689.06	&	308.06	&	-	&	-	\\
&	14	&	-	&	-	&	-	&	-	\\
  \midrule
  SortedList
&	3	&	0.01	&	0.01	&	0.24	&	0.94	\\
&	4	&	0.02	&	0.03	&	74.45	&	-	\\
&	5	&	0.05	&	0.05	&	-	&	-	\\
&	6	&	0.11	&	0.13	&	-	&	-	\\
&	7	&	0.23	&	0.27	&	-	&	-	\\
&	8	&	0.65	&	0.68	&	-	&	-	\\
&	9	&	02.07	&	2.24	&	-	&	-	\\
&	10	&	8.14	&	09.04	&	-	&	-	\\
  \bottomrule
\end{tabular}
\end{table}


\begin{table}[H]
\scriptsize
\centering
\label{tab:results-obj4}
\begin{tabular}{ l r | r | r | r | r  }
  \toprule
  \multicolumn{6}{c}{\textbf{Kiasan}} \\
  \midrule 
  \textbf{Class} & \textbf{Scope} & \textbf{SM/BLD} & \textbf{SM}  & \textbf{BLD} & \textbf{NoOPT}  \\
  \midrule
  BinTree
&	3	&	0.03	&	0.04	&	0.52	&	-	\\
&	4	&	0.08	&	0.10	&	116.71	&	-	\\
&	5	&	0.18	&	0.17	&	-	&	-	\\
&	6	&	0.48	&	0.40	&	-	&	-	\\
&	7	&	01.02	&	1.20	&	-	&	-	\\
&	8	&	2.80	&	4.80	&	-	&	-	\\
&	9	&	11.66	&	22.33	&	-	&	-	\\
&	10	&	50.82	&	112.73	&	-	&	-	\\
&	11	&	204.83	&	583.60	&	-	&	-	\\
&	12	&	1235.67	&	-	&	-	&	-	\\
  \midrule
  DLList
&	3	&	0.02	&	0.37	&	0.22	&	-	\\
&	4	&	0.11	&	1.59	&	72.60	&	-	\\
&	5	&	0.47	&	67.72	&	-	&	-	\\
&	6	&	1.68	&	-	&	-	&	-	\\
&	7	&	18.09	&	-	&	-	&	-	\\
&	8	&	257.42	&	-	&	-	&	-	\\
  \midrule
  DisjSet
&	3	&	0.02	&	0.02	&	0.05	&	1.44	\\
&	4	&	0.06	&	0.07	&	0.24	&	-	\\
&	5	&	0.30	&	0.29	&	2.69	&	-	\\
&	6	&	1.94	&	2.22	&	-	&	-	\\
&	7	&	28.94	&	31.67	&	-	&	-	\\
  \midrule
  RBT
&	3	&	0.05	&	0.04	&	25.34	&	-	\\
&	4	&	0.11	&	0.06	&	-	&	-	\\
&	5	&	0.25	&	0.14	&	-	&	-	\\
&	6	&	0.41	&	0.24	&	-	&	-	\\
&	7	&	0.78	&	0.43	&	-	&	-	\\
&	8	&	1.57	&	1.00	&	-	&	-	\\
&	9	&	4.12	&	2.80	&	-	&	-	\\
&	10	&	11.39	&	8.93	&	-	&	-	\\
&	11	&	34.88	&	29.22	&	-	&	-	\\
&	12	&	84.51	&	95.11	&	-	&	-	\\
  \midrule
  DisjSet
&	3	&	0.02	&	0.02	&	1.49	&	1.45	\\
&	4	&	0.06	&	0.07	&	-	&	-	\\
&	5	&	0.21	&	0.22	&	-	&	-	\\
&	6	&	0.89	&	0.99	&	-	&	-	\\
&	7	&	8.26	&	7.92	&	-	&	-	\\
  \midrule
  StackLi
&	3	&	0.03	&	0.06	&	0.24	&	-	\\
&	4	&	0.10	&	0.18	&	90.27	&	-	\\
&	5	&	0.55	&	0.84	&	-	&	-	\\
&	6	&	4.35	&	8.47	&	-	&	-	\\
&	7	&	83.06	&	158.26	&	-	&	-	\\
  \midrule
  BHeap
&	3	&	0.04	&	0.06	&	0.8	&	-	\\
&	4	&	0.13	&	0.15	&	-	&	-	\\
&	5	&	0.45	&	0.69	&	-	&	-	\\
&	6	&	3.14	&	44.413	&	-	&	-	\\
&	7	&	53.78	&	79.12	&	-	&	-	\\
&	8	&	1221.59	&	1756.71	&	-	&	-	\\
\midrule
  TreeMap
&	3	&	0.14	&	0.50	&	-	&	-	\\
&	4	&	2.15	&	2.59	&	-	&	-	\\
&	5	&	24.95	&	45.46	&	-	&	-	\\
&	6	&	866.71	&	-	&	-	&	-	\\
  \bottomrule
\end{tabular}
\end{table}




\subsection{RQ6: BEAPI para analizar invariantes de representación}
\label{sec:existing-specs-analysis}
\subsection{RQ6: BEAPI para analizar invariantes de representación}

Esta pregunta de investigación evalúa si \textsf{BEAPI} puede ser útil para asistir a los usuarios 
en la detección de fallas en los métodos \texttt{repOK}, a través de la comparación entre el conjunto 
de objetos que se pueden generar mediante la API y aquellos que se pueden generar a partir del 
invariante de representación.

Para ello, diseñamos un procedimiento automatizado en tres pasos. En primer lugar, ejecutamos 
\emph{BEAPI} para generar un conjunto de estructuras, denominado \texttt{SA}, utilizando únicamente 
la API de la clase, y ejecutamos \emph{Korat} para generar un segundo conjunto, \texttt{SR}, 
utilizando únicamente el método \texttt{repOK}. Ambas ejecuciones se realizaron con el mismo 
\emph{scope}. En segundo lugar, canonizamos todas las estructuras de ambos conjuntos utilizando el 
proceso de linearización (Sección~\ref{sec:stateMatching}), de modo que estructuras isomorfas puedan 
compararse de forma directa. Finalmente, comparamos los conjuntos \texttt{SA} y \texttt{SR} para 
identificar diferencias que indiquen discrepancias semánticas entre la definición operativa de la 
clase (la API) y su especificación declarativa (\texttt{repOK}).

El procedimiento puede producir tres resultados distintos. Cuando \texttt{SA} $\subset$ \texttt{SR}, 
puede deberse a que la API genera un subconjunto de las estructuras válidas, que \texttt{repOK} 
sufre de subespecificación (es decir, permite estructuras que no deberían ser válidas), o ambos. 
En estos casos, las estructuras que están en \texttt{SR} pero no en \texttt{SA} son evidencia potencial 
de errores, y fueron inspeccionadas manualmente para confirmar si se trataba efectivamente de fallas 
en \texttt{repOK}. Clasificamos estos errores como \texttt{under}, ya que el invariante permite más 
de lo que debería.

Por el contrario, cuando \texttt{SR} $\subset$ \texttt{SA}, puede indicar que \texttt{repOK} es demasiado 
restrictivo (sobreespecificado), que la API permite estructuras no válidas, o ambas cosas. En este 
caso, se analizan las estructuras que aparecen en \texttt{SA} pero no en \texttt{SR}. Los errores 
confirmados manualmente en esta categoría se etiquetan como \texttt{over}.

En algunos casos, se encontraron diferencias en ambas direcciones: estructuras válidas generadas por 
\textsf{Korat} que no están presentes en \texttt{SA}, y estructuras generadas por \emph{BEAPI} que 
no son aceptadas por \texttt{repOK}. Estas diferencias simultáneas pueden deberse a fallas en ambas 
fuentes (API y \texttt{repOK}), o a errores más profundos de especificación. En estos casos, cuando se 
confirma una falla manualmente, la clasificamos como \texttt{error}.

Cabe destacar que diferencias menores en la definición de \emph{scope} entre los enfoques pueden 
producir "falsos positivos" en esta comparación. Esto ocurrió solamente en las estructuras 
\texttt{RBT} y \texttt{FibHeap}, donde \emph{BEAPI} no generó ciertas estructuras válidas dentro del 
alcance debido a restricciones de balanceo interno que requieren secuencias más largas que el 
\emph{scope} establecido. Estos casos se identificaron y descartaron manualmente, observando que al 
aumentar el \emph{scope}, las estructuras de \textsf{Korat} sí aparecen en la salida de \emph{BEAPI}.

Los resultados del experimento se resumen en la Tabla~\ref{table:bugs}. Encontramos errores en 9 de 
los 26 métodos \texttt{repOK} analizados. Esta proporción evidencia la dificultad de escribir 
invariantes de representación correctos, incluso en bibliotecas diseñadas por expertos, y pone de 
manifiesto el valor de \textsf{BEAPI} como herramienta de apoyo para detectar inconsistencias.

En particular, se detectaron errores de subespecificación en clases como \texttt{RBTree}, \texttt{AVL}, 
\texttt{BinTree}, \texttt{FibHeap} y \texttt{NCL}, donde \texttt{repOK} permite estructuras 
incompletas, con campos nulos o claves no válidas, que no pueden ser generadas mediante la API 
original. También se identificaron casos de sobreespecificación, como en \texttt{FibHeap}, donde 
el método \texttt{repOK} rechaza heaps válidos cuyo nodo mínimo es nulo de forma transitoria. 
Finalmente, se confirmaron errores estructurales en las definiciones de altura y ordenamiento en 
clases como \texttt{AVL}, donde las hojas están mal inicializadas o los cálculos son inconsistentes 
con la implementación.

En conclusión, \emph{BEAPI} no solo es útil para generar estructuras exhaustivamente, sino que 
también actúa como un verificador complementario de especificaciones de invariante, permitiendo 
descubrir errores difíciles de detectar mediante revisión manual o prueba dinámica convencional. 
Este enfoque ofrece una vía práctica y automatizada para mejorar la calidad de las especificaciones 
de clase, lo que redunda en mayor confiabilidad del software verificado.

\begin{table}[H]
\scriptsize
\renewcommand{\arraystretch}{1.2} %
\setlength\dashlinedash{1pt}
\begin{tabular}{ll|ll}
\toprule
\textbf{Bench.} & \textbf{Clase}   & \textbf{Descripción del Error} &   \textbf{Tipo} \\
\midrule
Korat    &     RBTree   & El color de la raíz debería ser rojo & under  \\
\midrule
\multirow{12}{*}{Roops} & \multirow{4}{*}{NCL} & Los valores de las claves en la caché deben estar\\&& establecidos en nulo & under\\ 
\cdashline{3-4}    
                        &                  & El valor clave del nodo ficticio en la lista principal\\&& debería ser nulo & under \\ 
\cline{2-4} 
                        & BinTree              & El padre del nodo raíz debería ser nulo & under \\ 
\cline{2-4} 
                        & RBT                  & El color de la raíz no debería ser rojo  & under\\ 
\cline{2-4} 
                        & \multirow{3}{*}{AVL} & El cálculo de la altura es incorrecto\\&& (las hojas tienen asignado el valor incorrecto) & error\\ 
\cdashline{3-4} 
                        &                      & No se deben permitir valores clave repetidos  &   under\\ 
\cline{2-4} 
                        & \multirow{9}{*}{FibHeap} & Los campos izquierdo y derecho de los nodos \\&&no deben ser nulos & under\\ 
\cdashline{3-4} 
                        &                      & El nodo mínimo siempre debe contener el valor mínimo \\&&en el heap & under \\  
\cdashline{3-4}
                        
                        &                      & Si un nodo no tiene hijos, su grado debería ser cero & under \\ 
\cdashline{3-4} 
                        &                      & Los nodos hijos deben tener claves más pequeñas \\&&que sus padres &  under\\ 
\cdashline{3-4} 
                        &                      & Se obliga a que los campos de los padres de todos \\&&los nodos sean nulos & over\\
\cdashline{3-4} 
                        &                      & Se rechaza el heap con el nodo mínimo establecido en nulo    & over   \\ 
\midrule
\multirow{2}{*}{Kiasan} & DisjSetFast  & La clasificación de la raíz puede ser inválida  & under  \\ 
\cline{2-4} 

\cline{3-4} 
                        & BinaryHeap   & La primera posición de un arreglo (ficticio)\\&& puede contener un elemento &   under \\ 
\midrule
Fajita                  & AVL          & El cálculo de la altura es incorrecto \\&&(las hojas tienen asignado el valor incorrecto)  & error \\ 
\bottomrule
\end{tabular}
\caption{Resumen de los errores encontrados en las \texttt{repOKs} utilizando \textsf{BEAPI}}
\label{table:bugs}
\end{table}



\subsection{RQ7: Usando BEAPI para el testing parametrizado}
\subsection{RQ7: Usando BEAPI para el testing parametrizado}

En esta etapa de la evaluación, se llevó a cabo un análisis exhaustivo para determinar la 
utilidad de los métodos \emph{builders} identificados mediante nuestro enfoque, particularmente 
en el contexto de la generación automatizada de casos de prueba. Estos métodos permiten construir 
objetos que pueden utilizarse como entradas en test suites parametrizadas.

Los tests parametrizados constituyen una técnica eficiente para mejorar la cobertura en pruebas 
automatizadas. En lugar de definir casos de prueba específicos para cada configuración, se 
establece una única prueba que se ejecuta múltiples veces con distintos parámetros. En nuestro 
contexto, dichos parámetros son objetos generados por distintas técnicas, los cuales sirven como 
entrada a los métodos de la clase bajo prueba.

Para llevar a cabo este experimento, diseñamos una test suite parametrizada en la que cada método 
público de la clase bajo análisis es ejercitado con objetos generados por distintas técnicas. Las 
herramientas utilizadas para generar estos objetos fueron: la versión estándar de Randoop, que 
genera directamente una suite de pruebas sin reutilizar objetos previos; una variante de Randoop 
denominada \texttt{R-Serialize}, que serializa los objetos construidos durante la generación de 
tests, permitiendo su reutilización como entrada en tests parametrizados; una versión modificada 
de Randoop que prioriza el uso de métodos \emph{builders} identificados previamente mediante 
nuestro enfoque (denominada \texttt{R-Builders}); y por último, la herramienta \texttt{BEAPI}, que 
produce directamente un conjunto reducido pero válido de objetos a partir de su propia lógica de 
exploración exhaustiva acotada.

A fin de evaluar y comparar el impacto de cada enfoque, se utilizó un benchmark basado en clases 
del paquete \texttt{java.util}: \texttt{HashSet}, \texttt{HashMap}, \texttt{TreeMap}, 
\texttt{TreeSet} y \texttt{LinkedList}. Para cada clase se definió un invariante que permite 
distinguir objetos válidos de inválidos, permitiendo así evaluar la calidad de los generadores.

Las métricas consideradas fueron: la cantidad de objetos válidos e inválidos generados, el tiempo 
de generación (\texttt{GTime}), la cantidad de casos de prueba ejecutados (\texttt{Test}), el 
tiempo total de ejecución de la test suite (\texttt{T(Seg)}), la cobertura de ramas alcanzada 
(\texttt{Ramas}) y la cantidad de mutantes eliminados (\texttt{Mutacion}). Es importante destacar 
que la test suite parametrizada es común a los enfoques \texttt{R-Serialize}, \texttt{R-Builders} 
y \texttt{BEAPI}, permitiendo así una comparación justa sobre los objetos generados. En el caso de 
\texttt{Randoop}, la herramienta genera su propia test suite, por lo que los resultados deben 
interpretarse con cautela.

\begin{table}[H]
\scriptsize
\centering
\begin{tabular}{ c  r  |r | r | r|r|r|r  }
  \toprule
  \textbf{GTime} & \textbf{Tool} & \textbf{Valid}  & \textbf{Invalid} & \textbf{Test}&\textbf{T(Seg)} &\textbf{Ramas}  & \textbf{Mutacion} \\ 
  \midrule
5	&	Randoop	&	-	&	- & 380 & 4	& 53 & 67 \\
& R-Serialize	& 3851 & 264 & 184852 & 4	& 72 &	79 	\\
& R-Builders		& 12812 & 9008 & 614980 & 20 &	82	& 89  \\
&	BEAPI & 32 & - & 1540 & 2	& 82 & 89 \\
\cline{2-8}
10	&	Randoop	&	-	&	- & 380 & 4	& 67 & 67 \\
& R-Serialize	& 7655 & 532 & 367444 & 6	& 75 &	81 	\\
& R-Builders		& 23320 & 16482 & 1119364 & 35 &	82	& 89  \\
&	BEAPI & 32 & - & 1540 & 2	& 82 & 89 \\
\cline{2-8}
20	&	Randoop	&	-	&	- & 380 & 4	& 73 & 85 \\
& R-Serialize	& 14980 & 1015 & 719044 & 8	& 76 &	83 	\\
& R-Builders		& 38047 & 26904 & 1826260 & 55 &	82	& 89  \\
&	BEAPI & 32 & - & 1540 & 2	& 82 & 89 \\

\cline{2-8}
40	&	Randoop	&	-	&	- & 380 & 4	& 53 & 67 \\
& R-Serialize	& 28965 & 2020 & 1390324 & 12		& 76 &	85 	\\
& R-Builders		& 59016 & 41681 & 2832772 & 85 	& 82 &	89 	  \\
&	BEAPI & 32 & - & 1540 & 2	& 82 &	89 \\
\cline{2-8}
150	&	Randoop	&	-	&	- & 12452 & 29	& 81 & 89\\
& R-Serialize	& 97236 & 7088 & 4667332 & 35	& 82 &	89 	\\
& R-Builders		& 125992 & 89292 & 6047620 & 179 & 82 &	89  \\
&	BEAPI & 32 & - & 1540 & 2	& 82 &	89 \\
\midrule
\end{tabular}
\caption{Comparativa de diferentes tecnicas en HashSet}
\label{tab:hashSetTools}
\end{table}


\begin{table}[H]
\scriptsize
\centering
\begin{tabular}{ c  r  |r | r | r|r|r|r  }
  \toprule
  \textbf{GTime} & \textbf{Tool} & \textbf{Valid}  & \textbf{Invalid} & \textbf{Test}&\textbf{T(Seg)} &\textbf{Ramas}  & \textbf{Mutacion} \\ 
  \midrule

5	&	Randoop	&	-	&	-	&	15745	&	9	&	61	&	71	\\
	&	R-Serialize	&	7779	&	1468	&	606762	&	4	&	114	&	99	\\
	&	R-Builders	&	13791	&	2950	&	1075698	&	5	&	148	&	111	\\
	&	BEAPI	&	72	&	-	&	5616	&	2	&	150	&	109	\\
 \cline{2-8}															
10	&	Randoop	&	-	&	-	&	1138	&	5	&	77	&	96	\\
	&	R-Serialize	&	16444	&	3246	&	1282632	&	5	&	114	&	99	\\
	&	R-Builders	&	25569	&	5514	&	1994382	&	7	&	151	&	111	\\
	&	BEAPI	&	72	&	-	&	5616	&	2	&	150	&	109	\\
 \cline{2-8}															
20	&	Randoop	&	-	&	-	&	2374	&	7	&	80	&	100	\\
	&	R-Serialize	&	33188	&	6786	&	2588664	&	8	&	114	&	100	\\
	&	R-Builders	&	43670	&	9358	&	3406260	&	9	&	151	&	111	\\
	&	BEAPI	&	72	&	-	&	5616	&	2	&	150	&	109	\\
 \cline{2-8}															
40	&	Randoop	&	-	&	-	&	4736	&	11	&	80	&	100	\\
	&	R-Serialize	&	62678	&	13170	&	4888884	&	11	&	125	&	109	\\
	&	R-Builders	&	69110	&	14990	&	5390580	&	13	&	154	&	111	\\
	&	BEAPI	&	72	&	-	&	5616	&	2	&	150	&	109	\\
 \cline{2-8}															
150	&	Randoop	&	-	&	-	&	16932	&	33	&	105	&	100	\\
	&	R-Serialize	&	215136	&	47411	&	16780608	&	31	&	125	&	109	\\
	&	R-Builders	&	150637	&	32426	&	11749686	&	25	&	154	&	111	\\
	&	BEAPI	&	72	&	-	&	5616	&	2	&	150	&	109	\\
\midrule
\end{tabular}
\caption{Comparativa de diferentes tecnicas en TreeSet}
\label{tab:treeSetTools}
\end{table}

\begin{table}[H]
\scriptsize
\centering
\begin{tabular}{ c  r  |r | r | r|r|r|r  }
  \toprule
  \textbf{GTime} & \textbf{Tool} & \textbf{Valid}  & \textbf{Invalid} & \textbf{Test}&\textbf{T(Seg)} &\textbf{Ramas}  & \textbf{Mutacion} \\ 
  \midrule
5	&	Randoop	&		&		&	460	&	4	&	27	&	51	\\
	&	R-Serialize	&	3248	&	425	&	295568	&	4	&	115	&	100	\\
	&	R-Builders	&	7924	&	570	&	721084	&	3	&	159	&	152	\\
	&	BEAPI	&	1745	&		&	158795	&	3	&	173	&	140	\\
 \cline{2-8}															
10	&	Randoop	&		&		&	1068	&	4	&	31	&	58	\\
	&	R-Serialize	&	5966	&	775	&	542906	&	3	&	116	&	100	\\
	&	R-Builders	&	14454	&	742	&	1315314	&	3	&	159	&	152	\\
	&	BEAPI	&	1745	&		&	158795	&	3	&	173	&	140	\\
 \cline{2-8}															
20	&	Randoop	&		&		&	2152	&	6	&	31	&	58	\\
	&	R-Serialize	&	11209	&	1410	&	1020019	&	4	&	116	&	100	\\
	&	R-Builders	&	23067	&	936	&	2099097	&	4	&	159	&	152	\\
	&	BEAPI	&	34276	&		&	3119116	&	15	&	179	&	140	\\
 \cline{2-8}															
40	&	Randoop	&		&		&	4284	&	9	&	31	&	58	\\
	&	R-Serialize	&	21260	&	2566	&	1934660	&	5	&	119	&	100	\\
	&	R-Builders	&	33460	&	1141	&	3044860	&	4	&	162	&	152	\\
	&	BEAPI	&	34276	&		&	3119116	&	15	&	179	&	140	\\
 \cline{2-8}															
150	&	Randoop	&		&		&	15902	&	21	&	46	&	58	\\
	&	R-Serialize	&	69747	&	8181	&	6346977	&	7	&	120	&	100	\\
	&	R-Builders	&	68373	&	1645	&	6221943	&	5	&	162	&	152	\\
	&	BEAPI	&34276&		&	3119116	&	15	&	179	&	140	\\
\midrule
\end{tabular}
\caption{Comparativa de diferentes tecnicas en TreeMap}
\label{tab:treeMapTools}
\end{table}

\begin{table}[H]
\scriptsize
\centering
\begin{tabular}{ c  r  |r | r | r|r|r|r  }
  \toprule
  \textbf{GTime} & \textbf{Tool} & \textbf{Valid}  & \textbf{Invalid} & \textbf{Test}&\textbf{T(Seg)} &\textbf{Ramas}  & \textbf{Mutacion} \\ 
  \midrule
5	&	Randoop	&	-	&	-	&	356	&	4	&	39	&	38	\\
	&	R-Serialize	&	1945	&	2875	&	184852	&	4	&	49	&	56	\\
	&	R-Builders	&	15571	&	81103	&	6088262	&	12	&	49	&	57	\\
	&	BEAPI	&	781	&	-	&	305375	&	3	&	49	&	58	\\
 \cline{2-8}															
10	&	Randoop	&	-	&	-	&	794	&	5	&	43	&	43	\\
	&	R-Serialize	&	3996	&	6241	&	367444	&	4	&	49	&	56	\\
	&	R-Builders	&	26085	&	136050	&	10199236	&	19	&	49	&	58	\\
	&	BEAPI	&	781	&	-	&	305375	&	3	&	49	&	58	\\
 \cline{2-8}															
20	&	Randoop	&		&	-	&	1678	&	7	&	45	&	45	\\
	&	R-Serialize	&	7463	&	12437	&	719044	&	5	&	49	&	57	\\
	&	R-Builders	&	40989	&	214029	&	16026700	&	28	&	49	&	58	\\
	&	BEAPI	&	781	&	-	&	305375	&	3	&	49	&	58	\\
 \cline{2-8}															
40	&	Randoop	&		&	-	&	3432	&	11	&	47	&	50	\\
	&	R-Serialize	&	14156	&	24838	&	1390324	&	7	&	49	&	57	\\
	&	R-Builders	&	62051	&	323668	&	24261943	&	40	&	49	&	58	\\
	&	BEAPI	&	781	&	-	&	305375	&	3	&	49	&	58	\\
 \cline{2-8}															
150	&	Randoop	&		&	-	&	12950	&	31	&	47	&	55	\\
	&	R-Serialize	&	48428	&	88638	&	4667332	&	15	&	49	&	57	\\
	&	R-Builders	&	129365	&	677056	&	50581717	&	86	&	49	&	58	\\
	&	BEAPI	&	781	&	-	&	305375	&	3	&	49	&	58	\\
\midrule
\end{tabular}
\caption{Comparativa de diferentes tecnicas en LinkedList}
\label{tab:linkedListTools}
\end{table}

\begin{table}[H]
\scriptsize
\centering
\begin{tabular}{ c  r  |r | r | r|r|r|r  }
  \toprule
  \textbf{GTime} & \textbf{Tool} & \textbf{Valid}  & \textbf{Invalid} & \textbf{Test}&\textbf{T(Seg)} &\textbf{Ramas}  & \textbf{Mutacion} \\ 
  \midrule
5	&	Randoop	&		&	-	&	78	&	2	&	48	&	66	\\
	&	R-Serialize	&	3290	&	348	&	649448	&	9	&	95	&	118	\\
	&	R-Builders	&	4222	& 908		&	856088	&	15	&	119	&	127	\\
	&	BEAPI	&	1203	&	-	&	246623	&	4	&	102	&	119	 \\
 \cline{2-8}															
10	&	Randoop	&		&	-	&	80	&	3	&	48	&	66	\\
	&	R-Serialize	&	6317	&	686	&	1285563	&	17	&	104	&	120	\\
	&	R-Builders	&	5666	&	1218	&	1158668	&	19	&	119	&	127	\\
	&	BEAPI	&	10453	&	-	&	2142873	&	9	&	122	&	120	\\
 \cline{2-8}															
20	&	Randoop	&	-	&	-	&	82	&	3	&	48	&	66	\\
	&	R-Serialize	&	11762	&	1200	&	2415318	&	26	&	104	&	120	\\
	&	R-Builders	&	7249	&	1559	&	1483798	&	24	&	119	&	127	\\
	&	BEAPI	&	10453	&	-	&	2142873	&	9	&	122	&	120	 \\
 \cline{2-8}															
40	&	Randoop	&	-	&	-	&	90	&	3	&	48	&	66	\\
	&	R-Serialize	&	22284	&	2042	&	4557158	&	46	&	104	&	120	\\
	&	R-Builders	&	9318	& 2003	&	1909583	&	30	&	119	&	127	\\
	&	BEAPI	&	10453	&	-	&	2142873	&	9	&	122	&	120	\\ 
 \cline{2-8}															
150	&	Randoop	&	-	&	-	&	108	&	3	&	48	&	67	\\
	&	R-Serialize	&	76581	&	5590	&	15567913	&	149	&	110	&	120	\\
	&	R-Builders	&	14542	&	3127	&	2962873	&	46	&	119	&	128	\\
	&	BEAPI	&	10453	&	-	&	2142873	&	9	&	122	&	120		\\
\midrule
\end{tabular}
\caption{Comparativa de diferentes tecnicas en HashMap}
\label{tab:hashMapTools}
\end{table}

La Tabla~\ref{tab:hashSetTools} muestra una comparativa de diferentes técnicas aplicadas sobre 
\texttt{HashSet}. Se observa que al priorizar métodos \emph{builders} en Randoop, la cantidad de 
objetos generados se incrementa significativamente (más de 300\% en comparación con la versión 
estándar), lo cual impacta positivamente en la cobertura.

En contraste, \textsf{BEAPI} genera sólo 32 objetos válidos, pero logra una cobertura de ramas y 
mutación equivalente a \texttt{R-Builders}, y superior a la alcanzada por \texttt{R-Serialize} y 
\texttt{Randoop} en lapsos de tiempo cortos. Esta eficiencia evidencia que BEAPI puede generar un 
conjunto reducido de objetos, pero de alta calidad para testing, en muy poco tiempo (menos de 5 
segundos).

En términos de ejecución, mientras \texttt{R-Builders} y \texttt{R-Serialize} requieren ejecutar 
millones de tests para alcanzar altos niveles de cobertura, \texttt{BEAPI} logra el mismo nivel 
ejecutando apenas unos pocos miles. Esto indica que los objetos generados por \textsf{BEAPI} son 
altamente efectivos como entradas, permitiendo detectar fallas y explorar el espacio de estados 
de manera eficiente.

Los resultados obtenidos permiten concluir que los objetos generados por \textsf{BEAPI} son 
efectivos y eficientes para ser utilizados en tests parametrizados. Aunque la cantidad total de 
objetos generados es menor, su calidad y representatividad permiten alcanzar los máximos niveles 
de cobertura con una mínima ejecución. Además, se observa que guiar a generadores aleatorios como 
Randoop mediante la identificación previa de métodos \emph{builders} también mejora significativamente 
la cobertura. Sin embargo, \textsf{BEAPI} logra estos mismos resultados sin depender de técnicas 
aleatorias ni ejecutar millones de pruebas.

En contextos donde el tiempo de ejecución es un factor crítico, o donde se necesita eficiencia sin 
sacrificar cobertura, \textsf{BEAPI} se posiciona como una herramienta valiosa para la generación 
de objetos de prueba. Estos resultados también reafirman la importancia de identificar y utilizar 
adecuadamente los métodos \emph{builders} para mejorar la calidad de los tests automatizados.












\cacho{Esto es viejo}

\cacho{ya esta en el capitulo, lo movi para alla
}
Para implementar el algoritmo genético, utilizamos una biblioteca muy popular en Java llamada \emph{Jenetics}\footnote{https://jenetics.io/}. Esta biblioteca está diseñada específicamente para algoritmos evolutivos y nos proporcionó las herramientas necesarias para desarrollar nuestro enfoque genético.

\emph{Jenetics} es una biblioteca robusta y versátil que ofrece una amplia gama de funcionalidades para la implementación de algoritmos genéticos. Nos permitió definir y manipular genes, cromosomas y poblaciones, así como utilizar operadores genéticos como selección, cruzamiento y mutación. Además, cuenta con un sólido conjunto de herramientas de optimización y técnicas de evolución que nos permitieron adaptar el algoritmo a nuestras necesidades específicas.
Gracias a \emph{Jenetics}, pudimos implementar el algoritmo genético de manera eficiente y efectiva, lo que nos permitió explorar y encontrar subconjuntos óptimos de métodos \emph{builders} para las diferentes estructuras de datos en nuestro estudio. Además, \emph{Jenetics} es muy fácil de usar, con una documentación completa y una comunidad activa de usuarios que proporciona soporte y ayuda. 


% \subsubsection{Variacion de acuerdo a los Parametros}

% Ademas, examinamos la sensibilidad de los parámetros del Algoritmo Genético y exploramos los usos de los builders generados. Los resultados obtenidos proporcionaron información valiosa sobre el rendimiento y la utilidad de nuestra técnica en la generación de builders y su aplicación en diversas tareas.

% Nuestra comparacion se basa en medir la cantidad de tiempo que le lleva a cada Algortimo terminar la ejecucio, en la cantidad de candidatos que evalua la funcion de valoracion y cuan bueno es en eficacia para encontrar el minimo y suficiente subconjunto de metodos que pudimos observar en nuestro ground truth \cacho{Agregar seccion}. Ejecutamos el algoritmo 10 veces con el resto de los parametros que no esta en evaluacion con un valor promedio (Crossover=0.5, mutation 0.1, Tournanament 4).
% Tambien utilizamos 30 segundos para la fitness con randoop y scope 6 para BEAPI.

% En la tabla \ref{tab:CrossOverGA} se puede observar como se comporta el algoritmo cuando se utiliza diferente rate para el operador de CrossOver. 



\emph{Sobre el costo de la identificación de métodos builders.} 


\begin{table}[H]
\scriptsize

\centering
\begin{tabular}{ l r | r | r | r | r  }
  \toprule
  \multicolumn{6}{c}{\textbf{Real World}} \\
  \midrule 
  \textbf{Class} & \textbf{Scope} & \textbf{SM/BLD} & \textbf{SM}  & \textbf{BLD} & \textbf{NoOPT}  \\
  \midrule
  NCL
&	3	&	.10	&	.47	&	-	&	-	\\
&	4	&	.41	&	3.48	&	-	&	-	\\
&	5	&	3.33	&	-	&	-	&	-	\\
&	6	&	73.78	&	-	&	-	&	-	\\
  \midrule
  TSet
&	3	&	.03	&	.07	&	56.82	&	-	\\
&	4	&	.06	&	.13	&	-	&	-	\\
&	5	&	.11	&	.22	&	-	&	-	\\
&	6	&	.17	&	.42	&	-	&	-	\\
&	7	&	0.31	&	.91	&	-	&	-	\\
&	8	&	0.74	&	2.66	&	-	&	-	\\
&	9	&	2.23	&	7.80	&	-	&	-	\\
&	10	&	6.88	&	26.34	&	-	&	-	\\
&	11	&	21.52	&	86.06	&	-	&	-	\\
&	12	&	69.98	&	276.85	&	-	&	-	\\
&	13	&	226.66	&	887.83	&	-	&	-	\\
    \midrule
  TMap
&	3	&	.11	&	.25	&	-	&	-	\\
&	4	&	.75	&	2.36	&	-	&	-	\\
&	5	&	15.97	&	57.64	&	-	&	-	\\
&	6	&	839.87	&	2901.37	&	-	&	-	\\
  \midrule
  LList
&	3	&	.02	&	.13	&	.64	&	-	\\
&	4	&	.06	&	.38	&		&		\\
&	5	&	.20	&	3.80	&		&		\\
&	6	&	.96	&	258.85	&		&		\\
&	7	&	12.98	&		&		&		\\
&	8	&	286.21	&		&		&		\\
  \midrule
  HMap
&	3	&	.10	&	11.49	&	-	&	-	\\
&	4	&	.55	&		&		&		\\
&	5	&	5.33	&		&		&		\\
&	6	&	119.87	&		&		&		\\
  \midrule
  Schedule
&	3	&	.01	&	.01	&	59.27	&	-	\\
&	4	&	.82	&	45.55	&	-	&	-	\\
&	5	&	1.43	&	-	&	-	&	-	\\
&	6	&	06.01	&	-	&	-	&	-	\\
&	7	&	23.32	&	-	&	-	&	-	\\
  \bottomrule

\end{tabular}
\caption{Tiempos de generacion de acuerdo a las optimizaciones propuestas}
\label{tab:results-realWorld}

\end{table}


\begin{table}[H]
\scriptsize
\centering
\label{tab:results-obj1}
\begin{tabular}{ l r | r | r | r | r  }
  \toprule
  \multicolumn{6}{c}{\textbf{FAJITA}} \\
  \midrule 
  \textbf{Class} & \textbf{Scope} & \textbf{SM/BLD} & \textbf{SM}  & \textbf{BLD} & \textbf{NoOPT}  \\
  \midrule
  BinTree
&	3	&	0.06	&	0.05	&	0.63	&	0.886	\\
& 4	&	0.1	&	0.11	&	150.99	&	141.21	\\
&	5	&	0.26	&	0.31	&	-	&	-	\\
&	6	&	0.52	&	0.52	&	-	&	-	\\
&	7	&	01.01	&	1.15	&	-	&	-	\\
&	8	&	2.59	&	3.55	&	-	&	-	\\
&	9	&	12.38	&	12.96	&	-	&	-	\\
&	10	&	51.34	&	59.45	&	-	&	-	\\
&	11	&	265.57	&	204.13	&	-	&	-	\\
&	12	&	1578.72	&	1560.70	&	-	&	-	\\
  \midrule
  AVL
&	3	&	0.02	&	0.07	&	0.24	&	- \\
&	4	&	0.04	&	0.16	&	81.24	&	- \\
&	5	&	0.09	&	0.31	&	-	&	-	\\
&	6	&	0.14	&	0.76	&	-	&	-	\\
&	7	&	0.27	&	2.43	&	-	&	-	\\
&	8	&	0.46	&	7.71	&	-	&	-	\\
&	9	&	0.86	&	27.89	&	-	&	-	\\
&	10	&	1.92	&	104.60	&	-	&	-	\\
&	11	&	5.80	&	391.28	&	-	&	-	\\
&	12	&	14.58	&	1389.78	&	-	&	-	\\
&	13	&	45.45	&	-	&	-	&	-	\\
&	14	&	-	&	-	&	-	&	-	\\
  \midrule
  RBT
&	3	&	0.02	&	0.05	&	25.21	&	-	\\
&	4	&	0.05	&	0.10	&	-	&	-	\\
&	5	&	0.09	&	0.13	&	-	&	-	\\
&	6	&	0.16	&	0.25	&	-	&	-	\\
&	7	&	0.31	&	0.50	&	-	&	-	\\
&	8	&	0.72	&	1.30	&	-	&	-	\\
&	9	&	2.11	&	3.67	&	-	&	-	\\
&	10	&	6.18	&	11.70	&	-	&	-	\\
&	11	&	19.72	&	37.63	&	-	&	-	\\
&	12	&	63.16	&	122.18	&	-	&	-	\\
&	13	&	206.66	&	394.47	&	-	&	-	\\
  \midrule
  BinHeap
&	3	&	0.03	&	0.04	&	1.35	&	-	\\
&	4	&	0.08	&	0.18	&	-	&	-	\\
&	5	&	0.31	&	1.21	&	-	&	-	\\
&	6	&	1.28	&	18.41	&	-	&	-	\\
&	7	&	44452	&	391.53	&	-	&	-	\\
&	8	&	97.08	&	-	&	-	&	-	\\
  \midrule
  SLList
&	3	&	0.02	&	0.04	&	0.44	&	- \\
&	4	&	0.11	&	0.10	&	71.89	&	-	\\
&	5	&	0.33	&	0.38	&	-	&	-	\\
&	6	&	1.58	&	2.31	&	-	&	-	\\
&	7	&	15.38	&	291.53	&	-	&	-	\\
&	8	&	297.41	&	705.71	&	-	&	-	\\
  \midrule
  DLList
&	3	&	0.01	&	0.19	&	0.63	&	-	\\
&	4	&	0.05	&	0.55	&	155.70	&	-	\\
&	5	&	0.17	&	2.50	&	-	&	-	\\
&	6	&	0.83	&	27.58	&	-	&	-	\\
&	7	&	12.14	&	-	&	-	&	-	\\
&	8	&	269.47	&	-	&	-	&	-	\\
  \midrule
  NCL
&	3	&	0.06	&	0.07	&	08.07	&	-	\\
&	4	&	0.26	&	0.19	&	-	&	-	\\
&	5	&	0.84	&	0.65	&	-	&	-	\\
&	6	&	4.80	&	5.95	&	-	&	-	\\
&	7	&	58.75	&	99.29	&	-	&	-	\\
  \bottomrule
\end{tabular}
\end{table}

\begin{table}
\scriptsize
\centering
\label{tab:results-obj2}
\begin{tabular}{ l r | r | r | r | r  }
  \toprule
  \multicolumn{6}{c}{\textbf{ROOPS}} \\
  \midrule 
  \textbf{Class} & \textbf{Scope} & \textbf{SM/BLD} & \textbf{SM}  & \textbf{BLD} & \textbf{NoOPT}  \\
  \midrule
  AVL
&	3	&	0.02	&	0.04	&	.34	&	-	\\
&	4	&	0.03	&	0.07	&	102.16	&	-	\\
&	5	&	0.05	&	0.11	&	-	&	-	\\
&	6	&	.009	&	0.23	&	-	&	-	\\
&	7	&	0.17	&	0.47	&	-	&	-	\\
&	8	&	0.32	&	0.64	&	-	&	-	\\
&	9	&	0.65	&	2.43	&	-	&	-	\\
&	10	&	1.52	&	07.04	&	-	&	-	\\
&	11	&	5.11	&	24.58	&	-	&	-	\\
&	12	&	19.92	&	55.86	&	-	&	-	\\
&	13	&	46.71	&	657.17	&	-	&	-	\\
  \midrule
  NCL
&	3	&	0.04	&	1.31	&	1.37	&	7.96	\\
&	4	&	0.10	&	9.59	&	52.17	&	-	\\
&	5	&	0.34	&	40.54	&	-	&	-	\\
&	6	&	2.27	&	128.51	&	-	&	-	\\
&	7	&	33.89	&	659.40	&	-	&	-	\\
&	8	&	769.63	&	-	&	-	&	-	\\
  \midrule
  BinTree
&	3	&	0.02	&	0.04	&	0.23	&	105.46	\\
&	4	&	0.05	&	0.08	&	85.32	&	-	\\
&	5	&	0.11	&	0.16	&	-	&	-	\\
&	6	&	0.20	&	0.50	&	-	&	-	\\
&	7	&	0.52	&	1.00	&	-	&	-	\\
&	8	&	1.70	&	3.77	&	-	&	-	\\
&	9	&	7.11	&	16.30	&	-	&	-	\\
&	10	&	34.72	&	82.07	&	-	&	-	\\
&	11	&	181.11	&	431.63	&	-	&	-	\\
&	12	&	966.41	&	2281.42	&	-	&	-	\\
  \midrule
  LList
&	3	&	.03	&	0.09	&	0.26	&	-	\\
&	4	&	.07	&	0.48	&	115.27	&	-	\\
&	5	&	.23	&	11.45	&	-	&	-	\\
&	6	&	01.06	&	1169.05	&	-	&	-	\\
&	7	&	12.62	&	-	&	-	&	-	\\
&	8	&	295.94	&	-	&	-	&	-	\\
  \midrule
  RBT
&	3	&	0.04	&	0.04	&	39.11	&	-	\\
&	4	&	0.11	&	0.09	&	-	&	-	\\
&	5	&	0.22	&	0.14	&	-	&	-	\\
&	6	&	0.40	&	0.35	&	-	&	-	\\
&	7	&	0.70	&	0.52	&	-	&	-	\\
&	8	&	1.43	&	1.13	&	-	&	-	\\
&	9	&	4.11	&	3.42	&	-	&	-	\\
&	10	&	13.92	&	11.34	&	-	&	-	\\
&	11	&	31.02	&	51.05	&	-	&	-	\\
&	12	&	81.03	&	2379.44	&	-	&	-	\\
&	13	&	697.06	&	-	&	-	&	-	\\
  \midrule
  FibHeap
&	3	&	0.04	&	0.09	&	0.94	&	-	\\
&	4	&	0.13	&	0.20	&	-	&	-	\\
&	5	&	0.70	&	1.13	&	-	&	-	\\
&	6	&	6.51	&	12.80	&	-	&	-	\\
&	7	&	129.01	&	243.36	&	-	&	-	\\
  \midrule
  BinHeap
&	3	&	0.05	&	0.11	&	2.03	&	18.38	\\
&	4	&	0.09	&	0.34	&	-	&	-	\\
&	5	&	0.26	&	0.96	&	-	&	-	\\
&	6	&	1.31	&	2.96	&	-	&	-	\\
&	7	&	13.06	&	30.23	&	-	&	-	\\
&	8	&	96.94	&	220.18	&	-	&	-	\\
  \bottomrule
\end{tabular}
\end{table}
\vspace{0.3cm}
\begin{table}[H]
\scriptsize

\centering
\label{tab:results-obj3}
\begin{tabular}{ l r | r | r | r | r  }
  \toprule
  \multicolumn{6}{c}{\textbf{Korat}} \\
  \midrule 
  \textbf{Class} & \textbf{Scope} & \textbf{SM/BLD} & \textbf{SM}  & \textbf{BLD} & \textbf{NoOPT}  \\
  \midrule
  DLList
&	3	&	0.06	&	0.07	&	0.53	&	31.63	\\
&	4	&	0.25	&	0.36	&	113.34	&	-	\\
&	5	&	01.01	&	1.30	&	-	&	-	\\
&	6	&	7.11	&	9.97	&	-	&	-	\\
&	7	&	108.08	&	153.01	&	-	&	-	\\
&	8	&	-	&	-	&	-	&	-	\\
  \midrule
  FHeap
&	3	&	0.04	&	0.18	&	0.85	&	-	\\
&	4	&	0.11	&	0.45	&	-	&	-	\\
&	5	&	0.56	&	2.55	&	-	&	-	\\
&	6	&	5.95	&	26.8	&	-	&	-	\\
&	7	&	115.44	&	-	&	-	&	-	\\
&	8	&	-	&	-	&	-	&	-	\\
  \midrule
  BinHeap
&	3	&	0.11	&	0.10	&	0.68	&	-	\\
&	4	&	0.23	&	0.45	&	-	&	-	\\
&	5	&	0.68	&	5.24	&	-	&	-	\\
&	6	&	44503	&	176.12	&	-	&	-	\\
&	7	&	25.32	&	-	&	-	&	-	\\
&	8	&	163.39	&	-	&	-	&	-	\\
&	9	&	-	&		&		&		\\
  \midrule
  BST
&	3	&	0.04	&	0.02	&	0.24	&	0.55	\\
&	4	&	0.09	&	0.02	&	77.21	&	116.03	\\
&	5	&	0.25	&	0.20	&	-	&	-	\\
&	6	&	0.44	&	0.38	&	-	&	-	\\
&	7	&	0.86	&	0.95	&	-	&	-	\\
&	8	&	03.07	&	2.73	&	-	&	-	\\
&	9	&	11.77	&	10.70	&	-	&	-	\\
&	10	&	49.098	&	49.61	&	-	&	-	\\
&	11	&	199.46	&	171.50	&	-	&	-	\\
&	12	&	1341.86	&	1351.38	&	-	&	-	\\
&	13	&	-	&	-	&	-	&	-	\\
  \midrule
  SSList
&	3	&	0.04	&	0.04	&	0.49	&	14.02	\\
&	4	&	0.13	&	0.21	&	106.38	&	-	\\
&	5	&	0.42	&	0.65	&	-	&	-	\\
&	6	&	1.87	&	2.3	&	-	&	-	\\
&	7	&	17.87	&	25.22	&	-	&	-	\\
&	8	&	256.49	&	352.16	&	-	&	-	\\
&	9	&	-	&	-	&	-	&	-	\\
  \midrule
  RBT
&	3	&	0.05	&	.03	&	25.18	&	-	\\
&	4	&	0.10	&	.06	&	-	&	-	\\
&	5	&	0.25	&	.11	&	-	&	-	\\
&	6	&	0.43	&	.21	&	-	&	-	\\
&	7	&	0.91	&	.40	&	-	&	-	\\
&	8	&	1.31	&	1.01	&	-	&	-	\\
&	9	&	4.63	&	2.90	&	-	&	-	\\
&	10	&	13.88	&	9.19	&	-	&	-	\\
&	11	&	33.42	&	28.84	&	-	&	-	\\
&	12	&	79.45	&	94.46	&	-	&	-	\\
&	13	&	689.06	&	308.06	&	-	&	-	\\
&	14	&	-	&	-	&	-	&	-	\\
  \midrule
  SortedList
&	3	&	0.01	&	0.01	&	0.24	&	0.94	\\
&	4	&	0.02	&	0.03	&	74.45	&	-	\\
&	5	&	0.05	&	0.05	&	-	&	-	\\
&	6	&	0.11	&	0.13	&	-	&	-	\\
&	7	&	0.23	&	0.27	&	-	&	-	\\
&	8	&	0.65	&	0.68	&	-	&	-	\\
&	9	&	02.07	&	2.24	&	-	&	-	\\
&	10	&	8.14	&	09.04	&	-	&	-	\\
  \bottomrule
\end{tabular}
\end{table}


\begin{table}[H]
\scriptsize
\centering
\label{tab:results-obj4}
\begin{tabular}{ l r | r | r | r | r  }
  \toprule
  \multicolumn{6}{c}{\textbf{Kiasan}} \\
  \midrule 
  \textbf{Class} & \textbf{Scope} & \textbf{SM/BLD} & \textbf{SM}  & \textbf{BLD} & \textbf{NoOPT}  \\
  \midrule
  BinTree
&	3	&	0.03	&	0.04	&	0.52	&	-	\\
&	4	&	0.08	&	0.10	&	116.71	&	-	\\
&	5	&	0.18	&	0.17	&	-	&	-	\\
&	6	&	0.48	&	0.40	&	-	&	-	\\
&	7	&	01.02	&	1.20	&	-	&	-	\\
&	8	&	2.80	&	4.80	&	-	&	-	\\
&	9	&	11.66	&	22.33	&	-	&	-	\\
&	10	&	50.82	&	112.73	&	-	&	-	\\
&	11	&	204.83	&	583.60	&	-	&	-	\\
&	12	&	1235.67	&	-	&	-	&	-	\\
  \midrule
  DLList
&	3	&	0.02	&	0.37	&	0.22	&	-	\\
&	4	&	0.11	&	1.59	&	72.60	&	-	\\
&	5	&	0.47	&	67.72	&	-	&	-	\\
&	6	&	1.68	&	-	&	-	&	-	\\
&	7	&	18.09	&	-	&	-	&	-	\\
&	8	&	257.42	&	-	&	-	&	-	\\
  \midrule
  DisjSet
&	3	&	0.02	&	0.02	&	0.05	&	1.44	\\
&	4	&	0.06	&	0.07	&	0.24	&	-	\\
&	5	&	0.30	&	0.29	&	2.69	&	-	\\
&	6	&	1.94	&	2.22	&	-	&	-	\\
&	7	&	28.94	&	31.67	&	-	&	-	\\
  \midrule
  RBT
&	3	&	0.05	&	0.04	&	25.34	&	-	\\
&	4	&	0.11	&	0.06	&	-	&	-	\\
&	5	&	0.25	&	0.14	&	-	&	-	\\
&	6	&	0.41	&	0.24	&	-	&	-	\\
&	7	&	0.78	&	0.43	&	-	&	-	\\
&	8	&	1.57	&	1.00	&	-	&	-	\\
&	9	&	4.12	&	2.80	&	-	&	-	\\
&	10	&	11.39	&	8.93	&	-	&	-	\\
&	11	&	34.88	&	29.22	&	-	&	-	\\
&	12	&	84.51	&	95.11	&	-	&	-	\\
  \midrule
  DisjSet
&	3	&	0.02	&	0.02	&	1.49	&	1.45	\\
&	4	&	0.06	&	0.07	&	-	&	-	\\
&	5	&	0.21	&	0.22	&	-	&	-	\\
&	6	&	0.89	&	0.99	&	-	&	-	\\
&	7	&	8.26	&	7.92	&	-	&	-	\\
  \midrule
  StackLi
&	3	&	0.03	&	0.06	&	0.24	&	-	\\
&	4	&	0.10	&	0.18	&	90.27	&	-	\\
&	5	&	0.55	&	0.84	&	-	&	-	\\
&	6	&	4.35	&	8.47	&	-	&	-	\\
&	7	&	83.06	&	158.26	&	-	&	-	\\
  \midrule
  BHeap
&	3	&	0.04	&	0.06	&	0.8	&	-	\\
&	4	&	0.13	&	0.15	&	-	&	-	\\
&	5	&	0.45	&	0.69	&	-	&	-	\\
&	6	&	3.14	&	44.413	&	-	&	-	\\
&	7	&	53.78	&	79.12	&	-	&	-	\\
&	8	&	1221.59	&	1756.71	&	-	&	-	\\
\midrule
  TreeMap
&	3	&	0.14	&	0.50	&	-	&	-	\\
&	4	&	2.15	&	2.59	&	-	&	-	\\
&	5	&	24.95	&	45.46	&	-	&	-	\\
&	6	&	866.71	&	-	&	-	&	-	\\
  \bottomrule
\end{tabular}
\end{table}



\subsubsection{Uso de Builders en BEAPI}
Para las conclusiones de esta sección, es suficiente decir que se empleó el ámbito 5 para la identificación de métodos builders en todos los casos, y que el tiempo de ejecución máximo del enfoque fue de 132 segundos en las estructuras de datos del mundo real (\texttt{TreeMap}, 61 métodos). Verificamos manualmente que los métodos identificados incluyeran un conjunto suficiente de constructores en todos los casos. Ten en cuenta que BEG se realiza a menudo para ámbitos cada vez más grandes, y los métodos builders identificados se pueden reutilizar en diferentes ejecuciones. Por lo tanto, los tiempos de identificación de constructores se amortizan en diferentes ejecuciones, lo que dificulta calcular cuánto tiempo de identificación de constructores se agrega a los tiempos de ejecución de \textsf{BEAPI} en cada caso. Por lo tanto, no incluimos los tiempos de identificación de metodos builders en los tiempos de ejecución de \textsf{BEAPI} en ninguno de los experimentos. Observa que, para los scopes más grandes, que son los más importantes, el tiempo de identificación de constructores es insignificante en relación con los tiempos de generación.
Estos resultados de identificación de métodos builders, se pueden observar con detalles en la sección \ref{sec:buildersExp}.



\subsection{Uso de BEAPI para analizar especificaciones}
\label{sec:existing-specs-analysis}
\begin{table}[H]
\scriptsize
\renewcommand{\arraystretch}{1.2} %
\setlength\dashlinedash{1pt}
\begin{tabular}{ll|ll}
\toprule
\textbf{Bench.} & \textbf{Clase}   & \textbf{Descripción del Error} &   \textbf{Tipo} \\
\midrule
Korat    &     RBTree   & El color de la raíz debería ser rojo & under  \\
\midrule
\multirow{12}{*}{Roops} & \multirow{4}{*}{NCL} & Los valores de las claves en la caché deben estar\\&& establecidos en nulo & under\\ 
\cdashline{3-4}    
                        &                  & El valor clave del nodo ficticio en la lista principal\\&& debería ser nulo & under \\ 
\cline{2-4} 
                        & BinTree              & El padre del nodo raíz debería ser nulo & under \\ 
\cline{2-4} 
                        & RBT                  & El color de la raíz no debería ser rojo  & under\\ 
\cline{2-4} 
                        & \multirow{3}{*}{AVL} & El cálculo de la altura es incorrecto\\&& (las hojas tienen asignado el valor incorrecto) & error\\ 
\cdashline{3-4} 
                        &                      & No se deben permitir valores clave repetidos  &   under\\ 
\cline{2-4} 
                        & \multirow{9}{*}{FibHeap} & Los campos izquierdo y derecho de los nodos \\&&no deben ser nulos & under\\ 
\cdashline{3-4} 
                        &                      & El nodo mínimo siempre debe contener el valor mínimo \\&&en el heap & under \\  
\cdashline{3-4}
                        
                        &                      & Si un nodo no tiene hijos, su grado debería ser cero & under \\ 
\cdashline{3-4} 
                        &                      & Los nodos hijos deben tener claves más pequeñas \\&&que sus padres &  under\\ 
\cdashline{3-4} 
                        &                      & Se obliga a que los campos de los padres de todos \\&&los nodos sean nulos & over\\
\cdashline{3-4} 
                        &                      & Se rechaza el heap con el nodo mínimo establecido en nulo    & over   \\ 
\midrule
\multirow{2}{*}{Kiasan} & DisjSetFast  & La clasificación de la raíz puede ser inválida  & under  \\ 
\cline{2-4} 

\cline{3-4} 
                        & BinaryHeap   & La primera posición de un arreglo (ficticio)\\&& puede contener un elemento &   under \\ 
\midrule
Fajita                  & AVL          & El cálculo de la altura es incorrecto \\&&(las hojas tienen asignado el valor incorrecto)  & error \\ 
\bottomrule
\end{tabular}
\caption{Resumen de los errores encontrados en las \texttt{repOKs} utilizando \textsf{BEAPI}}
\label{table:bugs}
\end{table}


La RQ3 aborda si \textsf{BEAPI} puede ser útil para ayudar al usuario a encontrar fallas en \texttt{repOK}s, mediante la comparación del conjunto de objetos que se pueden generar utilizando la API y el conjunto de objetos generados a partir de utilizar un invariante, como es el caso del \texttt{repOK}. Diseñamos el siguiente procedimiento automatizado. Primero, ejecutamos \emph{BEAPI} para generar un conjunto, \texttt{SA}, de estructuras a partir de la API, y utilizamos \emph{Korat} para generar un conjunto, \texttt{SR}, a partir de \texttt{repOK}, utilizando el mismo ámbito para ambas herramientas. En segundo lugar, canonizamos las estructuras tanto en \texttt{SA} como en \texttt{SR} utilizando la linearización (Sección \ref{sec:stateMatching}). En tercer lugar, comparamos los conjuntos \texttt{SA} y \texttt{SR} en cuanto a igualdad. Las diferencias en esta comparación señalan una discrepancia entre \texttt{repOK} y la API. Existen tres posibles resultados para este procedimiento automatizado. Si \texttt{SA} $\subset$ \texttt{SR}, es posible que la API genere un subconjunto de las estructuras válidas, que \texttt{repOK} sufra de subespecificación (\texttt(under)) (restricciones faltantes), o ambos. En este caso, las estructuras en \texttt{SR} que no pertenecen a \texttt{SA} son evidencia del problema, y el usuario debe analizarlas manualmente para descubrir dónde está el error. Aquí, informamos los errores de subespecificación (confirmados manualmente) en \emph{repOK}s que son evidenciados por las estructuras mencionadas. En contraste, cuando \texttt{SR} $\subset$ \texttt{SA}, puede ser el caso de que la API genere un superconjunto de las estructuras válidas, que \texttt{repOK} sufra de sobreespecificación, \texttt(over), (\texttt{repOK} es demasiado restrictivo), o ambos. Las estructuras en \texttt{SA} que no pertenecen a \texttt{SR} podrían indicar la raíz del error, y nuevamente deben ser analizadas manualmente por el usuario. Informamos los errores de sobreespecificación (confirmados manualmente) en \texttt{repOK}s que son evidenciados por estas estructuras. Finalmente, puede darse el caso de que haya estructuras en \texttt{SR} que no pertenecen a \texttt{SA}, y que haya estructuras (distintas de las anteriores) en \texttt{SA} que no pertenecen a \texttt{SR}. Estos pueden ser debidos a fallos en la API, fallas en \texttt{repOK}, o ambos. Informamos las fallas confirmadas manualmente en \texttt{repOK}s que son evidenciadas por tales estructuras simplemente como errores (\texttt{repOK} describe un conjunto de estructuras diferente al que debería). Observa que las diferencias en las definiciones de ámbito de los enfoques pueden hacer que los conjuntos \texttt{SA} y \texttt{SR} difieran. Esto solo fue el caso en las estructuras \texttt{RBT} y \texttt{FibHeap}, donde \textsf{BEAPI} generó un conjunto más pequeño de estructuras para el mismo ámbito que \textsf{Korat} debido a restricciones de balance (como se explica en la Sección \ref{sec:evaluation-vs-korat}). Sin embargo, estos "falsos positivos" se pueden revelar fácilmente, ya que todas las estructuras generadas por \textsf{Korat} siempre estuvieron incluidas en las estructuras generadas por \textsf{BEAPI} si se utilizaba un ámbito más amplio para este último enfoque. Utilizando esta información, descartamos manualmente los "falsos positivos" debido a las diferencias de ámbito en \texttt{RBT} y \texttt{FibHeap}.

Los resultados de este experimento se resumen en la Tabla \ref{table:bugs}. Encontramos fallas en 9 de 26 \texttt{repOK}s utilizando el enfoque descrito anteriormente. El alto número de fallas descubiertas evidencia que los problemas en \texttt{repOK}s son difíciles de encontrar manualmente, y que \textsf{BEAPI} puede ser de gran ayuda para esta tarea.

\cacho{Agregar mas conclusiones}
\subsection{Comparativa de BEAPI con otras tecnicas de generacion de test}

\hspace{1cm}

En esta etapa de la evaluación, se llevó a cabo un análisis exhaustivo para determinar la utilidad de los métodos builders identificados en el contexto del análisis de programas, específicamente en la generación automatizada de casos de prueba. Estos builders se consideran objetos clave que pueden ser utilizados como entradas en test parametrizadas.

Los test parametrizados son una técnica utilizada en el campo de la generación automatizada de casos de prueba para aumentar la eficiencia y la cobertura de las pruebas. En lugar de escribir casos de prueba individuales para cada escenario posible, los test parametrizados permiten definir un conjunto de parámetros que se utilizan para generar automáticamente múltiples casos de prueba.

En el contexto de la evaluación experimental, se utilizó la técnica de test parametrizados para alimentar una test suite con objetos creados por diferentes técnicas. Esto significa que se definieron parámetros que representan diferentes características o propiedades de los objetos, y luego se generaron automáticamente casos de prueba utilizando estos parámetros.

Los test parametrizados son una técnica poderosa en la generación automatizada de casos de prueba, ya que permiten explorar diferentes combinaciones de parámetros y generar una variedad de casos de prueba de manera eficiente. En el contexto de la evaluación experimental, se utilizaron para evaluar y comparar el desempeño de diferentes técnicas en la generación de objetos y su impacto en la calidad de las pruebas.

Para llevar a cabo este experimento, se creó una test suite parametrizada que serviría como marco de prueba para evaluar los distintos enfoques. Esta test suite parametrizada fue, básicamente, crear un test por método que contiene la clase y ejercitarlos con los objetos creados por las diferentes técnicas. Se utilizaron varias técnicas y herramientas para generar los objetos necesarios. En primer lugar, se empleó la conocida herramienta \texttt{Randoop} utilizando como es su forma estander, con todos los métodos disponibles en su API, lo que da lugar a una suite de pruebas tradicional generada por Randoop. En este caso no se utiliza una test suite parametrizada, ya que \texttt{Randoop}  no genera objetos, sino que crea sus propias tests suite.

Ahora si, se utilizó una variante de Randoop llamada \texttt{R-Serialize} para serializar las secuencias de pruebas generadas anteriormente. Esto permitió generar objetos que, a su vez, se utilizaron para alimentar la test suite parametrizada, ampliando así el alcance de los casos de prueba. Estos objetos fueron generados utilizando todos los métodos de la API de Randoop, tal como se hizo en el enfoque anterior.

Otra herramienta utilizada en el experimento fue una versión modificada de \texttt{Randoop}, diseñada específicamente para utilizar únicamente los métodos builders identificados previamente mediante nuestro enfoque (\ref{cap:builders}). Los objetos generados por esta variante especial de Randoop también se serializaron y se incorporaron a la test suite parametrizada, permitiendo una comparación directa entre los objetos generados por los builders identificados y los generados sin utilizar la información de estos métodos builders.

Por último, se emplearon los objetos generados por la herramienta \texttt{BEAPI}. Estos objetos, creados utilizando su propia API, se integraron en la test suite parametrizada para evaluar su efectividad en la generación de casos de prueba. De esta manera, se obtuvo una visión completa y comparativa de las diferentes técnicas utilizadas en términos de generación de objetos y su impacto en la calidad de las pruebas.

Mediante este enfoque meticuloso y riguroso, se buscó determinar la capacidad de los builders identificados para mejorar la generación automatizada de casos de prueba y, en última instancia, contribuir a la mejora de la calidad del análisis de programas. Los resultados obtenidos en esta evaluación experimental proporcionaron información valiosa sobre la utilidad y efectividad de los builders en el contexto del análisis de programas, abriendo así nuevas oportunidades para futuras investigaciones y desarrollos en este campo.

A continuación, se realiza la comparativa de las diferentes herramientas utilizadas en el benchmark de \emph{java.util} previamente mencionado. Los casos de estudio son: \emph{HashSet}, \emph{HashMap}, \emph{TreeMap}, \emph{TreeSet} y \emph{LinkedList}. La tabla de resultados muestra varias métricas, como el tiempo en segundos (\texttt{GTime}) que es tiempo que lleva generar estos objectos/tests, la herramienta utilizada (\texttt{Tool}), la cantidad de objetos válidos generados (\texttt{Valid}) y la cantidad de objetos inválidos generados (\texttt{Invalid}).
Es importante destacar que se definió un invariante para cada clase bajo evaluación, el cual establece qué estructuras son válidas e inválidas. Todos los objetos generados fueron sometidos a la prueba de estos invariantes para determinar su validez. Sin embargo, es importante mencionar que la técnica \texttt{Randoop} genera directamente una test suite en lugar de objetos individuales, por lo que no se aplica directamente el concepto de validez e invalidez a los casos generados por esta herramienta.

Además, se realiza una comparación en función de la cantidad de tests generados por cada técnica. Esta medida se refleja en las columnas \texttt{Test}, \texttt{T(Seg)}, que indica la cantidad de tests que se ejecutan y sus respectivos segundos que son necesarios para ejecutar la test suite parametrizada, o no (en el caso de \texttt{Ranndop}), correspondiente.

Finalmente, se evalúa la calidad de las test suites generadas mediante la comparación de la cobertura de ramas y la cantidad de mutantes eliminados. Estas métricas permiten determinar qué tan efectivas son las test suites en términos de su capacidad para cubrir diferentes ramas del código y eliminar mutantes generados para introducir fallas.

A continuación analizaremos caso por caso con sus respectivas tablas.

\begin{table}[H]
\scriptsize
\centering
\begin{tabular}{ c  r  |r | r | r|r|r|r  }
  \toprule
  \textbf{GTime} & \textbf{Tool} & \textbf{Valid}  & \textbf{Invalid} & \textbf{Test}&\textbf{T(Seg)} &\textbf{Ramas}  & \textbf{Mutacion} \\ 
  \midrule
5	&	Randoop	&	-	&	- & 380 & 4	& 53 & 67 \\
& R-Serialize	& 3851 & 264 & 184852 & 4	& 72 &	79 	\\
& R-Builders		& 12812 & 9008 & 614980 & 20 &	82	& 89  \\
&	BEAPI & 32 & - & 1540 & 2	& 82 & 89 \\
\cline{2-8}
10	&	Randoop	&	-	&	- & 380 & 4	& 67 & 67 \\
& R-Serialize	& 7655 & 532 & 367444 & 6	& 75 &	81 	\\
& R-Builders		& 23320 & 16482 & 1119364 & 35 &	82	& 89  \\
&	BEAPI & 32 & - & 1540 & 2	& 82 & 89 \\
\cline{2-8}
20	&	Randoop	&	-	&	- & 380 & 4	& 73 & 85 \\
& R-Serialize	& 14980 & 1015 & 719044 & 8	& 76 &	83 	\\
& R-Builders		& 38047 & 26904 & 1826260 & 55 &	82	& 89  \\
&	BEAPI & 32 & - & 1540 & 2	& 82 & 89 \\

\cline{2-8}
40	&	Randoop	&	-	&	- & 380 & 4	& 53 & 67 \\
& R-Serialize	& 28965 & 2020 & 1390324 & 12		& 76 &	85 	\\
& R-Builders		& 59016 & 41681 & 2832772 & 85 	& 82 &	89 	  \\
&	BEAPI & 32 & - & 1540 & 2	& 82 &	89 \\
\cline{2-8}
150	&	Randoop	&	-	&	- & 12452 & 29	& 81 & 89\\
& R-Serialize	& 97236 & 7088 & 4667332 & 35	& 82 &	89 	\\
& R-Builders		& 125992 & 89292 & 6047620 & 179 & 82 &	89  \\
&	BEAPI & 32 & - & 1540 & 2	& 82 &	89 \\
\midrule
\end{tabular}
\caption{Comparativa de diferentes tecnicas en HashSet}
\label{tab:hashSetTools}
\end{table}


\begin{table}[H]
\scriptsize
\centering
\begin{tabular}{ c  r  |r | r | r|r|r|r  }
  \toprule
  \textbf{GTime} & \textbf{Tool} & \textbf{Valid}  & \textbf{Invalid} & \textbf{Test}&\textbf{T(Seg)} &\textbf{Ramas}  & \textbf{Mutacion} \\ 
  \midrule

5	&	Randoop	&	-	&	-	&	15745	&	9	&	61	&	71	\\
	&	R-Serialize	&	7779	&	1468	&	606762	&	4	&	114	&	99	\\
	&	R-Builders	&	13791	&	2950	&	1075698	&	5	&	148	&	111	\\
	&	BEAPI	&	72	&	-	&	5616	&	2	&	150	&	109	\\
 \cline{2-8}															
10	&	Randoop	&	-	&	-	&	1138	&	5	&	77	&	96	\\
	&	R-Serialize	&	16444	&	3246	&	1282632	&	5	&	114	&	99	\\
	&	R-Builders	&	25569	&	5514	&	1994382	&	7	&	151	&	111	\\
	&	BEAPI	&	72	&	-	&	5616	&	2	&	150	&	109	\\
 \cline{2-8}															
20	&	Randoop	&	-	&	-	&	2374	&	7	&	80	&	100	\\
	&	R-Serialize	&	33188	&	6786	&	2588664	&	8	&	114	&	100	\\
	&	R-Builders	&	43670	&	9358	&	3406260	&	9	&	151	&	111	\\
	&	BEAPI	&	72	&	-	&	5616	&	2	&	150	&	109	\\
 \cline{2-8}															
40	&	Randoop	&	-	&	-	&	4736	&	11	&	80	&	100	\\
	&	R-Serialize	&	62678	&	13170	&	4888884	&	11	&	125	&	109	\\
	&	R-Builders	&	69110	&	14990	&	5390580	&	13	&	154	&	111	\\
	&	BEAPI	&	72	&	-	&	5616	&	2	&	150	&	109	\\
 \cline{2-8}															
150	&	Randoop	&	-	&	-	&	16932	&	33	&	105	&	100	\\
	&	R-Serialize	&	215136	&	47411	&	16780608	&	31	&	125	&	109	\\
	&	R-Builders	&	150637	&	32426	&	11749686	&	25	&	154	&	111	\\
	&	BEAPI	&	72	&	-	&	5616	&	2	&	150	&	109	\\
\midrule
\end{tabular}
\caption{Comparativa de diferentes tecnicas en TreeSet}
\label{tab:treeSetTools}
\end{table}

\begin{table}[H]
\scriptsize
\centering
\begin{tabular}{ c  r  |r | r | r|r|r|r  }
  \toprule
  \textbf{GTime} & \textbf{Tool} & \textbf{Valid}  & \textbf{Invalid} & \textbf{Test}&\textbf{T(Seg)} &\textbf{Ramas}  & \textbf{Mutacion} \\ 
  \midrule
5	&	Randoop	&		&		&	460	&	4	&	27	&	51	\\
	&	R-Serialize	&	3248	&	425	&	295568	&	4	&	115	&	100	\\
	&	R-Builders	&	7924	&	570	&	721084	&	3	&	159	&	152	\\
	&	BEAPI	&	1745	&		&	158795	&	3	&	173	&	140	\\
 \cline{2-8}															
10	&	Randoop	&		&		&	1068	&	4	&	31	&	58	\\
	&	R-Serialize	&	5966	&	775	&	542906	&	3	&	116	&	100	\\
	&	R-Builders	&	14454	&	742	&	1315314	&	3	&	159	&	152	\\
	&	BEAPI	&	1745	&		&	158795	&	3	&	173	&	140	\\
 \cline{2-8}															
20	&	Randoop	&		&		&	2152	&	6	&	31	&	58	\\
	&	R-Serialize	&	11209	&	1410	&	1020019	&	4	&	116	&	100	\\
	&	R-Builders	&	23067	&	936	&	2099097	&	4	&	159	&	152	\\
	&	BEAPI	&	34276	&		&	3119116	&	15	&	179	&	140	\\
 \cline{2-8}															
40	&	Randoop	&		&		&	4284	&	9	&	31	&	58	\\
	&	R-Serialize	&	21260	&	2566	&	1934660	&	5	&	119	&	100	\\
	&	R-Builders	&	33460	&	1141	&	3044860	&	4	&	162	&	152	\\
	&	BEAPI	&	34276	&		&	3119116	&	15	&	179	&	140	\\
 \cline{2-8}															
150	&	Randoop	&		&		&	15902	&	21	&	46	&	58	\\
	&	R-Serialize	&	69747	&	8181	&	6346977	&	7	&	120	&	100	\\
	&	R-Builders	&	68373	&	1645	&	6221943	&	5	&	162	&	152	\\
	&	BEAPI	&34276&		&	3119116	&	15	&	179	&	140	\\
\midrule
\end{tabular}
\caption{Comparativa de diferentes tecnicas en TreeMap}
\label{tab:treeMapTools}
\end{table}

\begin{table}[H]
\scriptsize
\centering
\begin{tabular}{ c  r  |r | r | r|r|r|r  }
  \toprule
  \textbf{GTime} & \textbf{Tool} & \textbf{Valid}  & \textbf{Invalid} & \textbf{Test}&\textbf{T(Seg)} &\textbf{Ramas}  & \textbf{Mutacion} \\ 
  \midrule
5	&	Randoop	&	-	&	-	&	356	&	4	&	39	&	38	\\
	&	R-Serialize	&	1945	&	2875	&	184852	&	4	&	49	&	56	\\
	&	R-Builders	&	15571	&	81103	&	6088262	&	12	&	49	&	57	\\
	&	BEAPI	&	781	&	-	&	305375	&	3	&	49	&	58	\\
 \cline{2-8}															
10	&	Randoop	&	-	&	-	&	794	&	5	&	43	&	43	\\
	&	R-Serialize	&	3996	&	6241	&	367444	&	4	&	49	&	56	\\
	&	R-Builders	&	26085	&	136050	&	10199236	&	19	&	49	&	58	\\
	&	BEAPI	&	781	&	-	&	305375	&	3	&	49	&	58	\\
 \cline{2-8}															
20	&	Randoop	&		&	-	&	1678	&	7	&	45	&	45	\\
	&	R-Serialize	&	7463	&	12437	&	719044	&	5	&	49	&	57	\\
	&	R-Builders	&	40989	&	214029	&	16026700	&	28	&	49	&	58	\\
	&	BEAPI	&	781	&	-	&	305375	&	3	&	49	&	58	\\
 \cline{2-8}															
40	&	Randoop	&		&	-	&	3432	&	11	&	47	&	50	\\
	&	R-Serialize	&	14156	&	24838	&	1390324	&	7	&	49	&	57	\\
	&	R-Builders	&	62051	&	323668	&	24261943	&	40	&	49	&	58	\\
	&	BEAPI	&	781	&	-	&	305375	&	3	&	49	&	58	\\
 \cline{2-8}															
150	&	Randoop	&		&	-	&	12950	&	31	&	47	&	55	\\
	&	R-Serialize	&	48428	&	88638	&	4667332	&	15	&	49	&	57	\\
	&	R-Builders	&	129365	&	677056	&	50581717	&	86	&	49	&	58	\\
	&	BEAPI	&	781	&	-	&	305375	&	3	&	49	&	58	\\
\midrule
\end{tabular}
\caption{Comparativa de diferentes tecnicas en LinkedList}
\label{tab:linkedListTools}
\end{table}

\begin{table}[H]
\scriptsize
\centering
\begin{tabular}{ c  r  |r | r | r|r|r|r  }
  \toprule
  \textbf{GTime} & \textbf{Tool} & \textbf{Valid}  & \textbf{Invalid} & \textbf{Test}&\textbf{T(Seg)} &\textbf{Ramas}  & \textbf{Mutacion} \\ 
  \midrule
5	&	Randoop	&		&	-	&	78	&	2	&	48	&	66	\\
	&	R-Serialize	&	3290	&	348	&	649448	&	9	&	95	&	118	\\
	&	R-Builders	&	4222	& 908		&	856088	&	15	&	119	&	127	\\
	&	BEAPI	&	1203	&	-	&	246623	&	4	&	102	&	119	 \\
 \cline{2-8}															
10	&	Randoop	&		&	-	&	80	&	3	&	48	&	66	\\
	&	R-Serialize	&	6317	&	686	&	1285563	&	17	&	104	&	120	\\
	&	R-Builders	&	5666	&	1218	&	1158668	&	19	&	119	&	127	\\
	&	BEAPI	&	10453	&	-	&	2142873	&	9	&	122	&	120	\\
 \cline{2-8}															
20	&	Randoop	&	-	&	-	&	82	&	3	&	48	&	66	\\
	&	R-Serialize	&	11762	&	1200	&	2415318	&	26	&	104	&	120	\\
	&	R-Builders	&	7249	&	1559	&	1483798	&	24	&	119	&	127	\\
	&	BEAPI	&	10453	&	-	&	2142873	&	9	&	122	&	120	 \\
 \cline{2-8}															
40	&	Randoop	&	-	&	-	&	90	&	3	&	48	&	66	\\
	&	R-Serialize	&	22284	&	2042	&	4557158	&	46	&	104	&	120	\\
	&	R-Builders	&	9318	& 2003	&	1909583	&	30	&	119	&	127	\\
	&	BEAPI	&	10453	&	-	&	2142873	&	9	&	122	&	120	\\ 
 \cline{2-8}															
150	&	Randoop	&	-	&	-	&	108	&	3	&	48	&	67	\\
	&	R-Serialize	&	76581	&	5590	&	15567913	&	149	&	110	&	120	\\
	&	R-Builders	&	14542	&	3127	&	2962873	&	46	&	119	&	128	\\
	&	BEAPI	&	10453	&	-	&	2142873	&	9	&	122	&	120		\\
\midrule
\end{tabular}
\caption{Comparativa de diferentes tecnicas en HashMap}
\label{tab:hashMapTools}
\end{table}


% En esta etapa de la evaluación, se llevó a cabo un análisis exhaustivo para determinar la utilidad de los metodos builders identificados en el contexto del análisis de programas, específicamente en la generación automatizada de casos de prueba. Estos builders se consideran objetos clave que pueden ser utilizados como entradas en test parametrizadas. 
% Los test parametrizados son una técnica utilizada en el campo de la generación automatizada de casos de prueba para aumentar la eficiencia y la cobertura de las pruebas. En lugar de escribir casos de prueba individuales para cada escenario posible, los test parametrizados permiten definir un conjunto de parámetros que se utilizan para generar automáticamente múltiples casos de prueba.
% En el contexto de la evaluación experimental, se utilizó la técnica de test parametrizados para alimentar una test suite con objetos creados por diferentes técnicas. Esto significa que se definieron parámetros que representan diferentes características o propiedades de los objetos, y luego se generaron automáticamente casos de prueba utilizando estos parámetros.
% Los test parametrizados son una técnica poderosa en la generación automatizada de casos de prueba, ya que permiten explorar diferentes combinaciones de parámetros y generar una variedad de casos de prueba de manera eficiente. En el contexto de la evaluación experimental, se utilizaron para evaluar y comparar el desempeño de diferentes técnicas en la generación de objetos y su impacto en la calidad de las pruebas.

% Para llevar a cabo este experimento, se creó una test suite parametrizada que serviría como marco de prueba para evaluar los distintos enfoques. Esta test suite parametrizada fue, basicamente, crear un test por metodos que contiene la clase y ejercitalos con los objectos creados por las distintas técnicas. Se utilizaron varias técnicas y herramientas para generar los objetos necesarios. En primer lugar, se empleó la conocida herramienta \texttt{Randoop}, utilizando todos los métodos disponibles en su API estándar, lo que dio lugar a una suite de pruebas tradicional generada por Randoop.

% Además, se utilizó una variante de Randoop llamada \texttt{R-Serialize} para serializar las secuencias de pruebas generadas anteriormente. Esto permitió generar objetos que, a su vez, se utilizaron para alimentar la test suite parametrizada, ampliando así el alcance de los casos de prueba. Estos objetos fueron generados utilizando todos los métodos de la API de Randoop, tal como se hizo en el enfoque anterior.

% Otra herramienta utilizada en el experimento fue una versión modificada de \texttt{Randoop}, diseñada específicamente para utilizar únicamente los métodos builders identificados previamente mediante nuestro enfoque (\ref{cap:builders}). Los objetos generados por esta variante especial de Randoop también se serializaron y se incorporaron a la test suite parametrizada, permitiendo una comparación directa entre los objetos generados por los builders identificados y los generados sin utilizar la informacion de estos metodos builders.

% Por último, se emplearon los objetos generados por la herramienta \texttt{BEAPI}. Estos objetos, creados utilizando su propia API, se integraron en la test suite parametrizada para evaluar su efectividad en la generación de casos de prueba. De esta manera, se obtuvo una visión completa y comparativa de las diferentes técnicas utilizadas en términos de generación de objetos y su impacto en la calidad de las pruebas.

% Mediante este enfoque meticuloso y riguroso, se buscó determinar la capacidad de los builders identificados para mejorar la generación automatizada de casos de prueba y, en última instancia, contribuir a la mejora de la calidad del análisis de programas. Los resultados obtenidos en esta evaluación experimental proporcionaron información valiosa sobre la utilidad y efectividad de los builders en el contexto del análisis de programas, abriendo así nuevas oportunidades para futuras investigaciones y desarrollos en este campo.


%  A continuacion hacemos la comparativa de las diferentes tools, explicada en el parrafo anterior, en el benchmarks de \emph{java.util} utilizado previamente. Los casos de estudios son; \emph{HashSet},\emph{HashMap},\emph{TreeMap},\emph{TreeSet}, \emph{LinkedList}. La comparacion se realiza sobre, \texttt{GTime}, que representa el tiempo en milisegundos, \texttt{Tool} indica la herramienta utilizada, \texttt{Valid} muestra la cantidad de objectos válidos, \texttt{Invalid} muestra la cantidad de objetos inválidos que son generados. Para saber que objetos son válidos e inválidos, escribimos un invariante para cada clase bajo evaluacion, y este nos dice que estructura es valida y que estructura es invalida. Puede observar que para la tecnica \texttt{Randoo} esto no cuenta, ya que recuerde que aqui utilizamos la tool \texttt{Randoop} que generan test suite directamnete y no objectos. Ademas, BEAPI siempre va a generar objectos validos, esto se debe a que como estos objectos son construidos desde la API, no hay posibilidad de generar objectos invalidos. Vale aclarar que estos objectos tambien fueron puesto bajo prueba del mismo invariante que escribimos para cada clase. 
%  Ademas, comparamos de acuerdo a la cantidad de test que cada tecnica genera. Para realizar esta medida, en las tools que utilizan la test suite parametrizada, indica la cantidad de test que se ejecutan y es un valor de que tiene relacion con los objectos que se alimentan a la test suite. En base a esto \texttt{T(Seg)} es la cantidad de segundo que lleva ejecutar esta test suite.
%  Por ultimo, comparamos que tan buena es estas test suite, comprando la cobertura de ramas y de mutantes que matan.

 \cacho{Mejorar esto:}
 
La tabla \ref{tab:hashSetTools} muestra una comparativa de diferentes técnicas aplicadas en el contexto del caso de estudio de \emph{java.util}, HashSet.
% Test indica el total de casos de prueba, T(Seg) representa el tiempo en segundos, Ramas muestra la cantidad de ramas y Mutacion indica el valor correspondiente a la mutación.

% En general, se realizaron pruebas utilizando diferentes técnicas, como Randoop, R-Serialize, R-Builders y BEAPI. Estas técnicas fueron evaluadas en diferentes intervalos de tiempo, como 5, 10, 20, 40 y 150 unidades.

Como se puede observar en la tabla \ref{tab:hashSetTools}, al utilizar los métodos builders como prioridad en Randoop, se generan significativamente más objetos (un aumento del 300%) en comparación con Randoop sin utilizar estos métodos. Por otro lado, BEAPI genera un total de 32 objetos para un scope de 5, y lo logra en menos de 5 segundos.

La cantidad de tests ejecutados es una medida de la cobertura de la test suite. En las técnicas que utilizan los objetos generados para alimentar tests parametrizados, hay una correlación directa entre estos dos números. Es importante destacar que la test suite para tests parametrizados es la misma para R-Serialize, R-Builders y BEAPI.

En cuanto a la calidad de las test suites generadas, podemos observar que Randoop no logra alcanzar altos valores de cobertura en un corto período de tiempo. Solo después de 150 segundos comienza a alcanzar coberturas similares a las otras técnicas. Esto se debe a que Randoop genera muchos tests que resultan irrelevantes o que están subsumidos en otros. Cuando utilizamos estas secuencias de tests para serializar y crear objetos que se utilizan como inputs en nuestros tests parametrizados, se puede observar una mejora en la cobertura, pero aún no alcanza los valores obtenidos con la identificación previa de los métodos builders y su posterior uso con Randoop (R-Builders). Esta técnica logra la misma cobertura que BEAPI, tanto en términos de ramas como de mutación. Aquí podemos apreciar la importancia de guiar a Randoop mediante la identificación de los métodos builders.



\cacho{Como resultados finales:}
Los resultados indican que la utilización de los métodos builders como prioridad en Randoop genera una mayor cantidad de objetos y mejora la calidad de las test suites en comparación con Randoop sin utilizar esta prioridad. Además, se destaca el desempeño de BEAPI, que logra generar un número significativo de objetos válidos en poco tiempo. Estos hallazgos respaldan la importancia de guiar la generación de casos de prueba utilizando información específica de la API, como la identificación de los métodos builders, para mejorar la efectividad y la cobertura de las pruebas en el contexto del software testing.





