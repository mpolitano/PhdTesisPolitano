%!TEX root = main.tex
\begin{center}
    \Large
    \textbf{Generación Automática Eficiente de Objetos a partir de APIs de Clase}
        
    \vspace{0.4cm}
    \large
    por Mariano Politano
        
    \vspace{0.9cm}
    \textbf{Resumen}
\end{center}

El testing de software consiste en ejecutar código bajo diversas entradas para verificar su correctitud, 
siendo una de las actividades más costosas y críticas en el desarrollo de software, 
ya que puede representar más de la mitad del costo total del proceso.
Esta actividad es central en el ciclo de desarrollo de \emph{software} y, en proyectos reales, 
puede representar una porción sustancial del costo total tanto en tiempo como en recursos humanos, 
por lo que su automatización resulta clave \cite{}.
Entre las estrategias de automatización, la generación exhaustiva acotada (BEG, por sus siglas en inglés) \cite{}
ha mostrado ser altamente efectiva para descubrir fallas. Esta misma, propone ejercitar el programa 
bajo todas las entradas válidas dentro de cotas dadas. 
Sin embargo, su adopción enfrenta dos obstáculos prácticos. Primero, la necesidad de especificaciones de validez 
(invariantes de representación), difíciles y costosas de escribir. Segundo, la selección manual de subconjuntos de 
métodos de la API para construir drivers de generación, donde el espacio de combinaciones crece exponencialmente con el número de métodos.

En este trabajo abordamos ambos problemas mediante dos aportes. 
En primer lugar, presentamos un método para la identificación automática de métodos generadores de objectos.
a partir de la API pública de una clase. Estos métodos son aquellos que cumplen la caracteristica de ser 
suficientes (capaces de construir todas las instancias válidas dentro del scope) y minimales (sin métodos superfluos), esto quiere decir que nos permiten construir cualquier objecto de la API.
Modelamos esta selección de métodos como un problema de optimización y exploramos el espacio con dos algoritmos de búsqueda, \emph{Hill Climbing} (HC) y Algoritmo Genetico (GA). 
Para evaluar la calidad de un candidato (subconjunto de métodos), desarrollamos dos funciones objetivo. Una basada en generación exhaustiva acotada, 
que estima la cantidad de objetos distintos que puede producir el conjunto de métodos bajo un scope dado; y la otra función es 
basada en cobertura de código (líneas y ramas) obtenida al generar y ejecutar suites de tesst a partir de dicho conjunto de métodos. 
La evaluación experimental sobre un conjunto representativo de clases indica que nuestras técnicas identifican 
conjuntos de métodos, con tiempos de cómputo adecuados para su uso práctico.

En segundo lugar, introducimos BEAPI, un enfoque de generación exhaustiva acotada que no requiere invariante de representacion, (comunmente llamados, \emph{repOK}). 
BEAPI construye objetos ejecutando secuencias de método de la API de la clase bajo test 
y aplica tres optimizaciones esenciales. La primera es una poda por realizar un buen uso de la API 
(descarta tempranamente secuencias que producen excepciones o violaciones de contrato). 
La segunda es por coincidencia de estados (\emph{state matching}) 
para eliminar redundancias cuando diferentes secuencias generan el mismo objeto. 
Y la tercera, es haciendo uso de los métodos generadores de objetos (los detectados en el primer aporte de este trabajo) 
para restringir las combinaciones de métodos. 
Realizamos una comparación contra la herramienta muy conocida en el estado del arte, \emph{Korat}, 
y observamos que BEAPI alcanza niveles comparables de en cuanto a lo \cacho{exhaustivo} de la tecnica 
y con mejores tiempos en escenarios de estructuras que tienen varias restricciones y, lo mas importante, sin requerir 
los invariantes de representación. 
Además, al contrastar los objetos generados por BEAPI con los aceptados por \emph{repOKs} existentes de \emph{Korat}, 
mostramos que el enfoque puede revelar defectos en dichas especificaciones, subrayando la dificultad de escribir \emph{repOKs} correctos.

En conjunto, los resultados demuestran que la selección automática de métodos generadores y BEAPI reducen la intervención manual del programador y 
mejoran la eficiencia y escalabilidad de la generación de entradas a partir de APIs de clase. 
Estos aportes facilitan la adopción de técnicas exhaustivas en contextos reales, tanto para testing como para verificación.

\cacho{algo de lineas futura de trabajo, va aca?}



\vspace{0.5cm}

\newpage

\begin{center}
    \Large
    \textbf{Efficient Automatic Generation of Objects from Class APIs}
        
    \vspace{0.4cm}
    \large
    by Mariano Politano
        
    \vspace{0.6cm}
    \textbf{Abstract}
\end{center}