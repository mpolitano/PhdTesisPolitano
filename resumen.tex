%!TEX root = main.tex
\begin{center}
    \Large
    \textbf{Enfoques Eficientes de Generación Automática de Objetos a partir de APIs de Clase}
        
    \vspace{0.4cm}
    \large
    por Mariano Politano
        
    \vspace{0.9cm}
    \textbf{Resumen}
\end{center}
% Nuevo título: Enfoques Eficientes de Generación Automática de Objetos a partir de APIs de Clase

Garantizar la corrección funcional del software es objetivo crucial para la
Ingeniería de Software. Numerosas técnicas 
automáticas de análisis de software se han desarrollado para ayudar a los
desarrolladores a lograr este objetivo. Muchas de estas técnicas 
requieren producir un conjunto de objetos para alimentar el programa bajo
análisis (por ejemplo, técnicas de testing, de model checking de software,
etc.). Generar un conjunto diverso de objetos para ejercitar adecuadamente el software 
es un problema desafiante, sobre todo para programas que manipulan estructuras complejas 
(ej. estructuras encadenadas almacenadas en el heap).

En este trabajo, se proponen dos enfoques para contruibir a la generación de
objetos complejos a partir de la API de un módulo. Por un lado, observamos que 
la selección de los métodos apropiados de la API es crucial para 
la construcción eficiente de objetos. Definimos como \emph{métodos generadores
de objetos} a un subconjunto minimal de métodos de la API que son suficientes para 
construir cualquier objeto para el módulo. 
Identificar un conjunto suficiente y minimal de métodos generadores de objetos 
de forma manual es una tarea trabajosa, por lo que se proponen enfoques para 
llevar a cabo esta tarea de manera automática. 

Por otro lado, se define un enfoque eficiente de generación exhaustiva acotada
de objetos basado en la API del módulo, denominado BEAPI. La generación
exhaustiva acotada consiste en crear todos los objetos de tamaño acotado para
el módulo, con cotas provistas por el usuario. Enfoques previos
requieren de especificaciones formales de las propiedades que deben satisfacer
los objetos.
En contraste, BEAPI construye objetos creando y ejecutando secuencias de invocaciones a 
métodos (secuencias de test) de la API. BEAPI implementa tres 
técnicas de poda claves para su eficiencia. Primero, BEAPI
descarta secuencias de test que lanzan excepciones por el uso indebido de la
API. Segundo, BEAPI implementa coincidencia de estados para descartar
secuencias de test distintas que crean los mismos objetos. Tercero,
BEAPI emplea solo métodos generadores de objetos durante la generación 
(identificados automáticamente con los enfoques propuestos). 

Se evaluaron experimentalmente las técnicas propuestas, en clases que manipulan 
objetos complejos. 
Los resultados muestran que los enfoques de identificación de métodos generadores se ejecutan en tiempos razonables, y encuentran métodos generadores de objetos
suficientes y (casi) minimales (incluyen algunos métodos
innecesarios en pocos casos). Además, se muestran resultados preliminares que
indican que los métodos generadores de objetos son útiles para mejorar 
la generación de objetos usando técnicas de generación aleatoria y
exhaustiva acotada de tests, y para model checking de software.
La evaluación de BEAPI muestra que su eficiencia y escalabilidad respecto 
de los scopes es comparable al mejor enfoque existente (Korat). Además, 
se provee evidencia preliminar de que BEAPI sirve de ayuda para 
encontrar errores en especificaciones formales de invariantes de clase, y que 
los objetos que genera son de utilidad para el testing basado en propiedades.



% Versión larga

%A medida que el software se vuelve ubicuo gracias a los rápidos avances en la
%tecnología, garantizar la corrección funcional del mismo se vuelve crucial.
%Debido a esto se han desarrollado diversas técnicas automáticas de análisis de
%software (implementadas mediante herramientas de software), para asistir a los
%desarrolladores en distintas tareas, como por ejemplo, encontrar fallas en 
%el software. 
%
%Varias técnicas de análisis requieren producir un conjunto de
%estructuras (objetos en la terminología de la programación orientada a
%objetos) para alimentar el programa bajo test. Entre ellas,
%podemos mencionar el testing unitario, el testing basado en propiedades, y el model
%checking (acotado) de software. 
%Si los programas manipulan estructuras de datos complejas (ej. estructuras
%encadenadas almacenadas en el heap), generar un conjunto 
%diverso de objetos para ejercitar adecuadamente el software es un problema
%desafiante. 
%
%En este trabajo, se proponen dos enfoques para contruibir a la generación de
%objetos complejos a partir de la API de un módulo. Por un lado, observamos que 
%la selección de los métodos apropiados de la API de un módulo es crucial para 
%la construcción eficiente de objetos. Por esta razón, definimos el concepto
%de \emph{métodos generadores de objetos} como un subconjunto minimal de métodos de 
%la API que son suficientes para construir cualquier objeto para el módulo. 
%Identificar un conjunto suficiente y minimal de métodos generadores de objetos de forma manual 
%es una tarea trabajosa, que requiere de un análisis minucioso de la API y una comprensión profunda 
%de la semántica del módulo. 
%Para facilitar el trabajo del programador, se proponen distintos enfoques automáticos para 
%identificar conjuntos suficientes y minimales de métodos generadores de objetos
%a partir de la API. 
%
%Por otro lado, definimos un enfoque eficiente de generación exhaustiva acotada
%de objetos basado en la API del módulo, denominado BEAPI. La generación
%exhaustiva acotada consiste en crear todos los objetos de tamaño acotado para
%el módulo, con cotas provistas por el usuario (ej. máxima cantidad de nodos en
%estructuras enlazadas). La generación exhaustiva acotada ha mostrado ser un
%enfoque efectivo para revelar fallas durante el testing. Enfoques previos
%requiren de especificaciones formales de las propiedades que deben satisfacer
%los objetos válidos, y realizan la generación en base a estas especificaciones.
%En contraste, BEAPI construye objetos creando y ejecutando secuencias de invocaciones a
%métodos (secuencias de test) de la API. Para hacer posible la generación eficiente a partir de la
%API, BEAPI incluye tres técnicas de poda claves. En primer lugar, BEAPI
%descarta secuencias de test que lanzan excepciones debido al uso indebido de la
%API (ej. violaciones de precondiciones). 
%En segundo lugar, BEAPI implementa coincidencia de estados para descartar
%secuencias de test diferentes que crean los mismos objeto. En tercer lugar,
%BEAPI emplea solo métodos generadores de objetos para la creación de secuencias
%de test (identificados automáticamente con los algoritmos propuestos en una etapa 
%previa a la generación exhaustiva). 
%
%Se realizó una evaluación experimental preliminar de los enfoques
%propuestos, en clases cuyas instancias son objetos complejos. 
%Los resultados muestran que los enfoques de identificación de métodos
%generadores propuestos se ejecutan en tiempos
%razonables para esta tarea (algunos minutos), y computan generadores de objetos
%suficientes y minimales en la mayoría de los casos (incluyen algunos métodos
%innecesarios en pocos casos). Además, se muestran resultados preliminares que
%indican que los métodos generadores de objetos pueden ser útiles para la mejorar 
%la generación de objetos en herramientas de generación aleatoria de tests
%(Randoop), en herramientas de generación exhaustiva acotada (BEAPI), y en 
%bounded model checking de software (usando Java Pathfinder).
%Con respecto a BEAPI, los resultados experimentales muestran que
%su eficiencia y escalabilidad respecto de los scopes es comparable a la mejor 
%herramienta de generación exhaustiva acotada 
%basada en especificaciones (Korat). Además, se muestra que 
% BEAPI es útil para encontrar errores en 
% especificaciones formales de invariantes de clase, y que los objetos 
%generados por BEAPI son de utilidad en el contexto de testing basado en 
%propiedades.





%Our experimental assessment shows that BEAPI’s efficiency and scalabil-
%ity is competitive with existing BEG approaches, without the need for
%repOKs. We also show that BEAPI can assist the user in finding flaws in
%repOKs, by (automatically) comparing inputs generated by BEAPI with
%those generated from a repOK. Using this approach, we revealed several
%errors in repOKs taken from the assessment of related tools, demonstrat-
%ing the difficulties of writing precise repOKs for BEG.



%Para builders mostramos que las técnicas:
%- computan builders en tiempos razonables 
%- son precisas
%- y resultados preliminares que muestran que los builders pueden mejorar la generación de
%objetos en herramientas como Randoop (aleatoria), y BEAPI (generación exhaustiva
%acotada), y en la JPF (bounded model checking de software).
%
%Respecto a BEAPI mostramos que:
%- su eficiencia es comparable a la mejor herramienta de generación exhaustiva acotada basada
%en especificaciones (Korat)
%- es útil para encontrar discrepancias entre la capacidad de generación de
%objetos de la API y la especificación
%- y resultados preliminares que muestran que los objetos generados por BEAPI son 
%buenos para el testing basado en propiedades.


%Primero, la necesidad de especificaciones de validez 
%(invariantes de representación), difíciles y costosas de escribir. Segundo, la selección manual de subconjuntos de 
%métodos de la API para construir drivers de generación, donde el espacio de combinaciones crece exponencialmente con el número de métodos.
%
%En este trabajo abordamos ambos problemas mediante dos aportes. 
%En primer lugar, presentamos un método para la identificación automática de métodos generadores de objectos.
%a partir de la API pública de una clase. Estos métodos son aquellos que cumplen la caracteristica de ser 
%suficientes (capaces de construir todas las instancias válidas dentro del scope) y minimales (sin métodos superfluos), esto quiere decir que nos permiten construir cualquier objecto de la API.
%Modelamos esta selección de métodos como un problema de optimización y exploramos el espacio con dos algoritmos de búsqueda, \emph{Hill Climbing} (HC) y Algoritmo Genetico (GA). 
%Para evaluar la calidad de un candidato (subconjunto de métodos), desarrollamos dos funciones objetivo. Una basada en generación exhaustiva acotada, 
%que estima la cantidad de objetos distintos que puede producir el conjunto de métodos bajo un scope dado; y la otra función es 
%basada en cobertura de código (líneas y ramas) obtenida al generar y ejecutar suites de tesst a partir de dicho conjunto de métodos. 
%La evaluación experimental sobre un conjunto representativo de clases indica que nuestras técnicas identifican 
%conjuntos de métodos, con tiempos de cómputo adecuados para su uso práctico.
%
%En segundo lugar, introducimos BEAPI, un enfoque de generación exhaustiva acotada que no requiere invariante de representacion, (comunmente llamados, \emph{repOK}). 
%BEAPI construye objetos ejecutando secuencias de método de la API de la clase bajo test 
%y aplica tres optimizaciones esenciales. La primera es una poda por realizar un buen uso de la API 
%(descarta tempranamente secuencias que producen excepciones o violaciones de contrato). 
%La segunda es por coincidencia de estados (\emph{state matching}) 
%para eliminar redundancias cuando diferentes secuencias generan el mismo objeto. 
%Y la tercera, es haciendo uso de los métodos generadores de objetos (los detectados en el primer aporte de este trabajo) 
%para restringir las combinaciones de métodos. 
%Realizamos una comparación contra la herramienta muy conocida en el estado del arte, \emph{Korat}, 
%y observamos que BEAPI alcanza niveles comparables de en cuanto a lo \cacho{exhaustivo} de la tecnica 
%y con mejores tiempos en escenarios de estructuras que tienen varias restricciones y, lo mas importante, sin requerir 
%los invariantes de representación. 
%Además, al contrastar los objetos generados por BEAPI con los aceptados por \emph{repOKs} existentes de \emph{Korat}, 
%mostramos que el enfoque puede revelar defectos en dichas especificaciones, subrayando la dificultad de escribir \emph{repOKs} correctos.
%
%En conjunto, los resultados demuestran que la selección automática de métodos generadores y BEAPI reducen la intervención manual del programador y 
%mejoran la eficiencia y escalabilidad de la generación de entradas a partir de APIs de clase. 
%Estos aportes facilitan la adopción de técnicas exhaustivas en contextos reales, tanto para testing como para verificación.
%
%\cacho{algo de lineas futura de trabajo, va aca?}



\vspace{0.5cm}

\newpage

\begin{center}
    \Large
    \textbf{Efficient Automatic Generation of Objects from Class APIs}
        
    \vspace{0.4cm}
    \large
    by Mariano Politano
        
    \vspace{0.6cm}
    \textbf{Abstract}
\end{center}
