%!TEX root = main.tex
\chapter*{Resumen}

\begin{center}
    \Large
    \textbf{Generación Automática Eficiente de Objetos a partir de APIs de Clase}
        
    \vspace{0.4cm}
    \large
    por Mariano Politano
        
    \vspace{0.9cm}
    \textbf{Resumen}
\end{center}


El testing de software consiste en ejecutar código bajo diversas entradas para verificar su correctitud, 
siendo una de las actividades más costosas y críticas en el desarrollo de software, 
ya que puede representar más de la mitad del costo total del desarrollo del software \cacho{cita}\ref{}.
Por ello, la automatización del testing, especialmente la generación automática de entradas, 
es fundamental para reducir costos y mejorar la eficiencia. 
El testing consiste en ejecutar una pieza de código cuya correctitud se desea verificar, 
utilizando diversas entradas y verificando si los resultados observados coinciden con los esperados. 
Esta actividad es central en el ciclo de desarrollo de software y, en proyectos reales, 
puede representar una porción sustancial del costo total ya se en tiempo de desarrollo y en recursos humanos, 
por lo que su automatización resulta clave. Entre las estrategias de automatización del \emph{testing}, 
la generación exhaustiva acotada (BEG, por sus siglas en inglés) ha mostrado ser altamente efectiva para descubrir fallas, 
ya que propone ejercitar el programa bajo todas las entradas válidas dentro de cotas dadas. 
Sin embargo, su adopción enfrenta dos grandes obstáculos. Primero, la necesidad de escribir invariantes de representacion 
que son aquellos que nos especifica que debe hacer el programa. Estos son difíciles y costosas de escribir. 
Segundo, la selección manual de subconjuntos de métodos de la API que sea \cacho{Ver esto} \emph{importante} para construir todas las posibles entradas, 
donde el espacio de combinaciones crece exponencialmente con respecto al número de métodos.



\vspace{0.5cm}

\newpage

\begin{center}
    \Large
    \textbf{Efficient Automatic Generation of Objects from Class APIs}
        
    \vspace{0.4cm}
    \large
    by Mariano Politano
        
    \vspace{0.6cm}
    \textbf{Abstract}
\end{center}