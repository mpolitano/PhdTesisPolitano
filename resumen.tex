%!TEX root = main.tex
\begin{center}
    \Large
    \textbf{Enfoques Eficientes de Generación Automática de Objetos a partir de APIs de Clase}
        
    \vspace{0.4cm}
    \large
    por Mariano Politano
        
    \vspace{0.9cm}
    \textbf{Resumen}
\end{center}
% Nuevo título: Enfoques Eficientes de Generación Automática de Objetos a partir de APIs de Clase

Garantizar la corrección funcional del software es un objetivo crucial para la
Ingeniería de Software. Numerosas técnicas 
automáticas de análisis de software se han desarrollado para ayudar a los
desarrolladores a lograr este objetivo. Muchas de estas técnicas 
requieren producir un conjunto de objetos para alimentar el programa bajo
análisis (por ejemplo, técnicas de testing, de model checking de software,
etc.). Generar un conjunto diverso de objetos para ejercitar adecuadamente el software 
es un problema desafiante, sobre todo para programas que manipulan estructuras complejas 
(por ejemplo, estructuras encadenadas almacenadas en el heap).

En este trabajo, se proponen dos enfoques para contruibir a la generación de
objetos complejos a partir de la API de un módulo. Por un lado, observamos que 
la selección de los métodos apropiados de la API es crucial para 
la construcción eficiente de objetos. Definimos como \emph{métodos generadores
de objetos} a un subconjunto minimal de métodos de la API que son suficientes para 
construir cualquier objeto para el módulo. 
Identificar un conjunto suficiente y minimal de métodos generadores de objetos 
de forma manual es una tarea trabajosa, por lo que se proponen enfoques para 
llevar a cabo esta tarea de manera automática. 

Por otro lado, se define un enfoque eficiente de generación exhaustiva acotada
de objetos basado en la API del módulo, denominado BEAPI. La generación
exhaustiva acotada consiste en crear todos los objetos de tamaño acotado para
el módulo, con cotas provistas por el usuario. Enfoques previos
requieren de especificaciones formales de las propiedades que deben satisfacer
los objetos.
En contraste, BEAPI construye objetos creando y ejecutando secuencias de invocaciones a 
métodos de la API (secuencias de test). BEAPI implementa tres 
técnicas de poda claves para su eficiencia. Primero, BEAPI
descarta secuencias de test que lanzan excepciones por el uso indebido de la
API. Segundo, BEAPI implementa coincidencia de estados para descartar
secuencias de test distintas que crean los mismos objetos. Tercero,
BEAPI emplea solo métodos generadores de objetos durante la generación 
(identificados automáticamente con los enfoques propuestos). 

Se evaluaron experimentalmente las técnicas presentadas, en clases que manipulan 
objetos complejos. 
Los resultados muestran que los enfoques de identificación de métodos
generadores se ejecutan en tiempos razonables, y encuentran métodos generadores
de objetos con buena precisión (incluyen pocos métodos
irrelevantes en pocos casos). Además, se muestran resultados preliminares que
indican que los métodos generadores de objetos son útiles para mejorar 
la generación de objetos usando técnicas de generación aleatoria y
exhaustiva acotada de tests, y para model checking de software.
La evaluación de BEAPI muestra que su eficiencia y escalabilidad respecto 
de los scopes es comparable al mejor enfoque existente (Korat). Además, 
se provee evidencia preliminar de que BEAPI sirve de ayuda para 
encontrar errores en especificaciones formales de invariantes de clase, y que 
los objetos que genera son de utilidad para el testing basado en propiedades.



% Versión larga

%A medida que el software se vuelve ubicuo gracias a los rápidos avances en la
%tecnología, garantizar la corrección funcional del mismo se vuelve crucial.
%Debido a esto se han desarrollado diversas técnicas automáticas de análisis de
%software (implementadas mediante herramientas de software), para asistir a los
%desarrolladores en distintas tareas, como por ejemplo, encontrar fallas en 
%el software. 
%
%Varias técnicas de análisis requieren producir un conjunto de
%estructuras (objetos en la terminología de la programación orientada a
%objetos) para alimentar el programa bajo test. Entre ellas,
%podemos mencionar el testing unitario, el testing basado en propiedades, y el model
%checking (acotado) de software. 
%Si los programas manipulan estructuras de datos complejas (ej. estructuras
%encadenadas almacenadas en el heap), generar un conjunto 
%diverso de objetos para ejercitar adecuadamente el software es un problema
%desafiante. 
%
%En este trabajo, se proponen dos enfoques para contruibir a la generación de
%objetos complejos a partir de la API de un módulo. Por un lado, observamos que 
%la selección de los métodos apropiados de la API de un módulo es crucial para 
%la construcción eficiente de objetos. Por esta razón, definimos el concepto
%de \emph{métodos generadores de objetos} como un subconjunto minimal de métodos de 
%la API que son suficientes para construir cualquier objeto para el módulo. 
%Identificar un conjunto suficiente y minimal de métodos generadores de objetos de forma manual 
%es una tarea trabajosa, que requiere de un análisis minucioso de la API y una comprensión profunda 
%de la semántica del módulo. 
%Para facilitar el trabajo del programador, se proponen distintos enfoques automáticos para 
%identificar conjuntos suficientes y minimales de métodos generadores de objetos
%a partir de la API. 
%
%Por otro lado, definimos un enfoque eficiente de generación exhaustiva acotada
%de objetos basado en la API del módulo, denominado BEAPI. La generación
%exhaustiva acotada consiste en crear todos los objetos de tamaño acotado para
%el módulo, con cotas provistas por el usuario (ej. máxima cantidad de nodos en
%estructuras enlazadas). La generación exhaustiva acotada ha mostrado ser un
%enfoque efectivo para revelar fallas durante el testing. Enfoques previos
%requiren de especificaciones formales de las propiedades que deben satisfacer
%los objetos válidos, y realizan la generación en base a estas especificaciones.
%En contraste, BEAPI construye objetos creando y ejecutando secuencias de invocaciones a
%métodos (secuencias de test) de la API. Para hacer posible la generación eficiente a partir de la
%API, BEAPI incluye tres técnicas de poda claves. En primer lugar, BEAPI
%descarta secuencias de test que lanzan excepciones debido al uso indebido de la
%API (ej. violaciones de precondiciones). 
%En segundo lugar, BEAPI implementa coincidencia de estados para descartar
%secuencias de test diferentes que crean los mismos objeto. En tercer lugar,
%BEAPI emplea solo métodos generadores de objetos para la creación de secuencias
%de test (identificados automáticamente con los algoritmos propuestos en una etapa 
%previa a la generación exhaustiva). 
%
%Se realizó una evaluación experimental preliminar de los enfoques
%propuestos, en clases cuyas instancias son objetos complejos. 
%Los resultados muestran que los enfoques de identificación de métodos
%generadores propuestos se ejecutan en tiempos
%razonables para esta tarea (algunos minutos), y computan generadores de objetos
%suficientes y minimales en la mayoría de los casos (incluyen algunos métodos
%innecesarios en pocos casos). Además, se muestran resultados preliminares que
%indican que los métodos generadores de objetos pueden ser útiles para la mejorar 
%la generación de objetos en herramientas de generación aleatoria de tests
%(Randoop), en herramientas de generación exhaustiva acotada (BEAPI), y en 
%bounded model checking de software (usando Java Pathfinder).
%Con respecto a BEAPI, los resultados experimentales muestran que
%su eficiencia y escalabilidad respecto de los scopes es comparable a la mejor 
%herramienta de generación exhaustiva acotada 
%basada en especificaciones (Korat). Además, se muestra que 
% BEAPI es útil para encontrar errores en 
% especificaciones formales de invariantes de clase, y que los objetos 
%generados por BEAPI son de utilidad en el contexto de testing basado en 
%propiedades.







\vspace{0.5cm}

\newpage

\begin{center}
    \Large
    \textbf{Efficient Approaches for Automatic Generation of Objects from Class APIs}
        
    \vspace{0.4cm}
    \large
    by Mariano Politano
        
    \vspace{0.6cm}
    \textbf{Abstract}
\end{center}
Ensuring the functional correctness of software is a crucial goal in Software
Engineering. Numerous automated software analysis techniques have been developed
to assist developers in achieving this goal. Many of these techniques require
producing a set of objects to serve as inputs to the program under analysis
(e.g., testing techniques, software model checking, etc.). Generating a diverse
set of objects to adequately exercise a program is a challenging problem, 
particularly for programs that manipulate complex structures 
(e.g., linked structures stored in the heap).

In this work, we propose two approaches to contribute to the generation of
complex objects from a module’s API. On the one hand, we observe that selecting
the appropriate API methods is crucial for the efficient construction of
objects. We define object builders as a minimal subset of API methods that is
sufficient to construct any object of the module. Manually identifying a
sufficient and minimal set of object builders is a labor-intensive task; 
therefore, we propose approaches to perform this task automatically.

On the other hand, we define an efficient API-based bounded exhaustive object
generation approach, called BEAPI. Bounded exhaustive generation consists of
creating all objects of bounded size for a given module, with bounds provided by
the user. Previous approaches require formal specifications of the properties
that objects must satisfy. In contrast, BEAPI constructs objects by creating and
executing sequences of API method invocations (test sequences). BEAPI implements
three key pruning techniques to achieve efficiency. First, BEAPI discards test
sequences that throw exceptions due to improper use of the API. Second, BEAPI
implements state matching to discard different test sequences that produce the
same objects. Third, BEAPI uses only object builders during generation 
(automatically identified using the proposed approaches).

We experimentally assess our techniques on classes that manipulate complex
objects. The results show that the builder identification approaches run in
reasonable time and identify object builders with good precision (including only
a few irrelevant methods in a few cases). 
In addition, we present preliminary results indicating that object builders are useful for improving object generation using both random and bounded exhaustive test generation techniques, as well as for software model checking. 
Furthermore, the evaluation of BEAPI shows that its efficiency and scalability
with respect to the scopes are comparable to those of the best existing approach (Korat). 
Moreover, we provide preliminary evidence that BEAPI is helpful in finding errors in formal 
class invariant specifications, and that the objects it generates are useful for property-based testing.



