%\documentclass[10pt]{report} %twoside para impresion final 
\documentclass[11pt,a4paper,openany]{report} %a4paper,
\usepackage{array}
\makeatletter
\renewcommand*\@argarraycr{\let\@tempa\relax%
  \ifnum0=`{\fi}\@argarraycr@}
\makeatother

% Note: Make all your adjustments in here
\input{classicthesis-config}
\usepackage{xcolor}
\newcommand{\cacho}{\textcolor{red}}
\usepackage{tikz}
\usepackage{multirow}
\usepackage{underscore}
\usepackage[spanish]{babel}
\usepackage{titletoc}
\usepackage{tocloft}
\usepackage{fancyhdr}
\usepackage{lineno}     % Line numbers with e.g. the linenumbers environment
\usepackage{sectsty}    % \allsectionsfont{}, \chapterfont{}, \sectionfont{}


\usepackage{titletoc}
\usepackage{tocloft}
\renewcommand{\cftchapleader}{\cftdotfill{\cftdotsep}} % Agrega puntos para las secciones

\titlecontents{chapter}
  [0em]
  {\vspace{1em}\bfseries}
  {\thecontentslabel.\enspace}
  {}
  {\hfill\contentspage} % Mueve el número de página a la derecha

\titlecontents{section}
  [2em]
  {}
  {\thecontentslabel.\enspace}
  {}
  {\titlerule*[0.75em]{.}\contentspage} % Agrega líneas de puntos para las secciones

\titlecontents{subsection}
  [4.5em]
  {}
  {\thecontentslabel.\enspace}
  {}
  {\titlerule*[0.75em]{.}\contentspage} % Agrega líneas de puntos para las subsecciones


\titleformat{\chapter}[display]
  {\normalfont\huge\bfseries}{\chaptertitlename\ \thechapter}{20pt}{\Huge}
\titlespacing*{\chapter}{0pt}{50pt}{40pt}

\renewcommand{\cftsecleader}{\cftdotfill{\cftdotsep}} % Línea de puntos para secciones
\renewcommand{\cftsubsecleader}{\cftdotfill{\cftdotsep}} % Línea de puntos para subsecciones

% Eliminar encabezados
% Configuración del estilo de página
\pagestyle{fancy}
\fancyhf{} % Limpiar encabezados y pies de página
\fancyfoot[C]{\thepage} % Número de página centrado en el pie de página
\renewcommand{\headrulewidth}{0pt} % Eliminar línea del encabezado
\babelhyphenation[spanish]{Node-Caching-Linked-List}
\definecolor{dkgreen}{rgb}{0,0.6,0}
\usepackage{listings}
\lstset{ %this is the stype
        captionpos=b,
        mathescape=true,
        texcl=false,
        frame=tb,
        language=Java,
        numbers=left,
      	aboveskip=0mm,
      	belowskip=0mm,
      	showstringspaces=false,
      	columns=flexible,
      	basicstyle={\scriptsize\ttfamily}, 
        keywordstyle=\color{blue}\bfseries,
        commentstyle=\color{dkgreen},
        morekeywords={,input, output, return, datatype, function, in, if, then, else, foreach, while, begin, end, inputs, outputs, procedure, for, do, done, fi, break, not, local, continue, and,} %add the keywords you want, or load a language as Rubens explains in his comment above.
        numbers=left,
        xleftmargin=.05\textwidth
      }


\lstnewenvironment{algorithm1}[1][] %defines the algorithm listing environment
{   
    \lstset{ %this is the stype
        mathescape=true,
        texcl=false,
        frame=tb,
        language=Java,
      	numbers=left,
      	aboveskip=0mm,
      	belowskip=0mm,
      	showstringspaces=false,
      	columns=flexible,
      	basicstyle={\scriptsize\ttfamily}, 
%        numberstyle=\scriptsize,
%        numberstyle=\tiny,
%        basicstyle=\scriptsize, 
%        keywordstyle=\color{black}\bfseries\em,
        keywordstyle=\color{blue}\bfseries,
        commentstyle=\color{dkgreen},
        morekeywords={,input, output, return, datatype, function, in, if, then, else, foreach, while, begin, end, inputs, outputs, procedure, for, do, done, fi, break, not, local, continue, and,} %add the keywords you want, or load a language as Rubens explains in his comment above.
        numbers=left,
        xleftmargin=.05\textwidth
      }
}
{}


\usepackage[misc]{ifsym}
\usepackage{tikz}



%%%%  Contenido  %%%%%%%%%%%%%%%%%%%%%%%%%%%%%%%%%%%%%%%%%%%%%%%%%%%%%%%%
\begin{document}

% \raggedbottomn



% ********************************************************************
% Frontmatter
%\include{titlepage}
%!TEX root = main.tex
\begin{titlepage}
\begin{center}

\begin{center}
%\href{https://creativecommons.org/licenses/by-nc-sa/4.0/deed.es}{\includegraphics[scale=0.8]{by-nc-sa.png}}
{\includegraphics[scale=0.65]{images/logos.png}}
\end{center}
\medskip

{\LARGE \textbf{Generación eficiente de entradas exhaustivas acotadas basada en la API sin la necesidad de especificaciones}}\\ 
\vspace{1cm}
{\Large por Mariano Politano}\\\vspace{1cm}

{\large Presentado ante la Facultad de Matem\'atica, Astronom{\'\i}a, F{\'\i}sica y Computaci\'on como parte de los requisitos para la obtenci\'on del grado de Doctor en Ciencias de la Computación de la} \\ \vspace{1cm}

{\large UNIVERSIDAD NACIONAL DE C\'ORDOBA}\\ \vspace{1.5cm}
{\large  XXX, 2025}\\ \vspace{0.5cm} 
%{\large  \copyright\;FaMAF \/- UNC 2024}\\ \vspace{0.5cm}

{\Large Director: Pablo Daniel Ponzio}
\\  \vspace{0.5cm} 
{\Large Co-Director: Germán Enrique Regis} \vspace{0.5cm} 

Tribunal especial (titulares y suplentes):

\end{center}

\begin{center}
\href{https://creativecommons.org/licenses/by-nc-sa/4.0/}{\includegraphics[scale=0.8]{images/by-nc-sa.png}}
\end{center}

\begin{center}
Generación eficiente de entradas exhaustivas acotadas basada en la API sin la necesidad de especificaciones © 2024 por \href{https://mpolitano.github.io}{Mariano Politano} se distribuye bajo licencia \href{https://creativecommons.org/licenses/by-nc-sa/4.0/}{CC BY-NC-SA 4.0} 
\end{center}

\end{titlepage} 

%LICENCIA FAMAF
%<a rel="license" href="http://creativecommons.org/licenses/by-nc/2.5/ar/"><img alt="Licencia Creative Commons" style="border-width:0" src="https://i.creativecommons.org/l/by-nc/2.5/ar/88x31.png" /></a><br /><span xmlns:dct="http://purl.org/dc/terms/" property="dct:title">Técnicas Automáticas para la Elaboración, Validación y Verificación de Requisitos de Software</span> por <a xmlns:cc="http://creativecommons.org/ns#" href="http://dc.exa.unrc.edu.ar/staff/rdegiovanni/" property="cc:attributionName" rel="cc:attributionURL">Renzo Degiovanni</a> se distribuye bajo una <a rel="license" href="http://creativecommons.org/licenses/by-nc/2.5/ar/">Licencia Creative Commons Atribución-NoComercial 2.5 Argentina</a>.


%\textbf{Lugar de Trabajo}
% Departamento de Computaci\'on\\
%\hspace{3cm} Facultad de Ciencias Exactas F\'isico-Qu\'imicas y Naturales\\
%\hspace{3cm} Universidad Nacional de R\'io Cuarto\\
%\vspace{5mm}
%Director: \textbf{Dr.~Nazareno~Mat\'ias~Aguirre}\\
%\vspace{5mm}
%Jurado:\\
%\vspace{5mm}
%\noindent
%C\'ordoba, Argentina, 2014

\chapter*{}
\pagenumbering{Roman} % para comenzar la numeracion de paginas en numeros romanos
\begin{flushright}
\textit{}
\end{flushright}
%!TEX root = main.tex
\begin{center}
    \Large
    \textbf{Enfoques Eficientes de Generación Automática de Objetos a partir de APIs de Clase}
        
    \vspace{0.4cm}
    \large
    por Mariano Politano
        
    \vspace{0.9cm}
    \textbf{Resumen}
\end{center}
% Nuevo título: Enfoques Eficientes de Generación Automática de Objetos a partir de APIs de Clase

Garantizar la corrección funcional del software es objetivo crucial para la
Ingeniería de Software. Numerosas técnicas 
automáticas de análisis de software se han desarrollado para ayudar a los
desarrolladores a lograr este objetivo. Muchas de estas técnicas 
requieren producir un conjunto de objetos para alimentar el programa bajo
análisis (por ejemplo, técnicas de testing, de model checking de software,
etc.). Generar un conjunto diverso de objetos para ejercitar adecuadamente el software 
es un problema desafiante, sobre todo para programas que manipulan estructuras complejas 
(ej. estructuras encadenadas almacenadas en el heap).

En este trabajo, se proponen dos enfoques para contruibir a la generación de
objetos complejos a partir de la API de un módulo. Por un lado, observamos que 
la selección de los métodos apropiados de la API es crucial para 
la construcción eficiente de objetos. Definimos como \emph{métodos generadores
de objetos} a un subconjunto minimal de métodos de la API que son suficientes para 
construir cualquier objeto para el módulo. 
Identificar un conjunto suficiente y minimal de métodos generadores de objetos 
de forma manual es una tarea trabajosa, por lo que se proponen enfoques para 
llevar a cabo esta tarea de manera automática. 

Por otro lado, se define un enfoque eficiente de generación exhaustiva acotada
de objetos basado en la API del módulo, denominado BEAPI. La generación
exhaustiva acotada consiste en crear todos los objetos de tamaño acotado para
el módulo, con cotas provistas por el usuario. Enfoques previos
requieren de especificaciones formales de las propiedades que deben satisfacer
los objetos.
En contraste, BEAPI construye objetos creando y ejecutando secuencias de invocaciones a 
métodos (secuencias de test) de la API. BEAPI implementa tres 
técnicas de poda claves para su eficiencia. Primero, BEAPI
descarta secuencias de test que lanzan excepciones por el uso indebido de la
API. Segundo, BEAPI implementa coincidencia de estados para descartar
secuencias de test distintas que crean los mismos objetos. Tercero,
BEAPI emplea solo métodos generadores de objetos durante la generación 
(identificados automáticamente con los enfoques propuestos). 

Se evaluaron experimentalmente las técnicas propuestas, en clases que manipulan 
objetos complejos. 
Los resultados muestran que los enfoques de identificación de métodos generadores se ejecutan en tiempos razonables, y encuentran métodos generadores de objetos
suficientes y (casi) minimales (incluyen algunos métodos
innecesarios en pocos casos). Además, se muestran resultados preliminares que
indican que los métodos generadores de objetos son útiles para mejorar 
la generación de objetos usando técnicas de generación aleatoria y
exhaustiva acotada de tests, y para model checking de software.
La evaluación de BEAPI muestra que su eficiencia y escalabilidad respecto 
de los scopes es comparable al mejor enfoque existente (Korat). Además, 
se provee evidencia preliminar de que BEAPI sirve de ayuda para 
encontrar errores en especificaciones formales de invariantes de clase, y que 
los objetos que genera son de utilidad para el testing basado en propiedades.



% Versión larga

%A medida que el software se vuelve ubicuo gracias a los rápidos avances en la
%tecnología, garantizar la corrección funcional del mismo se vuelve crucial.
%Debido a esto se han desarrollado diversas técnicas automáticas de análisis de
%software (implementadas mediante herramientas de software), para asistir a los
%desarrolladores en distintas tareas, como por ejemplo, encontrar fallas en 
%el software. 
%
%Varias técnicas de análisis requieren producir un conjunto de
%estructuras (objetos en la terminología de la programación orientada a
%objetos) para alimentar el programa bajo test. Entre ellas,
%podemos mencionar el testing unitario, el testing basado en propiedades, y el model
%checking (acotado) de software. 
%Si los programas manipulan estructuras de datos complejas (ej. estructuras
%encadenadas almacenadas en el heap), generar un conjunto 
%diverso de objetos para ejercitar adecuadamente el software es un problema
%desafiante. 
%
%En este trabajo, se proponen dos enfoques para contruibir a la generación de
%objetos complejos a partir de la API de un módulo. Por un lado, observamos que 
%la selección de los métodos apropiados de la API de un módulo es crucial para 
%la construcción eficiente de objetos. Por esta razón, definimos el concepto
%de \emph{métodos generadores de objetos} como un subconjunto minimal de métodos de 
%la API que son suficientes para construir cualquier objeto para el módulo. 
%Identificar un conjunto suficiente y minimal de métodos generadores de objetos de forma manual 
%es una tarea trabajosa, que requiere de un análisis minucioso de la API y una comprensión profunda 
%de la semántica del módulo. 
%Para facilitar el trabajo del programador, se proponen distintos enfoques automáticos para 
%identificar conjuntos suficientes y minimales de métodos generadores de objetos
%a partir de la API. 
%
%Por otro lado, definimos un enfoque eficiente de generación exhaustiva acotada
%de objetos basado en la API del módulo, denominado BEAPI. La generación
%exhaustiva acotada consiste en crear todos los objetos de tamaño acotado para
%el módulo, con cotas provistas por el usuario (ej. máxima cantidad de nodos en
%estructuras enlazadas). La generación exhaustiva acotada ha mostrado ser un
%enfoque efectivo para revelar fallas durante el testing. Enfoques previos
%requiren de especificaciones formales de las propiedades que deben satisfacer
%los objetos válidos, y realizan la generación en base a estas especificaciones.
%En contraste, BEAPI construye objetos creando y ejecutando secuencias de invocaciones a
%métodos (secuencias de test) de la API. Para hacer posible la generación eficiente a partir de la
%API, BEAPI incluye tres técnicas de poda claves. En primer lugar, BEAPI
%descarta secuencias de test que lanzan excepciones debido al uso indebido de la
%API (ej. violaciones de precondiciones). 
%En segundo lugar, BEAPI implementa coincidencia de estados para descartar
%secuencias de test diferentes que crean los mismos objeto. En tercer lugar,
%BEAPI emplea solo métodos generadores de objetos para la creación de secuencias
%de test (identificados automáticamente con los algoritmos propuestos en una etapa 
%previa a la generación exhaustiva). 
%
%Se realizó una evaluación experimental preliminar de los enfoques
%propuestos, en clases cuyas instancias son objetos complejos. 
%Los resultados muestran que los enfoques de identificación de métodos
%generadores propuestos se ejecutan en tiempos
%razonables para esta tarea (algunos minutos), y computan generadores de objetos
%suficientes y minimales en la mayoría de los casos (incluyen algunos métodos
%innecesarios en pocos casos). Además, se muestran resultados preliminares que
%indican que los métodos generadores de objetos pueden ser útiles para la mejorar 
%la generación de objetos en herramientas de generación aleatoria de tests
%(Randoop), en herramientas de generación exhaustiva acotada (BEAPI), y en 
%bounded model checking de software (usando Java Pathfinder).
%Con respecto a BEAPI, los resultados experimentales muestran que
%su eficiencia y escalabilidad respecto de los scopes es comparable a la mejor 
%herramienta de generación exhaustiva acotada 
%basada en especificaciones (Korat). Además, se muestra que 
% BEAPI es útil para encontrar errores en 
% especificaciones formales de invariantes de clase, y que los objetos 
%generados por BEAPI son de utilidad en el contexto de testing basado en 
%propiedades.





%Our experimental assessment shows that BEAPI’s efficiency and scalabil-
%ity is competitive with existing BEG approaches, without the need for
%repOKs. We also show that BEAPI can assist the user in finding flaws in
%repOKs, by (automatically) comparing inputs generated by BEAPI with
%those generated from a repOK. Using this approach, we revealed several
%errors in repOKs taken from the assessment of related tools, demonstrat-
%ing the difficulties of writing precise repOKs for BEG.



%Para builders mostramos que las técnicas:
%- computan builders en tiempos razonables 
%- son precisas
%- y resultados preliminares que muestran que los builders pueden mejorar la generación de
%objetos en herramientas como Randoop (aleatoria), y BEAPI (generación exhaustiva
%acotada), y en la JPF (bounded model checking de software).
%
%Respecto a BEAPI mostramos que:
%- su eficiencia es comparable a la mejor herramienta de generación exhaustiva acotada basada
%en especificaciones (Korat)
%- es útil para encontrar discrepancias entre la capacidad de generación de
%objetos de la API y la especificación
%- y resultados preliminares que muestran que los objetos generados por BEAPI son 
%buenos para el testing basado en propiedades.


%Primero, la necesidad de especificaciones de validez 
%(invariantes de representación), difíciles y costosas de escribir. Segundo, la selección manual de subconjuntos de 
%métodos de la API para construir drivers de generación, donde el espacio de combinaciones crece exponencialmente con el número de métodos.
%
%En este trabajo abordamos ambos problemas mediante dos aportes. 
%En primer lugar, presentamos un método para la identificación automática de métodos generadores de objectos.
%a partir de la API pública de una clase. Estos métodos son aquellos que cumplen la caracteristica de ser 
%suficientes (capaces de construir todas las instancias válidas dentro del scope) y minimales (sin métodos superfluos), esto quiere decir que nos permiten construir cualquier objecto de la API.
%Modelamos esta selección de métodos como un problema de optimización y exploramos el espacio con dos algoritmos de búsqueda, \emph{Hill Climbing} (HC) y Algoritmo Genetico (GA). 
%Para evaluar la calidad de un candidato (subconjunto de métodos), desarrollamos dos funciones objetivo. Una basada en generación exhaustiva acotada, 
%que estima la cantidad de objetos distintos que puede producir el conjunto de métodos bajo un scope dado; y la otra función es 
%basada en cobertura de código (líneas y ramas) obtenida al generar y ejecutar suites de tesst a partir de dicho conjunto de métodos. 
%La evaluación experimental sobre un conjunto representativo de clases indica que nuestras técnicas identifican 
%conjuntos de métodos, con tiempos de cómputo adecuados para su uso práctico.
%
%En segundo lugar, introducimos BEAPI, un enfoque de generación exhaustiva acotada que no requiere invariante de representacion, (comunmente llamados, \emph{repOK}). 
%BEAPI construye objetos ejecutando secuencias de método de la API de la clase bajo test 
%y aplica tres optimizaciones esenciales. La primera es una poda por realizar un buen uso de la API 
%(descarta tempranamente secuencias que producen excepciones o violaciones de contrato). 
%La segunda es por coincidencia de estados (\emph{state matching}) 
%para eliminar redundancias cuando diferentes secuencias generan el mismo objeto. 
%Y la tercera, es haciendo uso de los métodos generadores de objetos (los detectados en el primer aporte de este trabajo) 
%para restringir las combinaciones de métodos. 
%Realizamos una comparación contra la herramienta muy conocida en el estado del arte, \emph{Korat}, 
%y observamos que BEAPI alcanza niveles comparables de en cuanto a lo \cacho{exhaustivo} de la tecnica 
%y con mejores tiempos en escenarios de estructuras que tienen varias restricciones y, lo mas importante, sin requerir 
%los invariantes de representación. 
%Además, al contrastar los objetos generados por BEAPI con los aceptados por \emph{repOKs} existentes de \emph{Korat}, 
%mostramos que el enfoque puede revelar defectos en dichas especificaciones, subrayando la dificultad de escribir \emph{repOKs} correctos.
%
%En conjunto, los resultados demuestran que la selección automática de métodos generadores y BEAPI reducen la intervención manual del programador y 
%mejoran la eficiencia y escalabilidad de la generación de entradas a partir de APIs de clase. 
%Estos aportes facilitan la adopción de técnicas exhaustivas en contextos reales, tanto para testing como para verificación.
%
%\cacho{algo de lineas futura de trabajo, va aca?}



\vspace{0.5cm}

\newpage

\begin{center}
    \Large
    \textbf{Efficient Automatic Generation of Objects from Class APIs}
        
    \vspace{0.4cm}
    \large
    by Mariano Politano
        
    \vspace{0.6cm}
    \textbf{Abstract}
\end{center}

% \pagenumbering{roman}
% \pagestyle{plain}
%!TEX root = main.tex
\chapter*{Agradecimientos}
\label{cap:agradecimientos}

\cacho{Completar}
\newpage

\tableofcontents % indice de contenidos

\cleardoublepage
% \listoffigures
%\cleardoublepage
% \listoftables
% \pagestyle{scrheadings}
%\cleardoublepage
% ********************************************************************
% Mainmatter
\lstset{basicstyle=\ttfamily,showstringspaces=false}
% \pagestyle{headings}
\cleardoublepage
\pagenumbering{arabic} % para empezar la numeración con números

\chapter{Introducci\'on}
\label{cap:introduccion}
\cleardoublepage
\chapter[Preliminares]{Preliminares}
\label{cap:preliminares.BE}

\section{Testing}
El testing constituye una de las estrategias más directas y ampliamente utilizadas para verificar el comportamiento de un programa.
Esta técnica consiste en ejecutar el software bajo distintas condiciones específicas, observando si los resultados obtenidos se corresponden con los resultados esperados. 
Dado que no es factible analizar el comportamiento de un programa frente a todas las posibles entradas (debido a que este conjunto puede ser extremadamente grande o incluso infinito, como sucede en programas que procesan números enteros arbitrarios), 
se recurre a la selección de un subconjunto representativo de casos de prueba, comúnmente denominado \emph{suite de tests}.

El objetivo de esta suite es brindar un nivel de confianza suficiente sobre la corrección funcional del software, 
incluso en escenarios que no han sido ejercitados directamente. 
El testing es una etapa fundamental en los procesos modernos de desarrollo de software. 
Tradicionalmente, se realiza después de la fase de implementación, pero en metodologías de desarrollo ágiles está integrado de manera más continua, 
acompañando activamente el ciclo de desarrollo e incluso guiando decisiones de diseño e implementación.



\section{Korat: Generación exhaustiva acotada basada en especificaciones}
\label{sec:korat}


Korat es uno de los enfoques más conocidos y eficientes para la generación exhaustiva acotada
de entradas para test \cite{Boyapati02}.
Para funcionar, Korat requiere de una especificación formal dada en términos de
un método booleano, denominado \texttt{repOK}, que sirve para determinar si una estructura
es válida o no. Es decir, si satisface o no las restricciones que la estructura
debe cumplir, como por ejemplo, un invariante de representación de una
estructura de datos, una precondición de un método, etc. Korat también requiere 
de la definición del alcance de la generación, (en inglés, el \emph{scope}),
esto es, límites en el tamaño de los dominios de datos a ser utilizados
para la generación de estructuras.

Así, Korat genera de manera exhuastiva todas las estructuras acotadas que
satisfacen las condiciones de validez definidas por el programador en el
\texttt{repOK}. 
Korat explora de manera exhaustiva el espacio de todas las posibles instancias
acotadas de una clase (por ejemplo, listas con hasta $k$ nodos), 
retornando únicamente aquellas que satisfacen el invariante especificado. 

La eficiencia de Korat radica en su estrategia de exploración, 
que evita examinar múltiples estructuras que se sabe, por construcciones previas, que no conducen a estructuras válidas. 
Para lograr esto, Korat realiza una exploración sistemática del espacio de
posibles asignaciones de valores de atributos (por ejemplo, referencias de
nodos, enteros, etc.) utilizando una técnica basada en el seguimiento de los
atributos  leídos por el \texttt{repOK}.
Cada vez que una estructura es evaluada por el método \texttt{repOK}, Korat
registra qué atributos han sido accedidos (conocidos como \emph{field
accesses}). Esta información se utiliza para podar el espacio de búsqueda: si
una configuración parcial del heap no puede dar lugar a una estructura válida
(\texttt{repOK} da false), todas sus extensiones también serán descartadas.
De esta forma, el algoritmo de Korat combina una búsqueda basada en backtracking con poda
guiada por la ejecución del \texttt{repOK}, que le permite explorar 
el espacio de búsqueda acotado eficientemente, evitando explorar estructuras innecesarias. 

Korat ha demostrado ser eficaz y eficiente para la generación de entradas para
estructuras complejas con invariantes con propiedades ricas, como
listas doblemente enlazadas, árboles binarios, heaps binomiales, etc. \cite{Boyapati02}.
%, y ha sido utilizado como base para múltiples trabajos posteriores en verificación y generación de tests. Su enfoque representa un punto de referencia importante para técnicas que requieren generación de entradas válidas y exhaustivas a partir de especificaciones formales, y constituye una herramienta clásica dentro del campo del \emph{bounded exhaustive testing}.









%La \textit{generación exhaustiva de entradas} para probar APIs es una tarea crítica en la prueba de software, ya que permite identificar errores y garantizar que los programas funcionen correctamente bajo diversas condiciones. Sin embargo, el número potencialmente enorme de valores de entrada genera una explosión combinatoria, lo que hace inviable probar todas las posibles combinaciones.
%
%Para abordar este problema, se utiliza la \textit{generación exhaustiva acotada de entradas} (\textit{Bounded Exhaustive Testing, BE}), que consiste en generar todas las combinaciones posibles dentro de un rango predefinido. Este enfoque equilibra cobertura y factibilidad al reducir el espacio de prueba a un subconjunto manejable, sin dejar de incluir casos críticos que podrían pasar desapercibidos con métodos menos sistemáticos, como la \textit{generación aleatoria de pruebas}. Si bien la generación aleatoria es menos costosa computacionalmente, puede omitir casos límite relevantes, afectando la detección de errores.
%
%Korat es una herramienta que implementa la generación exhaustiva acotada, explorando todas las estructuras válidas dentro de ciertas restricciones. Su eficiencia se basa en la poda de combinaciones redundantes y en la eliminacion de estructuras isomorfas redundantes.
%
%El proceso de Korat consta de tres pasos clave:
%
%\begin{enumerate}
%    \item \textbf{Especificación de invariantes}: El usuario define las propiedades que deben cumplir las estructuras de datos, como que un árbol binario no debe tener ciclos y que cada nodo puede tener hasta dos hijos.
%    \item \textbf{Generación de candidatos}: Se generan todas las configuraciones posibles dentro de los límites especificados. Por ejemplo, si se establece un máximo de tres nodos para un árbol binario, se generarán todas las combinaciones con 1, 2 y 3 nodos.
%    \item \textbf{Filtrado y validación}: Se descartan las configuraciones que no cumplen con las restricciones definidas.
%\end{enumerate}
%
%\subsubsection{Ejemplo: Árboles Binarios}
%
%Si queremos probar una función que opera sobre árboles binarios, podemos definir:
%
%\begin{itemize}
%    \item \textbf{Límite}: Máximo 3 nodos.
%    \item \textbf{Invariantes}:
%    \begin{itemize}
%        \item No hay ciclos en el árbol.
%        \item Cada nodo tiene como máximo dos hijos.
%        \item Todos los nodos están conectados.
%    \end{itemize}
%\end{itemize}
%
%Korat generará todas las configuraciones válidas dentro de estos límites, como:
%
%\begin{itemize}
%    \item Un árbol con un solo nodo raíz.
%    \item Un árbol con un nodo raíz y un hijo izquierdo.
%    \item Un árbol con un nodo raíz, un hijo izquierdo y un hijo derecho.
%\end{itemize}
%
%Estos casos de prueba permiten verificar que la función maneja correctamente distintas estructuras de árboles binarios.
%
%\subsection{Ventajas de Korat}
%
%\begin{itemize}
%    \item \textbf{Cobertura completa}: Garantiza la exploración de todas las configuraciones dentro de los límites establecidos.
%    \item \textbf{Detección de casos límite}: Aumenta la probabilidad de encontrar errores en escenarios extremos.
%    \item \textbf{Eficiencia}: Evita la generación de estructuras inválidas o redundantes mediante poda y filtrado.
%\end{itemize}
%
%En esta tesis, Korat se utiliza como referencia para evaluar nuevas técnicas de generación de pruebas. Su enfoque sistemático y basado en especificaciones permite compararlo con otros métodos en términos de cobertura y calidad de los casos generados.
%
%
%
%
%
%La generación exhaustiva de entradas para realizar pruebas a las APIs de los programas es una tarea crítica en la prueba del software, ya que permite identificar errores y garantizar que los programas se comporten correctamente bajo una amplia gama de condiciones. Sin embargo, este proceso es inherentemente complejo debido al número potencialmente enorme de valores de entrada posibles, lo que genera una explosión combinatoria que hace inviable la prueba exhaustiva de todas las posibles entradas.
%
%Para abordar esta problemática, se recurre a la generación exhaustiva acotada de entradas (\emph{Bounded Exhaustive Testing, BE}), una técnica que consiste en generar todas las combinaciones posibles de entradas dentro de un rango o límite predefinido. Este enfoque permite reducir el espacio de entradas a un subconjunto manejable que aún cubre una diversidad significativa de casos, aumentando así la probabilidad de descubrir errores que podrían pasar desapercibidos con otros métodos de prueba.
%
%Por otro lado, una alternativa común a la generación exhaustiva acotada es la generación aleatoria de pruebas. Aunque este método es menos costoso computacionalmente y puede ser útil en ciertos contextos, presenta el inconveniente de que podría no cubrir casos críticos o bordes del espacio de entradas, lo que podría resultar en la omisión de errores significativos.
%
%En resumen, la generación exhaustiva acotada de entradas representa un compromiso entre la cobertura exhaustiva y la factibilidad práctica, ofreciendo una estrategia eficaz para la prueba de APIs, especialmente en sistemas donde la calidad y la seguridad son primordiales.
%
%Korat es una herramienta que implementa la técnica de generación exhaustiva acotada. Korat es una herramienta poderosa para la generación exhaustiva acotada de casos de prueba, permitiendo explorar todas las posibles estructuras válidas dentro de las restricciones establecidas. Su enfoque basado en filtrado, combinado con técnicas de poda y evitando las estructuras isomorfas, garantiza la eficiencia y efectividad en la generación de casos de prueba.
%
%La idea fundamental detrás de Korat es generar todas las configuraciones válidas de una estructura de datos dentro de un rango acotado (por ejemplo, un número máximo de nodos en un árbol o elementos en una lista). A diferencia de la generación aleatoria de pruebas, Korat garantiza que se exploren todas las combinaciones posibles dentro de los límites establecidos, lo que aumenta la probabilidad de detectar errores sutiles o casos límite.
%
%El proceso de Korat se basa en tres pasos clave:
%\begin{enumerate}
%\item \textbf{Especificación de invariantes}: El usuario define las propiedades que deben cumplir las estructuras de datos mediante predicados lógicos. Por ejemplo, en un árbol binario, se puede especificar que no debe contener ciclos y que cada nodo debe tener como máximo dos hijos
%
%\item \textbf{Generación de candidatos}: Korat genera todas las posibles configuraciones de la estructura de datos dentro de los límites especificados. Por ejemplo, si se define un límite de 3 nodos para un árbol binario, Korat generará todas las combinaciones posibles de árboles con 1, 2 o 3 nodos.
%
%\item \textbf{Filtrado y validación}: Cada configuración generada es validada contra los invariantes especificados. Solo las configuraciones que cumplen con las restricciones son retenidas como casos de prueba válidos.
%
%\end{enumerate}
%
%Para ilustrar el funcionamiento de Korat, consideremos el siguiente ejemplo concreto:
%
%\subsubsection{Ejemplo: Árboles Binarios}
%Supongamos que queremos probar una función que opera sobre árboles binarios. Con Korat, podemos definir los siguientes límites y restricciones:
%\begin{itemize}
%\item Límite: Árboles con un máximo de 3 nodos.
%\item Invariantes:
%\begin{itemize}
%\item No hay ciclos en el árbol.
%\item Cada nodo tiene como máximo dos hijos.
%\item Todos los nodos están conectados (no hay subárboles desconectados).
%\end{itemize}
%\end{itemize}
%
%Korat generará todas las posibles configuraciones de árboles binarios con 1, 2 y 3 nodos que cumplan con estas restricciones. Por ejemplo:
%\begin{itemize}
%\item Un árbol con un solo nodo raíz.
%\item Un árbol con un nodo raíz y un hijo izquierdo.
%\item Un árbol con un nodo raíz, un hijo izquierdo y un hijo derecho.
%\end{itemize}
%
%Estos casos de prueba permiten verificar que la función bajo prueba maneja correctamente diferentes configuraciones de árboles binarios.
%
%\cacho{Pongo ejemplos de codigo? de repok? de dibujitos?}
%
%La generación exhaustiva acotada implementada por Korat ofrece varias ventajas:
%\begin{itemize}
%\item \textbf{Cobertura completa}: Garantiza que todas las configuraciones válidas dentro de los límites establecidos sean exploradas.
%\item \textbf{Detección de casos límite}: Al generar todas las combinaciones posibles, aumenta la probabilidad de detectar errores en casos extremos o inusuales.
%\item \textbf{Eficiencia}: Las técnicas de poda y filtrado evitan la generación de configucciones redundantes o inválidas, optimizando el proceso.
%\end{itemize}
%
%En esta tesis, Korat se utiliza como una herramienta de referencia para evaluar la efectividad de nuevas técnicas de generación de pruebas. Su enfoque sistemático y basado en especificaciones lo convierte en un estándar ideal para comparar la cobertura y la calidad de los casos de prueba generados. Además, su capacidad para explorar exhaustivamente el espacio de entradas dentro de límites acotados proporciona una base sólida para identificar deficiencias en otros métodos de generación de pruebas.
%
%%En resumen, la generación exhaustiva acotada de entradas representa un compromiso entre la cobertura exhaustiva y la factibilidad práctica, ofreciendo una estrategia eficaz para la prueba de APIs, especialmente en sistemas donde la calidad y la seguridad son primordiales.
%
%\pp{Hacer una única sección con estas dos}
%
%% Los enfoques de generación de pruebas automatizadas tienen como objetivo ayudar a los desarrolladores en tareas cruciales de prueba de software [TODO ???], como la generación automática o facilitar la creación de conjuntos de pruebas [TODO: ????] y la detección y reporte automáticos de fallas [TODO: ????] . Muchos de estos enfoques implican componentes aleatorios que evitan una exploración sistemática del espacio de comportamientos, pero mejoran la eficiencia de la generación de pruebas [TODO: ????] . Si bien estos enfoques han sido muy útiles para encontrar una gran cantidad de errores en el software, podrían perder la exploración de ciertos comportamientos defectuosos del software debido a su naturaleza aleatoria. Los enfoques alternativos tienen como objetivo explorar sistemáticamente un número muy grande de ejecuciones del software bajo prueba (SUT), con el objetivo de proporcionar garantías más sólidas sobre la ausencia de errores[TODO: ????] . Uno de estos enfoques es la generación exhaustiva acotada (BE) [TODO: ????] , que consiste en generar todas las estructuras factibles que se pueden construir utilizando dominios de datos acotados. Los objetivos comunes de los enfoques BE han sido implementaciones de estructuras de datos complejas y dinámicas con ricos y estructurados enlaces (por ejemplo, listas enlazadas, árboles, etc.). Los enfoques BE de caja negra [TODO: ????]  son los más utilizados y eficientes para probar software. Requieren que el usuario proporcione una especificación formal de las restricciones que las estructuras deben satisfacer, con mayor frecuencia una invariante de representación de la estructura (repOK), y los límites de los dominios de datos [TODO: ????] , a menudo llamados \emph{scope}. De este modo, los enfoques BE de caja negra generan todas las estructuras dentro de los ámbitos proporcionados que satisfacen repOK.
%% Varios estudios muestran que los enfoques BE son efectivos para revelar fallas en el software [TODO ???]. Además, la llamada hipótesis del cota pequeña [TODO ?], que establece que la mayoría de las fallas de software se pueden revelar ejecutando el SUT en "entradas pequeñas", sugiere que, si se utilizan ámbitos lo suficientemente grandes, los enfoques BE deberían ser capaces de revelar la mayoría (si no todas) las fallas en el SUT. El desafío que enfrentan los enfoques BE es cómo explorar eficientemente un gran espacio de búsqueda, que en el peor de los casos crece exponencialmente con respecto a los ámbitos. El espacio de búsqueda a menudo incluye un gran número de estructuras no válidas (que no satisfacen repOK) y estructuras isomórficas [TODO ??]. Por lo tanto, podar partes del espacio de búsqueda que involucran estructuras inválidas y redundantes es clave para hacer que los enfoques BE se escalen en la práctica [TODO ??]. Escribir especificaciones formales apropiadas para la generación de BE es una tarea desafiante y que consume mucho tiempo. Las especificaciones deben capturar precisamente el conjunto de restricciones previstas en las estructuras. Las especificaciones sobrerestringidas hacen que falte la generación de una parte de las estructuras válidas, lo que puede hacer que la etapa de prueba subsiguiente pierda la exploración de los comportamientos defectuosos del SUT. Las especificaciones subrestringidas pueden llevar a la generación de estructuras inválidas (es decir, estructuras que no cumplen con las restricciones previstas), lo que puede producir falsos negativos durante la prueba del SUT. Además, a veces el usuario tiene que tener en cuenta la forma en que opera el enfoque de generación y escribir las especificaciones de una manera muy específica, de manera que el enfoque pueda lograr un buen rendimiento [TODO ???]. Finalmente, tales especificaciones formales precisas rara vez están disponibles en el software, lo que dificulta la usabilidad de los enfoques BE de caja negra.
%

\section{Randoop: Generación aleatoria de tests guiada por retroalimentación}
\label{sec:feedback-directed-test-gen}

% \begin{lstlisting}[caption={Método generar pruebas},basicstyle=\ttfamily\scriptsize]
%     public Set<Test> generarPruebas(Set<Metodo> mds, int tLimite, int nPruebas) {
%         Set<Test> tests = new HashSet<>();
%         long inicio = System.currentTimeMillis();
        
%         while (System.currentTimeMillis() - inicio < tLimite 
%                && tests.size() < nPruebas) {
            
%             Metodo metSel = selMetodoAleatorio(mds);
%             List<Parametro> params = new ArrayList<>();
            
%             for (Parametro param : metSel.getParams()) {
%                 Tipo tipo = param.getTipo();
%                 Object valor;
                
%                 if (tipo.esPrimitivo()) {
%                     valor = genValorPrimitivo(tipo);
%                 } else {
%                     valor = genSecuenciaTests(tests, tipo);
%                 }
%                 params.add(new ParamInstanciado(param, valor));
%             }
%             tests.add(new Test(metSel, params));
%         }
%         return tests;
%     }
%     \end{lstlisting}
    
El testing es una técnica fundamental para mejorar la confiabilidad del software
y detectar errores en el código \cite{Ammann16}. 
Sin embargo, escribir los tests de manera manual es una tarea trabajosa, y
requiere un esfuerzo importante por parte de los desarrolladores. Es por esto
que en los últimos años la comunidad de la ingeniería de software ha puesto especial énfasis en el
el desarrollo de técnicas automáticas que permitan automatizar la generación de
tests \cite{Boyapati02,Khurshid01,Fraser11,Visser05,Pasareanu:2010,Cadar08,Tillmann:2010,Ma15,Ponzio:2016,Rosner14}.
Uno de los enfoques más importantes para esta tarea es la generación aleatoria
de tests \cite{Chen19,Li18,Ramler12,Shamshiri15}. 
%que explora el espacio de entrada del programa mediante la ejecución de combinaciones diversas de valores y secuencias de invocaciones. 
Dentro de estos enfoques, Randoop se destaca como una herramienta ampliamente
utilizada, debido a su capacidad para generar tests que han mostrado ser útiles para revelar
fallas en distintos tipos de software muy utilizados \cite{Pacheco07,Pacheco08,Shamshiri15,just2014mutants}

Randoop utiliza la API del SUT (por las siglas de \emph{software under test}) para generar secuencias de tests, es decir,
secuencias de invocaciones a métodos de la API. Randoop propone la técnica
de generación aleatoria con retroalimentación basada en la ejecución, que
consiste en ejecutar las secuencias de test y observar los resultados para 
determinar si se debe continuar extendiendo o no las secuencias de test. 
El algoritmo principal de Randoop se describe en la Figura \ref{fig:randoop-algorithm}.

\begin{figure}[H]
    \centering
    \begin{algorithm}[H]
        \SetAlgoLined
        \KwIn{Conjunto de métodos $M_1,\ldots,M_n$, tiempo límite $S$, número deseado de tests $W$}
        \KwOut{Tests de regresión}
        
        $prev \leftarrow \emptyset$\;
        $tests \leftarrow \emptyset$\;
        
        \While{tiempo transcurrido $< S$ \textbf{y} $|tests| < W$}{
            Seleccionar aleatoriamente $M(p_1:T_1, \ldots, p_m:T_m) \in \{M_1,\ldots,M_n\}$\;
            \For{$p_i:T_i$ de $M$}{
                \If{$T_i$ es primitivo}{
                    $S_i \leftarrow$ valor primitivo para $T_i$ tomado
                    aleatoriamente de las semillas\;
                }\Else{
                    $S_i \leftarrow$ secuencia aleatoria $\in prev$ que crea objeto de tipo $T_i$\;
                }
            }
            $test \leftarrow S_1; \ldots; S_m; M(v_1,\ldots,v_m)$\;
            $res \leftarrow ejecutar($test$)$\;
            \If{res = falla} {
                \Return{$\{ test \}$}\;
            }   
            \If{res = inválido} {
                // No se guarda test para futuras extensiones
            }       
            \If{res = exito} {
                $prev \leftarrow prev \cup \{test\}$\;
            }        
            $tests \leftarrow tests \cup \{test\}$\;
        }
        \Return{$tests$}\;
    \end{algorithm}
    \caption{Algoritmo de Randoop}
    \label{fig:randoop-algorithm}
\end{figure}


\textsf{Randoop} toma como entradas un conjunto de rutinas (por ejemplo, la API
de un módulo), y criterios de terminación para la generación de tests, como el tiempo
máximo de generación \emph{S}, y/o la cantidad máxima de tests a generar \emph{W}.
Como salida, \textsf{Randoop} produce tests de regresión para el módulo, que
ejercitan los métodos dados.

\textsf{Randoop} representa los tests como secuencias de invocaciones a 
rutinas. Estas se denominan \emph{secuencias de
test}. \textsf{Randoop} comienza con unos pocos valores ``semilla''
predeterminados para los tipos primitivos de Java, que se utilizarán para
instanciar los parámetros de tipos primitivos de las rutinas durante la
generación. El usuario también puede proporcionar valores semilla adicionales
para los tipos primitivos. 

Inicialmente, se inicializan dos conjuntos de secuencias de test (líneas 1 y 2). 
\emph{prev} almacena las secuencias de test generadas en iteraciones previas cuya
ejecución es exitosa, es decir, las secuencias de test que el algoritmo intentará
extender con métodos adicionales. \emph{tests} se utiliza para guardar los tests
de regresión que produce el algoritmo como resultado.

\textsf{Randoop} construye secuencias de test de manera
iterativa, hasta que se alcance alguno de los criterios de terminación
provistos por el usuario, en su ciclo principal (líneas 3-18). 
Cada iteración consiste en seleccionar aleatoriamente una rutina \texttt{M},
para usarla en la creación de una nueva secuencia de test (línea 4). 
Para lograr esto, \textsf{Randoop} debe seleccionar secuencias \texttt{S$_1$,..,S$_k$}
que le permitan crear valores (del tipo correcto) para instanciar los parámetros 
\texttt{p$_1$,..,p$_m$} de \texttt{M} (lineas 5-25). 
Si el parámetro \texttt{p$_i$} es de tipo primitivo, el algoritmo simplemente
crea una secuencia de test que produce un valor primitivo, eligiendo
aleatoriamente un valor entre las semillas del tipo (líneas 6-8). 
Por ejemplo, para \texttt{p$_i$} de tipo entero, 
\textsf{Randoop} puede crear la secuencia \texttt{S$_i$} definida como 
\texttt{int vi = -1;}.
En caso de que \texttt{p$_i$} sea de tipo referencia, el algoritmo elige
aleatoriamente una secuencia \texttt{S$_i$} que sirve para crear un objeto del
tipo de \texttt{p$_i$}, tomando \texttt{S$_i$} del conjunto de secuencias generadas 
previamente (\texttt{prev}) (líneas 9-11).

De esta manera, la nueva secuencia de test, \texttt{test}, se crea componiendo
secuencialmente las secuencias \texttt{S$_1$,..,S$_m$} y \texttt{M},
reemplazando los parámetros formales de \texttt{M} por las variables
correspondientes en \texttt{S$_1$,..,S$_m$} (línea 13). 

Las líneas 14-23 del algoritmo implementan la retroalimentación guiada por la ejecución
de las secuencias de test. La idea principal es ejecutar \texttt{test} (línea
14), y observar el comportamiento de la ejecución. Si \texttt{test} revela una
falla el SUT, \textsf{Randoop} termina su ejecución y retorna al usuario el test que
revela la falla (líneas 15-17). Esto sucede, por ejemplo, cuando el test lanza una excepción 
que se corresponde con una falla (\texttt{NullPointerException}), cuando viola un contrato implícito 
de Java, o una especificación dada por el usuario \cite{Pacheco07}.

En cambio, si el test lanza otros tipos de excepciones que no se corresponden
con fallas en el SUT, \textsf{Randoop} simplemente descarta el test, y no lo guarda para
extenderlo en iteraciones futuras del algoritmo (líneas 18-20). Ejemplos de
estos tipos de excepciones son las que representan un uso inválido de la API,
como las excepciones correspondientes a la violación de una
precondición (\texttt{IllegalStateException}).

Por último, si el test termina exitosamente (líneas 21-23), se lo guarda en el
conjunto \texttt{prev} de secuencias que usarán en el futuro para ser extendidas
y así generar nuevos tests. 

Notar que tanto los tests que lanzan excepciones (que no representan fallas) como los 
exitosos se guardan en el conjunto de tests de regresión creados por el
algoritmo (\texttt{tests} en la línea 24). Cuando el ciclo principal termina,
\textsf{Randoop} retorna los tests de regresión generados.

Las evaluaciones experimentales han encontrado que la generación aleatoria de tests dirigida 
por retroalimentación funciona significativamente mejor que la generación
aleatoria tradicional \cite{Pacheco07,Pacheco08}.
La principal ventaja de \textsf{Randoop} es su capacidad para generar
automáticamente una gran cantidad de casos de prueba muy rápidamente, y así 
detectar fallas que pueden pasar desapercibidas durante el testing
manual \cite{Pacheco07,Pacheco08}.

\pp{Poner ejemplo de ejecución de Randoop ilustrando el algoritmo acá abajo.}


Para ilustrar el funcionamiento de Randoop, consideremos la clase
\texttt{java.util.LinkedList}. El generador parte de un conjunto pequeño de
valores primitivos (por ejemplo, 0, 1, 2, $-1$, $-2$) y de los
constructores públicos disponibles. En una primera ejecución, Randoop
selecciona aleatoriamente el constructor \texttt{LinkedList()} y
ejecuta la secuencia:
\\
\begin{lstlisting}[language=Java,basicstyle=\ttfamily\small,columns=fullflexible,keepspaces=true]
LinkedList<Integer> l = new LinkedList<>();
\end{lstlisting}

Como la ejecución no produce fallos, la secuencia y el objeto resultante
se incorporan al \emph{pool} para futuras extensiones.

En una iteración posterior, Randoop elige al azar un método, por ejemplo,
\texttt{add(Object)}. Para obtener una lista sobre la cual invocar el
método, reutiliza la secuencia previa que construye \texttt{l}. Para el
parámetro, toma un entero disponible en el \emph{pool} (por ejemplo, 1).
La secuencia extendida queda:
\\
\begin{lstlisting}[language=Java,basicstyle=\ttfamily\small,columns=fullflexible,keepspaces=true]
LinkedList<Integer> l = new LinkedList<>();
l.add(1);
\end{lstlisting}

La ejecución es exitosa y la nueva secuencia pasa a estar disponible para
nuevas extensiones.

En otra iteración, Randoop genera una secuencia que incluye
\texttt{l.get(2)} sobre una lista de tamaño 1. La invocación arroja una
excepción por índice fuera de rango. Debido a su estrategia de
\emph{feedback-directed random testing}, Randoop descarta esa secuencia
fallida y deja de ampliarla. Solo las secuencias que ejecutan sin fallos
se siguen extendiendo, guiando así la exploración de las secuencias de la
API~\cite{Pacheco07,Pacheco08}.

\cacho{Ver si agregar algo mas }
%Una de las técnicas que presentamos en esta tesis, \textsf{BEAPI}, se basa en \textsf{Randoop}, pero en lugar de la generación aleatoria, tiene como objetivo generar \emph{todas} las secuencias de pruebas factibles (acotadas), dentro de un alcance dado. La retroalimentación de la ejecución permite a \textsf{BEAPI} descartar pruebas ilegales durante el proceso de generación. Además, al estar basado en una generación de pruebas dirigida por retroalimentación, \textsf{BEAPI} genera una secuencia de prueba para cada estructura en el conjunto exhaustivo acotado que construye y solo guarda los conjuntos de secuencias de llamadas a métodos que generan objetos válidos.
%

%Uno de los enfoques que presentamos en esta tesis, BEAPI, toma como base la idea de Randoop pero introduce una diferencia clave: en lugar de generar pruebas de manera aleatoria, 
%BEAPI busca generar todas las secuencias de prueba factibles dentro de un alcance acotado. Es decir, en lugar de depender de la aleatoriedad, 
%BEAPI explora sistemáticamente el espacio de búsqueda, asegurando una mayor cobertura.
%Además, BEAPI aprovecha la retroalimentación de la ejecución para descartar pruebas ilegales de manera eficiente, 
%reduciendo el número de combinaciones a evaluar. También genera una única secuencia de prueba para cada estructura válida, 
%evitando la repetición innecesaria de casos similares.



% \section{Algoritmos Perezosos? (Greedy)}
% \label{sec:greedyPrev}
% El algoritmo Greedy es una estrategia de resolución de problemas que sigue un enfoque voraz, tomando decisiones locales óptimas en cada etapa con la esperanza de llegar a una solución global óptima. A diferencia de otros enfoques, el algoritmo Greedy no realiza una búsqueda exhaustiva en el espacio de soluciones, sino que se centra en elegir la mejor opción disponible en cada momento.

% El enfoque Greedy se basa en la idea de que, al tomar decisiones óptimas en cada etapa, se puede obtener una solución aproximadamente óptima para el problema en general. Sin embargo, debido a su naturaleza voraz, el algoritmo Greedy puede no garantizar una solución óptima en todos los casos. En algunos casos, el algoritmo puede llegar a un óptimo local, pero no a un óptimo global. Por lo tanto, es importante tener en cuenta las limitaciones y restricciones del problema al aplicar el algoritmo Greedy.

% \begin{algorithm}[H]
%   \SetAlgoLined
%   \SetKwInOut{Input}{Input}
%   \SetKwInOut{Output}{Output}
%   \SetKwFunction{Greedy}{Greedy}

%   \caption{Greedy Algorithm}

%   \Input{Set $S$}
%   \Output{Solution $R$}

%   \BlankLine
%   $R \gets \emptyset$ 
%   $U \gets S$ \tcp*{Set of available elements}

%   \BlankLine
%   \While{$U \neq \emptyset$}{
%     $x \gets$ element selected according to Greedy criteria
    
%     $R \gets R \cup \{x\}$ \tcp*{Add $x$ to the solution}
%     $U \gets U \setminus \{x\}$ \tcp*{Remove $x$ from available elements}
%   }

%   \BlankLine
%   \Return $R$ 

% \end{algorithm}

% El algoritmo comienza con un conjunto de elementos disponibles y un conjunto solución vacío. En cada iteración, se selecciona el elemento óptimo según la función de heurística para los estados que se haya implementado y se agrega a la solución. Luego, se elimina ese elemento de los disponibles y se repite el proceso hasta que no queden elementos disponibles.

% Es importante tener en cuenta que el algoritmo Greedy no garantiza encontrar la solución óptima en todos los casos, ya que puede quedarse atrapado en mínimos locales. Sin embargo, en muchos casos, el enfoque Greedy proporciona soluciones rápidas y razonables.

% En esta tesis se personalizó el algoritmo para detectar subconjuntos óptimos de métodos \emph{builders}. 
% Para mas informacion sobre este tipo de algoritmos, invitamos al lector referirse a \cite{Cormen2009}, \cite{kleinberg2006}



\section{Criterios de Cobertura}
\label{sec:coverage}
La evaluación de técnicas de análisis automático de software (incluyendo las
técnicas de generación automática de tests), como las presentadas en esta tesis, requieren de métricas que cuantifiquen su capacidad para ejercitar el código y exponer defectos. 
Los criterios de cobertura tradicionales proveen formas relativamente sencillas de evaluar 
la calidad de una suite de tests, debido a que típicamente hay una alta correlación entre
la satisfacción de los criterios y la capacidad de detección de defectos de la
suite \cite{Ammann16,just2014mutants}. 
En particular, Just et al.~\cite{just2014mutants} muestran que existe una fuerte
correlación entre la detección de mutantes y la detección de fallas reales, lo
que respalda el uso de la mutación como sustituto práctico para evaluar la
eficacia de un conjunto de tests.

En esta sección describimos tres criterios fundamentales adoptados en esta tesis: 
(i) cobertura de código, considerando tanto líneas como ramas \cite{Ammann16,myzili2012coverage}, 
(ii) análisis de mutación \cite{jia2011analysis,Ammann16,just2014mutants}, 
y \cacho{Esto no me cierra:} (iii) la identificación de métodos observadores, propuesta en esta tesis como un criterio adicional
específico para APIs de estructuras de datos. 
Las dos primeras técnicas son ampliamente utilizadas en la literatura y proporcionan 
métricas objetivas para medir la calidad de los tests, mientras que la tercera constituye
una contribución original de este trabajo.

\subsection{Cobertura de Código}

La cobertura de líneas mide el porcentaje de
líneas de código del SUT que han sido ejecutadas por los tests. 
Esta técnica evalúa la exhaustividad de los tests al determinar si cada línea de código ha sido ejecutada al menos una vez. 
Sin embargo, la cobertura de líneas no mide si se han ejecutado todas las ramas
posibles en el programa, lo que puede limitar su capacidad para detectar fallas
en ciertos escenarios.

La cobertura de ramas mide qué porcentaje de las
decisiones lógicas de un programa ha sido ejercitado por los tests. En otras
palabras, se refiere a la medida en que se han ejecutado las ramas de decisión
(if-else, switch-case, ciclos, etc.) en el código fuente durante la ejecución de
los tests. La cobertura de ramas es especialmente útil para identificar áreas del 
código que no han sido ejercitadas y que podrían contener errores lógicos o comportamientos no deseados.
Al lograr una alta cobertura de ramas, se aumenta la confianza en la calidad de
los tests implementados, ya que se ha examinado exhaustivamente la lógica del programa 
en diferentes condiciones de ejecución.

Cabe destacar que los criterios anteriores no imponen ninguna obligación sobre
las aserciones que se deben escribir en los tests. Debido a esto, puede que los
tests no revelen una falla si no tienen las aserciones apropiadas, incluso
cuando cubren el defecto con la entrada adecuada.  

\subsection{Análisis de Mutación}

El análisis de mutación (mutation testing) es una técnica para medir la
capacidad de un conjunto de tests para detectar cambios sintácticos
introducidos deliberadamente en el código fuente. Consiste en la creación de versiones 
alteradas del código original, denominadas mutantes, mediante la realización de
cambios sintáticos pequeños (ej. reemplazar $>$ por $>=$). Cada mutante representa una 
posible variación o error que podría haberse introducido en el código original.
Una vez generados, se ejecutan los tests sobre los mutantes. 
Si algún test de la suite falla al ejecutar el mutante se considera que ha
\'matado\' al mutante. Esto significa que los tests son capaces de detectar el error 
introducido por el mutante. Un mutante sobrevive si ningún test de la suite 
detecta el cambio. Esto implica que es posible mejorar la suite para aumentar su capacidad 
para encontrar errores, por ejemplo, agregando un nuevo test que mate el mutante.

El puntaje de mutación de una suite (mutation score) se define como
el cociente entre la cantidad de mutantes muertos y el total de mutantes
generados. Así, un puntaje de mutación alto da una mayor confianza de que la suite 
de tests tiene una buena capacidad de revelar defectos en el SUT. 
Por otro lado, al identificar los mutantes que no son detectados por los tests 
se pueden identificar deficiencias en la suite, y se pueden tomar medidas 
correctivas para mejorar su calidad.

El análisis de mutación es uno de los criterios más fuertes para evaluar la 
calidad de los tests, ya que los mutantes generados por las operadores de 
mutación más usados 
tienen una alta correlación con las fallas que cometen los desarrolladores 
(incluso, una mayor correlación que los criterios anteriores) \cite{just2014mutants}. Sin
embargo, su costo computacional es elevado (requiere generar un
número usualmente grande de mutantes, y ejecutar los tests para todos ellos). 
Suele ser el criterio más trabajoso para el desarrollador, ya que usualmente 
se requiere un número relativamente grande de tests para lograr un buen puntaje 
de mutación (suele ser significativamente mayor que la cantidad de tests
    requeridos para lograr buena cobertura de código) \cite{papadakis2019}.

Las técnicas mencionadas anteriormente se aplican en esta tesis para evaluar las 
nuevas técnicas propuestas, y poder realizar una comparación con herramientas
relacionadas del estado de arte. Además, se utilizan como base para la
definición de algunas de las técnicas presentadas. Por ejemplo, la cobertura de
código se usa como parte de la función objetivo para el cómputo de métodos
generadores de objetos usando Randoop \ref{sec:fitnessRandoop}.


\section{Java PathFinder (JPF)}
\label{sec:jpf}

\emph{Java PathFinder} (JPF) es una herramienta de verificación automática de programas Java, que implementa técnicas con
exploración explícita del espacio de estados. Entre sus aplicaciones está el \emph{bounded model checking} (BMC), que
consiste en explorar sistemáticamente todas las ejecuciones del programa bajo entradas de tamaño acotado en busca de fallas
\cite{Visser05,Pasareanu:2010}. Los límites para el tamaño de las entradas (por ej., número máximo de nodos o rangos de primitivos)
son provistos por la persona usuaria. Cuando se detecta una falla, JPF produce un contrajemplo reproducible que facilita el
\emph{debugging}.\footnote{\url{https://github.com/javapathfinder/jpf-core}}

Para realizar verificación acotada, JPF se basa en \emph{drivers} (también llamados controladores): combinaciones de métodos que permiten construir todas las
entradas acotadas para ejecutar el código bajo análisis. En tipos de datos complejos (por ej., estructuras dinámicas en el \emph{heap}
como listas doblemente enlazadas) la construcción suele requerir varias rutinas de la API del módulo (ejemplo, \texttt{add},
\texttt{remove}, etc.), por lo que el diseño del \emph{driver} depende de dichas rutinas.

En módulos con muchas operaciones, seleccionar rutinas para el \emph{driver} no es trivial y puede impactar fuertemente en el
análisis posterior. Se busca un conjunto tan pequeño como sea posible cuyas combinaciones permitan construir todas las
estructuras acotadas del módulo. Incluir rutinas superfluas incrementa exponencialmente las combinaciones y degrada la
eficiencia. Un caso típico de rutina superflua son los métodos observadores (no modifican el estado) \cite{Huang:2012}. 
A la vez, el subconjunto debe ser suficiente para generar todas las instancias dentro del alcance. 
Esta selección suele hacerse manualmente, exige revisar toda la API y comprender su semántica, y se vuelve tediosa cuando
hay redundancia o muchas operaciones irrelevantes.

Las propiedades pueden expresarse con \emph{assert}; si la condición resulta falsa en una ejecución, JPF reporta un
contrajemplo. Por ejemplo, para cualquier lista de entrada \texttt{t}, insertar al inicio un entero \texttt{v} (\texttt{addFirst(v)}, con \texttt{v} en
\([0,b]\)) que no pertenezca a la lista, y luego eliminar el primer elemento (\texttt{removeFirst()}), debe dejar el mismo tamaño.

\begin{algorithm}[H]
  \caption{Propiedad a chequear}
  \label{alg:propiedad-a-chequear}
  $oldSize \gets t.\texttt{size()}$\;
  $value \gets \texttt{Verify.getInt}(0, b)$\;
  \texttt{Verify.ignoreIf}( $t.\texttt{contains}(value)$ )\;
  $t.\texttt{addFirst}(value)$\;
  $t.\texttt{removeFirst}()$\;
  \texttt{assert } $oldSize$ \texttt{ == } $t.\texttt{size}()$
  \texttt{ : "different size";}\;
\end{algorithm}


Para introducir elecciones dentro de un mismo programa, JPF ofrece primitivas de no determinismo, como:
\begin{itemize}
\item \texttt{Verify.ignoreIf(condicion)}: Evita explorar ejecuciones cuyo estado no
cumplen con el predicado condicion. En el programa anterior, permite ignorar todas las ejecuciones en las que la lista contiene el valor \emph{value} a insertar.
Esto es similar a agregar una precondición a la propiedad a verificar.
\item \texttt{int i = Verify.getInt(min,max)}: Esta construcción de JPF explora todas las posibles ejecuciones del programa que resultan de asignar 
a \emph{i} cualquier valor entre min y max. Esto significa que JPF ejecutará el código luego
de esta instrucción con \emph{i=min}, con \emph{i=min+1, ..., i=max}. Esto introduce no
determinismo, ya que en principio no hay garantías del orden en que se
asignarán valores a \emph{i} (aunque JPF tiene opciones para elejir un orden de
ejecución particular). 
\end{itemize}

Para analizar la propiedad del algortimo \ref{alg:propiedad-a-chequear} , JPF necesita mecanismos para generar
las listas del parámetro \texttt{t}. Los enteros se obtienen con
\texttt{Verify.getInt}, pero las estructuras requieren un
\emph{driver}. El objetivo es construir todas las listas de tamaño
máximo \(b\), con elementos en \([0,b]\) (verificación exhaustiva
acotada). El \emph{driver} selecciona una longitud
\(maxLength \in [0,b]\). Como cada método considerado agrega a lo sumo
un elemento, ejecutar a lo sumo \(b\) métodos produce listas de
tamaño \(\le b\). En cada iteración se elige no determinísticamente
una operación de la API y sus argumentos.

Si la persona usuaria no conoce los métodos generadores (y no desea seleccionarlos manualmente), la alternativa más segura es
usar todos los métodos del módulo, como se muestra en el algortimo \ref{lst:driverNCL}. En el
cuerpo del ciclo, a cada operación se le asigna un identificador y se elige no determinísticamente un valor en \([0, m)\) para
seleccionar la operación a ejecutar; por ejemplo, si
\texttt{methodN=2}, se ejecuta \texttt{clear()}. El número de ejecuciones crece exponencialmente con \(m\) (cantidad de
métodos): tras \(k\) pasos hay \(m^{k}\) secuencias potenciales.
\\
\begin{lstlisting}[language=Java,caption={Controlador con todos los métodos},label={lst:driverNCL},captionpos=b] 
private static NCL generateStructure(int scope) {
  int maxLength = Verify.getInt(0, scope);
  NCL t = new NCL();
  for (int i = 1; i <= maxLength; i++) {
    int n_methods = Verify.random(n_methods)
    switch (n_methods) {
      case 0:
        t.addFirst(Verify.getInt(0,scope),Verify.getInt(0,scope));
        break;
      case 1:
        t.removeFirst(Verify.getInt(0,scope));
        break;
      case 2:
        t.clear();
        break;
      case 3:
        t.containsValue(Verify.getInt(0,scope));
        break;
      // ...
      case 33:
        t.size(l);
        break;
      default:
        /* no-op */
    }
  }
  return t;
}
\end{lstlisting}


En esta tesis proponemos una técnica automática que, dada la API de un módulo, identifica un subconjunto suficiente y no superfluo de
métodos generadores (ver Capitulo \ref{cap:builders}). Esta técnica actúa como paso previo a la construcción del controlador
para BMC. Al exponer sólo los métodos generadores necesarios, reduce el factor de ramificación del controlador, disminuye el
espacio de búsqueda y mejora la eficiencia de la exploración, sin perder la capacidad de generar todas las instancias acotadas dentro
del \emph{scope}.
\chapter[Identificación de métodos builders]{Enfoques para la identificación de Métodos Builders}
\label{cap:builders}
\pp{Reemplazar builders por generadores de objetos o algo parecido}

\pp{Tenemos que hacer una meet para ponernos de acuerdo en como llamamos a los builders, armar un
 orden para la presentación de ideas, definir bien los conceptos y evitar
 mencionar cosas que no hay sido definidas.}

\pp{Lo primero que habría que hacer en esta sección es introducir la idea de
builders, y nunca se hace. El párrafo que sigue está fuera de contexto.}
El análisis de software es una tarea crucial en el campo de la Ingeniería de
Software, ya sea para generar tests para el software bajo test (SUT) o para
realizar verificación de modelos de software. Estos enfoques  requieren que los
programadores identifiquen manualmente un subconjunto de los métodos de un
módulo con el fin de dirigir el análisis \pp{Ojo, no hay muchos enfoques que requiran
builders}. En general, al analizar un módulo
\pp{los builders no se usan en general, quizás en algún análisis particular},
los programadores seleccionan un subconjunto de sus métodos que serán
considerados como constructores de objetos (\emph{builders}) para definir lo que
se conoce como un controlador o "driver", que se utilizará para construir
objetos automáticamente para el análisis, combinándolos de manera no
determinista, aleatoria, etc. Esto requiere una inspección minuciosa del módulo
y su estructura, ya que la exhaustividad relativa del análisis (omitir métodos
importantes puede evitar sistemáticamente la generación de diferentes objetos) y
su eficiencia (las diferentes combinaciones acotadas de métodos crecen de manera
exponencial a medida que aumenta el número de métodos) se ven afectados por la
selección \pp{acá estás asumiendo que el que lee sabe de builders, y todavía no
fueron introducidos}.

En este capitulo, proponemos tres enfoques para seleccionar automáticamente un
conjunto de constructores de objetos a partir de la API, basados en un algoritmo
evolutivo, un algoritmo greedy y un algoritmo de particionado en clases de equivalencias. Estos algoritmos favorecen conjuntos de métodos cuyas combinaciones conducen a la generación de conjuntos más grandes de objetos o a aquellos conjuntos que obtienen mejor cobertura de ramas del SUT. Además, los algoritmos tienen en cuenta otras características sobre estos conjuntos de métodos, tratando de priorizar la selección de aquellos con menor cantidad de métodos y con más simples parámetros.

% Para evaluar experimentalmente nuestra propuesta, hemos realizado una evaluación en un conjunto de clases con estado de referencia, que representan casos de uso típicos. Los resultados muestran que nuestro enfoque puede identificar automáticamente conjuntos de constructores de objetos que son suficientes (se pueden utilizar para crear cualquier instancia del módulo) y mínimos (no contienen métodos superfluos), en un tiempo razonable.

% En resumen, en este trabajo abordamos el desafío de seleccionar automáticamente un conjunto óptimo de constructores de objetos a partir de la API de un módulo, con el objetivo de mejorar tanto la exhaustividad como la eficiencia del análisis de software. Nuestros resultados experimentales demuestran la viabilidad y efectividad de nuestra propuesta, lo que abre nuevas perspectivas para la automatización de la selección de constructores de objetos en el análisis de software.


\section{Motivacion}
\label{sec:motivacion}
\begin{table}[H]
\center
{\scriptsize
\begin{tabular}{|l|l|l|l|}
\hline
No. &Return type & Method name & Obs? \\
\hline
    0 && NCL() & no \\
    1& & NCL(int) & no \\
    2&& NCL(Collection) & no \\
    3&boolean & add(Object) & no \\
    4&void&add(int,Object) & no \\
    5&boolean&addAll(Collection) & no\\
    6&boolean&addAll(int,Collection) & no \\
    7&boolean&addFirst(Object) & no \\
    8&boolean&addLast(Object) & no\\
    9&void&clear() & no\\
    10&boolean&contains(Object) & yes \\
    11&boolean&containsAll(Collection) & yes \\
    12&boolean&equals(Object) & yes \\
    13&Object&get(int) & yes\\
    14&Object&getFirst() &yes \\
    15&Object&getLast() & yes\\
    16&int&indexOf(Object) &yes\\
    17&boolean&isEmpty() & yes\\
    18&Iterator&iterator() & no\\
    19&int&lastIndexOf(Object) &yes \\
    20&ListIterator&listIterator() &no \\
    21&ListIterator&listIterator(int) & no\\
    22&Object&remove(int) &no\\
    23&boolean&remove(Object) & no \\
    24&boolean&removeAll(Collection) & no \\
    25&Object&removeFirst() &no\\
    26&Object&removeLast() &no\\
    27&boolean&retainAll(Collection) &no \\
    28&Object&set(int,Object) &no\\
    29&int&size() &yes\\
    30&List&subList(int,int) & no \\
    31&Object[]&toArray() & yes \\
    32&Object[]&toArray(Object[]) &yes\\
    33&String&toString() & yes \\
\hline
\end{tabular}
}
\caption{Apache's NodeCachingLinkedList API}
\label{tab:ncl-api}
\end{table} 
En esta sección, motivamos nuestro enfoque mediante un ejemplo práctico. La estructura de datos NodeCachingLinkedList (NCL) de Apache \cite{apache} consta de una lista principal circular doblemente enlazada  que contiene los elementos de la colección y una lista secundaria simplemente enlazada que actúa como caché para los nodos que se han eliminado de la lista principal. Los nodos almacenados en la caché pueden ser reutilizados y añadidos de nuevo a la lista principal al insertar elementos en ella. Gracias a su caché, en las aplicaciones en las que las inserciones y eliminaciones de la lista son muy frecuentes, NCL puede reducir significativamente la sobrecarga necesaria para la asignación de memoria y la recolección de basura de los nodos. 
Como ilustración, la Figura~\ref{fig:ncl-instances} muestra las tres instancias de NCL que se pueden construir con exactamente dos nodos.


\begin{figure}[H]
    \centering
    \includegraphics[width=0.85\textwidth]{NCL-instances.png}
    \caption{Tres instancias de NodeCachingLinkedList con exactamente dos nodos}
    \label{fig:ncl-instances}
\end{figure}


NCL tiene una API muy completa, como se muestra en la Tabla~\ref{tab:ncl-api}.
Sin embargo, para construir cualquier objeto de NCL, sólo se necesitan algunos
métodos de la API. Por ejemplo, las combinaciones de los métodos en la
Figura~\ref{fig:NCLbuilders}, cuando se instancian con los parámetros
apropiados, se pueden utilizar para construir cualquier objeto de NCL acotado
con el scope deseado \pp{Acá aparece scope y nunca hablaste de scope todavía}.
Por lo tanto, los siguientes métodos \pp{Los metiste en una figura, y venís
hablando de ellos, esto no se lee bien}:
\\
\begin{lstlisting}[numbers=none,label=fig:NCLbuilders, caption=Conjunto de metodos sufiente para NCL]
  (0)  NodeCachingLinkedList()
  (7)  addFirst(Object)
  (25) removeFirst()
\end{lstlisting}

 son un ejemplo de un conjunto suficiente de constructores (\emph{builders}) de
 NCL \pp{Nunca definiste lo que es un conjunto suficiente, ni builders}.
 Hay que tener en cuenta que, después de utilizar el método constructor \pp{hay
     que buscarle la vuelta para nombrar a builders algo que no sea
 constructores, porque sino se confunde con los constructores de la clase. Para
 mi métodos generadores de objetos suena bien}, la
 lista principal de NCL se puede completar simplemente utilizando el método
 \texttt{addFirst}. Sin embargo, si queremos generar instancias en las que la
 lista caché no esté vacía, podemos hacerlo a través del método
 \texttt{removeFirst}, como sugiere el conjunto suficiente de constructores.
 Para la mayoría de las técnicas de análisis automáticas \pp{no son la mayoría},
 nos gustaría considerar tantos escenarios (entradas) variables como sea
 posible, de ahí la motivación para construir estas con un un conjunto
 suficientes de métodos de la API \pp{acá hay un salto a generación de entradas
 que no se entiende}. Además, los metodos \emph{builders} de la
 figura~\ref{fig:NCLbuilders} también son mínimos \pp{no definido}, ya que la falta de alguno de
 ellos implicaría que algunos objetos de NCL ya no se pueden construir bajo el
 scope deseado \pp{no definido}.

Hay que tener en cuenta que puede haber muchos conjuntos de constructores suficientes \pp{no definido}. Por ejemplo, se pueden obtener constructores suficientes reemplazando el método \texttt{addFirst} en la Figura~\ref{fig:NCLbuilders} por cualquier otra variante de \texttt{add} que se muestra en la Figura~\ref{fig:NCLadds}, ya que para cualquier manera de llenar la lista principal de NCL con \texttt{addFirst}, existe una forma diferente de construir el mismo objeto utilizando otra variante de \texttt{add} (quizás invocada con diferentes parámetros y cambiando el orden de ejecución).
\\
\begin{lstlisting}[numbers=none,label=fig:NCLadds, caption=Variantes del método 'Add' que puedo ser utilizado para rellanar la lista principal en NCL, captionpos=b, frame=tb , basicstyle=\scriptsize]
  (3) add(Object)
  (4) add(int,Object)
  (7) addFirst(Object)
  (8) addLast(Object)
\end{lstlisting}

Además, es importante remarcar la importancia de obtener subconjuntos que sean
mínimos\pp{acá aparece recién la definición de minímos}, lo que significa que 
no contengan métodos adicionales \pp{qué signfica esto?}. Esto es crucial para
utilizar la combinación de estos métodos en la construcción de objetos. Cuantos
más métodos tenga el subconjunto de builders, más costosa será la combinación de
los mismos para generar los objetos \pp{no está definido como se piensa generar
los objetos}.

Los siguientes subconjuntos son suficientes pero no son mínimos para NCL:
\\

\begin{lstlisting}[numbers=none,label=fig:NCLnoMin, frame=tb , basicstyle=\scriptsize]
  (0)  NodeCachingLinkedList()
  (7)  addFirst(Object)
  (3)  addF(Object)
  (25) removeFirst()
\end{lstlisting}

\begin{lstlisting}[numbers=none,label=fig:NCLnoMin1, caption= Conjuntos de metodos builders suficientes pero no mínimos, captionpos=b, frame=tb , basicstyle=\scriptsize]
  (0)  NodeCachingLinkedList()
  (7)  addFirst(Object)
  (3)  remove(Object)
  (25) removeFirst()
\end{lstlisting}

También observamos que cuanto más simples sean los parámetros de una rutina, más
fácil será utilizarla para generar entradas en el contexto de un análisis de
programas. Por ejemplo, entre las alternativas de rutinas para insertar datos en
la lista principal de NCL (Figura.~\ref{fig:NCLadds}), \texttt{add(int,Object)}
recibe más parámetros que los otros tres métodos, por lo tanto es más difícil
generar parámetros para brindale a esta rutina a la hora de generar entradas
\pp{No está definido como se generarían entradas}. Esto hace que las otras tres alternativas tengan mas preferencia sobre esta. Así, nuestro enfoque tiene en cuenta el número de parámetros y sus complejidades para seleccionar los constructores mejores posibles.

Muchos métodos en la tabla \ref{tab:ncl-api} están marcados como observadores
(columna Obs?), lo que significa que no modifican los objetos en los que operan,
ni son útiles para crear objetos. Por lo tanto, los observadores siempre son
superfluos y nunca deben incluirse en un conjunto de constructores mínimos
\pp{no definido}. Nuestro enfoque intenta reconocerlos de antemano y
descartarlos de la búsqueda para reducir significativamente el espacio de
búsqueda \pp{Buscaría hablar del concepto de builders antes que de los enfoques}.

Todos estos problemas planteados en este capítulo fueron lo que nos motivó a desarrollar enfoques para obtener automáticamente métodos constructores de manera eficiente y mínima. Para lograrlo, utilizamos una variedad de técnicas y algoritmos que explicaremos en la próxima sección. 


\section{Algoritmos}


\label{sec:algorithms}
Para encontrar un conjunto suficiente \pp{y mínimo?} de constructores a partir
de una API de un programa, diseñamos tres algoritmos de búsqueda que se
describen a continuación. El primero es una modificación de los algoritmos
genéticos \pp{un algoritmo genético?}, el segundo es un algoritmo \emph{greedy}
más precisamente una variante de un algoritmo \emph{Hill Climbing}, y el tercero
es un algoritmo de búsqueda de acuerdo a las \emph{clases de equivalencias}  de
los subconjuntos respecto al valor de su función de valoración \pp{no definida}.
La función de evaluación, también conocida como función objetivo o función de
fitness, desempeña un papel crucial en los algoritmos de búsqueda
\pp{informada}. Esta función asigna un valor numérico a cada solución candidata
en función de su calidad o idoneidad \pp{idoneidad no se usa} para resolver el
problema en cuestión. El propósito de la función de evaluación es proporcionar
una medida cuantitativa de la calidad de cada solución en relación con los
objetivos y restricciones del problema \pp{no son estándar estas definiciones,
hay que buscar mejores}. Dependiendo del tipo de problema, la función de
evaluación puede buscar maximizar o minimizar ciertas variables, satisfacer
restricciones específicas o lograr un equilibrio entre múltiples criterios
. Es fundamental diseñar una función de evaluación adecuada que capture las
características esenciales del problema y que permita seleccionar y guiar de
manera efectiva la búsqueda hacia soluciones de alta calidad. Una función de evaluación bien
definida puede influir en la eficiencia y efectividad de los algoritmos de
búsqueda, ya que determina qué soluciones son consideradas prometedoras y
merecen ser exploradas en mayor detalle \pp{van como 5
definiciones distintas de función de fitness, ninguna muy acertada}. Sobre las funciones de evaluacion que hemos implementado lo explicaremos en la seccion \ref{sec:fitness}


Los métodos observadores son aquellos que no modifican el estado de un objeto al
ejecutarse. En el ejemplo motivador presentado en la sección
\ref{sec:motivacion} y la tabla \ref{tab:ncl-api}, se puede observar que existen
muchos métodos clasificados como observadores. Esto nos indica que estos métodos
no son relevantes para ser considerados como métodos builders. Antes de ejecutar
cualquiera de nuestros algoritmos, aplicamos una técnica de análisis estático
para eliminar de la búsqueda aquellos métodos que son irrelevantes para nuestros
métodos builders \pp{castellano}. En otras palabras, necesitamos identificar y eliminar aquellos métodos que son considerados observadores en la API.
Para realizar esta identificación de métodos observadores, utilizamos una
herramienta de análisis estático llamada
\emph{Infer}\footnote{https://fbinfer.com/}. Esta herramienta ha sido
desarrollada por \emph{Facebook} y ofrece varias funcionalidades. En el contexto
de nuestra tesis, utilizamos únicamente su analizador estático para detectar qué
métodos son considerados puros (observadores) \pp{falta cita a paper de
métodos puros}.

El análisis estático realizado con \emph{Infer} nos permite obtener una lista de
métodos que son clasificados como observadores en la API bajo prueba. Esta
información es utilizada como paso previo a la ejecución de nuestros algoritmos,
para asegurarnos de que solo consideramos los métodos relevantes y adecuados
para comenzar la búsqueda de los métodos \emph{builders}. \pp{este párrafo es
una repetición de lo que dice el anterior}

\subsection{Algoritmo Genético}
\label{alg:approachGA}

\pp{hay varios errores técnicos en esta sección que podemos charlar. Hay que
    mirar la explicación de algún libro y tratar de mejorar las explicaciones}

En esta sección presentamos los detalles para la detección de los subconjuntos
de métodos builders utilizando un algoritmo evolutivo. Para lograr realizar
esto, implementamos un algoritmo genético (\ref{sec:geneticoPrev}) que busca el
subconjunto de métodos que sean mínimo y suficientes \pp{constructores?}, que describiremos a continuación.


% % basados en una estrategia de escalada de colinas \cite{Russell:2009}.
% Durante esta sección explicaré en detalle cada algoritmo.


% Dare mas detalles de este algoritmo en la seccion {TODO!!!!}.


\subsubsection{Cromosomas}
\label{ge:cromosomas}

Los elementos de la población de nuestro algoritmo genético son subconjuntos de métodos de la API. Para representar esto, necesitamos codificarlos como \emph{cromosomas}. En nuestro caso, necesitamos alguna forma de representar los métodos de la clase como un vector. Por ejemplo, consideremos el ejemplo de NCL explicado en el capítulo anterior, que tiene 34 métodos (consultar Tabla \ref{tab:ncl-api}). Para representar nuestras posibles soluciones, crearemos un vector de genes booleanos, donde cada posición $i$ es verdadera si y solo si el cromosoma contiene el $i$-ésimo método de la API. Para hacer esto, enumeraremos los métodos de la API desde 0 hasta $n$, donde $n$ es el número total de métodos en la API. Cabe destacar que cada cromosoma tendrá la misma longitud, lo cual es común en los algoritmos genéticos.

Para el ejemplo de NCL, que tiene 34 métodos (Tabla \ref{tab:ncl-api}), tendríamos la siguiente representación del cromosoma:

\begin{center}
$c = \begin{array}{@{}cccccc@{}}
\scriptsize g_0 & \scriptsize g_1 & \scriptsize g_2 & \ldots & \scriptsize g_{34}
\end{array}$
\end{center}
\pp{este ejemplo quedó de más con el de abajo me parece, no aclara mucho así
como está}

Aquí, cada gen (\emph{g}) es un valor booleano que indica la presencia o
ausencia del método en ese cromosoma. \pp{está dicho arriba esto}

Como ejemplo más concreto, consideremos la representación del cromosoma que representa el conjunto de métodos suficientes y mínimos para NCL, como se muestra en la Figura \ref{fig:NCLbuilders}:
\begin{center}
$c = \begin{array}{|*{8}{c|}}
\hline
0 & 1 & \ldots & 7 & \ldots & 25 & \ldots & 33 \\
\hline
1 & 0 & \ldots & 1 & \ldots & 1 & \ldots & 0 \\
\hline
\end{array}$
\end{center}
\pp{los índices no son parte del cromosoma y deberían ir abajo, como está el dibujo parece una
matriz, y no coincide el formato con el dibujo anterior}

En este caso, las posiciones 0, 7 y 25 están establecidas como verdaderas (según el orden asignado a los métodos de la API), mientras que las demás posiciones están establecidas como falsas.


\subsubsection{Poblacion Inicial \pp{por qué solo la inicial?}}

El proceso comienza con un conjunto de individuos llamado Población. Recordemos
que cada individuo es una solución \pp{los individuos no son soluciones} al problema que deseas resolver.

En nuestro algoritmo genético, la población inicial se crea generando
aleatoriamente un conjunto de cromosomas, donde cada cromosoma representa una
solución potencial al problema \pp{los individuos no son soluciones}. Cada gen
en el cromosoma se establece aleatoriamente como 1 o 0, lo que indica la
presencia o ausencia del método que se codificó en esa posición \pp{repetición}.
Además, todos nuestros cromosomas iniciales tienen un tamaño igual a la cantidad
de métodos que tiene la API bajo test \pp{repetición}.

Es importante destacar que el número de cromosomas en la población inicial
desempeña un papel fundamental en el algoritmo genético \pp{solo en la
inicial?}, ya que proporciona la base para la evolución y mejora gradual de las
soluciones\pp{no son soluciones} a lo largo de las generaciones. En nuestro algoritmo, tenemos una
población inicial de 100 individuos \pp{por qué 100? No se debería buscar
experimentalmente?}. A medida que el algoritmo avanza, se
aplican operadores genéticos como la selección, el cruce \pp{mala traducción} y
la mutación para crear nuevas soluciones \pp{no son soluciones} a partir de la población actual. A continuación, se explican estos operadores en detalle.

\subsubsection{Operadores Geneticos \pp{Noté algunos errores ortográficos, usar un spell checker}}

A continuación explicaremos los operadores genéticos principales utilizados en
el algoritmo evolutivo. Cada uno desempeña un papel importante en la exploración
y explotación \pp{explotación?} del espacio de búsqueda y en la mejora de la
calidad de las soluciones \pp{no son soluciones}a lo largo de las generaciones
\pp{no definido}.

\paragraph{Cross-over (Cruce\pp{mala traducción})}

El cruzamiento (o \emph{cross-over} en inglés) es un operador genético que
combina una parte de dos cromosomas $c1$ y $c2$ elegidos aleatoriamente para
crear un nuevo cromosoma descendiente \pp{no se eligen aletoriamente, se eligen
con el procedimiento de selección. Selección debería explicarse antes}. Esa parte que toma de cada cromosoma padre
depende de la configuración con la que se lo setee. En nuestro algoritmo,
utilizamos un ratio de 0.30\pp{No se debería buscar
experimentalmente? Hay una mezcla entre la definición de algoritmos genéticos
en general y la implementación particular}. Además, un cruzamiento de dos puntos \pp{no
definido} de cruce dentro de los cromosomas de los padres. Su objetivo principal es explorar y explotar la información genética existente en la población.
El cruce de varios puntos permite una mayor variedad en los descendientes
generados, ya que se intercambian segmentos más largos de genes entre los
padres. Esto puede ayudar a explorar el espacio de búsqueda de soluciones de
manera más efectiva y encontrar soluciones potencialmente mejores \pp{errores
técnicos}.
\pp{un ejemplo hubiera estado bueno}

\paragraph{Mutacion\pp{spell checker}}

La mutación es un operador fundamental en un algoritmo genético que introduce
pequeños cambios aleatorios en los cromosomas para explorar nuevas soluciones
\pp{no son soluciones} y mantener la diversidad dentro de la población. Juega un
papel crucial en la prevención de la convergencia prematura \pp{no me convence}
y garantiza que el algoritmo explore de manera efectiva el espacio de búsqueda.
Existen varios operadores de mutación que se pueden utilizar en un algoritmo
genético. Uno de los operadores de mutación más comúnmente utilizados es la
Mutación por Inversión de Bits \pp{por qué mayúsculas?}, que es adecuada para algoritmos genéticos codificados en binario como en el caso de nuestra codificación.
En la Mutación por Inversión de Bits, se selecciona uno o más bits aleatorios en
el cromosoma y se invierten. Si un bit está inicialmente establecido en 1, se
cambia a 0, y viceversa. Esta alteración aleatoria introduce cambios sutiles en
la información del cromosoma, creando potencialmente nuevas soluciones \pp{no son soluciones}que no
estaban presentes en la población original. El número de bits invertidos y sus
posiciones se determinan típicamente mediante una baja probabilidad de mutación
\pp{en realidad se hace experimentalmente}.
\pp{un ejemplo hubiera estado bueno}


\paragraph{Selección}
La operación de selección es un componente clave en nuestro algoritmo genético,
ya que determina qué individuos serán preservados en la siguiente generación
\pp{error técnico}. En nuestro enfoque, hemos desarrollado un operador de
selección que se basa en el nivel de aptitud de cada individuo \pp{no
desarrollamos ningún operador nuevo}.

Utilizamos un enfoque de selección tipo torneo (\emph{tournament}) con un tamaño
de torneo de 4 en nuestro algoritmo. En este enfoque, se selecciona
aleatoriamente un grupo de individuos de la población y se \pp{los pone a
competir} compite entre ellos. El individuo con el mejor valor de fitness en el
torneo es seleccionado para formar parte de la próxima generación \pp{error técnico}. Este proceso se repite hasta que se hayan seleccionado todos los individuos necesarios\pp{error técnico}.

Es importante destacar que el operador de selección basado en torneo tiene varias ventajas. No solo permite seleccionar a los individuos más aptos\pp{error técnico}, sino que también introduce una presión de selección ajustable. Esto significa que los individuos más débiles tienen menos posibilidades de ser seleccionados en torneos más grandes, lo que ayuda a mejorar gradualmente la calidad de la población a lo largo de las generaciones.


El siguiente es el algoritmo genético que utilizamos para nuestra búsqueda
\\
\cacho{Algortimo? esta en preliminares, es standard para Genetico. Pongo los valores que utilizamos en cada parametros aca? Lo tengo en evaluacion.}
\\
%Pongo ejemplo de codigo???%
\begin{algorithm}[H]

\SetAlgoLined
\KwResult{Best solution}
Inicializar población inicial $P$ de cromosomas aleatorios\;

Evaluar la aptitud de cada cromosoma en $P$\;

\While{Criterio de terminación no se cumple}{ 
    Seleccionar cromosomas padres de $P$ para la reproducción\;
    Aplicar operador de cruce para generar descendencia\;
    Aplicar operador de mutación a la descendencia\;
    Evaluar la aptitud de los nuevos cromosomas\;
    % Reemplazar cromosomas menos aptos en $P$ por la descendencia\;
}
\caption{Algoritmo Genético}
\end{algorithm}
\pp{se queda muy corto este pseudocódigo, no explica casi nada.}
\pp{En este punto todavía no se sabe cuál es la función de fitness. No se
entiende que hace el algoritmo en el contexto de identificación de builders.}
\cacho{Explicar algortimo}

\subsection{Hill Climbing}
\label{alg:approachHC}
El algoritmo Hill Climbing, también conocido como búsqueda por ascenso de
colina\cite{Russell:2009}, es un algoritmo de búsqueda local que se utiliza para
optimizar una función de valoracion. El objetivo del algoritmo es encontrar la
solución óptima \pp{no es óptima} para un problema determinado.  Su enfoque se
basa en realizar movimientos ascendentes, es decir, buscar soluciones que
mejoren continuamente el valor objetivo o la función de evaluación \pp{o?}.
Este algoritmo es un algoritmo Greedy(Perezoso) explicado en la seccion de
preliminares, (\ref{sec:greedyPrev})\pp{no hay mucho que explicar en
preliminares sobre hill climbing, ya está todo acá por lo que veo}. Este selecciona un buen estado sucesor
para el momento actual, sin pensar en dónde ir a continuación \pp{no se entiende}. 
La idea principal detrás del algoritmo Hill Climbing es comenzar con una solución inicial y, en cada iteración, realizar un movimiento hacia una solución vecina que mejore el valor objetivo. Este proceso se repite hasta que no se pueda encontrar una solución vecina que mejore el valor actual. En ese punto, el algoritmo se detiene y devuelve la mejor solución encontrada hasta el momento.
Es importante destacar que el algoritmo Hill Climbing puede quedar atrapado en óptimos locales, es decir, puede converger hacia soluciones que son mejores en comparación con sus vecinos inmediatos, pero no son óptimas en el contexto global

La representación que hemos utilizado del problema fue igual a la que utilizamos para representar cromosomas en el Algoritmo genético explicado en la sección anterior. Esto quiere decir que utilizamos vectores de 0 y 1 para representar una posible solución. 

A continuación mostraremos un pseudocódigo del algoritmo \emph{Hill Climbing} que hemos implementado:
\pp{En este punto no se sabe cuál es la función de fitness, entonces no se
entiende que significa el algoritmo en el contexto de identificación de builders.}

\begin{algorithm}[H]
  \caption{Algoritmo de Hill Climbing}
  \label{algo:hill_climbing}
  \SetAlgoLined
  \KwResult{Solución óptima $curr$}
  $curr \gets c$\; 
  
  \While{existe un mejor candidato}{
    $S \gets$ GenerarSucesores($curr$)\;
    
    $best \gets$ SeleccionarMejorSucesor($S$)\;
    
    \If{$f(best) > f(curr)$}{
      $curr \gets best$\;
    }
  }
  \Return{$best$}\;
\end{algorithm}

Este algoritmo representa el esquema básico de Hill Climbing, comienza
calculando la funcion de valoración de todos los singletones ${c}$ de métodos
constructores.  El mejor de los singletones (mayor objectos puedo crear con ese
constructor) se establece como el candidato actual $curr$, y Hill Climbing
inicia un proceso de búsqueda típico e iterativo (Línea 1) \pp{Esto que se
explica acá no está en el algoritmo, hay una asignación nomás}.

En cada iteración (Línea 2 a 8), \emph{Hill Climbing} calcula $f(succ)$ para
cada succ $\in$ $GenerarSucesores(curr)$. El método que genera los sucesores nos
devuelve un conjunto de posibles soluciones que se crean a partir de sumarle un
método a la solución óptima corriente ($curr$). Es decir, los sucesores  $S$
generados con \emph{GenerarSucesores(curr)} de un candidato $curr$ son los
conjuntos {$curr\cup{mi}$}, para cada $mi$ $\in$ API (Línea 3) \pp{
Se podría poner el pseudocódigo de generar sucesores, y quizás poner un ejemplo
para que se entienda}.
Sea $best$ el sucesor con el valor de valoración más alto. Observe que $best$
tiene exactamente un método más que el mejor candidato de la iteración anterior,
$curr$ (Línea 4) \pp{hay que implementar con código el existe un mejor
candidato. Mirar el libro de AI.}.

Si $f(best) > f(curr)$, los métodos en $best$ se pueden utilizar para crear un
conjunto más grande de estructuras que los en $curr$ (Línea 5 a 7) \pp{f no está
definida}. Por lo tanto, \emph{Hill Climbing} asigna $best$ a $cur$r y continúa con la siguiente iteración. En cambio, si $f(best) <= f(curr)$, $curr$ ya genera el conjunto más grande de estructuras posible (no se puede agregar ningún método que aumente el número de estructuras generadas a partir de $curr$). En este punto, $curr$ se devuelve como el conjunto de constructores identificados. (Linea 9) 

% \cacho{
% En este algoritmo no se necesitan setear ningun parametros para guiar la busqueda a que sea mas efectiva}

Se puede observar que este algoritmo puede quedar atrapado en un máximo local y
no generar alguna combinación específica que podría ser aún mejor. Aquí se puede
ver que es un algoritmo greedy, ya que obtiene una solución rápida pero puede no
ser eficiente \pp{eficaz?} en encontrar la mejor solución global. 


\subsection{Clases de equivalencia}
\label{alg:approachCE}
 Las clases de equivalencia son una técnica utilizada para agrupar conjuntos de datos de entrada en categorías o clases que tienen un comportamiento similar o producen resultados equivalentes. Esta técnica es ampliamente utilizada en el diseño y la realización de pruebas de software.
 En términos generales, una clase de equivalencia representa  un conjunto de que se espera que produzcan resultados idénticos. La idea es que si un conjunto de la clase de equivalencia produce un resultado entonces todas las demás entradas de esa clase deberían producir el mismo resultado.
 Al trabajar con clases de equivalencia, se selecciona una entrada representativa, llamada caso de prueba, de cada clase para ser evaluada. En lugar de probar todas las posibles entradas, se eligen casos de prueba que representen cada clase de equivalencia para minimizar la cantidad total de pruebas necesarias.

 Bajo esta introducción, hemos desarrollado un tercer algoritmo donde agrupamos en clases de equivalencia aquellos subconjuntos de métodos que tengan el mismo valor de valoración con alguna de las funciones explicadas en \ref{sec:fitness}.


\begin{algorithm}[H]
  \caption{Algoritmo basado en Clases de Equivalencia}
  \label{algo:clases_equivalencia}
  \SetAlgoLined
  \KwResult{Conjunto de metodos builders $best$}
  $curr \gets c$\; 
  $equivalenceClasses \gets$ CrearClasesDeEquivalencia($curr$)\;
  
  \While{se ha creado una nueva clase de equivalencia}{
    $newCandidates \gets$ CandidatosPorClase($equivalenceClasses$)\;
    
    \ForEach{$candidate$ en $newCandidates$}{
      $successors \gets$ GenerarSucesores($candidate$)\;   
      
      \ForEach{$successor$ en $successors$}{
        $key \gets$ f($successor$)\;
        $equivalenceClasses[key]$.put($successor$)\;
      }
    }
        % $best \gets$ SeleccionarMejorCandidato($candidates$)\;
  }
  
  $result \gets$ obtenerMejor($best$)\;
  \Return{$result$}\;
\end{algorithm}

En este algoritmo, se comienza obteniendo los conjuntos singletons (Línea 1), que son los métodos constructores individuales de la misma manera que el algoritmo de \ref{algo:hill_climbing}. A partir de estos singletons, se selecciona el mejor candidato inicial $curr$, que servirá como punto de partida. Luego, se crean las clases de equivalencia basadas en la función de valoración y agrupando los candidatos que tienen el mismo valor de valoración. En el algortimo las clases de equivlencias estan guardadas en $equivalenceClasses$ (Línea 2).

A continuación, se itera (Línea 5 a 12) mientras se haya creado una nueva clase de equivalencia. En cada iteración, se generan nuevos candidatos por cada clase de equivalencia. Es decir, se selecciona un candidato de cada clase de equivalencia (el de menor métodos y menor parámetros) $CandidatosPorClase(equivalenceClasses)$ y se lo guarda en un cola ($newCandidates$) (Línea 5). A continuación, se generan los sucesores (línea 6) para cada candidato elegido, $GenerarSucesores(candidate)$, utilizando el mismo enfoque que en el algoritmo de \emph{Hill Climbing}. Estos se guardan en $successors$ Es decir, si tenemos $N$ clases de equivalencias vamos a tener $N$ candidatos a los cuales le vamos a calcular sus sucesores.  Luego, por cada uno de estos sucesores en  se le calcula su función de valoración y de acuerdo a esta se lo guarda en la clase de equivalencia representada por el valor de valoración.
El algoritmo itera hasta que no hayamos creado una nueva clase de equivalencia. Esto quiere decir que no hay cambios y hemos explorado todas las alternativas posibles. 
Luego vamos a obtener el mejor conjunto de métodos builders, accediendo a la clase de equivalencia con mayor key y obteniendo el representante de ese conjunto que tenga menor métodos y parámetros. Esto lo realiza el método $obtenerMejor(equivalenceClasses)$



\section{Función de Valoración}
\label{sec:fitness}


En la búsqueda de soluciones óptimas mediante algoritmos genéticos u otros algoritmos de búsqueda, la función de valoración \cite{goldberg1989genetic} desempeña un papel fundamental. Su diseño adecuado es crucial, ya que determina la dirección y el éxito del proceso de búsqueda.
La función de valoración evalúa la calidad de cada solución candidata y la compara con otras soluciones en la población. Proporciona una medida objetiva de la idoneidad de cada individuo en relación con los objetivos y restricciones del problema. La calidad se expresa generalmente mediante un valor numérico, donde valores más altos indican soluciones más deseables. Por ejemplo, si estamos resolviendo un problema de optimización en el que buscamos maximizar una función objetivo, la función de valoración puede asignar un valor más alto a las soluciones que se acercan más a la solución óptima. Por otro lado, si estamos resolviendo un problema de minimización, la función de valoración puede asignar un valor más alto a las soluciones que se alejan más de la solución óptima.
Definir una buena función de valoración implica considerar cuidadosamente los requisitos y características del problema en cuestión. Puede implicar ponderar diferentes objetivos, como maximizar o minimizar ciertas variables, satisfacer restricciones específicas o lograr un equilibrio entre múltiples criterios. 

Una función de valoración efectiva debe ser capaz de discriminar entre soluciones prometedoras y soluciones subóptimas. Debe proporcionar una evaluación objetiva y precisa, permitiendo la selección de soluciones de alta calidad y la eliminación de soluciones menos deseables.
Además, la función de valoración debe ser adecuada para el dominio y el tipo de problema abordado. Puede requerir conocimiento especializado y experiencia en el campo para definir correctamente las métricas y ponderaciones apropiadas. 

Es importante tener en cuenta que la función de valoración puede evolucionar y adaptarse a lo largo del proceso de búsqueda. A medida que el algoritmo de busqueda avanza y produce nuevas generaciones de soluciones, la función de valoración puede ser ajustada y refinada para enfocarse en las características más relevantes y deseadas.


Por lo tanto, en los algoritmos de búsqueda la función de valoración desempeña un papel crucial para evaluar y comparar soluciones candidatas en función de criterios específicos del problema. Proporciona una medida cuantitativa de la calidad de las soluciones y guía el proceso de selección y toma de decisiones en cada etapa del algoritmo. El diseño adecuado de la función de valoración es esencial para obtener resultados óptimos y eficientes en la resolución de problemas.
A continuación, explicaremos cada función de valoración implementada en nuestros algoritmos.


% En el caso de algoritmos greedy, la función de valoración juega un papel similar pero con un enfoque más local. En lugar de trabajar con una población de soluciones, los algoritmos greedy toman decisiones en cada paso basándose en una evaluación local de las opciones disponibles. La función de valoración en un algoritmo greedy se utiliza para seleccionar la mejor opción en cada iteración o paso del algoritmo, maximizando o minimizando la función de valoración según el objetivo del problema \cite{cormen2009introduction}.




\subsection{Fitness: Generador Exhaustivo}

Dado un candidato que representa un conjunto de métodos $M$, nuestra función de valoración intenta calcular una aproximación del número de objetos acotados que se pueden construir utilizando combinaciones de métodos habilitados en el cromosoma. Los candidatos con valores de aptitud más altos se estima que construyen más objetos que aquellos que tienen valores de aptitud más pequeños.
Nuestra función de valoración tiene como objetivo principal maximizar la cantidad de objetos que puedo generar utilizando el Generador exhaustivo que hemos desarrollado (explicado en el capítulo anterior,  \ref{cap:beapi}. Este enfoque nos permite evaluar la capacidad de los candidatos para construir una variedad de objetos y proporciona una base para la selección y mejora de los mejores candidatos en nuestro algoritmo.

Esta función de valoración en esta tesis se utiliza para evaluar la calidad de un candidato, el cual representa un conjunto de métodos $M$. Para cada $M$ , nuestra función devuelve un valor real que se construye de la siguiente manera:
{\small
\[
f(M) = \text{{\#Objectos}}(M) \ , \ (\text{{\#MétodosAPI}} - \text{{\#M}}) + w_1 \times \text{{PP}}(M)
\]
}

La función de valoración se compone de tres componentes. En primer lugar, $\text{{\#Objetos}}(M)$ representa el número de objetos generados por el conjunto de métodos $M$. La coma en la fórmula indica que este valor es independiente de los otros componentes, y se le da prioridad a la cantidad de objetos creados con el subconjunto de métodos $M$.

El segundo componente, $(\text{{\#MétodosAPI}} - \text{{\#M}})$, refleja la diferencia entre el número total de métodos disponibles en la API y el número de métodos en el conjunto $M$. Este término se utiliza para desempatar en caso de que otro subconjunto de métodos $M_1$ construya la misma cantidad de objetos que $M$.

\begin{lstlisting}[label=fig:NCLeqbuilders2, caption=Sufficient and minimal builders for NCL with more complex parameters than the ones in Figure \ref{fig:NCLbuilders}, captionpos=b, frame=tb, float=t]
  (0)  NodeCachingLinkedList()
  (4)  add(int,Object)
  (23)remove(Object)
\end{lstlisting}

Finalmente, $\text{{PP}}(M)$ es un factor de ponderación que permite ajustar la importancia de tener menor complejidad y cantidad de parámetros en el conjunto de métodos $M$. Este valor es entre 1 y 0 es solo para ajustar en caso de que haya empate de cantidad de objetos y métodos.
Los métodos con más parámetros o parámetros con tipos más complejos requieren más esfuerzo para generar entradas útiles, lo que los hace más exigentes para el análisis del programa. Por lo tanto, definimos un criterio de parámetro y adaptamos nuestra función de valoración para favorecer a los métodos con menos parámetros. Por ejemplo, ambos conjuntos de constructores en las Figuras \ref{fig:NCLbuilders} y \ref{fig:NCLeqbuilders2} son suficientes y mínimos (con 3 rutinas cada uno), pero el subconjunto de métodos en la Figura \ref{fig:NCLeqbuilders2} tienen más parámetros que deben ser instanciados. 

\begin{lstlisting}[label=fig:NCLeqbuilders2, caption=Sufficient and minimal builders for NCL with more complex parameters than the ones in Figure \ref{fig:NCLbuilders}, captionpos=b, frame=tb, float=t]
  (0)  NodeCachingLinkedList()
  (4)  add(int,Object)
  (23)remove(Object)
\end{lstlisting}

Al comparar las Figuras \ref{fig:NCLbuilders} y \ref{fig:NCLeqbuilders2}, podemos observar que \texttt{addFirst} ha sido reemplazado por \texttt{add}, que tiene un parámetro entero adicional, y que \texttt{removeFirst} se intercambió con \texttt{remove}, que tiene un parámetro no primitivo de tipo Object. Además, nuestro enfoque tiene en cuenta el número de parámetros y sus complejidades para seleccionar los constructores más adecuados. Hicimos un ranking de los tipos de parámetros más comunes para conocer su complejidad al instanciarlos.

\begin{lstlisting}[label=fig:rankParameters, caption=Ranking con los tipos de parametros, captionpos=b, frame=tb, float=t]
Boolean=1
Integer,Char,Object=2
Float,Double=3
String=6
Collection=10
Other Types=16
\end{lstlisting}

Siguiendo el criterio explicado, nuestro algoritmo elige el set de la Figure \ref{fig:NCLbuilders} sobre la Figura~\ref{fig:NCLeqbuilders2}



Esta función de valoración nos proporciona una medida cuantitativa de la calidad del candidato representado por el conjunto de métodos $M$. Los candidatos con valores de función de valoración más altos se consideran de mayor calidad y son preferidos en los algoritmos genéticos y en los algoritmos greedys que explicaremos en la sección \ref{sec:algorithms}.
Idealmente, nos gustaría explorar todos los objetos factibles dentro de un límite pequeño $k$ que se pueden construir utilizando los métodos del cromosoma actual. En otras palabras, necesitamos un generador exhaustivo acotado para el conjunto de métodos $BE(M, k)$. El límite $k$ representa el número máximo de objetos que se pueden crear para cada clase (en la Figura \ref{fig:ncl-instances}, el número de nodos en los objetos NCL está acotado por $k=2$) y el número máximo de valores primitivos disponibles (por ejemplo, enteros del 0 a $k-1$).

Para este propósito, desarrollamos la herramienta BEAPI, que se discute con más detalle en la Capítulo \ref{cap:beapi}. En resumen, primero exploramos exhaustivamente todas las posibles combinaciones de secuencias de los métodos de $M$. Luego, utilizamos un conjunto fijo de valores primitivos (enteros del 0 a $k-1$) con los cuales probar nuestros métodos cuando requieren valores primitivos.

En segundo lugar, descartamos las secuencias de métodos que crean objetos con más de $k$ objetos (de cualquier tipo) para evitar construir objetos más grandes de lo necesario. Para lograr esto, canonicamos los objetos generados por la ejecución de cada secuencia y descartamos la secuencia si algún objeto tiene un índice igual o mayor que $k$.

En tercer lugar, ampliamos esta generación con coincidencia de estado. Esto se debe a que, en la generación de pruebas, a menudo hay muchas secuencias de pruebas que producen el mismo objeto. Por ejemplo, insertar en una colección y luego eliminar el mismo elemento resulta en muchos casos en exactamente la misma estructura antes de la inserción. Nuestro enfoque asume que las ejecuciones de rutinas son deterministas con respecto a sus entradas. Bajo esta suposición, se deduce que, para generar un conjunto exhaustivo acotado de estructuras, solo necesitamos guardar una secuencia de prueba para crear cada estructura diferente en el conjunto, y que todas las siguientes secuencias de prueba que generen la misma estructura se pueden descartar.

Nuestra justificación para usar conjuntos acotados de objetos es similar a la \emph{hipótesis de la cota pequeña} \cite{Andoni:2003}.Si un conjunto de métodos no puede usarse para construir objetos pequeños que permitan diferenciarlo de otro conjunto de métodos, es poco probable que estos dos conjuntos puedan distinguirse con objetos más grandes. Esta hipótesis se mantuvo durante nuestra evaluación empírica en todos nuestros casos de estudio.

Para obtener más información sobre el algoritmo y más detalles sobre BEAPI, invitamos al lector a consultar el Capítulo \ref{cap:beapi}.


\subsection{Fitness: Cobertura de ramas con Randoop}
\
En el contexto de nuestra tesis, hemos realizado modificaciones en la herramienta \emph{Randoop} descrita en la sección \ref{sec:feedback-directed-test-gen}.

Nuestras modificaciones se centran en dar prioridad a un subconjunto específico de métodos, representado en la sección anterior por el conjunto $M$. El objetivo de esta modificación es generar secuencias de pruebas que se enfoquen principalmente en la ejecución de estos métodos seleccionados. Sin embargo, es importante destacar que esto no implica excluir por completo otros métodos, ya que aún buscamos ejercitar la API en su totalidad para obtener una buena cobertura de todos los métodos.

La motivación detrás de esta modificación radica en que los subconjuntos de métodos que obtienen una mayor cobertura de la API nos proporcionarán indicios valiosos sobre la calidad de dicho subconjunto. Por lo tanto, nuestra función de valoración tiene como objetivo maximizar la cobertura de ramas de la API bajo prueba, tal como se describe en la sección de preliminares \ref{sec:coverage}.

La función de valoración que utilizamos para evaluar la calidad de un candidato, que representa un conjunto de métodos $M$, se construye de la siguiente manera:

{\small
\[
f(M) = \text{{\#CoberturaRama}}(M) \ , \ (\text{{\#MétodosAPI}} - \text{{\#M}}) + w_1 \times \text{{PP}}(M)
\]
}

La función de valoración consta de tres componentes principales. En primer lugar, $\text{{\#CoberturaRama}}(M)$ representa el número de ramas cubiertas por las suites de prueba generadas por Randoop utilizando el subconjunto de métodos $M$. Este componente refleja la cantidad de ramas cubiertas específicamente por este subconjunto priorizado de métodos.

El segundo componente, $\text{{\#MétodosAPI}} - \text{{\#M}}$, tiene en cuenta la cantidad de métodos restantes en la API que no están presentes en el subconjunto $M$. Este componente incentiva a que el conjunto de métodos sea menor.

Por último, penalizamos sobre los parámetros de cada método en el subconjunto como lo hacíamos en la sección anterior.

En resumen, nuestras modificaciones en Randoop nos permiten priorizar un subconjunto de métodos durante la generación de secuencias de pruebas, y nuestra función de valoración se encarga de maximizar la cobertura de ramas de la API. Estas adaptaciones nos proporcionan una mejor comprensión de la calidad y efectividad de los subconjuntos de métodos generados.

\cacho{Tengo que agregar un cierre.}

\chapter{Generacion Exhaustiva acotada desde API de los programas}


\label{cap:beapi}

En adelante, se presenta la técnica  desarrollada llamada BEAPI, que tiene como objetivo mejorar la generación exhaustiva acotada (BEG, por sus siglas en inglés, \emph{Bounded Exhaustive Generation}). Este capítulo está organizado de la siguiente manera: en la Sección \ref{sec:motivation}, se presenta un ejemplo que ilustra el problema que se aborda con la técnica propuesta, la cual se explica en la Sección \ref{sec:beapiTecnique}. Luego, en las Secciones \ref{sec:scope}, \ref{sec:stateMatching} y \ref{sec:builders}, se describen las optimizaciones implementadas en BEAPI.


\section[Motivation]{motivation}
\label{sec:motivating-example}

Para motivar las dificultades de escribir especificaciones formales para la generación de estructuras exhaustiva acotadas,  considére el invariante de representación (conmunmente llamados \emph{repOK}) de la clase \emph{NodeCachingLinkedList} (NCL) de Apache, que se muestra en la Figura \ref{fig:NCL-repOK}. Este es un repOK del benchmarks \emph{ROOPS}. Los NCL se componen de una lista principal circular doblemente enlazada, utilizada para el almacenamiento de datos, y una caché de nodos previamente utilizados implementada como una lista enlazada simple. Los nodos eliminados de la lista principal se mueven a la caché, donde se guardan para su uso en el futuro. De esta manera, cuando se requiere un nodo para una operación de inserción, se reutiliza un nodo de la caché (si existe) en lugar de asignar un nuevo nodo. El objetivo es evitar la sobrecarga de recolección de basura para las aplicaciones que realizan una gran cantidad de inserciones y eliminaciones en la lista. El \emph{repOK} devuelve true si y solo si la estructura de entrada satisface las propiedades estructurales de NCL \cite{Liskov00}. Las líneas 1 a 20 verifican que la lista principal sea una lista circular doblemente enlazada con una cabeza ficticia; las líneas 21 a 33 verifican que la caché sea una lista enlazada simple terminada en nulo (y se verifica la consistencia de los campos de tamaño en el proceso). Se observa que repOK está escrito de la manera recomendada por los autores del enfoque de generación exhaustiva de BE, \textsf{Korat} \cite{Boyapati02}. Este devuelve False tan pronto como encuentra una violación de una propiedad prevista en la entrada actual. De lo contrario, devuelve Verdadero al final. Estos repOK permiten a \textsf{Korat} podar grandes porciones del espacio de búsqueda, lo que mejora en gran medida su eficiencia.


\begin{figure}[!thb]
\begin{lstlisting}
public boolean repOK() {
  if (this.header == null) return false;
  //  Missing constraint: the value of the sentinel node
  // must be null  
  // if (this.header.value != null) return false;
  if (this.header.next == null) return false;
  if (this.header.previous == null) return false;
  if (this.cacheSize > this.maximumCacheSize) return false;
  if (this.size < 0) return false;
  int cyclicSize = 0;
  LinkedListNode n = this.header;
  do {
      cyclicSize++;
      if (n.previous == null) return false;
      if (n.previous.next != n) return false;
      if (n.next == null) return false;
      if (n.next.previous != n) return false;
      if (n != null) n = n.next;
  } while (n != this.header && n != null);
  if (n == null) return false;
  if (this.size != cyclicSize - 1) return false;
  int acyclicSize = 0;
  LinkedListNode m = this.firstCachedNode;
  Set visited = new HashSet();
  visited.add(this.firstCachedNode);
  while (m != null) {
      acyclicSize++;
      if (m.previous != null) return false;
      // Missing constraint: the value of cache nodes
      // must be null
      // if (m.value != null) return false;
      m = m.next;
      if (!visited.add(m)) return false;
  }
  if (this.cacheSize != acyclicSize) return false;
  return true;
}
\end{lstlisting}
\caption{\texttt{NodeCachingLinkedList}'s \texttt{repOK} from \textsf{ROOPS}}
\label{fig:NCL-repOK}
\end{figure}

Este ejemplo muestra que escribir \texttt{repOK}s sólidos y precisos para BEG es difícil y consume tiempo. Afinar los \texttt{repOK}s para mejorar el rendimiento de BEG (por ejemplo, para \textsf{Korat}) es aún más difícil. 


\section[BEAPI]{BEAPI}
\label{sec:beapiIntro}
La técnica propuesta, llamada \textsf{BEAPI} (Bounded Exhaustive from API), introduce un enfoque novedoso para la generación exhaustiva acotada. Este enfoque se basa en la realización de llamadas a las rutinas de la API del Software bajo test (SUT). Al igual que otros enfoques de generación de pruebas basados en API, \textsf{BEAPI} crea secuencias de llamadas a métodos de la API, conocidas como secuencias de test, y las ejecuta para generar estructuras.

A diferencia de los enfoques de generación basados en caja negra, \textsf{BEAPI} no requiere una especificación formal de las propiedades de las estructuras. En lugar de eso, el usuario lo único que debe proporcionar son los alcances para la generación, lo cual se aborda en detalle en la sección \ref{ref:scope}.

La generación exhaustiva de todas las secuencias de prueba factibles a partir de rutinas hasta una longitud máxima, conocida como generación por fuerza bruta, es un enfoque intrínsecamente combinatorio que consume una gran cantidad de recursos computacionales, incluso para alcances pequeños. Por lo tanto, \textsf{BEAPI} utiliza varias técnicas de poda que son cruciales para mejorar su eficiencia y permitir la escalabilidad a alcances más grandes.

En primer lugar, \textsf{BEAPI} ejecuta secuencias de prueba y descarta aquellas que producen excepciones que violan las reglas de uso de la API, como \emph{IllegalArgumentException} e \emph{IllegalStateException} en Java.

En segundo lugar, \textsf{BEAPI} implementa la técnica de coincidencia de estado, la cual descarta secuencias de métodos que generan estructuras que ya han sido creadas por secuencias de métodos exploradas previamente. Esta técnica se describe en detalle en la sección \ref{sec:state-matching}.

En tercer lugar, \textsf{BEAPI} utiliza un subconjunto de las rutinas de la API para crear las secuencias de prueba. Este subconjunto se identifica mediante un algoritmo de búsqueda greedy y una función de valoración que tiene en cuenta qué subconjunto permite crear la mayor cantidad de objetos utilizando \textsf{BEAPI} en el menor tiempo posible. Este proceso se detalla en el capítulo \ref{cap:builders}.

La principal ventaja de \textsf{BEAPI} es que requiere un esfuerzo mínimo de especificación para realizar la generación exhaustiva acotada (BEG). Si los métodos de la API utilizados en la generación son correctos, todas las estructuras generadas serán válidas para su construcción. El programador solo necesita asegurarse de que los métodos de la API lancen excepciones cuando se violen las reglas de uso, siguiendo un estilo de programación defensivo \cite{Liskov00}. En la mayoría de los casos, esto implica verificar condiciones muy simples en las entradas. Por ejemplo, en el ejemplo mencionado, el método para agregar un elemento a un \texttt{NCL} lanza una \texttt{IllegalArgumentException} cuando se llama con un elemento \texttt{null}, mientras que la implementación del método se encarga de cumplir con las demás propiedades del \texttt{NCL}.

\cacho{TODO: no se si agregar arbolitos de exploracion para explicarlos}
\\
\\
% \begin{tikzpicture}
%     [level 1/.style={sibling distance=27mm},
%    level 2/.style={sibling distance=25mm},
%    every node/.style={rectangle,draw,fill=white,minimum size=10mm,align=center,font=\tiny},
%    edge from parent/.style={draw}]
   
%   % Raíz
%   \node {n = new NCL()}
%     % Hijos
%     child {node {n = new NCL() \\
%                 n.add(int)}
%       child {node {n = new NCL() \\
%                     n.method()}}
%       child {node {n = new NCL() \\
%                     n.anotherMethod()}}
%     }
%    child {node {n = new NCL() \\
%                 n.addFirst()}}
%    child {node {n = new NCL() \\
%                 n.remove()}}
%     child {node {...}}
%     % child {node {Método N}};
% \end{tikzpicture}



\section{Scope}
\label{sec:scope}

\begin{figure}[t!]
\begin{lstlisting}[keywordstyle=\scriptsize\ttfamily]
max.objects=3
int.range=0:2
strings=str1,str2,str3
omit.fields=NCL.DEFAULT_MAXIMUM_CACHE_SIZE
\end{lstlisting}
\caption{\textsf{BEAPI}'s scope definition for \texttt{NCL} (max. nodes 3)}
\label{fig:NCL-fin-BEAPI}
\end{figure}

\textsf{BEAPI} explora el espacio de búsqueda de secuencias de prueba (acotadas) que se pueden formar mediante llamadas a la API de \texttt{NCL}. Por lo tanto, debemos proporcionar dominios de datos para los tipos primitivos utilizados en esas llamadas y establecer un límite en el tamaño máximo de las estructuras que deseamos generar a partir de esas llamadas de API.

A continuación se muestra un ejemplo de archivo de configuración que define los alcances de \textsf{BEAPI} para el estudio de caso de \texttt{NCL}:

\begin{figure}[h!]
    \centering
    \begin{verbatim}
max.objects = 3
int.range = 0, 2
str1 = "value1"
str2 = "value2"
str3 = "value3"
omit.fields = maxCacheSize, modCounts
    \end{verbatim}
    \caption{Archivo de configuración de \textsf{BEAPI} para el estudio de caso de \texttt{NCL}}
    \label{fig:NCL-fin-BEAPI}
\end{figure}

El parámetro \texttt{max.objects} especifica el número máximo de objetos diferentes (alcanzables desde la raíz) permitidos en una estructura. Las secuencias de prueba que crean una estructura con más objetos diferentes (de cualquier clase) que \texttt{max.objects} se descartarán, junto con la estructura actual. En nuestro ejemplo, esto implica que \textsf{BEAPI} no creará NCL con más de 3 nodos.

Además, debemos especificar los valores que \textsf{BEAPI} utilizará para invocar las rutinas de la API que toman parámetros de tipo primitivo. El parámetro \texttt{int.range} permite especificar un rango de enteros, que va desde 0 hasta 2 en la Figura~\ref{fig:NCL-fin-BEAPI}. También se pueden especificar dominios para otros tipos primitivos, como flotantes, dobles y cadenas, describiendo sus valores por extensión. Por ejemplo, en la figura se definen \texttt{str1}, \texttt{str2} y \texttt{str3} como los valores factibles para los parámetros de tipo \texttt{String}.

Además, podemos informar a \textsf{BEAPI} qué campos relevantes considerar para la canonización de la estructura o qué campos omitir (\texttt{omit.fields}). Esto permite al usuario controlar el proceso de coincidencia de estados. Por ejemplo,  la línea 4 haría que \textsf{BEAPI} omita el tamaño máximo predeterminado de la caché en la coincidencia de estados, que en nuestro ejemplo es una constante inicializada en 20 en el constructor de la clase. Además, esta linea omitirá el campo \emph{modCount} que hemos hablado en el capitulo de identificación de métodos builders. En algunos casos, omitir estos tipos de campos puede ser importante.

La configuración mostrada en la Figura~\ref{fig:NCL-fin-BEAPI} es suficiente para que \textsf{BEAPI} genere NCL con un máximo de 3 nodos, que contienen enteros del 0 al 2 como valores. Los archivos de configuración utilizados en nuestra sección experimental son similares a los mostrados aquí.


\section{State Matching}
\label{sec:state-matching}

En la generación de pruebas con \textsf{BEAPI} a menudo muchas secuencias de prueba producen la misma estructura, por ejemplo, insertar un elemento en una lista y luego eliminarlo. \textsf{BEAPI} asume que las ejecuciones de métodos son deterministas: cualquier ejecución de una rutina con las mismas entradas produce los mismos resultados. Observamos que, para cada estructura distinta s, solo necesitamos guardar la primera secuencia de prueba que genera \texttt{s} (y la estructura en sí). Todas las secuencias de prueba generadas posteriormente que también crean \texttt{s} pueden ser descartadas. Nótese que, como \textsf{BEAPI} trabaja extendiendo secuencias de prueba previamente generadas %(ver seccion TODO ????)
, si guardamos muchas secuencias de prueba para la misma estructura, todas estas secuencias tendrían que ser extendidas con nuevas rutinas en iteraciones posteriores de \textsf{BEAPI}, lo que resultaría en muchos cálculos innecesarios. Por lo tanto, implementamos la coincidencia de estados en \textsf{BEAPI} de la siguiente manera. Almacenamos todas las estructuras producidas hasta ahora por \textsf{BEAPI} en una forma canónica (ver más abajo). Después de ejecutar la última rutina \texttt{r(p$_1$,..,p$_k$)} de una nueva secuencia de prueba generada \texttt{T}, comprobamos si alguno de los parámetros de \texttt{r} tiene una estructura no vista antes (no almacenada). Si T no crea ninguna estructura nueva, se descarta. De lo contrario, \texttt{T} y las nuevas estructuras que genera son almacenadas por \textsf{BEAPI}.

Representamos las estructuras asignadas en el montículo como grafos etiquetados. Después de la ejecución de un método, un parámetro $p$  (de tipo no primitivo) contiene una referencia al objeto raíz $r$ de un montículo con raíz (es decir,  $p=r$), definido a continuación.

\begin{definition}
    Sea $O$ un conjunto de Objectos y $P$ un set de valores primitivos (incluido $null$). Sea $F$ blos campos de todos los objectos en $O$. 
    \begin{itemize}
        \item A \emph{heap} es un grafp etiquetado $H = \langle O,E\rangle$ con $E = \{(o,f,v) | o \in O, f \in F, v \in O \cup P\}$.
        \item Un \emph{heap con raíz} es un par $RH = \langle r, H\rangle$ donde
            $r \in O$, $H = \langle O,E\rangle$ es un heap, y para cada $v' \in O \cup P$, $v'$ es alcanzable desde $r$ a través de campos en $F$.
    \end{itemize}
\end{definition}

El caso especial $p=null$  se puede representar con un montículo con raíz con un nodo ficticio y un campo ficticio que apunta a null. En lenguajes como Java, cada objeto se identifica por la dirección de memoria donde se encuentra. Cambiar las direcciones de memoria donde se asignan los objetos no tiene efecto desde el punto de vista del programa, ya que el programador no tiene control sobre la representación de bajo nivel de la memoria (a diferencia de otros lenguajes como C). Los montículos obtenidos por permutaciones de las direcciones de memoria de sus objetos componentes se llaman \emph{heap isomorfos}. Evitamos la generación de \emph{heaps isomorfos} empleando una representación canónica para heaps \cite{Iosif02,Boyapati02}. Los heaps enraizados se pueden canonizar eficientemente mediante un enfoque llamado \emph{linearización} \cite{Iosif02,Xie04}, que transforma un heap enraizado en una secuencia única de valores. 

\cacho{Lo cambio a espanol?}
\bigbreak

\begin{figure}[!th]
\begin{lstlisting}
int[] linearize(O root, Heap<O, E> heap, int scope, 
Regex omitFields) {
  Map ids = new Map(); // maps nodes into their unique ids 
  return lin(root, heap, scope, ids, omitFields); 
}
int[] lin(O root, Heap<O, E> heap, int scope, Map ids, 
Regex omitFields) { 
  if (ids.containsKey(root))
    return singletonSequence(ids.get(root)); 
  if (ids.size() == scope) 
    throw new ScopeExceededException();
  int id = ids.size() + 1;
  ids.put(root, id);
  int[] seq = singletonSequence(id);
  Edge[] fields = 
            sortByField({ <root, f, o> in E }, omitFields); 
  foreach (<root, f, o> in fields) {
    if (isPrimitive(o)) 
      seq.add(uniqueRepresentation(o));
    else
      seq.append(lin(o, heap, scope, ids, omitFields));
  }
  return seq; 
}
\end{lstlisting}
\caption{Linearization algorithm}
\label{alg:linearization}
\end{figure}


Se muestra en la Figura~\ref{alg:linearization} una versión personalizada para \textsf{BEAPI} del algoritmo de linearización de  \cite{Xie04} .El mismo fue modificado para informar cuando los objetos exceden los alcances y para admitir la omisión de campos de objeto. \texttt{linearize} invoca a \texttt{lin} y  comienza un recorrido en profundidad heap desde la raíz (línea 3). \texttt{linearize} asigna identificadores de objeto diferentes a cada objeto visitado. Cuando se visita un objeto por primera vez, se le asigna un nuevo identificador único (líneas 10-11) y se crea una secuencia de un solo elemento, seq, con el identificador de objeto para representar el objeto (línea 12). \texttt{ids} almacena el mapeo entre los objetos y los identificadores de objeto únicos. Luego, los campos del objeto, ordenados en un orden predefinido (por ejemplo, por nombre), se recorren y se construye la linearización de cada valor de campo y se agrega a la secuencia que representa el objeto, seq (líneas 13-19). Si un campo almacena un valor primitivo (línea 15), se agrega una secuencia de un solo elemento que representa el valor a seq (línea 16). Si el campo hace referencia a un objeto, se realiza una llamada recursiva a \texttt{lin}  para transformarlo en una secuencia, que se agregará a seq (línea 18). Al final del bucle, \texttt{seq}  contiene la representación canónica de todo el heap que comienza en la \texttt{raíz} , y se devuelve por \texttt{lin} (línea 20). Cuando se encuentra un objeto ya visitado por una llamada recursiva, es decir, que ya tiene un identificador asignado en  \texttt{ids}, el algoritmo devuelve la secuencia de un solo elemento con el identificador único del objeto (líneas 6-7). Si hay más de \texttt{scope} objetos alcanzables desde el heap enraizado, \texttt{linearize} devuelve una excepción para informar que se han superado los alcances (líneas 9-10). \texttt{linearize} también toma como parámetro una expresión regular \texttt{omitFields}, que coincide con los nombres de los campos que deben ser omitidos durante la canonicación (ver Sección~TODO REF). Para omitir tales campos, implementamos \texttt{sortByField} (línea 13) de tal manera que no devuelve las aristas correspondientes a los campos cuyos nombres coinciden con \texttt{omitFields}. Este comportamiento es particular de nuestro enfoque, ya que la excepción será empleada por BEAPI para descartar secuencias de prueba que creen objetos más grandes de lo permitido por los alcances. Nótese que la linearización permite la comparación eficiente de objetos: dos objetos son iguales si y solo si sus secuencias correspondientes generadas por \texttt{linearize} son iguales.


\section{Uso de Builders en BEAPI}
\label{sec:builders}

Dado que las combinaciones factibles de métodos crecen de manera exponencial con el número de métodos, es crucial reducir la cantidad de métodos que \textsf{BEAPI} utiliza para producir secuencias de prueba. Para evitar este problema, utilizamos el enfoque de identificación automático de métodos \emph{builders} presentado en el capitulo anterior (Capitulo \ref{cap:builders}) para encontrar un subconjunto de métodos de la API que son suficientes para la generación de conjuntos de estructuras acotadas y exhaustivas. Para el propósito que necesitamos en \textsf{BEPAI} el enfoque  para identificar un subconjunto de builders suficientes de una API que se basa en un algoritmo Greedy, nos fue suficiente. Recordemos que tenemos otros dos enfoques, pero computacionalmente son mucho mas costoso. Aquí consideramos un enfoque más simple, el basado en \emph{hill climbing} (HC), que logra un mejor rendimiento. Es posible que HC sea menos preciso, ya que puede incluir algunos métodos en el conjunto resultante de builders que podrían no ser necesarios para producir un conjunto acotado y exhaustivo de estructuras. Sin embargo, HC funcionó muy bien y calculó de manera consistente conjuntos mínimos de builders en nuestros experimentos (verificamos que el conjunto de builders calculados por HC coincidía con el conjunto de builders identificados manualmente para cada estudio de caso). Nuestro objetivo aquí es evaluar el impacto de utilizar builders para BEG (generación exhaustiva acotada) a partir de una API. 
En este caso utilizamos la función de valoración que utiliza un generador exhaustivo para generar objetos de la API (ver \ref{sec:fitness})
Observa que el algoritmo Hill Climbing realiza muchas invocaciones a un generador exhaustivo para la identificación de métodos builders. La idea clave que hace posible la identificación de builders es que a menudo los builders identificados para un alcance relativamente pequeño son el mismo conjunto de métodos que se necesitan para crear estructuras de cualquier tamaño. En otras palabras, una vez que el alcance para la computación de builders es lo suficientemente grande, aumentar el alcance dará como resultado el mismo conjunto de builders. Este resultado se asemeja a la hipótesis del alcance pequeño para la detección de errores \cite{Andoni02} (y \emph{transcoping} \cite{Rosner13}). Un alcance de 5 fue suficiente para la computación de builders en todos nuestros estudios de caso (verificamos manualmente que los builders calculados fueran los correctos en todos los casos). Después de que los builders se identifican de manera eficiente utilizando un alcance pequeño, podemos ejecutar \textsf{BEAPI} con los builders identificados utilizando un alcance mayor, por ejemplo, para generar objetos más grandes y poner a prueba el SUT. En la mayoría de nuestros estudios de caso, los builders comprenden un constructor y un solo método para agregar elementos a la estructura. Sin embargo, nuestro enfoque automatizado de identificación de builders mostró que, para los árboles Rojo-Negro, también se requería un método de eliminación (para alcances mayores que 3), ya que existen árboles con una configuración de balance particular (coloreando en rojo y negro para los nodos) que no se pueden construir solo agregando elementos al árbol. En contraste, los árboles AVL, que también están balanceados, no requieren el método de eliminación como builder, solo con el constructor de la clase y una rutina de adicción son suficientes. Esto demuestra que la identificación de builders no es trivial de realizar manualmente.

\section{Algoritmo de BEAPI}
\label{sec:beapiTechnique}

A continuación se muestra un pseudocódigo de \textsf{BEAPI} en la Figura \ref{alg:beapi}. \textsf{BEAPI} toma como entradas una lista de métodos de una API,  \texttt{methods} (la API completa o los métodos \emph{builders} previamente identificados); el alcance de objetos para la generación, \texttt{scope}; una lista para crear valores de cada tipo primitivo proporcionado en la descripción del alcance, \texttt{primitives} (creados automáticamente a partir de opciones de configuración como \texttt{int.range}, \texttt{string}, etc., ver Figura~\ref{fig:NCL-fin-BEAPI}); y una expresión regular que coincide con los campos que se deben omitir en la canonicación de las estructuras, \texttt{omitFields}. Observa que se pueden pasar métodos de más de una clase en \texttt{methods} si se desean generar objetos para varias clases en la misma ejecución de \textsf{BEAPI}, por ejemplo, cuando los métodos de una clase toman objetos de otra clase como parámetros. La estructura de datos de tipo mapa, \texttt{currSeqs} de \textsf{BEAPI}  almacena, para cada tipo, la lista de secuencias de test que se sabe que generan estructuras del tipo correspondiente. \texttt{currSeqs} se inicia con todas las secuencias de tipos primitivos en \texttt{primitives} (líneas 2-3). En cada iteración del bucle principal (líneas 5-34), \textsf{BEAPI}  crea nuevas secuencias para cada método disponible \texttt{m} (línea 8), explorando exhaustivamente todas las posibilidades para crear secuencias de prueba utilizando \texttt{m} e inputs generados en iteraciones anteriores y almacenados en \texttt{currSeqs} (líneas 9-30). Las secuencias de prueba recién creadas que generan nuevas estructuras en la iteración actual se guardan en el mapa \texttt{newSeqs} (inicializado vacío en la línea 6); todas las secuencias generadas se agregan a currSeqs al final de la iteración (línea 33). Si no se producen nuevas estructuras en la iteración actual (\texttt{newStrs} es falso en la línea 32), el bucle principal de \textsf{BEAPI}  termina y se devuelve la lista de todas las secuencias en \texttt{currSeqs} (línea 35).


\begin{figure}[t!]
\cacho{cambio a espanol?}

\begin{lstlisting}[language=Java]
public BEAPI(List methods, int scope, Map<Type, 
    List<Seq>> primitives, Regex omitFields) {
    Map<Type, List<Seq>> currSeqs = new Map();
    currSeqs.addAll({ T->L | T->L in primitives });
    Set canonicalStrs = new Set();
    for (int it=0; true; it++) {
      Map<Type, List<Seq>> newSeqs = new Map();
      boolean newStrs = false;
      for (m(T1,...,Tn):Tr: methods) {
        Map<Type, List<Seq>> seqsT1 = 
            currSeqs.getSequencesForType(T1);
        ...
        Map<Type, List<Seq>> seqsTn = 
            currSeqs.getSequencesForType(Tn);
        for ((s1,...,sn): seqsT1 x ... x seqsTn) {
          Seq newSeq = createNewSeq(s1,...,sn,m);
          o1,...,on,or,failure,exception = execute(newSeq);
          if (failure) 
            throw new ExecutionFailedException(newSeq);
          if (exception) 
            continue;
          c1,...,cn,cr,outOfScope = 
            makeCanonical(o1,...,on,or,scope,omitFields);
          if (outOfScope) 
            continue;
          if (isReferenceType(T1) and 
            !canonicalStrs.contains(c1)) {
                canonicalStrs.add(c1);
                newSeqs.addSeqForType(T1, newSeq);
                newStrs = true;
          }
          ...
          if (isReferenceType(Tr) and 
            !canonicalStrs.contains(cr)) {
                canonicalStrs.add(cr);
                newSeqs.addSeqForType(Tr, newSeq);
                newStrs = true;
          }
        }
      }
      if (!newStrs) 
        break;
      currSeqs.addAll(newSeqs);
    }
    return currSeqs.getAllSeqsAsList();
}
\end{lstlisting}
\caption{\textsf{BEAPI} algorithm}
\label{alg:beapi}
\end{figure}

A continuación, comentaré los detalles del bucle for en las líneas 9-30. En primer lugar, se obtienen todas las secuencias que se pueden utilizar para construir entradas para \texttt{m} en \texttt{seqsT$_1$}, ..., \texttt{seqsT$_n$}. \textsf{BEAPI} explora cada tupla \texttt{(s$_1$}, ..., \texttt{s$_n$)} de entradas factibles para \texttt{m}. A continuación, se ejecuta \texttt{createNewSeq} (línea 13), que construye una nueva secuencia de prueba \texttt{newSeq} realizando la composición secuencial de las secuencias de prueba \texttt{s$_1$}, ..., \texttt{s$_n$} y la rutina \texttt{m}, y reemplazando los parámetros formales de \texttt{m} por las variables que crean los objetos requeridos en \texttt{s$_1$}, ..., \texttt{s$_n$}. Luego, se ejecuta \texttt{newSeq} (línea 14) y produce un fallo (\texttt{failure} se establece en verdadero), genera una excepción que representa un uso no válido de la API (\texttt{exception} se establece en verdadero) o su ejecución tiene éxito y crea nuevos objetos \texttt{o$_1$,$\ldots$,o$_n$,o$_r$}. En caso de fallo, se lanza una excepción y \texttt{newSeq} se presenta al usuario como evidencia del fallo (línea 15). Si se lanza un tipo diferente de excepción, \textsf{BEAPI} asume que corresponde a un mal uso de la API (ver más abajo), descarta la secuencia de prueba (línea 16) y continúa con la siguiente secuencia candidata. De lo contrario, la ejecución de \texttt{newSeq} genera nuevos objetos \texttt{o$_1$,$\ldots$,o$_n$,o$_r$} (o valores de tipos primitivos) que se canonizan mediante \texttt{makeCanonical} (línea 17) --ejecutando \texttt{linearize} de la Figura~\ref{alg:linearization} en cada estructura. Si alguna de las estructuras producidas por \texttt{newSeq} excede el ámbito, \texttt{makeCanonical} establece \texttt{outOfScope} en verdadero, \textsf{BEAPI} descarta \texttt{newSeq} y continúa con la siguiente (línea 18).
%Esto garantiza que \textsf{BEAPI} nunca crea objetos más grandes que el ámbito dado.
Si ninguna de las situaciones anteriores ocurre, \texttt{makeCanonical} devuelve versiones canónicas de \texttt{o$_1$,$\ldots$,o$_n$,o$_r$} en las variables \texttt{c$_1$,$\ldots$,c$_n$,c$_r$}, respectivamente. A continuación, \textsf{BEAPI} realiza una coincidencia de estado comprobando que la estructura canónica \texttt{c$_1$} sea de tipo de referencia y que no haya sido creada por ninguna secuencia de prueba anterior (línea 19). Observa que \texttt{canonicalStrs} almacena todas las estructuras ya visitadas. Si \texttt{c$_1$} es una nueva estructura, se agrega a \texttt{canonicalStrs} (línea 27) y se agrega la secuencia que crea \texttt{c$_1$}, \texttt{newSeq}, al conjunto de secuencias de prueba que producen estructuras de tipo \texttt{T$_1$} (\texttt{newSeqs} en la línea 27). Además, se establece \texttt{newStrs} en verdadero para indicar que al menos se ha creado un nuevo objeto en la iteración actual (línea 22). Este proceso se repite para los objetos canónicos \texttt{c$_2$,$\ldots$,c$_n$,c$_r$} (líneas 24-29).

\textsf{BEAPI} distingue los fallos del mal uso de la API en función del tipo de excepción (similarmente a las técnicas anteriores de generación de pruebas basadas en API \cite{Pacheco07}). Por ejemplo,\\
\texttt{IllegalArgumentException} y \texttt{IllegalStateException} corresponden a usos incorrectos de la API, y el resto de las excepciones se consideran fallos de manera predeterminada. La implementación de \textsf{BEAPI} permite al usuario seleccionar las excepciones que corresponden a fallos y aquellas que no, configurando los parámetros correspondientes. Como se mencionó en la Sección~\ref{sec:motivating-example}, \textsf{BEAPI} asume que los métodos de la API lanzan excepciones cuando no se pueden ejecutar con entradas inválidas. Sostenemos que esta es una práctica común, llamada programación defensiva \cite{Liskov00}, que todos los programadores deberían seguir, ya que resulta en un código más robusto y mejora las pruebas de software en general \cite{Ammann16} (además de ayudar a las herramientas de generación de pruebas automatizadas). También argumentamos en la Sección~\ref{sec:motivating-example} que el esfuerzo de especificación requerido para la programación defensiva es mucho menor que escribir \texttt{repOK}s precisos (y eficientes) para BEG, y esto era cierto después de inspeccionar manualmente el código fuente de nuestros casos de estudio. Por otro lado, ten en cuenta que \textsf{BEAPI} puede utilizar especificaciones formales para revelar errores en la API, por ejemplo, ejecutando \texttt{repOK} y comprobando que devuelve verdadero en cada objeto generado del tipo correspondiente (como en Randoop \cite{Pacheco07}). Sin embargo, las especificaciones utilizadas para encontrar errores no necesitan ser muy precisas (por ejemplo, el \texttt{repOK} subespecificado \texttt{NCL} de la Sección~\ref{sec:motivating-example} es válido para encontrar errores), ni estar escritas de una manera particular (como lo requiere \textsf{Korat}). \textsf{BEAPI} también puede utilizar otros tipos de especificaciones más débiles y más simples de escribir para revelar errores, como violaciones de contratos específicos del lenguaje (por ejemplo, \texttt{equals} es una relación de equivalencia en Java), propiedades metamórficas \cite{Chen19}, afirmaciones proporcionadas por el usuario (\texttt{assert}), etc.

Otra ventaja de \textsf{BEAPI} es que, para cada objeto generado, proporciona una secuencia de prueba que se puede ejecutar para crear el objeto. Esto contrasta con los enfoques basados en especificaciones (que generan un conjunto de objetos a partir de \texttt{repOK}). Encontrar una secuencia de invocaciones a métodos de la API que creen una estructura específica es un problema difícil en sí mismo, que puede ser bastante costoso computacionalmente \cite{Braione17} o requerir un esfuerzo significativo para realizarlo manualmente. Por lo tanto, a menudo los objetos generados por enfoques basados en especificaciones están "incrustados" cuando se utilizan para probar un SUT (por ejemplo, mediante el uso de reflexión en Java), lo que hace que las pruebas sean muy difíciles de entender y mantener, ya que dependen de los detalles de implementación de bajo nivel de las estructuras \cite{Braione17}.


%!TEX root = main.tex
\chapter[Evaluaci\'on]{Evaluaci\'on}
\label{cap:evaluation}



En los capítulos \ref{cap:builders} y \ref{cap:beapi} presentamos varios enfoques para mejorar la generación exhaustiva acotada:

\begin{itemize}
\item Identificación de métodos \emph{builders} para construir todos los objetos disponibles hasta una cota en una API. En este sentido, presentamos los siguientes algoritmos para encontrar estos métodos:
\begin{itemize}
\item Algoritmo Genético.
\item Algoritmo Greedy.
\item Algoritmo basado en clases de equivalencia.
\end{itemize}
\item Generación exhaustiva acotada a partir de métodos de la API.
\end{itemize}

En este capítulo, realizamos una evaluación experimental de estas técnicas con el objetivo de analizar la eficiencia y efectividad de las suites producidas en comparación con la generación exhaustiva acotada. En otras palabras, las siguientes preguntas de investigación guían esta experimentación:

\cacho{TODO: pensar mejor las RQ}
\begin{itemize}
\item \emph{RQ1}: ¿Qué tan eficientes son los algoritmos (Algoritmo Genético, Greedy, Clases de Equivalencia) para identificar conjuntos de métodos suficientes? 
\item \emph{RQ2}: ¿Qué tan efectivos son los algoritmos presentados para identificar conjuntos de métodos \emph{builders} suficientes de cada API?
\emph{builders} y cuánto tiempo computacional requiere cada enfoque?
\item \emph{RQ3}: ¿Cuál es el impacto de utilizar métodos Builders identificados en comparación con el uso de todos los métodos de la API en el contexto de verificación de software?
\item \emph{RQ4}: ¿Puede realizarse eficientemente la generación exhaustiva utilizando métodos de la API (BEAPI)?
\item\emph{RQ4}: ¿Cuál es el impacto de las optimizaciones propuestas en el rendimiento de BEG a partir de la API?
\item\emph{RQ6}: ¿Puede BEAPI ayudar a encontrar discrepancias entre las especificaciones repOK y la capacidad de generación de objetos de la API?
\item\emph{RQ7}: ¿Qué tan efectivas son las suites de prueba reducidas en comparación con la generación exhaustiva acotada?
\end{itemize}

% \section{Algoritmos de Identificacion de metodos builders}

\section{Evaluación Experimental en identificación de métodos builders}
En esta sección, evaluamos experimentalmente nuestros enfoques presentados en el capítulo \ref{cap:builders}. Describiremos como es la eficiencia y efectividad de cada algoritmo con las funciones de valoración presentada en la sección \ref{sec:fitness}. Además, mostraremos el uso de estos métodos identificados por nuestras técnicas durante el uso de una herramienta de verificación de software.

La evaluación se basa en un conjunto de implementaciones de estructuras de datos que se utilizan como referencia, incluyendo: \verb"NCL" de Apache Collections \cite{apache}; \verb"BinaryTree", \verb"BinomialHeap", \verb"FibonacciHeap" tomados de \cite{Visser:2006}; \verb"UnionFind", una implementación de conjuntos disjuntos tomada de JGrapht \cite{jgrapht}. También evaluamos nuestra técnica en componentes de proyectos de software reales, como \verb"Lits" de la implementación de Sat4j \cite{sat4j}, tomada de \cite{Loncaric:2018}, que consiste en un almacén de variables que controla cuándo se realizó la última suposición sobre el valor de una variable y si hay oyentes observando el estado de esa variable; \verb"Scheduler", una implementación de un planificador de procesos tomada de \cite{sir}; y estructuras de datos conocidas del paquete \verb"java.util" \footnote{https://docs.oracle.com/javase/8/docs/api/java/util/package-summary.html} como: \verb"TreeMap", \verb"TreeSet", \verb"HashMap", \verb"HashSet" y \verb"LinkedList". 

Para corroborar nuestros resultados, necesitamos conocer cuáles son los métodos builders mínimos y suficiente de cada caso de estudio. Para esto hemos creado un \emph{ground Truth} manualmente. Fue necesario inspeccionar cada caso de manera manual. Esto llevo un largo trabajo para estudiar los casos de estudio más complejo. El \emph{ground truth} de nuestros casos de estudios se pueden observar en la tabla \ref{tab:groundTruth}. Esta incluye el número de rutinas en la API completa (\#API), y el conjunto de los \emph{builders} identificados manualmente (algunos métodos pueden intercambiarse en diferentes ejecuciones, por ejemplo, \texttt{addFirst} y \texttt{addLast} en NCL)
    
Todos los experimentos se ejecutaron en máquinas Intel Core i7-6700 de 3.4 GHz con 8GB de RAM, utilizando GNU/Linux como sistema operativo. 

\begin{table}[t!]
\centering
{\scriptsize
\begin{tabular}{l l}
\hline
&Métodos generadores de objectos  \\
\hline
\multirow{2}{*}{\textbf{NCL}} 
 & NCLinkedList(int)  \\
 & addFirst(Object)    \\
 {\scriptsize \#API: 34} & removeFirst()  \\
\hline

\multirow{2}{*}{\textbf{UFind}} 
 & UnionFind()  \\
 & addElement(int)   \\
 {\scriptsize \#API: 9} & union(int,int)   \\
\hline

\multirow{2}{*}{\textbf{FHeap}} 
 & FibonacciHeap()  \\
 & insert(int)  \\
 {\scriptsize \#API: 7} & removeMin()  \\
\hline
\multirow{1}{*}{\textbf{BTree}} 
 & BinTree()  \\
 {\scriptsize \#API: 7} & add(int) \\
\hline

\multirow{1}{*}{\textbf{BHeap}} 
 & BinomialHeap()  \\
 {\scriptsize \#API: 10} & insert(int) \\
 & decreaseKeyValue(int,int)   \\
\hline

\multirow{5}{*}{\textbf{Lits}} 
 & Lits() \\
 & getFromPool(int) \\
 & satisfies(int)  \\
 & forgets(int) \\
 {\scriptsize \#API: 26} & setLevel(int,int)  \\
 & setReason(int)\\
\hline

\multirow{3}{*}{\textbf{Sched.}} 
 & Schedule() \\
 & addProcess(int) \\
{\scriptsize \#API: 10} & blockProcess() \\
%  & quantumExpire() & \\
%   & finishProcess() & \\

\hline

\multirow{1}{*}{\textbf{LinkedList}} 
 & LinkedList() \\
 {\scriptsize \#API: 67} & addFirst(Object)  \\
 \hline

\multirow{2}{*}{\textbf{TreeMap}} 
 & TreeMap() \\
 & put(Object,Object) \\
{\scriptsize \#API: 61} & remove(Object) \\
\hline

\multirow{2}{*}{\textbf{TreeSet}} 
 & TreeSet() \\
 & add(Object) \\
{\scriptsize \#API: 34} & remove(Object) (int) \\
\hline

\multirow{1}{*}{\textbf{HashSet}} 
 & HashSet(int,float) \\
 {\scriptsize \#API: 31} & add(Object) \\
\hline

\multirow{1}{*}{\textbf{HashMap}} 
 & HashMap(int,float)  \\
{\scriptsize \#API: 45} & put(Object,Object)  \\
\hline

\end{tabular}%
}
\caption{Ground truth de métodos generadores por clase y la cantidad de métodos en la API de cada clase.}
\label{tab:groundTruth}
\end{table}


\subsection{Parámetros y Evaluación}
Como se discutió en el capítulo \ref{cap:beapi}, es necesario configurar ciertos parámetros para el funcionamiento de nuestros algoritmos y funciones de valoración. Los parámetros más importantes que afectan la eficiencia y efectividad de nuestro algoritmo genético son la tasa de mutación y de cruce, el tamaño de la población y el número de torneos en la selección. 

Para nuestra función de evaluación de Randoop, es importante conocer el tiempo disponible para generar objetos. Ademas, Randoop  necesita de una semilla. Esta semilla en Randoop se utiliza para controlar la generación de pruebas aleatorias y permite reproducir los mismos resultados en ejecuciones posteriores si utilizas la misma semilla. Esto puede ser útil para la reproducibilidad y comparación de resultados en el contexto de pruebas de software. Los resultados de Randoop, siempre tiene una aleatoriedad que, entre otras cosas, depende de esta semilla. En nuestros experimentos, cada vez que ejecutamos nuestra técnica, elegimos aleatoriamente una semilla para poder ver como se comporta. 
En cuanto a la función de valoración de Bounded Exhaustive, es crucial determinar el alcance de las estructuras generadas. Además, debemos omitir ciertos valores de los campos de nuestras clases que son irrelevantes para la generación de objetos. Estos campos son aquellos que cambian la representación del objeto pero no su estructura en sí. Un ejemplo común en varias clases de prueba es el campo \emph{ModCount}, que es un contador de modificaciones. Esta variable se utiliza a menudo en estructuras de datos para realizar un seguimiento de las modificaciones o cambios realizados en los datos almacenados, pero claramente no afecta la estructura generada.

Para establecer estos valores en la experimentación, realizamos una tarea de prueba y error para determinar los valores que mejor se ajustan a nuestros algoritmos. Esto es comúnmente realizado en algoritmos genéticos para determinar los parámetros óptimos para un problema específico.

Los mejores valores que hemos obtenido para nuestros problemas son los siguientes:
\begin{itemize}
    \item Algoritmo Genético:
    \begin{itemize}
        \item Tasa de cruce: 0.4
        \item Tasa de mutación: 0.05
        \item Selección: Torneo (4)
        \item Tamaño de población: 100
        \item Número de evoluciones: 20
    \end{itemize}
    \item Función de valoración, Cobertura de Randoop:
    \begin{itemize}
        \item Tiempo de ejecución de Randoop: 30 segundos
        \item Seeds: 2 (aleatorias para cada ejecución)
    \end{itemize}
     \item Función de valoración, Generador BE:
    \begin{itemize}
        \item Scope: 5
        \item omisión de campos: modCount
    \end{itemize}
\end{itemize}

Para realizar la evaluación de la técnica de identificación de métodos \emph{builders}, cada resultado que se muestra en la siguiente sección corresponde a la ejecución del algoritmo 5 veces para cada función de valoración con los parámetros mencionados anteriormente. Luego de estas evaluaciones, se calcula el promedio de los datos obtenidos.
Vale aclarar que para la función de valoración que utiliza el generador exhaustivo, no hace falta ejecutarlo diferentes veces, debido a que este resultado es determinístico si se utiliza los mismos parámetros.


Para implementar el algoritmo genético, utilizamos una biblioteca muy popular en Java llamada \emph{Jenetics}\footnote{https://jenetics.io/}. Esta biblioteca está diseñada específicamente para algoritmos evolutivos y nos proporcionó las herramientas necesarias para desarrollar nuestro enfoque genético.

\emph{Jenetics} es una biblioteca robusta y versátil que ofrece una amplia gama de funcionalidades para la implementación de algoritmos genéticos. Nos permitió definir y manipular genes, cromosomas y poblaciones, así como utilizar operadores genéticos como selección, cruzamiento y mutación. Además, cuenta con un sólido conjunto de herramientas de optimización y técnicas de evolución que nos permitieron adaptar el algoritmo a nuestras necesidades específicas.
Gracias a \emph{Jenetics}, pudimos implementar el algoritmo genético de manera eficiente y efectiva, lo que nos permitió explorar y encontrar subconjuntos óptimos de métodos \emph{builders} para las diferentes estructuras de datos en nuestro estudio. Además, \emph{Jenetics} es muy fácil de usar, con una documentación completa y una comunidad activa de usuarios que proporciona soporte y ayuda. 

% necesarias para obtener las soluciones. Comparando los resultados obtenidos con la técnica evolutiva y la técnica greedy, pudimos evaluar la eficiencia de ambos enfoques y determinar cuál es más rápido en términos de tiempo de ejecución.
\subsection{Eficiencia}
La eficiencia de nuestros algoritmos es la cantidad de segundos que le lleva a cada uno de nuestros algoritmos con cada función de valoración para encontrar el subconjunto de métodos \emph{builders} suficientes. Para evaluar la eficiencia de nuestro algoritmo, llevamos a cabo experimentos utilizando conjuntos de datos de muestra y medimos el tiempo de promedio de las 5 ejecuciones necesarias para obtener las soluciones. Comparamos los resultados obtenidos con la técnica evolutiva (\emph{GA}), la técnica greedy (\emph{HC}) y la técnica de clase de equivalencia (\emph{SubSet}) para evaluar la eficiencia de ambos enfoques en función de la función de valoración utilizada. Los resultados se presentan en la Tabla \ref{tab:eficiencia}.


Como se observa en la tabla, el algoritmo Hill Climbing (\emph{HC}) es más efectivo que el algoritmo genético (\emph{GA}) y el algoritmo de subconjuntos (\emph{SubSet}) para calcular los constructores de métodos en todos los casos de estudio considerados. El enfoque de Hill Climbing es razonablemente eficiente, tardando solo 20 minutos en el peor de los casos (NCL, debido a la cantidad y complejidad de su API), mientras que el algoritmo genético puede tardar mas de una hora en el peor de los casos. Esto se debe a que los algoritmos de Hill Climbing comienzan desde abajo hacia arriba, considerando menos métodos antes de considerar más métodos, a diferencia de los algoritmos genéticos, que generan sucesores mediante operadores de cruce y mutación. Esta estrategia favorece al algoritmo Hill Climbing para encontrar los métodos mínimos en tiempos más cortos. Cabe mencionar que en casos de estudio donde el algoritmo genético supera al Hill Climbing, se debe a que la clase bajo prueba tiene pocos métodos, lo que permite que los algoritmos evolutivos converjan más rápidamente que el algoritmo Hill Climbing.

En cuanto a nuestro último algoritmo, el de subconjuntos, suele tener una menor eficiencia en casos donde la API bajo prueba tiene una gran cantidad de métodos, como se observa en el caso de \emph{LinkedList} que es el caso que tiene más métodos. En este caso puede tardar hasta 3 horas con la fitness que utiliza cobertura de ramas como métrica. Esto se debe a que el algoritmo de subconjuntos debe realizar muchas combinaciones de métodos, y cuanto mayor sea la variedad de valores de la función de valoración, más lento será su proceso de convergencia.

En relación con las funciones de valoración, observamos que la función de valoración basada en el generador exhaustivo (GE) logra finalizar la búsqueda de métodos \emph{builders} en mucho menos tiempo en comparación con la función de valoración basada en la cobertura generada por la suite de pruebas modificada del Randoop (RC). Esto se aplica a todos los algoritmos implementados. La función de valoración GE converge antes en los algoritmos debido a que genera objetos hasta el alcance especificado (en este caso, 5 para todos los casos) y luego finaliza la ejecución del candidato actual para continuar con los siguientes. Todas tienen un \emph{timeout} de 30 segundos para el caso de pueda generar muchas estrucutras con el scope dado. Esto permite una convergencia más rápida en comparación con la función de valoración basada en Randoop, que agota su presupuesto de tiempo para cada candidato ejecutado en el algoritmo. Es importante destacar que todos los algoritmos se ejecutan con 10 hilos simultáneamente, evaluando así 10 candidatos a la vez para cada algoritmo.

En conclusión, en términos de eficiencia de nuestro algoritmo, para los casos estudiados en esta tesis, es conveniente utilizar el algoritmo Hill Climbing con una función de valoración que cuente los objetos generados por el generador exhaustivo (\emph{GE}).

\setlength{\tabcolsep}{4pt} 

\begin{table}[H]
\centering
\begin{tabular}{cccccc}
\hline
\multicolumn{2}{c}{\textbf{Casos}} & \multicolumn{2}{c}{\textbf{GA}} & \multicolumn{2}{c}{\textbf{HC}} \\
\cline{3-6}
\multicolumn{2}{c}{} & \textbf{\tiny FE} & \textbf{\tiny FC} & \textbf{\tiny FE} & \textbf{\tiny FC} \\
\hline
\multicolumn{2}{c}{\textbf{NCL}}            & 292   & 3622 & 40   & 645  \\
\multicolumn{2}{c}{\tiny \#API: 34, \#MGO: 3} & & & & \\

\multicolumn{2}{c}{\textbf{UFind}}          & 41     & 700    & 16  & 267  \\
\multicolumn{2}{c}{\tiny \#API: 9, \#MGO: 3}  & & & & \\

\multicolumn{2}{c}{\textbf{FHeap}}          & 9      & 346   & 4     & 225  \\
\multicolumn{2}{c}{\tiny \#API: 7, \#MGO: 3}  & & & & \\

\multicolumn{2}{c}{\textbf{BTree}}          & 4       & 174   & 3    & 168  \\
\multicolumn{2}{c}{\tiny \#API: 7, \#MGO: 2}  & & & & \\

\multicolumn{2}{c}{\textbf{BHeap}}          & 24    & 989  & 6    & 273  \\
\multicolumn{2}{c}{\tiny \#API: 10, \#MGO: 3} & & & & \\

\multicolumn{2}{c}{\textbf{Lits}}           & 25   & 3429 & 7    & 394  \\
\multicolumn{2}{c}{\tiny \#API: 26, \#MGO: 6} & & & & \\

\multicolumn{2}{c}{\textbf{Sched.}}         & 73     & 1685 & 15   & 262  \\
\multicolumn{2}{c}{\tiny \#API: 10, \#MGO: 3} & & & & \\

\multicolumn{2}{c}{\textbf{LinkedList}}     & 69     & 1637 & 8    & 302 \\
\multicolumn{2}{c}{\tiny \#API: 67, \#MGO: 2} & & & & \\

\multicolumn{2}{c}{\textbf{TreeMap}}        & 974  & 3872 & 68  & 484  \\
\multicolumn{2}{c}{\tiny \#API: 61, \#MGO: 3} & & & & \\

\multicolumn{2}{c}{\textbf{TreeSet}}        & 75     & 2266 & 7    & 561  \\
\multicolumn{2}{c}{\tiny \#API: 34, \#MGO: 3} & & & & \\

\multicolumn{2}{c}{\textbf{HashSet}}        & 115  & 428   & 12  & 2690 \\
\multicolumn{2}{c}{\tiny \#API: 31, \#MGO: 2} & & & & \\

\multicolumn{2}{c}{\textbf{HashMap}}        & 1406 & 2566 & 51   & 1202 \\
\multicolumn{2}{c}{\tiny \#API: 45, \#MGO: 2} & & & & \\
\hline
\end{tabular}

\caption{Tiempo en segundos para hallar el subconjunto suficiente de métodos generadores.}
\label{tab:eficiencia}
\end{table}


\subsection{Efectividad}
\cacho{Poner que beapi va hasta el scope o haste el tiempo dado. Pero casi siempre es scope porq es pequeno}

La efectividad de nuestros algoritmos se basa en qué tan cerca están de encontrar el conjunto mínimo de métodos \emph{builders} para las diferentes estructuras de datos. Es importante tener en cuenta que siempre encontramos al menos un subconjunto de métodos \emph{builders} que es suficiente para su utilización. Sin embargo, cuanto más mínimo sea este subconjunto, mejores resultados obtendremos en términos de generación de inputs para testing.

Realizamos experimentos utilizando las estructuras de datos especificadas en nuestra primera sección de este capítulo. Evaluamos tres algoritmos: Hill Climbing (\emph{HC}), Subset (\emph{SubSet}). Para cada algoritmo, utilizamos dos funciones de valoración: una basada en el generador exhaustivo (\emph{BE}) y otra basada en la cobertura de ramas (\emph{RC}).
Es importante destacar que la función de valoración que utiliza la test suite generada por Randoop, la cual hemos modificado para medir la cobertura de ramas, produce resultados variables según la semilla utilizada. La semilla tiene importancia en la aleatoriedad que posee Randoop, lo que puede afectar los resultados de la función de valoración.

Para la estructura \emph{NCL}, el algoritmo \emph{HC} con la función de valoración que cuenta objectos de nuestro generador exhaustivo, \emph{BE} encuentra un subconjunto de métodos \emph{builders} que coincide con el conjunto mínimo, es decir, no agrega métodos adicionales. Sin embargo, al utilizar la función de valoración \emph{RC}, el algoritmo \emph{HC} encuentra un subconjunto con 0.6 métodos adicionales en promedio. Esto indica que en 3 de las 5 ejecuciones se agrega un método extra.
En cuanto al algoritmo \emph{SubSet}, se comporta de manera similar al algoritmo \emph{HC} para ambas funciones de valoración.
El algoritmo evolutivo, en las 5 ejecuciones, tiene encuentra un método de más con la función de valoración que utiliza cobertura, mientras que la función de valoración con el generador exhaustivo en 4 de las 5 ejecuciones encuentra un método de más. Esto se debe a que el algoritmo genético es muy sensible a los parámetros que se utiliza y puede depender de cada caso. Si se le da el tiempo, los parámetros y las evoluciones necesarias, este algoritmo siempre encontraría el mínimo sin ningún método extra. Así mismo con la función de valoración, tiene un método de más en todas las ejecuciones.


En el caso de las estructuras \emph{UnionFind, FibonacciHeap, BTree, BinomialHeap y Scheduler}, todos los algoritmos logran encontrar el conjunto mínimo de métodos \emph{builders}, ya que no se encuentran métodos adicionales en ninguno de los casos. Esto tambien tiene que ver con que tienen menos cantidad de metodos en sus API y la complejidad de los mismos es diferente al resto de las clases que forman parte de nuestro benchmarks.

En la estructura \emph{Lits}, la función de valoración basada en el generador exhaustivo siempre encuentra al menos un método adicional. Esto se debe a que el generador exhaustivo depende de los valores con los que están inicializados los arreglos en la API de List, lo que puede generar un método adicional en el conjunto mínimo de métodos \emph{builders}. 
En la API de \emph{List}, la cual, es una biblioteca Java para resolver problemas booleanos de satisfacibilidad (SAT). Esta contiene una colección de arreglos que almacena los literales utilizados en un problema SAT. Proporciona métodos para manipular y acceder a literales, como agregar literales a la estructura, recuperar literales por sus índices, negar literales, etc. Nuestro generador exhaustivo depende de con que valores estén inicializados estos arreglos, y el scope que le hemos pasado. Es por esto que agrega un método para crear arreglos que no tiene posibilidad de hacerlo con los métodos builders y el scope que se le dio. Esto no quiere decir que sea builders el método que se agrega, solo que con el scope dado, no tiene posibilidad de crearlo con métodos alternativos. Si aumentamos el scope y modificamos la clase, nos encuentra siempre el subconjunto mínimo.
En cambio, la función de valoración basada en la cobertura de ramas no depende del scope dado y nos encuentra siempre el mismo subconjunto de métodos que es el mínimo y suficiente.

Para las estructuras de \emph{java.util}, el algoritmo evolutivo evaluado a través de la cobertura de ramas depende mucho del tiempo y los parámetros utilizados. Esto se debe a que estas estructuras tienen una mayor cantidad de métodos en su API, lo que hace más desafiante encontrar el conjunto mínimo de métodos \emph{builders}. Por un lado, la función de valoración basada en el generador exhaustivo encuentra el conjunto mínimo en la mayoría de las ejecuciones, con la excepción de HashSet, que agrega un método adicional que es un constructor de la clase que permite generar estructuras que no son posibles con los otros métodos hasta el scope dado.
Por otro lado, la función de valoración de de randoop \cacho{TODO, puede no encontrar el suficiente porq satura coverage.}


En resumen, la efectividad de nuestros algoritmos varía dependiendo de la estructura de datos y la función de valoración utilizada. Algunos algoritmos y funciones de valoración logran encontrar el conjunto mínimo de métodos \emph{builders}, mientras que otros pueden agregar métodos adicionales en ciertos casos.


% La efectividad de nuestros algoritmos se basa en qué tan cerca están de encontrar el conjunto mínimo de métodos \emph{builders}. Es importante destacar que nuestros algoritmos siempre encuentran al menos un subconjunto de métodos \emph{builders} que es suficiente. Sin embargo, cuanto más mínimo sea este subconjunto, mejores resultados obtendremos al utilizarlo, por ejemplo, durante la generación de inputs para testing.

\begin{table}[H]
\centering
\label{tab:t1}
\scriptsize
\begin{tabular}{|c c|cc|cc|cc|}
\midrule
\multicolumn{2}{|c|}{\multirow{3}{*}{\textbf{Cases}}} & \multicolumn{6}{c|}{\textbf{Efectividad}} \\
\cline{3-8}
\multicolumn{2}{|c|}{} & \multicolumn{2}{c}{\textbf{GA}} & \multicolumn{2}{c}{\textbf{HC}} & \multicolumn{2}{c|}{\textbf{SubSet}} \\
\multicolumn{2}{|c|}{} & \textbf{\tiny{BE}} & \textbf{\tiny{RC}} & \textbf{\tiny{BE}} & \textbf{\tiny{RC}} & \textbf{\tiny{BE}} & \textbf{\tiny{RC}} \\
\midrule
\multicolumn{2}{|c|}{\textbf{NCL}} & 3.80  &  4 & 3 &  3.40 &3 &  3.40 \\
\multicolumn{2}{|c|}{\tiny \#API: 34} &  &   & &   & & \\
\multicolumn{2}{|c|}{\tiny \#Builders: 3} &  &   & &   & & \\

\midrule
\multicolumn{2}{|c|}{\textbf{UFInd}}& 3 & 3  & 3  & 3  &3   & 3     \\
\multicolumn{2}{|c|}{\tiny \#API: 9} &  &   & &   & & \\
\multicolumn{2}{|c|}{\tiny \#Builders: 3} &  &   & &   & & \\
\midrule

\multicolumn{2}{|c|}{\textbf{FHeap}}& 3 & 3  &  3 &  3 &  3 &  0   \\
\multicolumn{2}{|c|}{\tiny \#API: 7} &  &   & &   & & \\
\multicolumn{2}{|c|}{\tiny \#Builders: 3} &  &   & &   & & \\
\midrule
\multicolumn{2}{|c|}{\textbf{BTree}} & 2 & 2  &  2 &  2 &  2 &  2  \\
\multicolumn{2}{|c|}{\tiny \#API: 8} &  &   & &   & & \\
\multicolumn{2}{|c|}{\tiny \#Builders: 2} &  &   & &   & & \\
\midrule
\multicolumn{2}{|c|}{\textbf{BHeap}}& 3 & 3 &  3 &  3 &  3 &  3   \\
\multicolumn{2}{|c|}{\tiny \#API: 10} &  &   & &   & & \\
\multicolumn{2}{|c|}{\tiny \#Builders: 3} &  &   & &   & & \\
\midrule
\multicolumn{2}{|c|}{\textbf{Lits}} &  6 & 5 & 6 &  5  & 6  &5   \\
\multicolumn{2}{|c|}{\tiny \#API: 26} &  &   & &   & & \\
\multicolumn{2}{|c|}{\tiny \#Builders: 6} &  &   & &   & & \\
\midrule
\multicolumn{2}{|c|}{\textbf{Sched.}} &  3 & 3   & 3  &  3&  3 & 3 \\

\multicolumn{2}{|c|}{\tiny \#API: 10} &  &   & &   & & \\
\multicolumn{2}{|c|}{\tiny \#Builders: 3} &  &   & &   & & \\
\midrule\multicolumn{2}{|c|}{\textbf{LinkedList}} & 2 &  2.80 &  2 &  3 &2  &  2.80  \\
\multicolumn{2}{|c|}{\tiny \#API: 67} &  &   & &   & & \\
\multicolumn{2}{|c|}{\tiny \#Builders: 2} &  &   & &   & & \\
\midrule
\multicolumn{2}{|c|}{\textbf{TreeMap}} &  3 & \cellcolor{gray!25}\emph{5.60}  &  3 &  \cellcolor{gray!25} \emph{4}  &3 & \cellcolor{gray!25}\emph{5}   \\
\multicolumn{2}{|c|}{\tiny \#API: 61} &  &   & &   & & \\
\multicolumn{2}{|c|}{\tiny \#Builders: 3} &  &   & &   & & \\
\midrule
\multicolumn{2}{|c|}{\textbf{TreeSet}} &  0&\cellcolor{gray!25} \emph{3.20}  &  3 &   \cellcolor{gray!25}\emph{4}  & 3 & \cellcolor{gray!25}\emph{4}   \\
\multicolumn{2}{|c|}{\tiny \#API: 34} &  &   & &   & & \\
\multicolumn{2}{|c|}{\tiny \#Builders: 3} &  &   & &   & & \\
\midrule
\multicolumn{2}{|c|}{\textbf{HashSet}} &  2 &3  &  2 &  3& 2  &  3 \\
\multicolumn{2}{|c|}{\tiny \#API: 31} &  &   & &   & & \\
\multicolumn{2}{|c|}{\tiny \#Builders: 2} &  &   & &   & & \\
\midrule
\multicolumn{2}{|c|}{\textbf{HashMap}} & 2.60  & 2  &2   &2   &  2 & 2   \\
\multicolumn{2}{|c|}{\tiny \#API: 45} &  &   & &   & & \\
\multicolumn{2}{|c|}{\tiny \#Builders: 2} &  &   & &   & & \\
\hline
\end{tabular}

\caption{Tabla de casos de estudio y efectividad de encontra el subconjunto optimo}
\label{tab:efectividad}
\end{table}




% Además de analizar la eficiencia, también investigamos la sensibilidad de algunos parámetros del GA. Ajustamos parámetros como el tamaño de la población, la tasa de mutación y el número de generaciones, y observamos cómo estos cambios afectaron el rendimiento del algoritmo y los resultados obtenidos. Esto nos permitió identificar las configuraciones óptimas de los parámetros y entender cómo influyen en el proceso de búsqueda.

Adicionalmente, examinamos los usos de los builders generados por nuestra técnica. Analizamos las características y las funcionalidades de los builders aprendidos y evaluamos su utilidad en diferentes tareas. Estudiamos cómo los builders pueden ser aplicados en el análisis y la manipulación de datos, y evaluamos su efectividad en términos de rendimiento y calidad de los resultados obtenidos.



% \paragraph{Randoop Objects}
% \label{sec:randoopObjectsExp}




% \paragraph{Bounded Exhuastive}
% \label{sec:BEExp}



% \subsubsection{Variacion de acuerdo a los Parametros}

% Ademas, examinamos la sensibilidad de los parámetros del Algoritmo Genético y exploramos los usos de los builders generados. Los resultados obtenidos proporcionaron información valiosa sobre el rendimiento y la utilidad de nuestra técnica en la generación de builders y su aplicación en diversas tareas.

% Nuestra comparacion se basa en medir la cantidad de tiempo que le lleva a cada Algortimo terminar la ejecucio, en la cantidad de candidatos que evalua la funcion de valoracion y cuan bueno es en eficacia para encontrar el minimo y suficiente subconjunto de metodos que pudimos observar en nuestro ground truth \cacho{Agregar seccion}. Ejecutamos el algoritmo 10 veces con el resto de los parametros que no esta en evaluacion con un valor promedio (Crossover=0.5, mutation 0.1, Tournanament 4).
% Tambien utilizamos 30 segundos para la fitness con randoop y scope 6 para BEAPI.

% En la tabla \ref{tab:CrossOverGA} se puede observar como se comporta el algoritmo cuando se utiliza diferente rate para el operador de CrossOver. 





% \begin{table}[t!]
\centering
\begin{tabular}{l|cc|cc|cc}
\hline

\textbf{Parameters}& \multicolumn{2}{c}{\textbf{Time}} & \multicolumn{2}{c}{\textbf{Candidates}} & \multicolumn{2}{c}{\textbf{Eficacia}} \\
 &RO&BE&RO&BE&RO&BE \\

\hline

0.01 & &  &\\

0.5 & &  &\\
0.1 & &  &\\
0.1 & &  &\\
\midrule

\end{tabular}%
\caption{Diferentes valores de Mutatacion}
\label{tab:MutationGA}
\end{table}

\begin{table}[t!]
\centering
\begin{tabular}{l l ccc}
\hline

\textbf{Parameters}& \textbf{Time} & \textbf{ObjectsCalculate} & \textbf{Eficacia} \\
\midrule

0.01 & &  &\\

0.5 & &  &\\
0.1 & &  &\\
1 & &  &\\
\midrule

\end{tabular}%
\caption{Diferentes valores de mutatacion}
\label{tab:CrossOverGA}
\end{table}

\begin{table}[t!]
\centering
\begin{tabular}{l l ccc}
\hline

\textbf{Parameters}& \textbf{Time} & \textbf{ObjectsCalculate} & \textbf{Eficacia} \\
\midrule

3 & &  &\\

4 & &  &\\
5 & &  &\\
6 & &  &\\
\midrule

\end{tabular}%
\caption{Diferentes valores de Tournaments para seleccion}
\label{tab:SelectionGA}
\end{table}
\subsection{Uso de Builders para generar inputs de programas}
En esta parte de la evaluación nos referimos a cuán útiles son los métodos \emph{builders} identificados en el contexto de un análisis de programa, en particular, la generación automatizada de casos de prueba. Estos objetos podrían utilizarse, por ejemplo, como entradas en pruebas unitarias parametrizadas. Para los estudios de caso que proporcionan mecanismos para medir el tamaño de los objetos y comparar objetos por igualdad (es decir, los métodos size y equals de las estructuras de datos), generamos pruebas con Randoop utilizando todos los métodos disponibles en la API (API), y luego generamos pruebas con Randoop utilizando solo los métodos \emph{builders} (BLD) identificados por nuestro enfoque en el experimento anterior (Tabla \ref{tab:results-compute-bld}). Luego, comparamos el número de objetos diferentes (No. of Objs.) y el tamaño del objeto más grande (Max Obj. Size) creado por las pruebas generadas a partir de la API, en comparación con las pruebas generadas utilizando solo los métodos de BLD. Configuramos tres presupuestos diferentes para la generación de pruebas: 60, 120 y 180 segundos (Budget). Los resultados se resumen en la Tabla \ref{tab:results-obj}. Los resultados muestran que, en el mismo presupuesto de pruebas, BLD genera en promedio un 500 porciento más de objetos que API. En todos los casos, BLD también genera objetos significativamente más grandes que API. A la luz de estos resultados, queda claro que la identificación automatizada de \emph{builders} vale la pena para la generación automatizada de estructuras para clases con estado.
\begin{table}[H]
\centering
\scriptsize
\begin{tabular}{ c l c c}
\hline
Class & Budget &
\multicolumn{2}{c}{\textsf{No. of Objs}} \\
&& \tiny{\textbf{Builders}} & \tiny{\textbf{AllMethods}} \\
\hline
\multirow{3}{*}{\textbf{NCL}} 
&	60	&	6648	&	470	\\
&	120	&	9436	&	612	\\
&	180	&	11441	&	703	\\
\hline
\multirow{3}{*}{\textbf{UFind}} 
&	60	&	1033	&	372	\\
&	120	&	1342	&	483	\\
&	180	&	1534	&	555	\\
\hline
\multirow{3}{*}{\textbf{FibHeap}}
&	60	&	6541	&	1766	\\
&	120	&	9270	&	2347	\\
&	180	&	10923	&	2745	\\
\hline
\multirow{3}{*}{\textbf{RBT}}
&	60	&	2634	&	515	\\
&	120	&	3410	&	611	\\
&	180	&	3938	&	676	\\
\hline
\multirow{3}{*}{\textbf{BTree}}
&	60	&	2937	&	975	\\
&	120	&	3820	&	1196	\\
&	180	&	4367	&	1354	\\
\hline
\multirow{3}{*}{\textbf{BHeap}}
&	60	&	6455	&	971	\\
&	120	&	8665	&	1230	\\
&	180	&	10093	&	1401	\\
\hline
\multirow{3}{*}{\textbf{Lits}}
&	60	&	3968	&	3174	\\
&	120	&	5109	&	4142	\\
&	180	&	5848	&	4783	\\
\hline
\multirow{3}{*}{\textbf{Schedule}}
&	60	&	2176	&	2901	\\
&	120	&	2756	&	2901	\\
&	180	&	3140	&	3437	\\
\hline
\multirow{3}{*}{\textbf{LinkedList}} 
&	60	&	8121	&	790	\\
&	120	&	11503	&	1095	\\
&	180	&	13905	&	1323	\\
\hline
\multirow{3}{*}{\textbf{TreeMap}} 
&	60	&	2750	&	748	\\
&	120	&	3754	&	953	\\
&	180	&	4496	&	1107	\\
\hline\multirow{3}{*}{\textbf{TreeSet}}
&	60	&	1129	&	291	\\
&	120	&	1527	&	343	\\
&	180	&	1816	&	381	\\
\hline
\multirow{3}{*}{\textbf{HashSet}}
&	60	&	8208	&	1498	\\
&	120	&	11467	&	2008	\\
&	180	&	13548	&	2366	\\
\hline
\multirow{3}{*}{\textbf{HashMap}}
&	60	&	9581	&	2103	\\
&	120	&	13044	&	3173	\\
&	180	&	15514	&	3784	\\
\hline

\end{tabular}%

\caption{Evaluación del uso de los builders identificados (BLD) frente a toda la API (API) en la generación de casos de test.}
\label{tab:results-obj}
\end{table}


\subsection{Uso de Builders en Verificación}

En el ultimo experimento sobre la identificación de \emph{builders}, utilicé Java PathFinder \cite{Visser:2005} (JPF) para realizar pruebas de generación de entradas de software para estructuras de datos de \emph{java.util}. JPF \footnote{https://github.com/javapathfinder/jpf-core}  es un verificador de modelos de estado explícito para programas escritos en Java. Para realizar la verificación, las técnicas de versificación de modelos de software se basan en la definición de controladores de métodos: combinaciones de métodos que permiten construir las entradas con las que se ejecutará el programa. Intuitivamente, es deseable seleccionar el menor conjunto de métodos posible, cuyas combinaciones permitan construir todas las estructuras acotadas para el módulo (para analizar el software con todas las entradas posibles). La dificultad de escribir controladores de pruebas es un obstáculo importante para el uso de un verificador de modelos. Esta selección de métodos, que generalmente se realiza manualmente, no es una tarea fácil: requiere un análisis exhaustivo de las rutinas disponibles en el módulo y una comprensión profunda de su semántica.
Es posible construir un método no determinista (harness de test) que genere todas las secuencias de llamadas a métodos de la API hasta un tamaño especificado por el usuario (scope). JPF se utiliza para enumerar todas estas secuencias. JPF almacena todos los estados explorados y retrocede cuando visita un estado previamente explorado.

JPF admite anotaciones de programa que se agregan a los programas a través de llamadas a métodos de una clase especial Verify.
Utilizamos los siguientes métodos de la biblioteca JPF \verb"Verify":
\\
\begin{itemize}
\item El método \verb"Verify.getInt(int lo, int hi)" devuelve un valor entre \verb"lo" y \verb"hi", inclusive. Crea un punto de elección no determinista: JPF necesita explorar las ejecuciones para todos los valores en el rango.
\item \verb"random(int n)" devuelve valores de 0 a \verb"n", de manera no determinista.
\end{itemize}

Si se desea verificar que el método \emph{put}  de TreeMap cumple con el repOK (predicado imperativo que verifica las invariantes de clase) de la estructura de datos, es necesario escribir algo como: 
\\

\begin{lstlisting}[caption={Probando el método put de TreeMap con JPF},label={lst:label},language=Java,captionpos=b]
public static void main(String[] args) {
   int scope = 3;
   TreeMap t = generateStructure(scope);
   t.put(Verify.getInt(0,scope),Verify.getInt(0,scope));
   assert t.repOK();
}
\end{lstlisting}

Para realizar el análisis de esta propiedad, es necesario proporcionar a JPF los mecanismos para generar todo el árbol de entrada (\textit{generateStructure}). 
En el siguiente ejemplo, mostramos un controlador de prueba construido con todos los métodos de la estructura de datos \textit{TreeMap}:
\\
\\
\begin{lstlisting}[caption={Controlador con todos los métodos},label={lst:driverAPI},language=Java,captionpos=b]
private static TreeMap generateStructure(int scope) {
   int maxLength = Verify.getInt(0, scope);
   TreeMap t = new TreeMap();
   for (int i = 1; i <= maxLength; i++) {
      switch (Verify.random(n_methods)) {
         case 0:
            t.put(Verify.getInt(0,scope),Verify.getInt(0,scope));
            break;
         case 1:
            t.remove(Verify.getInt(0,scope));
            break;						
         case 2:
            t.clear();
            break;
         case 3:
            t.containsValue(Verify.getInt(0,scope));
            break;
         ...
         case 11: 
            t.putAll(l);
            break;
      }
   }
   return t;
}
\end{lstlisting}

El método controlador anterior, en primer lugar, selecciona el número de métodos a ejecutar, \textit{maxLength}, un número entre 0 y \textit{scope}. (Linea 2). Cada iteración del ciclo (Linea 4 a 23) corresponde a la ejecución de un solo método, seleccionado de manera no determinista entre todos los disponibles. En el caso de que el usuario no conozca el conjunto de métodos builders (y no quiera hacer el complicado trabajo de seleccionarlos manualmente), la solución más segura para evitar descartar métodos importantes es utilizar todos los métodos disponibles en el módulo, como se muestra en el método controlador descrito anteriormente. En el cuerpo del bucle, a cada método se le asigna un número entero entre 0 y \textit{n\_methods}. Se elige de manera no determinista el método a ejecutar en el ciclo actual. Por ejemplo, si \textit{n\_methods}=1, se ejecuta el método \textit{remove}. Es fácil ver que el número de ejecuciones posibles que se deben explorar por JPF crece exponencialmente con el número de métodos disponibles.

%(for every method that is executed in one iteration, there are \textit{n_methods} possible methods to execute in the next iteration).

Para evitar este crecimiento exponencial en este experimento, se propone utilizar únicamente los constructores detectados por nuestro enfoque explicado en la sección \ref{sec:builders}.
\\
\\
\begin{lstlisting}[caption={Controlador con métodos constructores},label={lst:driverBLD},language=Java,captionpos=b]
private static TreeMap generateStructure(int scope) {
   int maxLength = Verify.getInt(0,scope);
   TreeMap t = new TreeMap();
   for (int i = 1; i <= maxLength; i++) {
      switch (Verify.random(11)) {
         case 0:
            t.put(Verify.getInt(0,scope),Verify.getInt(0,scope));
            break;
         case 1:
            t.remove(Verify.getInt(0,scope));
            break;						
      }
   }
   return t;
}
\end{lstlisting}

Como se muestra en el controlador anterior, solo 2 métodos componen un conjunto mínimo y suficiente para construir controladores para TreeMap de \textit{java.util}. Utilizando solo esos métodos, generamos exactamente los mismos objetos TreeMap que antes (porque los constructores son suficientes y mínimos), y por lo tanto JPF explora las mismas ejecuciones de propiedades con ambos controladores.

Los resultados de la tabla \ref{tab:results-jpf1} y \ref{tab:results-jpf2} muestran que la construcción del controlador a partir de nuestro enfoque presentado permite aumentar la eficiencia y escalabilidad a estructuras más grandes en el análisis utilizando JPF. Esto se debe a la reducción de los métodos utilizados en los controladores (evitando métodos superfluos) y manteniendo la capacidad de construir todos los objetos acotados posibles (debido a la suficiencia de los métodos elegidos) en menos tiempo.


\begin{table}
\scriptsize
\begin{tabular}{ c l c c c c c}
\hline
Clase & Método & Scope &
\multicolumn{2}{c}{\textsf{MGO}} &
\multicolumn{2}{c}{\textsf{API}} \\
&&&
\tiny{\textbf{estados}} & \tiny{\textbf{tiempo (S)}} &
\tiny{\textbf{estdaods}} & \tiny{\textbf{tiempo (S)} }\\
\hline
\multirow{14}{*}{LinkedList} 
& add
  & 1 & 11  & 0 & 42  & 0 \\
& & 2 & 58  & 0 & 715 & 0 \\
& & 3 & 453 & 0 & 11064 & 2 \\
& & 4 & 4881  & 1 & 153247  & 14  \\
& & 5 & 67183 & 13  & 2291803 & 171 \\
& & 6 & 1120932 & 221 & 39759491  & 2771  \\
& & 7 & 21913097  & 4588  &TO  & \\

\cline{2-7}
 &remove 
  & 1 & 11  & 0 & 42  & 0 \\
& & 2 & 58  & 0 & 715 & 0 \\
& & 3 & 453 & 0 & 11064 & 2\\
& & 4 & 4881  & 1 & 153247  & 14\\
& & 5 & 67183 & 11  & 2291803 & 168\\
& & 6 & 1120932 & 188 & 39759491  & 2713\\
& & 7 & 21913097  & 3998  & TO \\ 

\hline
\multirow{10}{*}{TreeMap} 
% Check
& put
  & 1 & 45  & 0 & 112 & 0 \\
& & 2 & 991 & 0 & 2821  & 0 \\
& & 3 & 16071 & 2 & 71678 & 8 \\
& & 4 & 552807  & 75  & 1757976 & 204 \\
& & 5 & 20601447  & 3007  & TO& \\
\cline{2-7}

& remove
  & 1 & 25  & 0 & 61  & 0 \\
& & 2 & 406 & 0 & 1345  & 0 \\
& & 3 & 8359  & 1 & 33845 & 3 \\
& & 4 & 182732  & 19  & 768564  & 66  \\
& & 5 & 7176075 & 713 & 24775111  & 2341  \\



%  && 5 &  & MO & - & - &MO  &  &-  & -  \\


\hline
\multirow{19}{*}{TreeSet} 
&put
  & 1 & 11  & 0 & 48  & 0 \\
& & 2 & 49  & 0 & 595 & 0 \\
& & 3 & 197 & 0 & 5275  & 1 \\
& & 4 & 806 & 0 & 36427 & 4 \\
& & 5 & 3115  & 0 & 204877  & 19  \\
& & 6 & 12062 & 1 & 1024038 & 90  \\
& & 7 & 47241 & 5 & 4966224 & 452 \\
& & 8 & 501666  & 45  & 23976826  & 2094  \\
& & 9 & 2047285 & 183 &TO  &  \\
\cline{2-7}

& remove
  & 1 & 17  & 0 & 48  & 0 \\
& & 2 & 88  & 0 & 595 & 0 \\
& & 3 & 423 & 0 & 5275  & 1 \\
& & 4 & 1842  & 0 & 36427 & 4 \\
& & 5 & 7455  & 1 & 204877  & 19  \\
& & 6 & 30197 & 3 & 1024038 & 88  \\
& & 7 & 122717  & 10  & 4966224 & 431 \\
& & 8 & 501666  & 43  & 23976826  & 2088  \\
& & 9 & 2047285 & 181 & TO    \\
& & 10  & 8182166 & 719 &    TO  \\
& & 11  & 31473779  & 2738  &  TO     \\
\hline

\end{tabular}%
\caption{Tiempo de ejecución y estados explorados por JPF utilizando los
enfoques API y MGO para construir los drivers.}
\label{tab:results-jpf}
 \end{table}


\begin{table}[H]
\scriptsize
\begin{tabular}{ c| l| c c c c c}
\hline
Class & Method & Scope &
\multicolumn{2}{c}{\textsf{Builders}} &
\multicolumn{2}{c}{\textsf{AllMethods}} \\
&&&
\tiny{\textbf{states}} & \tiny{\textbf{time (S)}} &
\tiny{\textbf{states}} & \tiny{\textbf{time (S)} }\\
\hline
\multirow{11}{*}{HM} 
& put
 & 1 & 21& 0& 100 & 0 \\
& & 2 & 136& 0& 235 & 0 \\
& & 3 & 2945& 0& 59494 & 9 \\
& & 4 & 64626& 11&  1808536& 299 \\
& & 5 & 1512217& 267& TO&  \\

 \cline{2-7}

& remove
 & 1 &25 & 0& 56 & 0 \\
& & 2 &325 & 0& 1144 &  0\\
& & 3 & 5479& 1& 23966 & 3 \\
& & 4 & 105607& 10& 627361 &  78\\
& & 5 & 2289075& 213& 22547086 &3011  \\
& & 6 & 55335111 & 5017&TO &  \\

\hline

\multirow{14}{*}{HS} 
& put
  & 1 & 11  & 0 & 30  & 0 \\
& & 2 & 40  & 0 & 248 & 0 \\
& & 3 & 149 & 0 & 1405  & 0 \\
& & 4 & 531 & 0 & 6259  & 1 \\
& & 5 & 1765  & 0 & 24107 & 3 \\
& & 6 & 5496  & 0 & 84617 & 8 \\
& & 7 & 16217 & 2 & 278241  & 24  \\


\cline{2-7}

& remove
  & 1 & 11  & 0 & 30  & 0 \\
& & 2 & 40  & 0 & 248 & 0 \\
& & 3 & 149 & 0 & 1405  & 0 \\
& & 4 & 531 & 0 & 6259  & 1 \\
& & 5 & 1765  & 0 & 24107 & 3 \\
& & 6 & 5496  & 0 & 84617 & 8 \\
& & 7 & 16217 & 1 & 278241  & 24  \\

\hline

\end{tabular}%
\caption{Estados generados JPF utilizando el método driver con todos los métodos de la API vs. solo utilizando los métodos builders.}
\label{tab:results-jpf2}
 
\end{table}





\section{BEAPI}

\subsection{Eficiencia}

\begin{table}[H]
\begin{center}
\tiny
\renewcommand{\arraystretch}{0.89} % Default value: 1
\begin{tabular}{clr|rr|rr|rr}
\toprule
& \textbf{Class} & \textbf{S} &
\multicolumn{2}{c}{\textbf{Time}} & \multicolumn{2}{c}{\textbf{Generated}} & \multicolumn{2}{c}{\textbf{Explored}} \\
&&& \multicolumn{1}{c}{\textbf{Korat}} & \multicolumn{1}{c}{\textbf{BEAPI}} & \multicolumn{1}{c}{\textbf{Korat}} & \multicolumn{1}{c}{\textbf{BEAPI}} & \multicolumn{1}{c}{\textbf{Korat}} & \multicolumn{1}{c}{\textbf{BEAPI}} \\
\midrule 
\multirow{18}{*}{\rotatebox[origin=c]{90}{\textbf{KORAT}}}
& DLList


    &   6   &  0.24     & 7.11     & 55987 & 55987 & 521904 & 335930 \\

&   &	7	&	2.31	&	\textbf{108.08} & 960800 & 960800 &	9875550	&	6725609	\\
%&	&	8	&	50.59	& TO & 19173961	& &	215616158 &	\\
&	&	9	&	\textbf{1333.88}	& TO & 435848050 & & 5325611829	& \\
\cmidrule{2-9}
 &{\textbf{FibHeap}}
	&	6	&	1.26 &	5.95 &	573223	&	54159 & 1641562 &	379125	\\
    &	&	7	&	32.87 &	\textbf{115.44} & 17858246 & 898394 & 54268866 & 7187167	\\
    &	&	8	&	\textbf{1415.77}	& TO & 654214491 & & 2105008180	&		\\
\cmidrule{2-9}
 &BinHeap

                       
      &   7   &   0.26     &  25.32      & 107416 &107416 & 261788 &859337 \\  
    &    &   8   &   0.85     &  \textbf{163.39}     & 603744 &603744 & 1323194&5433706\\

%&	&	9	&	7.65	& TO &	8746120	& & 11778107	&	\\
%&	&	10	&	107.30	& TO & 117157172 &	& 150727471	 &		\\
    &	&	11	&	\textbf{2558.32}	& TO & 2835325296 & & 2985116257 &		\\
\cmidrule{2-9}
 &BST 
	&	10	&	131.18	& 49.10 & 223191	& 223191 &	216680909 &	2231922	\\
    &	&	11	&	\textbf{1137.17}	& 199.46 &	974427	& 974427 & 1679669258 & 10718710	\\
    &	&	12	&	TO	    & \textbf{1341.86} &		& 4302645	& &	51631754	\\
\cmidrule{2-9}
 &SLList 
	&	7	&	5.76 & 17.87 & 137257 & 137257 & 2055596 & 960807	\\
    &	&	8	&	8.16 & \textbf{256.49} & 2396745	& 2396745 & 40701876 &	19173969	\\
    &	&	9	&	\textbf{190.45}	& TO & 48427561	& &	919451065 & \\
\cmidrule{2-9}
\hightlight
 &{\textbf{RBT}}
	&	11	&	40.54	& 33.42 &	51242	& 39942 &	53141999 & 878743	\\
\hightlight
&	&	12	&	220.77	& 79.45 & 146073	&	112237 & 276868584 &	2693710	\\
\hightlight
    &	&	13	&	\textbf{1277.67}	& \textbf{689.06} & 428381	&	314852 & 1454153331	&	8186175	\\
\midrule
 &BinTree 
	&	10	&	73.73	&	51.34 & 223191	& 223191 &	218675679 & 2231922	\\
    &	&	11	&	\textbf{634.114}	&	265.57 & 974427	& 974427 & 1689480455 &	10718710	\\
    &	&	12	&	TO	& \textbf{1578.72} &		&	4302645 &	& 51631754	\\
\cmidrule{2-9}
\hightlight
 &AVL

    &   10  &   163.50  &  1.92	 &  7393 & 7393 & 349178307 & 73942 \\
\hightlight

    &	&	11	&	\textbf{1271.23}	&	5.80 & 20267 & 20267 &	2504382415 &	222950	\\
\hightlight
%&	&	12	&	TO	& 14.58 &		&	54761	&		&	657146	\\


\hightlight
\multirow{-5}{*}{\rotatebox[origin=c]{90}{\textbf{FAJITA}}} 
&	&	13	&	TO	& \textbf{45.45}	&	&	145206	&	&	1887693	\\

\cmidrule{2-9}
 &{\textbf{RBT}}
	&	11	&	58.74	& 19.72 &	51242	& 39942 &	75814869 &	878743	\\
&	&	12	&	318.57	& 63.16 & 146073 & 112237 	&	385422689	&	2693710	\\
&	&	13	&	\textbf{1779.83}	& \textbf{206.66} &	428381	& 314852 &	1957228527	&	8186175	\\
\cmidrule{2-9}
 &BinHeap 

    &   7   & .77	& 44.452 & 107416 & 107416 & 1447594 & 859337 \\
&   &	8	&	5.96	& \textbf{97.08} &	603744	& 603744 & 13329584	&	5433706	\\
%&	&	9	&	65.21	& TO &	8746120	& &	139623323	&		\\
&	&	10	&	\textbf{1174.91}	& TO &	117157172 &	&	2064639445	&		\\
% \cmidrule{2-9}
%  &SLList 
% % 	&	3	&	0.118	&	13	&	28	&	0.02	&	13	&	43	\\
% % &	&	4	&	0.1	&	85	&	127	&	0.11	&	85	&	345	\\
% % &	&	5	&	0.13	&	781	&	996	&	0.33	&	781	&	3911	\\
% % % &	&	6	&	0.15	&	9331	&	11224	&	1.58	&	9331	&	55993	\\
% % 	&	7	&	0.24	&	137257	&	160168	&	15.38	&	137257	&	960807	\\
% 	&	8	&	1.36	&	2396745	&	2739181	&	297.41	&	2396745	&	19173969	\\
% &	&	9	&	30.47	&	48427561	&	54481060	&	TO	&		&		\\
% &	&	10	&	681.89	&	1111111111	&	1234567966	&	TO	&		&		\\
% \cmidrule{2-9}
%  &DLList 
% % 		&	3	&	0.101	&	13	&	42	&	0.01	&	13	&	43	\\
% % &	&	4	&	0.105	&	85	&	152	&	0.05	&	85	&	345	\\
% % &	&	5	&	0.113	&	781	&	1035	&	0.17	&	781	&	3911	\\
% % &	&	6	&	0.129	&	9331	&	11280	&	0.83	&	9331	&	55993	\\
% 	% &	7	&	0.254	&	137257	&	160244	&	12.14	&	137257	&	960807	\\
% 	&	8	&	1.734	&	2396745	&	2739280	&	269.47	&	2396745	&	19173969	\\
% &	&	9	&	32.748	&	48427561	&	54481185	&	TO	&		&		\\
% &	&	10	&	802.047	&	1111111111	&	1234568120	&	TO	&		&		\\
% \cmidrule{2-9}
%  &NCL 
% 		% &	3	&	0.12	&	102	&	4344	&	0.06	&	18	&	114	\\
% 	% &	4	&	0.16	&	1252	&	64269	&	0.26	&	112	&	904	\\
% 	&	5	&	0.63	&	18555	&	1205794	&	0.84	&	975	&	9760	\\
% &	&	6	&	7.12	&	324726	&	26462391	&	4.80	&	11196	&	134364	\\
% &	&	7	&	182.25	&	6565468	&	657989516	&	58.75	&	160132	&	2241862	\\
\midrule 
\hightlight
&	AVL

     &   5   &  3.54   &  0.05 & 1107 & 62 &  12277946 & 317 \\
\hightlight
&   &   6   &  \textbf{213.63} &  .009 & 3969 & 157 & 701862289 &  950 \\
\hightlight
%&	&	11	&	TO	& 5.11 &		&	20267	&	&	222950	\\
\hightlight
%&	&	12	&	TO	& 19.92	&		&	54761	&	&	657146	\\
\hightlight
&	&	13	&	TO	& \textbf{46.71}	&		&	145206	&	&	1887693	\\
\cmidrule{2-9}	
\hightlight
&	NCL														
	&	6	&	0.65	& 2.27 &	800667	& 11196 &	805921	& 134364	\\
\hightlight
&	&	7	&	8.797	& 33.89 & 	2739128	&	160132 & 16443824 &	2241862	\\
\hightlight
&	&	8	&	\textbf{205.596}	& \textbf{769.63} &	381367044	& 2739136 &	381381493 & 43826192	\\
\cmidrule{2-9}															
\hightlight
  &	BinTree


    &   3   &   0.173	& 0.02	& 65376 & 15 & 65596 &50 \\
\hightlight

&   &   4   &  \textbf{37.546}	& 0.05  & 121853251 & 51 & 121855507 & 210 \\
%\hightlight

%&	&	10	&	TO	& 34.72		&		&	223191	&		&	2231922	\\
%\hightlight
%&	&	11	&	TO	& 181.11		&		&	974427	&		&	10718710	\\
\hightlight
&	&	12	&	TO	& \textbf{966.41}		&		&	4302645	&		&	51631754	\\
\cmidrule{2-9}															
&	LList		
	&	7	&	0.51 & 12.62 	&	137257	&	137257 & 1410799 &	960807	\\
&	&	8	&	7.64 & \textbf{295.94}	&	2396745	&	2396745 & 26952027	&	19173969	\\
&	&	9	&	\textbf{176.69}	& TO &	48427561	& &	591734656	&		\\
\cmidrule{2-9}															
\hightlight
&	{\textbf{RBT}}												
	&	11	&	69.87	& 31.02 &	51242	& 39942 &	75814869 &	878743	\\
\hightlight
&	&	12	&	361.88	& 81.03 &	146073	& 112237 &	385422689 &	2693710	\\
\hightlight
&	&	13	&	\textbf{2007.29}	& \textbf{697.06} &	428381	& 314852 &	1957228527 & 8186175	\\
\cmidrule{2-9}															
\hightlight
&	{\textbf{FibHeap}}

%    &  3    & 0.214	    & 0.04 & 1317           &48    & 39213        & 199 \\
%\hightlight

   &  4    & 1.851	    &0.13  &   131444       &  335 & 5681553 & 1683\\

\hightlight

&	&	5	&	\textbf{346.275}	& 0.70 &	21629930	& 4381 &	1295961583	&	26297	\\
%\hightlight
%&	&	6	&	TO	&	6.51	&		&	54159	&		&	379125	\\

\hightlight
\multirow{-18}{*}{\rotatebox[origin=c]{90}{\textbf{ROOPS}}}
&	&	7	&	TO	&	\textbf{129.01}	&		&	898394	&		&	7187167	\\

\cmidrule{2-9}															
&	BinHeap														
	&	6	&	1.04	& 1.31 &	7602	& 7602 &	3202245	&	53222	\\
&	&	7	&	17.47	& 13.06 & 	107416	& 107416 &	64592184 & 859337	\\
&	&	8	&	\textbf{448.48}	& \textbf{96.94} &	603744	& 603744 &	1483194820	&	5433706	\\
\midrule
&	BST	


    &   11  &   12.184  & 204.83 & 974427 & 974427 & 62669069 & 10718710\\
&	&	12	&	65.305	& \textbf{1235.67} &	4302645	& 4302645 &	308229505 &	51631754	\\
%&	&	13	&	326.524	& TO & 19181100	& &	1514612776	&		\\
&	&	14	&	\textbf{1751.4}	& TO & 86211885	& &	7438853941  &		\\
\cmidrule{2-9}															
&	DLL	
	&	7	&	0.614 & 18.09	&	137257	& 137257 &	2326622	&	960807	\\
&	&	8	&	9.824 &	\textbf{257.42} & 2396745	& 2396745 &	45449534 &	19173969	\\
&	&	9	&	\textbf{245.787}	& TO &	48427561 & 	&	1015587001	&		\\
% \cmidrule{2-9}															
% &	DisjSet	
% % 	&	3	&	0.079	&	4	&	18	&	0.02	&	4	&	45	\\
% % &	&	4	&	0.08	&	20	&	108	&	0.06	&	20	&	336	\\
% % &	&	5	&	0.085	&	145	&	915	&	0.30	&	145	&	3650	\\
% % &	&	6	&	0.099	&	1441	&	10428	&	1.94	&	1441	&	51912	\\
% 	% &	7	&	0.205	&	18248	&	149422	&	28.94	&	18248	&	894201	\\
% 	&	8	&	0.812	&	280392	&	2567512	&	TO	&		&		\\
% &	&	9	&	10.568	&	5063361	&	51316011	&	TO	&		&		\\
% &	&	10	&	248.014	&	105063361	&	1168128900	&	TO	&		&		\\
\cmidrule{2-9}															
&	{\textbf{RBT}}


    &   7   &  10.76 &	0.78	& 911   &   561     & 44832139  & 7866\\
&   &   8   &  \textbf{283.33} &	1.57	&  2489 &   1657    & 1044561963    & 26526\\
%&   &   9   &   TO    & 4.12	&       &   4748    &               & 85478 \\


%&	&	10	&	TO	&	11.39	&		&	14077	&		&	281558	\\
%&	&	11	&	TO	&	34.88	&		&	39942	&		&	878743	\\
&	&	12	&	TO  &	\textbf{84.51} 	&		&	112237	&		&	2693710	\\
\cmidrule{2-9}															
\hightlight
&	DisjSetFast


    &   6   &   0.198	& 0.89	& 52165     & 544 & 117456      & 22890 \\
\hightlight
&	&	7	&	1.209	& \textbf{8.26} &  1545157	& 4397 &	3398383	&	246288	\\
%\hightlight
%&	&	8	&	32.075	& TO &	54251909	& &	117487014	&		\\


\hightlight
\multirow{-6}{*}{\rotatebox[origin=c]{90}{\textbf{KIASAN}}}

&	&	9	&	\textbf{1402.376} & TO	&	2201735557	& &	4715569321	&		\\
\cmidrule{2-9}															
&	StackList

    &   6   &   0.128	& 4.35	& 55987 & 55987 & 56008 & 335930 \\
&	&	7	&	0.517	& \textbf{83.06} &	960800	& 960800 &	960828	&	6725609	\\
%&	&	8	&	7.952	& TO	& 19173961	&	& 19173997	&		\\
&	&	9	&	\textbf{212.919}	& TO	& 435848050	&	& 435848095	&		\\

% \cmidrule{2-9}															
% &	StackAr
% % 	&	3	&	0.091	&	57	&	63	&	0.05	&	57	&	303	\\
% % &	&	4	&	0.091	&	586	&	625	&	0.23	&	586	&	3548	\\
% 	% &	5	&	0.101	&	7465	&	7775	&	1.07	&	7465	&	52305	\\
% 	&	6	&	0.183	&	114381	&	117648	&	13.58	&	114381	&	915120	\\
% &	&	7	&	0.596	&	2097151	&	2097151	&	TO	&		&		\\
% &	&	8	&	9.885	&	43046720	&	43046720	&	TO	&		&		\\
\cmidrule{2-9}															
\hightlight
&	BHeap
	&	7	&	0.654	& 53.78 &	3206861	& 458123 &	3221407	& 3665089	\\
\hightlight
&	&	8	&	8.98	& \textbf{1221.59} &	64014472 & 8001809 & 64124432 &	72016409	\\
\hightlight
&	&	9	&	\textbf{202.804}	& TO &	1447959627	& &	1449279657	&		\\
\cmidrule{2-9}															
&	{\textbf{TreeMap}}


    &   5   &   .55     & 24.95	 & 40526 & 34276 & 162375   &1028287\\

&	&	6	&	2.85	& \textbf{866.71} &	1207261	& 1098397 &	3381725	&	46132686	\\
%&	&	7	&	74.27	& TO &	43539441	& &	88359041 &		\\
&	&	8	&	\textbf{1980.70}	& TO &	1626500673	& &	2671020961	&		\\
\bottomrule

\end{tabular}%
\end{center}
\caption{Eficiencia de BEAPI comparado con Korat}

\label{table:korat-beapi}

\end{table}

 


\subsection{Impacto de las Optimizaciones realizas en BEAPI}



\begin{table}[H]
\scriptsize
\centering
\begin{tabular}{ l r  r  r  r  r  }
  \toprule
  \multicolumn{6}{c}{\textbf{Real World}} \\
  \midrule 
  \textbf{Class} & \textbf{Scope} & \textbf{CI/MGO} & \textbf{CI}  & \textbf{MGO} & \textbf{NoOPT}  \\
  \midrule
  NCL
&	3	&	.10	&	.47	&	-	&	-	\\
&	4	&	.41	&	3.48	&	-	&	-	\\
&	5	&	3.33	&	-	&	-	&	-	\\
&	6	&	73.78	&	-	&	-	&	-	\\
  \midrule
  TSet
&	3	&	.03	&	.07	&	56.82	&	-	\\
&	4	&	.06	&	.13	&	-	&	-	\\
&	5	&	.11	&	.22 	&	-	&	-	\\
&	6	&	.17	&	.42	&	-	&	-	\\
&	7	&	0.31	&	.91	&	-	&	-	\\
&	8	&	0.74	&	2.66	&	-	&	-	\\
&	9	&	2.23	&	7.80	&	-	&	-	\\
&	10	&	6.88	&	26.34	&	-	&	-	\\
&	11	&	21.52	&	86.06	&	-	&	-	\\
&	12	&	69.98	&	276.85	&	-	&	-	\\
&	13	&	226.66	&	887.83	&	-	&	-	\\
    \midrule
  TMap
&	3	&	.11	&	.25	&	-	&	-	\\
&	4	&	.75	&	2.36	&	-	&	-	\\
&	5	&	15.97	&	57.64	&	-	&	-	\\
&	6	&	839.87	&	2901.37	&	-	&	-	\\
  \midrule
  LList
&	3	&	.02	&	.13	&	.64	&	-	\\
&	4	&	.06	&	.38	&	-	&	-	\\
&	5	&	.20	&	3.80	&	-		&	-	\\
&	6	&	.96	&	258.85	&	-	&	-	\\
&	7	&	12.98	&	-	&	-	&	-	\\
&	8	&	286.21	&	-	&	-	&	-	\\
  \midrule
  HMap
&	3	&	.10	&	11.49	&	-	&	-	\\
&	4	&	.55	&	-	&	-	&	- \\
&	5	&	5.33	&	-	&	-	&	-	\\
&	6	&	119.87	&	-	&	-	&	-	\\
  \midrule
  Schedule
&	3	&	.01	&	.01	&	59.27	&	-	\\
&	4	&	.82	&	45.55	&	-	&	-	\\
&	5	&	1.43	&	-	&	-	&	-	\\
&	6	&	06.01	&	-	&	-	&	-	\\
&	7	&	23.32	&	-	&	-	&	-	\\
  \bottomrule

\end{tabular}
\caption{Impacto de las optimizaciones de BEAPI en su tiempo de
ejecución}
\label{tab:results-realWorld}

\end{table}


\begin{table}[H]
\scriptsize
\centering
\begin{tabular}{ l r  r  r  r  r  }
  \toprule
  \multicolumn{6}{c}{\textbf{FAJITA}} \\
  \midrule 
  \textbf{Clase} & \textbf{Scope} & \textbf{CI/MGO} & \textbf{CI}  & \textbf{MGO} & \textbf{NoOPT}  \\
  \midrule
  BinTree
&	3	&	0.06	&	0.05	&	0.63	&	0.886	\\
& 4	&	0.1	&	0.11	&	150.99	&	141.21	\\
&	5	&	0.26	&	0.31	&	-	&	-	\\
&	6	&	0.52	&	0.52	&	-	&	-	\\
&	7	&	01.01	&	1.15	&	-	&	-	\\
&	8	&	2.59	&	3.55	&	-	&	-	\\
&	9	&	12.38	&	12.96	&	-	&	-	\\
&	10	&	51.34	&	59.45	&	-	&	-	\\
&	11	&	265.57	&	204.13	&	-	&	-	\\
&	12	&	1578.72	&	1560.70	&	-	&	-	\\
  \midrule
  AVL
&	3	&	0.02	&	0.07	&	0.24	&	- \\
&	4	&	0.04	&	0.16	&	81.24	&	- \\
&	5	&	0.09	&	0.31	&	-	&	-	\\
&	6	&	0.14	&	0.76	&	-	&	-	\\
&	7	&	0.27	&	2.43	&	-	&	-	\\
&	8	&	0.46	&	7.71	&	-	&	-	\\
&	9	&	0.86	&	27.89	&	-	&	-	\\
&	10	&	1.92	&	104.60	&	-	&	-	\\
&	11	&	5.80	&	391.28	&	-	&	-	\\
&	12	&	14.58	&	1389.78	&	-	&	-	\\
&	13	&	45.45	&	-	&	-	&	-	\\
&	14	&	-	&	-	&	-	&	-	\\
  \midrule
  RBT
&	3	&	0.02	&	0.05	&	25.21	&	-	\\
&	4	&	0.05	&	0.10	&	-	&	-	\\
&	5	&	0.09	&	0.13	&	-	&	-	\\
&	6	&	0.16	&	0.25	&	-	&	-	\\
&	7	&	0.31	&	0.50	&	-	&	-	\\
&	8	&	0.72	&	1.30	&	-	&	-	\\
&	9	&	2.11	&	3.67	&	-	&	-	\\
&	10	&	6.18	&	11.70	&	-	&	-	\\
&	11	&	19.72	&	37.63	&	-	&	-	\\
&	12	&	63.16	&	122.18	&	-	&	-	\\
&	13	&	206.66	&	394.47	&	-	&	-	\\
  \midrule
  BinHeap
&	3	&	0.03	&	0.04	&	1.35	&	-	\\
&	4	&	0.08	&	0.18	&	-	&	-	\\
&	5	&	0.31	&	1.21	&	-	&	-	\\
&	6	&	1.28	&	18.41	&	-	&	-	\\
&	7	&	44452	&	391.53	&	-	&	-	\\
&	8	&	97.08	&	-	&	-	&	-	\\
  \midrule
  SLList
&	3	&	0.02	&	0.04	&	0.44	&	- \\
&	4	&	0.11	&	0.10	&	71.89	&	-	\\
&	5	&	0.33	&	0.38	&	-	&	-	\\
&	6	&	1.58	&	2.31	&	-	&	-	\\
&	7	&	15.38	&	291.53	&	-	&	-	\\
&	8	&	297.41	&	705.71	&	-	&	-	\\
  \midrule
  DLList
&	3	&	0.01	&	0.19	&	0.63	&	-	\\
&	4	&	0.05	&	0.55	&	155.70	&	-	\\
&	5	&	0.17	&	2.50	&	-	&	-	\\
&	6	&	0.83	&	27.58	&	-	&	-	\\
&	7	&	12.14	&	-	&	-	&	-	\\
&	8	&	269.47	&	-	&	-	&	-	\\
  \midrule
  NCL
&	3	&	0.06	&	0.07	&	08.07	&	-	\\
&	4	&	0.26	&	0.19	&	-	&	-	\\
&	5	&	0.84	&	0.65	&	-	&	-	\\
&	6	&	4.80	&	5.95	&	-	&	-	\\
&	7	&	58.75	&	99.29	&	-	&	-	\\
  \bottomrule
\end{tabular}
\caption{Impacto de las optimizaciones de BEAPI en su tiempo de
ejecución}
\label{tab:results-fajita}

\end{table}

\begin{table}
\scriptsize
\centering
\begin{tabular}{ l r  r  r  r  r  }
  \toprule
  \multicolumn{6}{c}{\textbf{ROOPS}} \\
  \midrule 
  \textbf{Clase} & \textbf{Scope} & \textbf{CI/MGO} & \textbf{CI}  & \textbf{MGO} & \textbf{NoOPT}  \\
  \midrule
  AVL
&	3	&	0.02	&	0.04	&	.34	&	-	\\
&	4	&	0.03	&	0.07	&	102.16	&	-	\\
&	5	&	0.05	&	0.11	&	-	&	-	\\
&	6	&	.009	&	0.23	&	-	&	-	\\
&	7	&	0.17	&	0.47	&	-	&	-	\\
&	8	&	0.32	&	0.64	&	-	&	-	\\
&	9	&	0.65	&	2.43	&	-	&	-	\\
&	10	&	1.52	&	07.04	&	-	&	-	\\
&	11	&	5.11	&	24.58	&	-	&	-	\\
&	12	&	19.92	&	55.86	&	-	&	-	\\
&	13	&	46.71	&	657.17	&	-	&	-	\\
  \midrule
  NCL
&	3	&	0.04	&	1.31	&	1.37	&	7.96	\\
&	4	&	0.10	&	9.59	&	52.17	&	-	\\
&	5	&	0.34	&	40.54	&	-	&	-	\\
&	6	&	2.27	&	128.51	&	-	&	-	\\
&	7	&	33.89	&	659.40	&	-	&	-	\\
&	8	&	769.63	&	-	&	-	&	-	\\
  \midrule
  BinTree
&	3	&	0.02	&	0.04	&	0.23	&	105.46	\\
&	4	&	0.05	&	0.08	&	85.32	&	-	\\
&	5	&	0.11	&	0.16	&	-	&	-	\\
&	6	&	0.20	&	0.50	&	-	&	-	\\
&	7	&	0.52	&	1.00	&	-	&	-	\\
&	8	&	1.70	&	3.77	&	-	&	-	\\
&	9	&	7.11	&	16.30	&	-	&	-	\\
&	10	&	34.72	&	82.07	&	-	&	-	\\
&	11	&	181.11	&	431.63	&	-	&	-	\\
&	12	&	966.41	&	2281.42	&	-	&	-	\\
  \midrule
  LList
&	3	&	.03	&	0.09	&	0.26	&	-	\\
&	4	&	.07	&	0.48	&	115.27	&	-	\\
&	5	&	.23	&	11.45	&	-	&	-	\\
&	6	&	01.06	&	1169.05	&	-	&	-	\\
&	7	&	12.62	&	-	&	-	&	-	\\
&	8	&	295.94	&	-	&	-	&	-	\\
  \midrule
  RBT
&	3	&	0.04	&	0.04	&	39.11	&	-	\\
&	4	&	0.11	&	0.09	&	-	&	-	\\
&	5	&	0.22	&	0.14	&	-	&	-	\\
&	6	&	0.40	&	0.35	&	-	&	-	\\
&	7	&	0.70	&	0.52	&	-	&	-	\\
&	8	&	1.43	&	1.13	&	-	&	-	\\
&	9	&	4.11	&	3.42	&	-	&	-	\\
&	10	&	13.92	&	11.34	&	-	&	-	\\
&	11	&	31.02	&	51.05	&	-	&	-	\\
&	12	&	81.03	&	2379.44	&	-	&	-	\\
&	13	&	697.06	&	-	&	-	&	-	\\
  \midrule
  FibHeap
&	3	&	0.04	&	0.09	&	0.94	&	-	\\
&	4	&	0.13	&	0.20	&	-	&	-	\\
&	5	&	0.70	&	1.13	&	-	&	-	\\
&	6	&	6.51	&	12.80	&	-	&	-	\\
&	7	&	129.01	&	243.36	&	-	&	-	\\
  \midrule
  BinHeap
&	3	&	0.05	&	0.11	&	2.03	&	18.38	\\
&	4	&	0.09	&	0.34	&	-	&	-	\\
&	5	&	0.26	&	0.96	&	-	&	-	\\
&	6	&	1.31	&	2.96	&	-	&	-	\\
&	7	&	13.06	&	30.23	&	-	&	-	\\
&	8	&	96.94	&	220.18	&	-	&	-	\\
  \bottomrule
\end{tabular}
\caption{Impacto de las optimizaciones de BEAPI en su tiempo de
ejecución}
\label{tab:results-roops}
\end{table}
\vspace{0.3cm}
\begin{table}[H]
\scriptsize

\centering
\begin{tabular}{ l r  r  r  r  r  }
  \toprule
  \multicolumn{6}{c}{\textbf{Korat}} \\
  \midrule 
  \textbf{Clase} & \textbf{Scope} & \textbf{CI/MGO} & \textbf{CI}  & \textbf{MGO} & \textbf{NoOPT}  \\
  \midrule
  DLList
&	3	&	0.06	&	0.07	&	0.53	&	31.63	\\
&	4	&	0.25	&	0.36	&	113.34	&	-	\\
&	5	&	01.01	&	1.30	&	-	&	-	\\
&	6	&	7.11	&	9.97	&	-	&	-	\\
&	7	&	108.08	&	153.01	&	-	&	-	\\
&	8	&	-	&	-	&	-	&	-	\\
  \midrule
  FHeap
&	3	&	0.04	&	0.18	&	0.85	&	-	\\
&	4	&	0.11	&	0.45	&	-	&	-	\\
&	5	&	0.56	&	2.55	&	-	&	-	\\
&	6	&	5.95	&	26.8	&	-	&	-	\\
&	7	&	115.44	&	-	&	-	&	-	\\
&	8	&	-	&	-	&	-	&	-	\\
  \midrule
  BinHeap
&	3	&	0.11	&	0.10	&	0.68	&	-	\\
&	4	&	0.23	&	0.45	&	-	&	-	\\
&	5	&	0.68	&	5.24	&	-	&	-	\\
&	6	&	44503	&	176.12	&	-	&	-	\\
&	7	&	25.32	&	-	&	-	&	-	\\
&	8	&	163.39	&	-	&	-	&	-	\\
&	9	&	-	&	-	&	-	&	-	\\
  \midrule
  BST
&	3	&	0.04	&	0.02	&	0.24	&	0.55	\\
&	4	&	0.09	&	0.02	&	77.21	&	116.03	\\
&	5	&	0.25	&	0.20	&	-	&	-	\\
&	6	&	0.44	&	0.38	&	-	&	-	\\
&	7	&	0.86	&	0.95	&	-	&	-	\\
&	8	&	03.07	&	2.73	&	-	&	-	\\
&	9	&	11.77	&	10.70	&	-	&	-	\\
&	10	&	49.098	&	49.61	&	-	&	-	\\
&	11	&	199.46	&	171.50	&	-	&	-	\\
&	12	&	1341.86	&	1351.38	&	-	&	-	\\
&	13	&	-	&	-	&	-	&	-	\\
  \midrule
  SSList
&	3	&	0.04	&	0.04	&	0.49	&	14.02	\\
&	4	&	0.13	&	0.21	&	106.38	&	-	\\
&	5	&	0.42	&	0.65	&	-	&	-	\\
&	6	&	1.87	&	2.3	&	-	&	-	\\
&	7	&	17.87	&	25.22	&	-	&	-	\\
&	8	&	256.49	&	352.16	&	-	&	-	\\
&	9	&	-	&	-	&	-	&	-	\\
  \midrule
  RBT
&	3	&	0.05	&	.03	&	25.18	&	-	\\
&	4	&	0.10	&	.06	&	-	&	-	\\
&	5	&	0.25	&	.11	&	-	&	-	\\
&	6	&	0.43	&	.21	&	-	&	-	\\
&	7	&	0.91	&	.40	&	-	&	-	\\
&	8	&	1.31	&	1.01	&	-	&	-	\\
&	9	&	4.63	&	2.90	&	-	&	-	\\
&	10	&	13.88	&	9.19	&	-	&	-	\\
&	11	&	33.42	&	28.84	&	-	&	-	\\
&	12	&	79.45	&	94.46	&	-	&	-	\\
&	13	&	689.06	&	308.06	&	-	&	-	\\
&	14	&	-	&	-	&	-	&	-	\\
  \midrule
%   SortedList
% &	3	&	0.01	&	0.01	&	0.24	&	0.94	\\
% &	4	&	0.02	&	0.03	&	74.45	&	-	\\
% &	5	&	0.05	&	0.05	&	-	&	-	\\
% &	6	&	0.11	&	0.13	&	-	&	-	\\
% &	7	&	0.23	&	0.27	&	-	&	-	\\
% &	8	&	0.65	&	0.68	&	-	&	-	\\
% &	9	&	02.07	&	2.24	&	-	&	-	\\
% &	10	&	8.14	&	09.04	&	-	&	-	\\
%   \bottomrule
\end{tabular}
\caption{Impacto de las optimizaciones de BEAPI en su tiempo de
ejecución}
\label{tab:results-korat}
\end{table}


\begin{table}[H]
\scriptsize
\centering
\begin{tabular}{ l r  r  r  r  r  }
  \toprule
  \multicolumn{6}{c}{\textbf{Kiasan}} \\
  \midrule 
  \textbf{Clase} & \textbf{Scope} & \textbf{CI/MGO} & \textbf{CI}  & \textbf{MGO} & \textbf{NoOPT}  \\
  \midrule
  BinTree
&	3	&	0.03	&	0.04	&	0.52	&	-	\\
&	4	&	0.08	&	0.10	&	116.71	&	-	\\
&	5	&	0.18	&	0.17	&	-	&	-	\\
&	6	&	0.48	&	0.40	&	-	&	-	\\
&	7	&	01.02	&	1.20	&	-	&	-	\\
&	8	&	2.80	&	4.80	&	-	&	-	\\
&	9	&	11.66	&	22.33	&	-	&	-	\\
&	10	&	50.82	&	112.73	&	-	&	-	\\
&	11	&	204.83	&	583.60	&	-	&	-	\\
&	12	&	1235.67	&	-	&	-	&	-	\\
  \midrule
  DLList
&	3	&	0.02	&	0.37	&	0.22	&	-	\\
&	4	&	0.11	&	1.59	&	72.60	&	-	\\
&	5	&	0.47	&	67.72	&	-	&	-	\\
&	6	&	1.68	&	-	&	-	&	-	\\
&	7	&	18.09	&	-	&	-	&	-	\\
&	8	&	257.42	&	-	&	-	&	-	\\
  \midrule
  DisjSet
&	3	&	0.02	&	0.02	&	0.05	&	1.44	\\
&	4	&	0.06	&	0.07	&	0.24	&	-	\\
&	5	&	0.30	&	0.29	&	2.69	&	-	\\
&	6	&	1.94	&	2.22	&	-	&	-	\\
&	7	&	28.94	&	31.67	&	-	&	-	\\
  \midrule
  RBT
&	3	&	0.05	&	0.04	&	25.34	&	-	\\
&	4	&	0.11	&	0.06	&	-	&	-	\\
&	5	&	0.25	&	0.14	&	-	&	-	\\
&	6	&	0.41	&	0.24	&	-	&	-	\\
&	7	&	0.78	&	0.43	&	-	&	-	\\
&	8	&	1.57	&	1.00	&	-	&	-	\\
&	9	&	4.12	&	2.80	&	-	&	-	\\
&	10	&	11.39	&	8.93	&	-	&	-	\\
&	11	&	34.88	&	29.22	&	-	&	-	\\
&	12	&	84.51	&	95.11	&	-	&	-	\\
  \midrule
  DisjSet
&	3	&	0.02	&	0.02	&	1.49	&	1.45	\\
&	4	&	0.06	&	0.07	&	-	&	-	\\
&	5	&	0.21	&	0.22	&	-	&	-	\\
&	6	&	0.89	&	0.99	&	-	&	-	\\
&	7	&	8.26	&	7.92	&	-	&	-	\\
  \midrule
  StackLi
&	3	&	0.03	&	0.06	&	0.24	&	-	\\
&	4	&	0.10	&	0.18	&	90.27	&	-	\\
&	5	&	0.55	&	0.84	&	-	&	-	\\
&	6	&	4.35	&	8.47	&	-	&	-	\\
&	7	&	83.06	&	158.26	&	-	&	-	\\
  \midrule
  BHeap
&	3	&	0.04	&	0.06	&	0.8	&	-	\\
&	4	&	0.13	&	0.15	&	-	&	-	\\
&	5	&	0.45	&	0.69	&	-	&	-	\\
&	6	&	3.14	&	44.413	&	-	&	-	\\
&	7	&	53.78	&	79.12	&	-	&	-	\\
&	8	&	1221.59	&	1756.71	&	-	&	-	\\
\midrule
  TreeMap
&	3	&	0.14	&	0.50	&	-	&	-	\\
&	4	&	2.15	&	2.59	&	-	&	-	\\
&	5	&	24.95	&	45.46	&	-	&	-	\\
&	6	&	866.71	&	-	&	-	&	-	\\
  \bottomrule
\end{tabular}
\caption{Impacto de las optimizaciones de BEAPI en su tiempo de
ejecución}
\label{tab:results-kiasan}
\end{table}



\subsection{Uso de BEAPI para analizar especificaciones}

\subsection{Comparativa de BEAPI con otras tecnicas de generacion de test}

\begin{table}[H]
\scriptsize
\centering
\label{tab:results-obj1}
\begin{tabular}{ l c  r  |r | r | r|r  }
  \toprule
  \textbf{Caso} & \textbf{Scope/GenTime} & \textbf{Técnica} & \textbf{Valid}  & \textbf{Invalid} & \textbf{Test}&\textbf{T Time(S)}  \\
  \midrule
  HashSet		&	5	&	Randoop	&	-	&	- & 380&4	\\
&		&	R-Serialize	&	3851	&	264 & 184852&4	\\
&		&	R-builders	&	12812	&	9008 & 614980&20	\\
&		&	BEAPI	&	32	&	- & 1540&2	\\
\cline{2-7}
		&	10 (5)	&	Randoop	&	-	&	-&800&4\\
&		&	R-Serialize	&	7655	&	532 &367444&6	\\
&		&	R-builders	&	23320	&	16482 &1119364&35	\\
&		&	BEAPI	&	32	&	- &1540&2\\
\cline{2-7}
		&	20 (5)	&	Randoop	&	-&-&\\
&		&	R-Serialize	&	23320	&	1015 &719044&8	\\
&		&	R-builders	&	38047	&	26904 &1826260&55	\\
&		&	BEAPI	&	32	&	- &1540&2\\
\cline{2-7}
		&	40 (5)	&	Randoop	&	78	&	8&3286&10\\
&		&	R-Serialize	&	28965	&	2020 &1390324&12	\\
&		&	R-builders	&	59016	&	41681 &2832772&85\\
&		&	BEAPI	&	32	&	-&1540&2 \\
\cline{2-7}
		&	150 (5)	&	Randoop	&	78	&	89&12452&29\\
&		&	R-Serialize	&	97236	&	7088 &4667332&35	\\
&		&	R-builders	&	125992	&	89292 &6047620&179	\\
&		&	BEAPI	&	32	&	- &1540&2\\\end{tabular}
\end{table}

\begin{table}[H]
\scriptsize
\centering
\label{tab:results-obj1}
\begin{tabular}{ l c  r  |r | r  }
  \toprule
  \textbf{Caso} & \textbf{Scope/GenTime} & \textbf{Técnica} & \textbf{Branch}  & \textbf{Mutation} \\
  \midrule
  HashSet		&	5 (5)	&	Randoop	&	53	&	67\\
&		&	R-Serialize	&	78	&	89 	\\
&		&	R-builders	&	78	&	89 	\\
&		&	BEAPI	&	78	&	89 \\
\cline{2-5}
		&	10 (5)	&	Randoop	&	67	&	89\\
&		&	R-Serialize	&	78	&	89 	\\
&		&	R-builders	&	78	&	89 	\\
&		&	BEAPI	&	78	&	89 \\
\cline{2-5}
		&	20 (5)	&	Randoop	&	78	&	89\\
&		&	R-Serialize	&	78	&	89 	\\
&		&	R-builders	&	78	&	89 	\\
&		&	BEAPI	&	78	&	89 \\
\cline{2-5}
		&	40 (5)	&	Randoop	&	78	&	89\\
&		&	R-Serialize	&	78	&	89 	\\
&		&	R-builders	&	78	&	89 	\\
&		&	BEAPI	&	78	&	89 \\
\cline{2-5}
		&	150 (5)	&	Randoop	&	78	&	89\\
&		&	R-Serialize	&	78	&	89 	\\
&		&	R-builders	&	78	&	89 	\\
&		&	BEAPI	&	78	&	89 \\
%   HashMap		&	5	&	Randoop	&	-	&	- & 380	\\
% &		&	Randoop-Serialize	&	-	&	- & -	\\
% &		&	Randoop-builders	&	-	&	- & -	\\
% &		&	BEAPI	&	-	&	- & -	\\
\end{tabular}
\end{table}


\hspace{1cm}





\subsection{Uso de Builders en BEAPI}
% En la segunda parte de la evaluación, analizamos qué tan útiles son los builders identificados en el contexto de un análisis de programas, específicamente en la generación automatizada de casos de prueba. Estos objetos pueden utilizarse, por ejemplo, como entradas en test parametrizados. Para los estudios de caso que proporcionan mecanismos para medir el tamaño de los objetos y compararlos por igualdad (es decir, los métodos size y equals de las estructuras de datos), generamos pruebas con Randoop utilizando todos los métodos disponibles en la API (API), y luego generamos pruebas con Randoop utilizando solo los métodos builders (BLD) identificados por nuestro enfoque en el experimento anterior (Tabla \ref{tab:results-compute-bld}). Luego comparamos el número de objetos diferentes (No. de Objs.) y el tamaño del objeto más grande (Max Obj. Size) generados por las pruebas generadas a partir de la API, en comparación con las pruebas generadas utilizando solo los métodos de BLD. Establecimos tres budget diferentes para la generación de pruebas: 60, 120 y 180 segundos (Budget). Los resultados se resumen en la Tabla \ref{tab:results-obj}. Los resultados muestran que, en el mismo presupuesto de pruebas, BLD genera en promedio un 500 porciento más de objetos que la API. En todos los casos, BLD también genera objetos significativamente más grandes que la API. A la luz de estos resultados, queda claro que la identificación automatizada de builders es beneficiosa para la generación automatizada de estructuras para clases con estado.

% Además, comparamos las suites de pruebas generadas con Randoop midiendo la cobertura de ramas y líneas de código. Al igual que en el experimento anterior, la suite de pruebas utilizada para la comparación se generó a partir de los métodos builders (BLD) en comparación con el uso de todos los métodos disponibles de la API con un Randoop predeterminado. Establecimos cinco presupuestos diferentes para la generación de pruebas: 60, 120, 180, 300 y 600 segundos (Budget). Los resultados se resumen en la Tabla \ref{tab:results-coverage}.

% % Cabe destacar que la tabla \ref{tab:results-coverage} muestra que, para los casos que manipulan estructuras de datos complejas (por ejemplo, java.util.TreeMap), las pruebas generadas solo con los métodos BLD obtienen una mejor cobertura tanto en ramas como en líneas de código
.
% \begin{table}[!thb]
% \scriptsize
% \centering
% \caption{Tiempo de ejecución de BEAPI con diferentes configuración.}


% \end{table}

% \begin{table}[!thb]
% \scriptsize

% \centering
% \begin{tabular}{ l r | r | r | r | r  }
%   \toprule
%   \multicolumn{6}{c}{\textbf{Real World}} \\
%   \midrule 
%   \textbf{Class} & \textbf{S} & \textbf{SM/BLD} & \textbf{SM}  & \textbf{BLD} & \textbf{NoOPT}  \\
%   \midrule
%   NCL
%   & 3 & .10 & .47 & -  & -  \\
%   & 4 & .41 & 3.48  &  - & -  \\
%   & 5 & 3.33  &  - &  - & -  \\
%   & 6 & 73.78 &  - &  - &  - \\
%   \midrule
%   TSet
%   & 3 & .03 & .07 & 56.82 & - \\
%   & 11  & 21.52 & 86.06 &  - & -  \\
%   & 12  & 69.98 & 276.85  &  - & -  \\
%   & 13  & 226.66  & 887.83  &  - & -  \\
%     \midrule
%   TMap
%   & 3 & .11 & .25 & - & - \\
%   & 4 & .75 & 2.36  &  - &  - \\
%   & 5 & 15.97 & 57.64 &  - & -  \\
%   & 6 & 839.87  & 2901.37 &  - &  - \\
%   \midrule
%   LList
%   & 3 & .02 & .13 & .64 & - \\
%   & 6 & .96 & 258.85  & -  &  - \\
%   & 7 & 12.98 &  - &  - &  - \\
%   & 8 & 286.21  & -  & -  & -  \\
%   \midrule
%   HMap
%   & 3 & .10 & 11.49 & - & - \\
%   & 4 & .55 & -  & -  & -  \\
%   & 5 & 5.33  & -  &  - &  - \\
%   & 6 & 119.87  &  - &  - &  - \\
%   \bottomrule

% \end{tabular}

% \label{table:beapi}

% \end{table}





\include{Builders Identification}
%!TEX root = main.tex
\chapter{Conclusiones}
\label{cap:conclutions}

La calidad del software se puede mejorar considerablemente gracias a las
técnicas modernas de análisis de software. Muchas de las técnicas de análisis,
como el testing basado en propiedades, o el model checking de software, requieren
que el usuario provea manualmente mecanismos para generar objetos para alimentar el análisis.
En este trabajo presentamos enfoques eficientes para la generación automática de objetos a
partir de APIs de clase. 

Por un lado, observamos que los métodos seleccionados para la generación de
objetos tienen un impacto crucial en la eficiencia de la generación. De esta
observación surge la idea de definir algoritmos para la identificación
automática de métodos generadores de objetos a partir de la API pública de un módulo 
(presentados en el Capítulo~\ref{cap:builders}). Se evaluaron experimentalmente estos enfoques
 en varios casos de estudio tomados de la literatura, y los resultados muestran
 que estos son capaces de identificar métodos generadores de objetos suficientes
 (y minimales en la mayoría de los casos; con muy pocos métodos superfluos), 
 en tiempos de ejecución razonables.
Hasta donde sabemos, este es el primer trabajo que aborda este problema, que típicamente se resuelve de manera manual.

También mostramos que los generadores de objetos computados automáticamente
por nuestros enfoques pueden ser aprovechados por herramientas de generación
aleatoria de tests (\textsf{Randoop}) para producir objetos más diversos, y que
los objetos producidos usando los generadores de objetos logran mayor cobertura
de código y puntaje de mutación en el testing basado en propiedades.
Adicionalmente, mostramos que técnicas como el \emph{model checking} de software pueden 
beneficiarse de los métodos generadores de objetos para la construcción
automática de \emph{drivers} eficientes. 
Si bien los experimentos realizados son preliminares, los resultados obtenidos son alentadores 
y sugieren que esta línea de trabajo es prometedora.

Por otro lado, en este trabajo presentamos \textsf{BEAPI}
(Capítulo~\ref{cap:beapi}), cuyo objetivo es
facilitar la aplicación de una técnica sistemática —la generación exhaustiva
acotada de entradas— mediante la producción de objetos exclusivamente a partir
de la API de un componente, sin necesidad de contar con una especificación formal 
de las propiedades que deben satisfacer los objetos. \textsf{BEAPI} puede generar 
objetos de manera exhaustiva acotada a partir de componentes con invariantes implícitos, 
reduciendo la carga de proporcionar especificaciones formales y de adaptarlas para 
hacerlas apropiadas para la generación. Gracias a sus optimizaciones, que
incluyen la identificación automática de métodos generadores de objetos y un
mecanismo de coincidencia de estados, \textsf{BEAPI} puede generar objetos de manera exhaustiva acotada 
con un rendimiento comparable al de la técnica basada en especificaciones
más rápida (\textsf{Korat}). 
También mostramos cómo los enfoques basados en especificaciones y \textsf{BEAPI} pueden complementarse entre sí, 
ilustrando cómo \textsf{BEAPI} puede utilizarse para evaluar la corrección de
especificaciones formales de invariantes de representación (\texttt{repOK}).
Mediante este enfoque, encontramos varios errores sutiles en especificaciones
\texttt{repOK} tomadas de la literatura. En consecuencia, las técnicas que
requieren especificaciones de \texttt{repOK}s (por ejemplo, \cite{Rosner15}),
así como aquellas que requieren objetos generados de manera exhaustiva acotada (por ejemplo, \cite{Molina+2021}), pueden beneficiarse de la técnica \textsf{BEAPI}.
Finalmente, mostramos en un experimento preliminar que los objetos de generados
por \textsf{BEAPI} dan buenos resultados cuando son usados como entradas para
tests basados en propiedades (en términos de cobertura de código y
puntaje de mutación).

\pp{Tenemos que agregar trabajos futuros!}
% Pensar en generación de objetos en Pex.









% %!TEX root = main.tex
\chapter{Trabajo futuro}
\label{cap:futurework}


% ********************************************************************
% Backmatter
\cleardoublepage\null
\bibliographystyle{acm} % estilo de la bibliografía.
% \bibliographystyle{apalike}
% \bibliographystyle{apacite}
% \bibliographystyle{unsrtnat} % Estilo de bibliografía

%\bibliographystyle{alpha}
%\cleardoublepage\phantomsection
\bibliography{Bibliografia}

%\appendix
%\cleardoublepage
%\part{Appendix}
\end{document}