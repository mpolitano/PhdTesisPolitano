%!TEX root = main.tex

%USEFULL CODE
%\begin{definition}[Sintaxis de f\'ormulas LP]
%\label{def:preliminares.lp-sintaxis}
%Sea $PA$ un conjunto de variables proposicionales; luego: 
%\begin{itemize}
%\item Los valores de verdad \True (verdadero) y \False (falso) son f\'ormulas de LP; 
%\item Toda proposici\'on $p \in PA$ es una f\'ormula de LP; 
%\item Si $\varphi_1, \varphi_2$ son f\'ormulas de LP, entonces tambi\'en son f\'ormulas de LP: $\neg \varphi_1$ y $\varphi_1 \lor \varphi_2$.
%\item Podemos definir los operadores derivados $\land$, $\Implies$ y $\Iff$ de la siguiente manera: $\varphi_1 \land \varphi_2 \equiv \neg(\neg \varphi_1 \lor \neg \varphi_2)$, $\varphi_1 \Implies \varphi_2 \equiv \neg \varphi_1 \lor \varphi_2$ y $\varphi_1 \Iff \varphi_2 \equiv (\varphi_1 \Implies \varphi_2) \land (\varphi_2 \Implies \varphi_1)$.
%\end{itemize}
%\end{definition}

%\begin{algorithm}[htp]
%\caption{DPLL}
%\label{alg:preliminares.DPLL}
%\begin{algorithmic}[1]
%\Function{DPLL}{$\varphi$: formula proposicional CNF}
%\State $\varphi' \gets $ BCP($\varphi$)
%\If{$\varphi' = \True$}
%  \State \Return \textit{SAT}
%\ElsIf{$\varphi' = \False$} 
%  \State \Return \textit{UNSAT}
%\Else 
%	\State $P \gets $ \textit{Choose}(\textbf{vars}($\varphi'$))
%	\If{DPLL($\varphi'[P \mapsto \True]$) =  \textit{UNSAT}}
%	\If{DPLL($\varphi'[P \mapsto \False]$) =  \textit{UNSAT}} 
%		\State \Return \textit{UNSAT}
%	\EndIf
%	\EndIf
%	\State \Return \textit{SAT}
%\EndIf
%\EndFunction
%\end{algorithmic}
%\end{algorithm}

%\begin{figure}[htp]
%\begin{minipage}[b]{0.5\linewidth}
%\tikzstyle{line} = [draw, -latex, line width=0.2ex]
%\centering
%\begin{tikzpicture}[node distance = 1.5cm, auto] % ,scale=0.5, every node/.style={scale=0.5}]
%    % Place nodes
%\node (s0) {$\Box$};
%\node (s1) [above of=s0, left of=s0] {$\neg p$};
%\node (s2) [above of=s0, right of=s0] {$p$};
%\node at (0,0.8) {$\{p\}$};
%\node (s3) [above of=s0, left of=s2] {$(p \lor \neg q)$};
%\node (s4) [above of=s0, right of=s2] {$q$};
%\node at (1.5, 2.3) {$\{q\}$};
%\node (s5) [above of=s0, left of=s4] {$(\neg p \lor q)$};
%\node (s6) [above of=s0, right of=s4] {$p$};
%\node at (3,3.8) {$\{p\}$};
%
%\path [line] (s1) -- (s0);
%\path [line] (s2) -- (s0);
%\path [line] (s3) -- (s2);
%\path [line] (s4) -- (s2);
%\path [line] (s5) -- (s4);
%\path [line] (s6) -- (s4);
%\end{tikzpicture}
%\caption[Prueba de Insatisfactibilidad $(\Pi, \varphi \cup \psi)$]{$(\Pi, \varphi \cup \psi)$}
%\label{fig:preliminares.unsat-proof}
%\end{minipage}
%\end{figure}
\chapter[Preliminares]{Preliminares}
\label{cap:preliminares}

\section{Introducci\'on}
\label{sec:preliminares.intro}

\input{testing}

%\section{Mutaci\'on}
%\label{sec:preliminares.mutation}
%
%\subsection{Operadores}
%
%\subsubsection{Operadores suficientes}
%
%\subsection{Propiedades deseadas en mutaci\'on}
%%EQ, TRIVIAL, COUPPLING -> SUBSUMPTION
%
%\subsection{Herramientas de mutaci\'on}
%
%\section{Expresiones de navegaci\'on}
%\label{sec:preliminares.navigationalExpressions}
%
%\subsection{Encadenamiento de m\'etodos}
%\subsubsection{Interfaces fluentes}

\section{Resumen}
\label{sec:preliminares.resumen}