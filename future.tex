%!TEX root = main.tex
\chapter{Trabajos Futuros}
\label{cap:future}

Para concluir, describiremos brevemente a continuación algunas líneas de trabajos futuros. 

Por un lado, los enfoques de identificación de métodos generadores de objetos
presentados aquí se basan en análisis dinámicos. Sin embargo, otros enfoques como el de
\cite{HuangMDE2012} para la identificación de métodos observadores (puros) son
estáticos. Como trabajo futuro se propone estudiar la definición de enfoques  
basados en análisis estático para la identificación de generadores de objetos, 
o la definición de enfoques que combinen componentes estáticos y dinámicos.  
Estos trabajos pueden dar lugar a la definición de enfoques más precisos para la
identificación de métodos observadores, por ejemplo, mediante la combinación de análisis
estático y dinámico.

Algunos aspectos de nuestros enfoques de identificación de métodos generadores
de objetos podrían mejorarse. Por ejemplo, realizar una sintonía fina de
los parámetros del algoritmo genético, y quizás de sus componentes, 
podría resultar en mejores tiempos de ejecución y en mayor precisión en los
resultados. Además, los algoritmos presentados podrían beneficiarse de una 
clasificación más general y más poderosa de los tipos de los
argumentos que toman los métodos (la clasificación definida aquí es preliminar, y 
fue suficiente para mostrar las ideas principales de los enfoques propuestos).
%Esto sería importante para extender la evaluación experimental de los enfoques 
%a un conjunto más amplio de casos de estudio.
Por ejemplo, podrían incorporarse otras dimensiones, como la complejidad del código,
para favorecer la selección de métodos más simples y/o eficientes.  

También se deja como trabajo futuro buscar mejores formas de implementar la 
función objetivo basada en cobertura (\textsf{FRC}). Por ejemplo, en lugar de
usar un enfoque aleatorio de generación de tests como \textsf{Randoop}, se podría 
adaptar un enfoque basado en algoritmos genéticos (como \textsf{EvoSuite}
\cite{Fraser:2011}) para determinar si los candidatos a métodos generadores de objetos 
logran una buena cobertura de código (o no). 

Otras técnicas podrían beneficiarse del uso de métodos generadores de objetos.
Por ejemplo, en el contexto de generación de entradas basado en ejecución
simbólica, en la herramienta Pex \cite{Tillmann:2008} el usuario debe proveer ``object factories'',
definidas manualmente, para construir los objetos que se usarán para crear los
tests. Por otro lado, \textsf{Sushi} \cite{Braione17,DBLP:conf/icse/BraioneDMP18} busca construir 
objetos especificos para cubrir condiciones de camino, también en el contexto de
ejecución simbólica, usando la API. Definir enfoques eficientes de generación de
objetos basados en los generadores de objetos podría dar lugar a mejoras en
estas técnicas (y en muchas otras).

Por último, las evaluaciones experimentales realizadas son preliminares, y utilizan un 
conjunto relativamente pequeño de casos de estudio. Se planea extender las
evaluaciones para incorporar casos de estudio más diversos y más complejos, y
más herramientas (por ejemplo, \textsf{EvoSuite} \cite{Fraser:2011}), y actualizar 
las conclusiones en base a los resultados.


