%!TEX root = main.tex
\chapter{Trabajo Futuro}
\label{cap:future}

Creemos que una dirección de investigación prometedora, que planeamos explorar más a fondo en trabajos futuros, 
es adaptar nuestro enfoque para la generación exhaustiva acotada de pruebas, buscando mejorar su eficiencia, aplicabilidad y capacidad de integración con otras técnicas de prueba automatizadas.

Una primera línea relevante consiste en ampliar las capacidades del análisis estático en el proceso de identificación de métodos constructores. 
Actualmente, el análisis se basa en información superficial sobre el tipo de retorno y los argumentos de los métodos. 
Sin embargo, una extensión que incorpore información sobre el flujo de datos, el uso de variables internas y los efectos colaterales permitiría una caracterización más precisa, 
especialmente en casos donde existen inicializaciones indirectas o dependencias implícitas.

En paralelo, algunos aspectos del algoritmo evolutivo desarrollado pueden mejorarse de manera sustancial. 
Planeamos explorar nuevas heurísticas que tengan en cuenta métricas estáticas del código para guiar mejor la selección y combinación de métodos en el proceso evolutivo. 
Actualmente, nuestra implementación se basa principalmente en una configuración básica de la librería Jenetics para Java \cite{jenetics}. 
Como parte del trabajo futuro, prevemos diseñar un algoritmo genético específico, adaptado a las particularidades del problema de identificación de constructores a partir de una API. 
Esto incluiría una representación más adecuada de los individuos, así como operadores de cruce y mutación que preserven estructuras válidas en el espacio de soluciones. 
Además, la sintonización de parámetros evolutivos, como las tasas de cruce y mutación, constituye otra línea de trabajo que puede contribuir significativamente a mejorar la eficiencia del proceso.

Asimismo, resulta fundamental mejorar el manejo de dependencias entre clases en el proceso de generación, 
ya que en muchos módulos las estructuras de interés dependen fuertemente de la colaboración entre múltiples clases. 
El tratamiento explícito de estas dependencias permitiría una generación más completa y representativa de los posibles estados del sistema. 
Además, se deben considerar las dependencias de variables o constantes seteadas en el código, tanto en el análisis de cobertura como en el proceso de generación de entradas con BEAPI. 
Estas dependencias influyen directamente en los caminos ejecutables del software y, por lo tanto, en la capacidad de los tests generados para alcanzar comportamientos relevantes.

Finalmente, planeamos evaluar el desempeño de BEAPI frente a herramientas de generación de pruebas no exhaustivas, como Randoop, 
utilizando benchmarks modernos propuestos en la literatura. Esta evaluación permitirá comparar las capacidades de nuestra técnica con las de enfoques alternativos más ampliamente difundidos, 
poniendo en perspectiva las ventajas y limitaciones de la generación exhaustiva acotada frente a otras estrategias de prueba automatizada.

En conjunto, estas líneas de trabajo futuro apuntan a consolidar y extender las capacidades de BEAPI, 
tanto desde el punto de vista de su eficiencia como de su aplicabilidad en contextos reales. 
La evolución de estas técnicas contribuirá a la construcción de herramientas más potentes y versátiles para el aseguramiento de la calidad del software.
