%!TEX root = main.tex
\chapter{Conclusiones}
\label{cap:conclutions}

La garantía de calidad del software puede mejorar considerablemente gracias a las técnicas modernas de análisis, 
entre las cuales las técnicas de generación automática de tests juegan un papel destacado \cite{Cadar08, Luckow14, Fraser11, Pacheco07, Ma15, goGodefroid05, Marinov01, Boyapati02,Godefroid12}. 
Los enfoques aleatorios y basados en búsqueda han demostrado gran éxito en la generación automática de suites de prueba con excelentes métricas de cobertura y mutación. 
Sin embargo, su naturaleza aleatoria impide caracterizar con precisión las familias de comportamientos del software que cubren los tests generados. 
En contraste, las técnicas sistemáticas —como el \emph{model checking}, la ejecución simbólica o la generación exhaustiva acotada— permiten cubrir un conjunto preciso de comportamientos 
y, por lo tanto, ofrecer garantías de corrección más específicas.

En esta tesis abordamos el desafío de la generación automática de datos de prueba de forma exhaustiva y sistemática, 
especialmente en contextos donde no se dispone de especificaciones formales de las estructuras de datos a probar. 
Nuestro objetivo fue diseñar y desarrollar técnicas que permitan generar entradas de prueba exhaustivas a partir de la API pública de un módulo, 
reduciendo así la necesidad de intervención manual y maximizando la cobertura de comportamientos relevantes del software bajo prueba.

Para lograr este objetivo, desarrollamos dos técnicas. La primera, presentada en el Capítulo~\ref{cap:builders}, 
introduce dos algoritmos para la identificación automática de métodos generadores de objetos a partir de la API pública de un módulo: un algoritmo evolutivo y un algoritmo \emph{greedy} o de escalada.
Ambos algoritmos buscan detectar conjuntos de métodos constructores que permitan generar estructuras desde cero, 
sin requerir intervención manual ni especificaciones formales. 
Evaluamos estas técnicas en casos de estudios extraídos de la literatura, 
y los resultados obtenidos muestran que los conjuntos identificados son suficientes y mínimos, con tiempos de ejecución razonables. 
Hasta donde tenemos conocimiento, este constituye el primer trabajo que aborda de manera sistemática y automática el problema de la identificación de constructores, el cual ha sido tradicionalmente resuelto de manera manual.

Además, mostramos que los conjuntos de constructores obtenidos pueden ser aprovechados por herramientas de generación de pruebas, facilitando la creación de objetos más grandes y diversos. 
También técnicas como \emph{model checking} pueden beneficiarse de estos métodos mediante la construcción automática de \emph{drivers} eficientes. 
Los resultados obtenidos son alentadores y sugieren que esta línea de trabajo sea prometedora.

Uno de los principales desafíos fue desarrollar una herramienta capaz de generar todas las estructuras posibles hasta un tamaño máximo \(k\), 
utilizando únicamente los métodos de la API. Si bien la solución implementada funcionó satisfactoriamente en los estudios de caso analizados, 
sería deseable eliminar elementos de aleatoriedad en la generación para garantizar una cobertura más sistemática. 
Aunque existen herramientas para generación exhaustiva acotada, ninguna de ellas produce entradas directamente a partir de la API de un módulo, 
lo cual subraya la originalidad y el valor del enfoque propuesto en este trabajo.

La segunda técnica, presentada en el Capítulo~\ref{cap:beapi}, se centra precisamente en la necesidad de contar con una herriamienta
de generación exhaustiva acotada que produzca entradas de pruebas y que opere directamente desde la API pública del software bajo prueba (SUT), sin requerir especificaciones formales.
Surge entonces BEAPI, que permite generar suites exhaustivas acotadas incluso cuando existen invariantes implícitos, 
reduciendo la carga de tener que escribir y ajustar especificaciones manualmente.
Al no existir herramientas de generación exhaustiva que trabajen directamente sobre APIs, BEAPI llena un vacío importante en el área.

Gracias a una serie de optimizaciones —incluyendo la identificación automática de rutinas de construcción y un mecanismo de canonización de estados—, 
BEAPI logra un rendimiento comparable al de la técnica más eficiente basada en especificaciones, Korat \cite{Boyapati02}. 
Además, identificamos ciertas características de los componentes que los hacen más adecuados para una generación basada en especificaciones o bien para una generación basada en API.

A pesar de que la generación exhaustiva acotada de pruebas es intrínsecamente combinatoria, 
el proceso puede volverse eficiente gracias a diversas optimizaciones que propusimos e implementamos. 
Estas incluyen la generación de secuencias de métodos guiada por retroalimentación, la eliminación de estados redundantes mediante comparación, 
y el uso de métodos constructores para reducir la explosión combinatoria. 
Demostramos experimentalmente que estas optimizaciones son cruciales para la escalabilidad de la herramienta, 
permitiendo a BEAPI alcanzar una eficiencia comparable a la de los enfoques exhaustivos basados en especificaciones.

Finalmente, mostramos cómo los enfoques basados en especificaciones y BEAPI pueden complementarse entre sí. 
En particular, ilustramos cómo BEAPI puede utilizarse para evaluar implementaciones de \texttt{repOK}. 
Siguiendo este enfoque, identificamos errores sutiles en especificaciones tomadas de la literatura. 
Esto indica que tanto las técnicas que requieren \texttt{repOK} (por ejemplo, \cite{Rosner15}) como aquellas que requieren suites exhaustivas acotadas (por ejemplo, \cite{Molina+2021}) 
pueden beneficiarse de la técnica presentada.

En conclusión, este trabajo sienta las bases para una generación exhaustiva acotada de entradas de prueba que, 
partiendo únicamente de la API pública de un módulo, permite construir conjuntos de pruebas completos y representativos. 
Las contribuciones desarrolladas en esta tesis aportan significativamente al aseguramiento de la calidad del software, 
y abren nuevas perspectivas para la investigación y el desarrollo de herramientas de prueba automatizadas, 
más efectivas y confiables en contextos reales.


\cacho{Traduccion del paper Efficient Bounded Exhaustive Input Generationfrom Program APIs }
La garantía de calidad del software puede mejorar considerablemente gracias a las técnicas modernas de análisis de software, 
entre las cuales las técnicas de generación automática de tests juegan un papel destacado \cite{Cadar08, Luckow14, Fraser11, Pacheco07, Ma15, goGodefroid05, Marinov01, Boyapati02,Godefroid12}. 
Los enfoques aleatorios y basados en búsqueda han demostrado un gran éxito en la generación automática de suites de tests con muy buenas métricas de cobertura y mutación, 
pero su naturaleza aleatoria no permite que estas técnicas caractericen con precisión las familias de comportamientos de software 
que cubren los tests generados. Las técnicas sistemáticas, como las basadas en \emph{model checking}, ejecución simbólica 
o generación exhaustiva acotada, cubren un conjunto preciso de comportamientos, y por lo tanto pueden proporcionar garantías 
de corrección específicas.

En este artículo, presentamos BEAPI, una técnica que busca facilitar la aplicación de una técnica sistemática, 
la generación exhaustiva acotada de entradas, produciendo estructuras únicamente a partir de la API de un componente, 
sin necesidad de una especificación formal de las propiedades de las estructuras. BEAPI puede generar suites exhaustivas acotadas 
a partir de componentes con invariantes implícitos, y reduce la carga de proporcionar especificaciones formales y de ajustar 
las especificaciones para mejorar la generación. 

Gracias a una serie de optimizaciones, incluyendo la identificación automática de 
rutinas de construcción y un mecanismo de canonización/igualación de estados, BEAPI puede generar suites exhaustivas acotadas con 
un rendimiento comparable al de la técnica basada en especificaciones más rápida, Korat \cite{Boyapati02}. También hemos identificado las
características de un componente que pueden hacerlo más adecuado para una generación basada en especificaciones o para una generación basada en API.
Al ejercitar directamente la API del software bajo prueba, BEAPI resulta fácil de usar y habilita la generación exhaustiva acotada para software que carece de especificaciones formales de entradas válidas. 
Al mismo tiempo, la herramienta necesariamente genera entradas positivas verdaderas, en el sentido de que cada entrada producida es efectivamente construible a partir de los métodos públicos del SUT.


Finalmente, hemos mostrado cómo los enfoques basados en especificaciones y BEAPI pueden complementarse entre sí, 
ilustrando cómo BEAPI puede ser utilizado para evaluar implementaciones de \texttt{repOK}. Utilizando este enfoque, 
encontramos una serie de errores sutiles en especificaciones \texttt{repOK} tomadas de la literatura. Por lo tanto, 
las técnicas que requieren especificaciones \texttt{repOK} (por ejemplo, \cite{Rosner15}), así como las técnicas que requieren 
suites exhaustivas acotadas (por ejemplo, \cite{Molina+2021}), pueden beneficiarse de nuestra técnica de generación presentada.


\cacho{Traduccion del paper Automatically Identifying Sufficient Object Builders from Module APIs }

En este trabajo, presentamos un algoritmo evolutivo para detectar automáticamente conjuntos de constructores a partir de la API de un módulo. 
Evaluamos nuestro algoritmo en varios estudios de caso tomados de la literatura, y encontramos que es capaz de identificar de manera precisa conjuntos de constructores que son suficientes y mínimos, 
en tiempos de ejecución razonables. Hasta donde sabemos, este es el primer trabajo que aborda este problema, el cual suele resolverse de manera manual.

También mostramos resultados preliminares que indican que nuestro enfoque puede ser aprovechado por herramientas de generación de casos de prueba para producir objetos más grandes y diversos. 
Otras técnicas, como el \emph{software model checking}, también pueden beneficiarse al usar el conjunto identificado de constructores para construir automáticamente drivers eficientes.
Se necesita realizar más experimentación, pero dado los resultados de este artículo, nuestro enfoque parece muy prometedor.

Uno de los mayores desafíos de este trabajo fue la construcción de una herramienta que nos permitiera generar todas las estructuras acotadas, 
para un número máximo $k$ de objetos, a partir de los métodos de la API del programa. La solución propuesta funcionó lo suficientemente bien para nuestros estudios de caso, 
pero sería deseable evitar la aleatoriedad en el proceso. El uso de herramientas de generación exhaustiva acotada en lugar de generación aleatoria se ajustaría mejor a nuestros propósitos \cite{Boyapati02}, 
pero desafortunadamente ninguna de las herramientas para generación exhaustiva acotada de tests produce entradas a partir de la API de un módulo.

Creemos que una dirección de investigación prometedora, que planeamos explorar más a fondo en trabajos futuros, 
es adaptar nuestro enfoque presentado para la generación exhaustiva acotada de tests.

Algunos aspectos de nuestro algoritmo genético pueden mejorarse aún más. 
Por ejemplo, se podría definir una clasificación más poderosa para los tipos de argumentos, en la priorización de métodos de acuerdo con sus complejidades. Además, se podrían incorporar otras dimensiones, 
como la complejidad del código, para favorecer métodos más simples. Exploraremos esta dirección como trabajo futuro. 
Asimismo, nuestra implementación del algoritmo genético es, en su mayor parte, una implementación evolutiva por defecto de la librería JGap para Java \cite{jgrapht}. 
Por supuesto, mejoras al algoritmo evolutivo y un ajuste fino de sus parámetros (por ejemplo, tasa de cruce/mutación) podrían generar tiempos de ejecución más rápidos, 
por lo que planeamos investigar esto más a fondo en el futuro.

s
\cacho{Traduccion del paper BEAPI: A Tool for Bounded Exhaustive Input Generation from APIs }

Presentamos BEAPI, una herramienta para la generación exhaustiva acotada de entradas de prueba que se basa únicamente en la API pública del SUT para la generación de pruebas. 
Al ejercitar directamente la API del software bajo prueba, BEAPI resulta fácil de usar y habilita la generación exhaustiva acotada para software que carece de especificaciones formales de entradas válidas. 
Al mismo tiempo, la herramienta necesariamente genera entradas positivas verdaderas, en el sentido de que cada entrada producida es efectivamente construible a partir de los métodos públicos del SUT.

A pesar de que la generación exhaustiva acotada de pruebas es intrínsecamente combinatoria, 
el proceso de generación puede hacerse eficiente gracias a diversas optimizaciones en el algoritmo de generación exhaustiva acotada de BEAPI, que hemos propuesto e implementado. 
Estas optimizaciones incluyen la generación de secuencias de métodos basada en retroalimentación, la eliminación de estados redundantes mediante comparación de estados, y la identificación y posterior uso de métodos constructores para reducir la explosión de secuencias de métodos. 
Mostramos experimentalmente que las optimizaciones propuestas son cruciales para la eficiencia y escalabilidad de la herramienta, y que permiten a BEAPI alcanzar una eficiencia comparable a la de los enfoques exhaustivos acotados más eficientes basados en especificaciones existentes.

Como trabajo futuro, planeamos evaluar BEAPI frente a otras herramientas de generación de pruebas de última generación que no son exhaustivas (por ejemplo, Randoop), 
utilizando benchmarks modernos de la literatura, para evaluar mejor las capacidades de BEAPI (y de la generación exhaustiva acotada en general) 
frente a otras técnicas de prueba automatizadas.


% Como proyección a futuro, planeamos evaluar BEAPI frente a técnicas modernas de generación de pruebas no exhaustivas, como Randoop, utilizando benchmarks recientes de la literatura. 
% Este análisis permitirá comparar su desempeño y precisión frente a otras técnicas automatizadas, 
% y evaluar con mayor profundidad las ventajas de la generación exhaustiva acotada de entradas de prueba. 
% Además, consideramos que mejorar el algoritmo evolutivo para la clasificación de argumentos y la priorización de métodos, 
% incluyendo dimensiones como la complejidad del código, podría potenciar aún más el rendimiento de BEAPI. 
% Por último, la sintonización de parámetros del algoritmo evolutivo, como las tasas de cruce y mutación, constituye otra línea de trabajo futura para mejorar la eficiencia y la aplicabilidad de la herramienta.


