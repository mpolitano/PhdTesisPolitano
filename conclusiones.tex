%!TEX root = main.tex
\chapter{Conclusiones}
\label{cap:conclutions}

La calidad del software se puede mejorar considerablemente gracias a las
técnicas modernas de análisis de software. Muchas de las técnicas de análisis,
como el testing basado en propiedades, o el model checking de software, requieren
que el usuario provea manualmente mecanismos para generar objetos para alimentar el análisis.
En este trabajo presentamos enfoques eficientes para la generación automática de objetos a
partir de APIs de clase. 

Por un lado, observamos que los métodos seleccionados para la generación de
objetos tienen un impacto crucial en la eficiencia de la generación. De esta
observación surge la idea de definir algoritmos para la identificación
automática de métodos generadores de objetos a partir de la API pública de un módulo 
(presentados en el Capítulo~\ref{cap:builders}). Se evaluaron experimentalmente estos enfoques
 en varios casos de estudio tomados de la literatura, y los resultados muestran
 que estos son capaces de identificar métodos generadores de objetos suficientes
 (y minimales en la mayoría de los casos; con muy pocos métodos superfluos), 
 en tiempos de ejecución razonables.
Hasta donde sabemos, este es el primer trabajo que aborda este problema, que típicamente se resuelve de manera manual.

También mostramos que los generadores de objetos computados automáticamente
por nuestros enfoques pueden ser aprovechados por herramientas de generación
aleatoria de tests (\textsf{Randoop}) para producir objetos más diversos, y que
los objetos producidos usando los generadores de objetos logran mayor cobertura
de código y puntaje de mutación en el testing basado en propiedades.
Adicionalmente, mostramos que técnicas como el \emph{model checking} de software pueden 
beneficiarse de los métodos generadores de objetos para la construcción
automática de \emph{drivers} eficientes. 
Si bien los experimentos realizados son preliminares, los resultados obtenidos son alentadores 
y sugieren que esta línea de trabajo es prometedora.

Por otro lado, en este trabajo presentamos \textsf{BEAPI}
(Capítulo~\ref{cap:beapi}), cuyo objetivo es
facilitar la aplicación de una técnica sistemática —la generación exhaustiva
acotada de entradas— mediante la producción de objetos exclusivamente a partir
de la API de un componente, sin necesidad de contar con una especificación formal 
de las propiedades que deben satisfacer los objetos. \textsf{BEAPI} puede generar 
objetos de manera exhaustiva acotada a partir de componentes con invariantes implícitos, 
reduciendo la carga de proporcionar especificaciones formales y de adaptarlas para 
hacerlas apropiadas para la generación. Gracias a sus optimizaciones, que
incluyen la identificación automática de métodos generadores de objetos y un
mecanismo de coincidencia de estados, \textsf{BEAPI} puede generar objetos de manera exhaustiva acotada 
con un rendimiento comparable al de la técnica basada en especificaciones
más rápida (\textsf{Korat}). 
También mostramos cómo los enfoques basados en especificaciones y \textsf{BEAPI} pueden complementarse entre sí, 
ilustrando cómo \textsf{BEAPI} puede utilizarse para evaluar la corrección de
especificaciones formales de invariantes de representación (\texttt{repOK}).
Mediante este enfoque, encontramos varios errores sutiles en especificaciones
\texttt{repOK} tomadas de la literatura. En consecuencia, las técnicas que
requieren especificaciones de \texttt{repOK}s (por ejemplo, \cite{Rosner15}),
así como aquellas que requieren objetos generados de manera exhaustiva acotada (por ejemplo, \cite{Molina+2021}), pueden beneficiarse de la técnica \textsf{BEAPI}.
Finalmente, mostramos en un experimento preliminar que los objetos de generados
por \textsf{BEAPI} dan buenos resultados cuando son usados como entradas para
tests basados en propiedades (en términos de cobertura de código y
puntaje de mutación).

\pp{Tenemos que agregar trabajos futuros!}
% Pensar en generación de objetos en Pex.








