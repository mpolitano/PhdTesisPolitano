\chapter[Preliminares]{Preliminares}
\label{cap:preliminares.BE}



%\section{Generacion Exhaustiva Acotada}
%\label{sec:be}
%
%\section{Korat: Generación exhaustiva acotada de entradas basada en especificaciones}
%\label{sec:korat}
%
%La \textit{generación exhaustiva de entradas} para probar APIs es una tarea crítica en la prueba de software, ya que permite identificar errores y garantizar que los programas funcionen correctamente bajo diversas condiciones. Sin embargo, el número potencialmente enorme de valores de entrada genera una explosión combinatoria, lo que hace inviable probar todas las posibles combinaciones.
%
%Para abordar este problema, se utiliza la \textit{generación exhaustiva acotada de entradas} (\textit{Bounded Exhaustive Testing, BE}), que consiste en generar todas las combinaciones posibles dentro de un rango predefinido. Este enfoque equilibra cobertura y factibilidad al reducir el espacio de prueba a un subconjunto manejable, sin dejar de incluir casos críticos que podrían pasar desapercibidos con métodos menos sistemáticos, como la \textit{generación aleatoria de pruebas}. Si bien la generación aleatoria es menos costosa computacionalmente, puede omitir casos límite relevantes, afectando la detección de errores.
%
%Korat es una herramienta que implementa la generación exhaustiva acotada, explorando todas las estructuras válidas dentro de ciertas restricciones. Su eficiencia se basa en la poda de combinaciones redundantes y en la eliminacion de estructuras isomorfas redundantes.
%
%El proceso de Korat consta de tres pasos clave:
%
%\begin{enumerate}
%    \item \textbf{Especificación de invariantes}: El usuario define las propiedades que deben cumplir las estructuras de datos, como que un árbol binario no debe tener ciclos y que cada nodo puede tener hasta dos hijos.
%    \item \textbf{Generación de candidatos}: Se generan todas las configuraciones posibles dentro de los límites especificados. Por ejemplo, si se establece un máximo de tres nodos para un árbol binario, se generarán todas las combinaciones con 1, 2 y 3 nodos.
%    \item \textbf{Filtrado y validación}: Se descartan las configuraciones que no cumplen con las restricciones definidas.
%\end{enumerate}
%
%\subsubsection{Ejemplo: Árboles Binarios}
%
%Si queremos probar una función que opera sobre árboles binarios, podemos definir:
%
%\begin{itemize}
%    \item \textbf{Límite}: Máximo 3 nodos.
%    \item \textbf{Invariantes}:
%    \begin{itemize}
%        \item No hay ciclos en el árbol.
%        \item Cada nodo tiene como máximo dos hijos.
%        \item Todos los nodos están conectados.
%    \end{itemize}
%\end{itemize}
%
%Korat generará todas las configuraciones válidas dentro de estos límites, como:
%
%\begin{itemize}
%    \item Un árbol con un solo nodo raíz.
%    \item Un árbol con un nodo raíz y un hijo izquierdo.
%    \item Un árbol con un nodo raíz, un hijo izquierdo y un hijo derecho.
%\end{itemize}
%
%Estos casos de prueba permiten verificar que la función maneja correctamente distintas estructuras de árboles binarios.
%
%\subsection{Ventajas de Korat}
%
%\begin{itemize}
%    \item \textbf{Cobertura completa}: Garantiza la exploración de todas las configuraciones dentro de los límites establecidos.
%    \item \textbf{Detección de casos límite}: Aumenta la probabilidad de encontrar errores en escenarios extremos.
%    \item \textbf{Eficiencia}: Evita la generación de estructuras inválidas o redundantes mediante poda y filtrado.
%\end{itemize}
%
%En esta tesis, Korat se utiliza como referencia para evaluar nuevas técnicas de generación de pruebas. Su enfoque sistemático y basado en especificaciones permite compararlo con otros métodos en términos de cobertura y calidad de los casos generados.
%
%
%
%
%
%La generación exhaustiva de entradas para realizar pruebas a las APIs de los programas es una tarea crítica en la prueba del software, ya que permite identificar errores y garantizar que los programas se comporten correctamente bajo una amplia gama de condiciones. Sin embargo, este proceso es inherentemente complejo debido al número potencialmente enorme de valores de entrada posibles, lo que genera una explosión combinatoria que hace inviable la prueba exhaustiva de todas las posibles entradas.
%
%Para abordar esta problemática, se recurre a la generación exhaustiva acotada de entradas (\emph{Bounded Exhaustive Testing, BE}), una técnica que consiste en generar todas las combinaciones posibles de entradas dentro de un rango o límite predefinido. Este enfoque permite reducir el espacio de entradas a un subconjunto manejable que aún cubre una diversidad significativa de casos, aumentando así la probabilidad de descubrir errores que podrían pasar desapercibidos con otros métodos de prueba.
%
%Por otro lado, una alternativa común a la generación exhaustiva acotada es la generación aleatoria de pruebas. Aunque este método es menos costoso computacionalmente y puede ser útil en ciertos contextos, presenta el inconveniente de que podría no cubrir casos críticos o bordes del espacio de entradas, lo que podría resultar en la omisión de errores significativos.
%
%En resumen, la generación exhaustiva acotada de entradas representa un compromiso entre la cobertura exhaustiva y la factibilidad práctica, ofreciendo una estrategia eficaz para la prueba de APIs, especialmente en sistemas donde la calidad y la seguridad son primordiales.
%
%Korat es una herramienta que implementa la técnica de generación exhaustiva acotada. Korat es una herramienta poderosa para la generación exhaustiva acotada de casos de prueba, permitiendo explorar todas las posibles estructuras válidas dentro de las restricciones establecidas. Su enfoque basado en filtrado, combinado con técnicas de poda y evitando las estructuras isomorfas, garantiza la eficiencia y efectividad en la generación de casos de prueba.
%
%La idea fundamental detrás de Korat es generar todas las configuraciones válidas de una estructura de datos dentro de un rango acotado (por ejemplo, un número máximo de nodos en un árbol o elementos en una lista). A diferencia de la generación aleatoria de pruebas, Korat garantiza que se exploren todas las combinaciones posibles dentro de los límites establecidos, lo que aumenta la probabilidad de detectar errores sutiles o casos límite.
%
%El proceso de Korat se basa en tres pasos clave:
%\begin{enumerate}
%\item \textbf{Especificación de invariantes}: El usuario define las propiedades que deben cumplir las estructuras de datos mediante predicados lógicos. Por ejemplo, en un árbol binario, se puede especificar que no debe contener ciclos y que cada nodo debe tener como máximo dos hijos
%
%\item \textbf{Generación de candidatos}: Korat genera todas las posibles configuraciones de la estructura de datos dentro de los límites especificados. Por ejemplo, si se define un límite de 3 nodos para un árbol binario, Korat generará todas las combinaciones posibles de árboles con 1, 2 o 3 nodos.
%
%\item \textbf{Filtrado y validación}: Cada configuración generada es validada contra los invariantes especificados. Solo las configuraciones que cumplen con las restricciones son retenidas como casos de prueba válidos.
%
%\end{enumerate}
%
%Para ilustrar el funcionamiento de Korat, consideremos el siguiente ejemplo concreto:
%
%\subsubsection{Ejemplo: Árboles Binarios}
%Supongamos que queremos probar una función que opera sobre árboles binarios. Con Korat, podemos definir los siguientes límites y restricciones:
%\begin{itemize}
%\item Límite: Árboles con un máximo de 3 nodos.
%\item Invariantes:
%\begin{itemize}
%\item No hay ciclos en el árbol.
%\item Cada nodo tiene como máximo dos hijos.
%\item Todos los nodos están conectados (no hay subárboles desconectados).
%\end{itemize}
%\end{itemize}
%
%Korat generará todas las posibles configuraciones de árboles binarios con 1, 2 y 3 nodos que cumplan con estas restricciones. Por ejemplo:
%\begin{itemize}
%\item Un árbol con un solo nodo raíz.
%\item Un árbol con un nodo raíz y un hijo izquierdo.
%\item Un árbol con un nodo raíz, un hijo izquierdo y un hijo derecho.
%\end{itemize}
%
%Estos casos de prueba permiten verificar que la función bajo prueba maneja correctamente diferentes configuraciones de árboles binarios.
%
%\cacho{Pongo ejemplos de codigo? de repok? de dibujitos?}
%
%La generación exhaustiva acotada implementada por Korat ofrece varias ventajas:
%\begin{itemize}
%\item \textbf{Cobertura completa}: Garantiza que todas las configuraciones válidas dentro de los límites establecidos sean exploradas.
%\item \textbf{Detección de casos límite}: Al generar todas las combinaciones posibles, aumenta la probabilidad de detectar errores en casos extremos o inusuales.
%\item \textbf{Eficiencia}: Las técnicas de poda y filtrado evitan la generación de configucciones redundantes o inválidas, optimizando el proceso.
%\end{itemize}
%
%En esta tesis, Korat se utiliza como una herramienta de referencia para evaluar la efectividad de nuevas técnicas de generación de pruebas. Su enfoque sistemático y basado en especificaciones lo convierte en un estándar ideal para comparar la cobertura y la calidad de los casos de prueba generados. Además, su capacidad para explorar exhaustivamente el espacio de entradas dentro de límites acotados proporciona una base sólida para identificar deficiencias en otros métodos de generación de pruebas.
%
%%En resumen, la generación exhaustiva acotada de entradas representa un compromiso entre la cobertura exhaustiva y la factibilidad práctica, ofreciendo una estrategia eficaz para la prueba de APIs, especialmente en sistemas donde la calidad y la seguridad son primordiales.
%
%\pp{Hacer una única sección con estas dos}
%
%% Los enfoques de generación de pruebas automatizadas tienen como objetivo ayudar a los desarrolladores en tareas cruciales de prueba de software [TODO ???], como la generación automática o facilitar la creación de conjuntos de pruebas [TODO: ????] y la detección y reporte automáticos de fallas [TODO: ????] . Muchos de estos enfoques implican componentes aleatorios que evitan una exploración sistemática del espacio de comportamientos, pero mejoran la eficiencia de la generación de pruebas [TODO: ????] . Si bien estos enfoques han sido muy útiles para encontrar una gran cantidad de errores en el software, podrían perder la exploración de ciertos comportamientos defectuosos del software debido a su naturaleza aleatoria. Los enfoques alternativos tienen como objetivo explorar sistemáticamente un número muy grande de ejecuciones del software bajo prueba (SUT), con el objetivo de proporcionar garantías más sólidas sobre la ausencia de errores[TODO: ????] . Uno de estos enfoques es la generación exhaustiva acotada (BE) [TODO: ????] , que consiste en generar todas las estructuras factibles que se pueden construir utilizando dominios de datos acotados. Los objetivos comunes de los enfoques BE han sido implementaciones de estructuras de datos complejas y dinámicas con ricos y estructurados enlaces (por ejemplo, listas enlazadas, árboles, etc.). Los enfoques BE de caja negra [TODO: ????]  son los más utilizados y eficientes para probar software. Requieren que el usuario proporcione una especificación formal de las restricciones que las estructuras deben satisfacer, con mayor frecuencia una invariante de representación de la estructura (repOK), y los límites de los dominios de datos [TODO: ????] , a menudo llamados \emph{scope}. De este modo, los enfoques BE de caja negra generan todas las estructuras dentro de los ámbitos proporcionados que satisfacen repOK.
%% Varios estudios muestran que los enfoques BE son efectivos para revelar fallas en el software [TODO ???]. Además, la llamada hipótesis del cota pequeña [TODO ?], que establece que la mayoría de las fallas de software se pueden revelar ejecutando el SUT en "entradas pequeñas", sugiere que, si se utilizan ámbitos lo suficientemente grandes, los enfoques BE deberían ser capaces de revelar la mayoría (si no todas) las fallas en el SUT. El desafío que enfrentan los enfoques BE es cómo explorar eficientemente un gran espacio de búsqueda, que en el peor de los casos crece exponencialmente con respecto a los ámbitos. El espacio de búsqueda a menudo incluye un gran número de estructuras no válidas (que no satisfacen repOK) y estructuras isomórficas [TODO ??]. Por lo tanto, podar partes del espacio de búsqueda que involucran estructuras inválidas y redundantes es clave para hacer que los enfoques BE se escalen en la práctica [TODO ??]. Escribir especificaciones formales apropiadas para la generación de BE es una tarea desafiante y que consume mucho tiempo. Las especificaciones deben capturar precisamente el conjunto de restricciones previstas en las estructuras. Las especificaciones sobrerestringidas hacen que falte la generación de una parte de las estructuras válidas, lo que puede hacer que la etapa de prueba subsiguiente pierda la exploración de los comportamientos defectuosos del SUT. Las especificaciones subrestringidas pueden llevar a la generación de estructuras inválidas (es decir, estructuras que no cumplen con las restricciones previstas), lo que puede producir falsos negativos durante la prueba del SUT. Además, a veces el usuario tiene que tener en cuenta la forma en que opera el enfoque de generación y escribir las especificaciones de una manera muy específica, de manera que el enfoque pueda lograr un buen rendimiento [TODO ???]. Finalmente, tales especificaciones formales precisas rara vez están disponibles en el software, lo que dificulta la usabilidad de los enfoques BE de caja negra.
%

\section{Randoop: Generación aleatoria de tests guiada por retroalimentación}
\label{sec:feedback-directed-test-gen}

% \begin{lstlisting}[caption={Método generar pruebas},basicstyle=\ttfamily\scriptsize]
%     public Set<Test> generarPruebas(Set<Metodo> mds, int tLimite, int nPruebas) {
%         Set<Test> tests = new HashSet<>();
%         long inicio = System.currentTimeMillis();
        
%         while (System.currentTimeMillis() - inicio < tLimite 
%                && tests.size() < nPruebas) {
            
%             Metodo metSel = selMetodoAleatorio(mds);
%             List<Parametro> params = new ArrayList<>();
            
%             for (Parametro param : metSel.getParams()) {
%                 Tipo tipo = param.getTipo();
%                 Object valor;
                
%                 if (tipo.esPrimitivo()) {
%                     valor = genValorPrimitivo(tipo);
%                 } else {
%                     valor = genSecuenciaTests(tests, tipo);
%                 }
%                 params.add(new ParamInstanciado(param, valor));
%             }
%             tests.add(new Test(metSel, params));
%         }
%         return tests;
%     }
%     \end{lstlisting}
    

El testing es una técnica fundamental para mejorar la confiabilidad del software
y detectar errores en el código \cite{} \pp{Libros de testing}. 
Sin embargo, escribir los tests de manera manual es una tarea trabajosa, y
requiere un esfuerzo importante por parte de los desarrolladores. Es por esto
que en los últimos años la comunidad de la ingeniería de software ha puesto especial énfasis en el
el desarrollo de técnicas automáticas que permitan automatizar la generación de
tests \cite{} \pp{papers sobre herramientas de testing automático}. 
Uno de los enfoques más importantes para esta tarea es la generación aleatoria
de tests \cite{} \pp{papers de generación aleatoria}. 
%que explora el espacio de entrada del programa mediante la ejecución de combinaciones diversas de valores y secuencias de invocaciones. 
Dentro de estos enfoques, Randoop se destaca como una herramienta ampliamente
utilizada, debido a su capacidad para generar tests que han mostrado ser útiles para revelar
fallas en distintos tipos de software muy utilizados \cite{} \pp{papers de
Randoop que muestran que revela fallas}
\pp{Para algunas las referencias ver sección relevancia del problema de mi PICT}.

Randoop utiliza la API del SUT para generar secuencias de tests, es decir,
secuencias de invocaciones a métodos de la API. Randoop propone la técnica
de generación aleatoria con retroalimentación basada en la ejecución, que
consiste en ejecutar las secuencias de test y observar los resultados para 
determinar si se debe continuar extendiendo o no las secuencias de test. 
El algoritmo principal de Randoop se describe en la Figura \ref{} \pp{Poner una
figura para después citarlo en los próximos capítulos}.

\begin{algorithm}[H]
    \SetAlgoLined
    \KwIn{Conjunto de métodos $M_1,\ldots,M_n$, tiempo límite $S$, número deseado de tests $W$}
    \KwOut{Tests de regresión}
    
    $prev \leftarrow \emptyset$\;
    $tests \leftarrow \emptyset$\;
    
    \While{tiempo transcurrido $< S$ \textbf{y} $|tests| < W$}{
        Seleccionar aleatoriamente $M(p_1:T_1, \ldots, p_m:T_m) \in \{M_1,\ldots,M_n\}$\;
        \For{$p_i:T_i$ de $M$}{
            \If{$T_i$ es primitivo}{
                $S_i \leftarrow$ valor primitivo para $T_i$ tomado
                aleatoriamente de las semillas\;
            }\Else{
                $S_i \leftarrow$ secuencia aleatoria $\in prev$ que crea objeto de tipo $T_i$\;
            }
        }
        $test \leftarrow S_1; \ldots; S_m; M(v_1,\ldots,v_m)$\;
        $res \leftarrow ejecutar($test$)$\;
        \If{res = falla} {
            \Return{$\{ test \}$}\;
        }   
        \If{res = exito} {
            $prev \leftarrow prev \cup \{test\}$\;
        }        
        \If{res = inválido} {
            // Ignorar el test 
        }        
        $tests \leftarrow tests \cup \{test\}$\;
    }
    \Return{$tests$}\;
    \end{algorithm}

\textsf{Randoop} toma como entradas un conjunto de rutinas (por ejemplo, la API
de un módulo), y criterios de terminación para la generación de tests, como el tiempo
máximo de generación \emph{S}, y/o la cantidad máxima de tests a generar \emph{W}.
Como salida, \textsf{Randoop} produce tests de regresión para el módulo, que
ejercitan los métodos dados.

\textsf{Randoop} representa los tests como secuencias de invocaciones a 
rutinas. Estas se denominan \emph{secuencias de
test}. \textsf{Randoop} comienza con unos pocos valores ``semilla''
predeterminados para los tipos primitivos de Java, que se utilizarán para
instanciar los parámetros de tipos primitivos de las rutinas durante la
generación. El usuario también puede proporcionar valores semilla adicionales
para los tipos primitivos. 

Inicialmente, se inicializan dos conjuntos de secuencias de test (líneas 1 y 2). 
\emph{prev} almacena las secuencias de test generadas en iteraciones previas cuya
ejecución es exitosa, es decir, las secuencias de test que el algoritmo intentará
extender con métodos adicionales. \emph{tests} se utiliza para guardar los tests
de regresión que produce el algoritmo como resultado.

\textsf{Randoop} construye secuencias de test de manera
iterativa, hasta que se alcance alguno de los criterios de terminación
provistos por el usuario, en su ciclo principal (líneas 3-18). 
Cada iteración consiste en seleccionar aleatoriamente una rutina \texttt{M},
para usarla en la creación de una nueva secuencia de test (línea 4). 
Para lograr esto, \textsf{Randoop} debe seleccionar secuencias \texttt{S$_1$,..,S$_k$}
que le permitan crear valores (del tipo correcto) para instanciar los parámetros 
\texttt{p$_1$,..,p$_m$} de \texttt{M} (lineas 5-25). 
Si el parámetro \texttt{p$_i$} es de tipo primitivo, el algoritmo simplemente
crea una secuencia de test que produce un valor primitivo, eligiendo
aleatoriamente un valor entre las semillas del tipo (líneas 6-8). 
Por ejemplo, para \texttt{p$_i$} de tipo entero, 
\textsf{Randoop} puede crear la secuencia \texttt{S$_i$} definida como 
\texttt{int vi = -1;}.
En caso de que \texttt{p$_i$} sea de tipo referencia, el algoritmo elige
aleatoriamente una secuencia \texttt{S$_i$} que sirve para crear un objeto del
tipo de \texttt{p$_i$}, tomando \texttt{S$_i$} del conjunto de secuencias generadas 
previamente (\texttt{prev}) (líneas 9-11).

De esta manera, la nueva secuencia de test, \texttt{test}, se crea componiendo
secuencialmente las secuencias \texttt{S$_1$,..,S$_m$} y \texttt{M},
reemplazando los parámetros formales de \texttt{M} por las variables
correspondientes en \texttt{S$_1$,..,S$_m$} (línea 13). 

\pp{TODO: voy por acá}
\texttt{T} podría ser una prueba ilegal debido al uso no válido de la API y/o fallas en el código en prueba. También podría ser redundante en el sentido de que no produce ninguna estructura para ejercitar el código de manera diferente que las pruebas generadas anteriormente.

La idea principal de la generación de pruebas dirigida por retroalimentación es ejecutar \texttt{T} y observar su comportamiento en tiempo de ejecución. Si \texttt{T} es una prueba legal, es decir, su ejecución termina con éxito,
y \texttt{T} no es redundante, entonces, \texttt{T} se agrega al conjunto de pruebas generadas previamente. Esto hace que \texttt{T} sea un candidato para su posible extensión para generar nuevas pruebas cuando se seleccione un nuevo método en una iteración futura del algoritmo. De lo contrario, si \texttt{T} es una prueba ilegal, por ejemplo, porque genera una excepción o viola un contrato, o si es redundante (consulte la sección~\ref{sec:state-matching})), entonces \texttt{T} no se considera para su posible extensión. \textsf{Randoop} continúa este proceso iterativo hasta que agota su presupuesto. Las evaluaciones experimentales han encontrado que la generación de pruebas dirigida por retroalimentación funciona significativamente mejor que la generación aleatoria simple de pruebas \cite{Pacheco07,Pacheco08}.
La principal ventaja de \textsf{Randoop} es su capacidad para generar automáticamente una gran cantidad de casos de prueba que cubren diferentes partes del programa bajo prueba. Esto permite detectar errores y comportamientos inesperados que pueden surgir en situaciones de entrada no contempladas durante el desarrollo manual de pruebas.

\pp{Poner ejemplo de ejecución de Randoop ilustrando el algoritmo acá abajo.}

%Una de las técnicas que presentamos en esta tesis, \textsf{BEAPI}, se basa en \textsf{Randoop}, pero en lugar de la generación aleatoria, tiene como objetivo generar \emph{todas} las secuencias de pruebas factibles (acotadas), dentro de un alcance dado. La retroalimentación de la ejecución permite a \textsf{BEAPI} descartar pruebas ilegales durante el proceso de generación. Además, al estar basado en una generación de pruebas dirigida por retroalimentación, \textsf{BEAPI} genera una secuencia de prueba para cada estructura en el conjunto exhaustivo acotado que construye y solo guarda los conjuntos de secuencias de llamadas a métodos que generan objetos válidos.
%

%Uno de los enfoques que presentamos en esta tesis, BEAPI, toma como base la idea de Randoop pero introduce una diferencia clave: en lugar de generar pruebas de manera aleatoria, 
%BEAPI busca generar todas las secuencias de prueba factibles dentro de un alcance acotado. Es decir, en lugar de depender de la aleatoriedad, 
%BEAPI explora sistemáticamente el espacio de búsqueda, asegurando una mayor cobertura.
%Además, BEAPI aprovecha la retroalimentación de la ejecución para descartar pruebas ilegales de manera eficiente, 
%reduciendo el número de combinaciones a evaluar. También genera una única secuencia de prueba para cada estructura válida, 
%evitando la repetición innecesaria de casos similares.



% \section{Algoritmos Perezosos? (Greedy)}
% \label{sec:greedyPrev}
% El algoritmo Greedy es una estrategia de resolución de problemas que sigue un enfoque voraz, tomando decisiones locales óptimas en cada etapa con la esperanza de llegar a una solución global óptima. A diferencia de otros enfoques, el algoritmo Greedy no realiza una búsqueda exhaustiva en el espacio de soluciones, sino que se centra en elegir la mejor opción disponible en cada momento.

% El enfoque Greedy se basa en la idea de que, al tomar decisiones óptimas en cada etapa, se puede obtener una solución aproximadamente óptima para el problema en general. Sin embargo, debido a su naturaleza voraz, el algoritmo Greedy puede no garantizar una solución óptima en todos los casos. En algunos casos, el algoritmo puede llegar a un óptimo local, pero no a un óptimo global. Por lo tanto, es importante tener en cuenta las limitaciones y restricciones del problema al aplicar el algoritmo Greedy.

% \begin{algorithm}[H]
%   \SetAlgoLined
%   \SetKwInOut{Input}{Input}
%   \SetKwInOut{Output}{Output}
%   \SetKwFunction{Greedy}{Greedy}

%   \caption{Greedy Algorithm}

%   \Input{Set $S$}
%   \Output{Solution $R$}

%   \BlankLine
%   $R \gets \emptyset$ 
%   $U \gets S$ \tcp*{Set of available elements}

%   \BlankLine
%   \While{$U \neq \emptyset$}{
%     $x \gets$ element selected according to Greedy criteria
    
%     $R \gets R \cup \{x\}$ \tcp*{Add $x$ to the solution}
%     $U \gets U \setminus \{x\}$ \tcp*{Remove $x$ from available elements}
%   }

%   \BlankLine
%   \Return $R$ 

% \end{algorithm}

% El algoritmo comienza con un conjunto de elementos disponibles y un conjunto solución vacío. En cada iteración, se selecciona el elemento óptimo según la función de heurística para los estados que se haya implementado y se agrega a la solución. Luego, se elimina ese elemento de los disponibles y se repite el proceso hasta que no queden elementos disponibles.

% Es importante tener en cuenta que el algoritmo Greedy no garantiza encontrar la solución óptima en todos los casos, ya que puede quedarse atrapado en mínimos locales. Sin embargo, en muchos casos, el enfoque Greedy proporciona soluciones rápidas y razonables.

% En esta tesis se personalizó el algoritmo para detectar subconjuntos óptimos de métodos \emph{builders}. 
% Para mas informacion sobre este tipo de algoritmos, invitamos al lector referirse a \cite{Cormen2009}, \cite{kleinberg2006}



\section{Cobertura de Código y Mutación}
\label{sec:coverage}
La evaluación de la calidad y efectividad de las técnicas de generación de pruebas en el campo de la ingeniería de software es un aspecto fundamental para garantizar la confiabilidad y robustez de los sistemas. Dos criterios ampliamente utilizados para evaluar la calidad de las pruebas son la \emph{cobertura de líneas}, la \emph{cobertura de ramas} y \emph{mutacion} (cobertura de mutantes matado. Estas técnicas proporcionan métricas objetivas para medir la exhaustividad de las pruebas y la capacidad para detectar errores en el software.

La cobertura de líneas es una métrica utilizada para medir qué porcentaje de las líneas de código ha sido ejecutado durante las pruebas. 
Esta técnica evalúa la exhaustividad de las pruebas al determinar si cada línea de código ha sido ejecutada al menos una vez. 
Sin embargo, la cobertura de líneas no mide si se han ejecutado todas las rutas posibles en el programa, lo que puede limitar su capacidad para detectar errores en situaciones complejas.

La cobertura de ramas es una métrica utilizada para medir qué porcentaje de las posibles rutas de ejecución en un programa ha sido cubierto por las pruebas. En otras palabras, se refiere a la medida en que se han ejecutado todas las ramas de decisión (if-else, switch-case, bucles, etc.) en el código fuente durante la ejecución de las pruebas. La cobertura de ramas es especialmente útil para identificar áreas del código que no han sido ejercitadas y que podrían contener errores lógicos o comportamientos no deseados.
Al lograr una alta cobertura de ramas, se aumenta la confianza en la calidad de las pruebas realizadas, ya que se ha examinado exhaustivamente la lógica del programa en diferentes condiciones de ejecución. Sin embargo, es importante destacar que la cobertura de ramas no garantiza la ausencia de errores, ya que no evalúa la calidad o validez de las pruebas en sí mismas. Es decir, una alta cobertura de ramas indica que se han ejecutado diferentes rutas de ejecución, pero no necesariamente que se hayan utilizado entradas significativas o se hayan probado casos límite relevantes.

La mutación es una técnica avanzada de evaluación de pruebas que se utiliza para medir la capacidad de un conjunto de pruebas para detectar cambios sintácticos o semánticos en el código fuente. Consiste en introducir cambios deliberados, llamados mutantes, en el programa original y luego ejecutar el conjunto de pruebas sobre estos mutantes. El objetivo es evaluar si las pruebas son lo suficientemente sensibles como para detectar estos cambios y, por lo tanto, indicar la presencia de posibles errores en el código.
Cada mutante representa una posible variación o error que podría haberse introducido en el código original. Si una prueba es capaz de detectar un mutante, se considera que la prueba ha "matado" al mutante, lo que indica que el conjunto de pruebas tiene la capacidad de encontrar y reportar posibles errores. Por otro lado, si un mutante no es detectado por ninguna prueba, se considera que el conjunto de pruebas es inadecuado y podría necesitar mejoras para aumentar su capacidad para encontrar errores.
La evaluación de la capacidad de detección de mutantes proporciona información valiosa sobre la efectividad de las pruebas en términos de su capacidad para encontrar posibles errores. Al identificar los mutantes que no son detectados por las pruebas, se pueden identificar deficiencias en el conjunto de pruebas y se pueden tomar medidas correctivas para mejorar su calidad.

Estas técnicas se aplican en esta tesis para evaluar las nuevas técnicas implementadas y poder realizar una comparación con herramientas que se utilizan en el estado de arte, como \emph{Randoop} y \emph{Korat}. Mediante la utilización de estas técnicas, se busca mejorar la confiabilidad y calidad del análisis de software, permitiendo a los investigadores y profesionales tomar decisiones más informadas sobre las técnicas de generación de pruebas a utilizar. Al proporcionar medidas comparativas de cobertura de ramas y evaluación de mutantes, se puede determinar la efectividad de las nuevas herramientas y enfoques propuestos, y se pueden identificar deficiencias en las pruebas existentes. Esto conduce a mejoras en la calidad y confiabilidad de los sistemas de software, lo que es fundamental en el desarrollo de software de alta calidad.

